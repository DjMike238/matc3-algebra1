% (c) 2012-2013 Claudio Carboncini - claudio.carboncini@gmail.com
% (c) 2012-2014 Dimitrios Vrettos - d.vrettos@gmail.com

\chapter{Equazioni di grado superiore al primo riducibili al primo grado}

Nel capitolo \ref{cap:equazioni_I_grado} abbiamo affrontato le equazioni di primo grado. Adesso consideriamo le equazioni di grado superiore al primo che possono essere ricondotte ad equazioni di primo grado,
utilizzando la legge di annullamento del prodotto (legge \ref{legge:annullamento_del_prodotto} a pagina \pageref{legge:annullamento_del_prodotto}).

\begin{exrig}
 \begin{esempio}
Risolvere~$x^{2}-4=0$.

Il polinomio al primo membro può essere scomposto in fattori:~$(x-2)(x+2)=0$.
Per la legge di annullamento, il prodotto dei due binomi si annulla se~$x-2=0$ oppure se~$x+2=0$.
Di conseguenza si avranno le soluzioni:~$x=2$ e $x=-2$.
 \end{esempio}
\end{exrig}

In generale, se si ha un'equazione di grado~$n$ scritta in forma normale~$P(x)=0$ e se il polinomio~$P(x)$ è
fattorizzabile nel prodotto di~$n$ fattori di primo grado:
\begin{equation*}
(x-a_{1})(x-a_{2})(x-a_{3})\ldots (x-a_{n-1})(x-a_{n})=0
\end{equation*}
applicando la legge di annullamento del prodotto, le soluzioni dell'equazione si ottengono determinando le soluzioni delle singole~$n$
equazioni di primo grado, cioè risolvendo:
\begin{equation*}
x-a_{1}=0\text{,~~~}x-a_{2}=0\text{,~~~}x-a_{3}=0\text{,~~~}\ldots\text{,~~~}x-a_{n-1}=0\text{,~~~}x-a_{n}=0.
\end{equation*}
Pertanto l'insieme delle soluzioni dell'equazione data sarà:~$\IS=\{a_{1}\text{,~}a_{2}\text{,~}a_{3}\text{,~}\ldots\text{,~}a_{n-1}\text{,~}a_{n}\}$.

\begin{exrig}
 \begin{esempio}
Risolvere~$x^{2}-x-2=0$.

Scomponendo in fattori il polinomio al primo membro, ricercando quei due numeri la cui somma è pari a~$-1$ e il cui prodotto è pari a~$-2$, 
si ha:~$(x+1)(x-2)=0$.
Utilizzando la legge di annullamento del prodotto, si ottiene il seguente insieme di soluzioni:~$\IS=\{-1\text{,~}2\}$.
 \end{esempio}

 \begin{esempio}
Risolvere~$x^{4}-5x^{2}+4=0$.

Scomponendo in fattori il polinomio al primo membro, utilizzando la regola della scomposizione del particolare trinomio di secondo grado,
si ottiene:~$(x^{2}-1)(x^{2}-4)=0$. Scomponendo ulteriormente in fattori si ha:
\begin{equation*}
(x-1)(x+1)(x-2)(x+2)=0.
\end{equation*}
Per la legge di annullamento del prodotto è necessario risolvere le equazioni:
\begin{equation*}
x-1=0\: \Rightarrow\: x=1\text{,}\quad x+1=0\: \Rightarrow\: x=-1\text{,}\quad x-2=0\: \Rightarrow\: x=2\text{,}\quad x+2=0\: \Rightarrow\: x=-2.
\end{equation*}
L'insieme delle soluzioni:~$\IS=\{+1\text{,~}-1\text{,~}+2\text{,~}-2\}$.
 \end{esempio}

\end{exrig}

\ovalbox{\risolvii \ref{ese:17.1}, \ref{ese:17.2}, \ref{ese:17.3}, \ref{ese:17.4}, \ref{ese:17.5}, \ref{ese:17.6}, \ref{ese:17.7}, \ref{ese:17.8},
\ref{ese:17.9}, \ref{ese:17.10}}

\vspazio\ovalbox{\ref{ese:17.11}, \ref{ese:17.12}, \ref{ese:17.13}}

\newpage
% (c) 2012-2013 Claudio Carboncini - claudio.carboncini@gmail.com
% (c) 2012-2014 Dimitrios Vrettos - d.vrettos@gmail.com

\section{Esercizi}

%\subsection{Esercizi dei singoli paragrafi}

%\subsubsection*{17.1 - Equazioni di grado superiore al primo riducibili al primo grado}

\begin{esercizio}[\Ast]
\label{ese:17.1}
Risolvere le seguenti equazioni riconducendole a equazioni di primo grado.
\begin{multicols}{2}
\begin{enumeratea}
 \item $x^{2}+2x=0$;
 \item $x^{2}+2x-9x-18=0$;
 \item $2x^{2}-2x-4=0$;
 \item $4x^{2}+16x+16=0$.
\end{enumeratea}
\end{multicols}
\end{esercizio}

\begin{esercizio}[\Ast]
\label{ese:17.2}
Risolvere le seguenti equazioni riconducendole a equazioni di primo grado.
\begin{multicols}{2}
\begin{enumeratea}
 \item $x^{2}-3x-10=0$;
 \item $x^{2}+4x-12=0$;
 \item $3x^{2}-6x-9=0$;
 \item $x^{2}+5x-14=0$.
\end{enumeratea}
\end{multicols}
\end{esercizio}

\begin{esercizio}[\Ast]
\label{ese:17.3}
Risolvere le seguenti equazioni riconducendole a equazioni di primo grado.
\begin{multicols}{2}
\begin{enumeratea}
 \item $-3x^{2}-9x+30=0$;
 \item $-{\dfrac{3}{2}}x^{2}+\dfrac{3}{2}x+63=0$;
 \item $7x^{2}+14x-168=0$;
 \item $\dfrac{7}{2}x^{2}+7x-168=0$.
\end{enumeratea}
\end{multicols}
\end{esercizio}

\begin{esercizio}[\Ast]
\label{ese:17.4}
Risolvere le seguenti equazioni riconducendole a equazioni di primo grado.
\begin{multicols}{2}
\begin{enumeratea}
 \item $x^{4}-16x^{2}=0$;
 \item $2x^{3}+2x^{2}-20x+16=0$;
 \item $-2x^{3}+6x+4=0$;
 \item $-x^{6}+7x^{5}-10x^{4}=0$.
\end{enumeratea}
\end{multicols}
\end{esercizio}

\begin{esercizio}[\Ast]
\label{ese:17.5}
Risolvere le seguenti equazioni riconducendole a equazioni di primo grado.
\begin{multicols}{2}
\begin{enumeratea}
 \item $x^{3}-3x^{2}-13x+15=0$;
 \item $x^{2}+10x-24=0$;
 \item $2x^{3}-2x^{2}-24x=0$;
 \item $x^{4}-5x^{2}+4=0$.
\end{enumeratea}
\end{multicols}
\end{esercizio}

\begin{esercizio}[\Ast]
\label{ese:17.6}
Risolvere le seguenti equazioni riconducendole a equazioni di primo grado.
\begin{multicols}{2}
\begin{enumeratea}
 \item $-x^{3}-5x^{2}-x-5=0$;
 \item $\dfrac{3}{4}x^{3}-\dfrac{3}{4}x=0$;
 \item $-4x^{4}-28x^{3}+32x^{2}=0$;
 \item $-{\dfrac{6}{5}}x^{3}-\dfrac{6}{5}x^{2}+\dfrac{54}{5}x+\dfrac{54}{5}=0$.
\end{enumeratea}
\end{multicols}
\end{esercizio}

\begin{esercizio}[\Ast]
\label{ese:17.7}
Risolvere le seguenti equazioni riconducendole a equazioni di primo grado.
\begin{multicols}{2}
\begin{enumeratea}
 \item $-4x^{3}+20x^{2}+164x-180=0$;
 \item $5x^{3}+5x^{2}-80x-80=0$;
 \item $-3x^{3}+18x^{2}+3x-18=0$;
 \item $4x^{3}+8x^{2}-16x-32=0$.
\end{enumeratea}
\end{multicols}
\end{esercizio}

\begin{esercizio}[\Ast]
\label{ese:17.8}
Risolvere le seguenti equazioni riconducendole a equazioni di primo grado.
\begin{multicols}{2}
\begin{enumeratea}
 \item $x^{3}+11x^{2}+26x+16=0$;
 \item $2x^{3}+6x^{2}-32x-96=0$;
 \item $2x^{3}+16x^{2}-2x-16=0$;
 \item $-2x^{3}+14x^{2}-8x+56=0$.
\end{enumeratea}
\end{multicols}
\end{esercizio}
%\newpage
\begin{esercizio}[\Ast]
\label{ese:17.9}
Risolvere le seguenti equazioni riconducendole a equazioni di primo grado.
\begin{multicols}{2}
\begin{enumeratea}
 \item $2x^{3}+12x^{2}+18x+108=0$;
 \item $x^{4}-10x^{3}+35x^{2}-50x+24=0$;
 \item $-2x^{3}-12x^{2}+18x+28=0$;
 \item $-5x^{4}+125x^{2}+10x^{3}-10x-120=0$.
\end{enumeratea}
\end{multicols}
\end{esercizio}

\begin{esercizio}[\Ast]
\label{ese:17.10}
Risolvere le seguenti equazioni riconducendole a equazioni di primo grado.
\begin{multicols}{2}
\begin{enumeratea}
 \item $\dfrac{7}{6}x^{4}-\dfrac{161}{6}x^{2}-21x+\dfrac{140}{3}=0$;
 \item $(x^{2}-6x+8)(x^{5}-3x^{4}+2x^{3})=0$;
 \item $\left(25-4x^{2}\right)^{4}\left(3x-2\right)^{2}=0$;
 \item $(x-4)^{3}\left(2x^{3}-4x^{2}-8x+16\right)^{9}=0$.
\end{enumeratea}
\end{multicols}
\end{esercizio}

\begin{esercizio}[\Ast]
\label{ese:17.11}
Risolvere le seguenti equazioni riconducendole a equazioni di primo grado.
\begin{multicols}{2}
\begin{enumeratea}
 \item $(x^{3}-x)(x^{5}-9x^{3})(x^{2}+25)=0$;
 \item $x^{5}+3x^{4}-11x^{3}-27x^{2}+10x+24=0$;
 \item $2x^{2}-x-1=0$;
 \item $3x^{2}+5x-2=0$.
\end{enumeratea}
\end{multicols}
\end{esercizio}

\begin{esercizio}[\Ast]
\label{ese:17.12}
Risolvere le seguenti equazioni riconducendole a equazioni di primo grado.
\begin{multicols}{2}
\begin{enumeratea}
 \item $6x^{2}+x-2=0$;
 \item $2x^{3}-x^{2}-2x+1=0$;
 \item $3x^{3}-x^{2}-8x-4=0$;
 \item $8x^{3}+6x^{2}-5x-3=0$.
\end{enumeratea}
\end{multicols}
\end{esercizio}

\begin{esercizio}[\Ast]
\label{ese:17.13}
Risolvere le seguenti equazioni riconducendole a equazioni di primo grado.
\begin{multicols}{2}
\begin{enumeratea}
 \item $6x^{3}+x^{2}-10x+3=0$;
 \item $4x^{4}-8x^{3}-13x^{2}+2x+3=0$;
 \item $8x^{4}-10x^{3}-29x^{2}+40x-12=0$;
 \item $-12x^{3}+68x^{2}-41x+5=0$.
\end{enumeratea}
\end{multicols}
\end{esercizio}

\begin{esercizio}[\Ast]
\label{ese:17.14}
Risolvere la seguente equazione riconducendola a una equazione di primo grado.

$(x^{4}+3x^{3}-3x^{2}-11x-6)(4x^{6}-216x^{3}+2916)=0$;
\end{esercizio}

\subsection{Risposte}
%\begin{multicols}{2}
 \paragraph{17.1.}
a)~$\{0\text{,~}-2\}$;\quad b)~$\{-2\text{,~}+9\}$;\quad c)~$\{2\text{,~}-1\}$;\quad d)~$\{-2\}$.
\paragraph{17.2.}
a)~$\{5\text{,~}-2\}$;\quad b)~$\{2\text{,~}-6\}$;\quad c)~$\{3\text{,~}-1\}$;\quad d)~$\{2\text{,~}-7\}$.
\paragraph{17.3.}
a)~$\{2\text{,~}-5\}$;\quad b)~$\{7\text{,~}-6\}$;\quad c)~$\{4\text{,~}-6\}$;\quad d)~$\{6\text{,~}-8\}$.
\paragraph{17.4.}
a)~$\{0\text{,~}+4\text{,~}-4\}$;\quad b)~$\{1\text{,~}+2\text{,~}-4\}$;\quad c)~$\{2\text{,~}-1\}$;\quad d)~$\{0\text{,~}+2\text{,~}+5\}$.
\paragraph{17.5.}
a)~$\{1\text{,~}+5\text{,~}-3\}$;\quad b)~$\{2\text{,~}-12\}$;\quad c)~$\{0\text{,~}-3\text{,~}+4\}$;\quad d)~$\{1\text{,~}-1\text{,~}+2\text{,~}-2\}$.
\paragraph{17.6.}
a)~$\{-5\}$;\quad b)~$\{0\text{,~}+1\text{,~}-1\}$;\quad c)~$\{0\text{,~}+1\text{,~}-8\}$;\quad d)~$\{-1\text{,~}+3\text{,~}-3\}$.
\paragraph{17.7.}
a)~$\{1\text{,~}+9\text{,~}-5\}$;\quad b)~$\{-1\text{,~}+4\text{,~}-4\}$;\quad c)~$\{1\text{,~}-1\text{,~}+6\}$;\quad d)~$\{2\text{,~}-2\}$.
\paragraph{17.8.}
a)~$\{-1\text{,~}-2\text{,~}-8\}$;\quad b)~$\{4\text{,~}-4\text{,~}-3\}$,\quad c)~$\{1\text{,~}-1\text{,~}-8\}$;\quad d)~$\{7\}$.
\paragraph{17.9.}
a)~$\{-6\}$;\quad b)~$\{1\text{,~}2\text{,~}3\text{,~}4\}$;\quad c)~$\{-1\text{,~}2\text{,~}-7\}$;\quad d)~$\{1\text{,~}-1\text{,~}-4\text{,~}+6\}$.

\paragraph{17.10.}
a)~$\{1\text{,~}-2\text{,~}+5\text{,~}-4\}$;\quad b)~$\{0\text{,~}1\text{,~}2\text{,~}4\}$;\quad c)~$\left\{\frac{5}{2}\text{,~}-\frac{5}{2}\text{,~}\frac{2}{3}\right\}$;\quad d)~$\{4\text{,~}+2\text{,~}-2\}$.

\paragraph{17.11.}
a)~$\{0\text{,~}1\text{,~}-1\text{,~}3\text{,~}-3\}$;\quad
b)~$\{1\text{,~}{-1}\text{,~}{-2}\text{,~}3\text{,~}{-4}\}$;\quad c)~$\left\{1\text{,~}-\frac{1}{2}\right\}$;\quad d)~$\left\{-2\text{,~}\frac{1}{3}\right\}$.

\paragraph{17.12.}
a)~$\left\{\frac{1}{2}\text{,~}-\frac{2}{3}\right\}$;\quad b)~$\left\{1\text{,~}-1\text{,~}\frac{1}{2}\right\}$;\quad c)~$\left\{-1\text{,~}2\text{,~}-\frac{2}{3}\right\}$;\quad d)~$\left\{-1\text{,~}-\frac{1}{2}\text{,~}\frac{3}{4}\right\}$.

\paragraph{17.13.}
a)~$\left\{1\text{,~}\frac{1}{3}\text{,~}-\frac{3}{2}\right\}$;\quad b)~$\left\{3\text{,~}-1\text{,~}\frac{1}{2}\text{,~}-\frac{1}{2}\right\}$;\quad c)~$\left\{2\text{,~}-2\text{,~}\frac{3}{4}\text{,~}\frac{1}{2}\right\}$;\quad d)~$\left\{5\text{,~}\frac{1}{2}\text{,~}\frac{1}{6}\right\}$.

\paragraph{17.14.}
$\{-1\text{,~}+2\text{,~}+3\text{,~}-3\}$.
%\end{multicols}
\cleardoublepage
