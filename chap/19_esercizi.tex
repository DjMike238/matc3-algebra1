% (c) 2012-2013 Claudio Carboncini - claudio.carboncini@gmail.com
% (c) 2012-2014 Dimitrios Vrettos - d.vrettos@gmail.com

\section{Esercizi}

\subsection{Esercizi dei singoli paragrafi}

\subsubsection*{19.1 - Equazioni letterali con un parametro}

\begin{esercizio}[\Ast]
\label{ese:19.1}
Risolvi e discuti le seguenti equazioni letterali nell'incognita~$x$.
\begin{multicols}{2}
\begin{enumeratea}
 \item $1+2x=a+1-2x$;
 \item $2x-\dfrac{7}{2}=ax-5$;
 \item $b^{2}x=2b+bx$;
 \item $ax+2=x+3$.
\end{enumeratea}
\end{multicols}
\end{esercizio}

\begin{esercizio}[\Ast]
\label{ese:19.2}
Risolvi e discuti le seguenti equazioni letterali nell'incognita~$x$.
\begin{multicols}{2}
\begin{enumeratea}
 \item $k(x+2)=k+2$;
 \item $(b+1)(x+1)=0$;
 \item $k^{2}x+2k=x+2$;
 \item $(a-1)(x+1)=x+1$.
\end{enumeratea}
\end{multicols}
\end{esercizio}

\begin{esercizio}
\label{ese:19.3}
Risolvi e discuti le seguenti equazioni letterali nell'incognita~$x$.
\begin{multicols}{2}
\begin{enumeratea}
 \item $ax+x-2a^{2}-2ax=0$;
 \item $3ax-2a=x\cdot (1-2a)+a\cdot (x-1)$;
 \item $x (3-5a)+2 (a-1)=(a-1) (a+1)$;
 \item $x+2a\cdot (x-2a)+1=0$.
\end{enumeratea}
\end{multicols}
\end{esercizio}
%\newpage
\begin{esercizio}[\Ast]
\label{ese:19.4}
Risolvi e discuti le seguenti equazioni letterali nell'incognita~$x$.
\begin{multicols}{2}
\begin{enumeratea}
 \item $(a-1)(x+1)=a-1$;
 \item $2k(x+1)-2=k(x+2)$;
 \item $a(a-1)x=2a(x-5)$;
 \item $3ax+a=2a^{2}-3a$.
\end{enumeratea}
\end{multicols}
\end{esercizio}

\begin{esercizio}[\Ast]
\label{ese:19.5}
Risolvi e discuti le seguenti equazioni letterali nell'incognita~$x$.
\begin{multicols}{2}
\begin{enumeratea}
 \item $3x-a=a(x-3)+6$;
 \item $2+2x=3ax+a-a^{2}x$;
 \item $x(a^{2}-4)=a+2$;
 \item $(x-m)(x+m)=(x+1)(x-1)$.
\end{enumeratea}
\end{multicols}
\end{esercizio}

\begin{esercizio}[\Ast]
\label{ese:19.6}
Risolvi e discuti le seguenti equazioni letterali nell'incognita~$x$.
\begin{multicols}{2}
\begin{enumeratea}
 \item $(a-2)^{2}x+(a-2)x+a-2=0$;
 \item $\left(9a^{2}-4\right)x=2(x+1)$;
 \item $(a-1)x=a^{2}-1$;
 \item $(a+2)x=a^{2}+a-1$.
\end{enumeratea}
\end{multicols}
\end{esercizio}

\begin{esercizio}[\Ast]
\label{ese:19.7}
Risolvi e discuti le seguenti equazioni letterali nell'incognita~$x$.
\begin{multicols}{2}
\begin{enumeratea}
 \item $a(x-1)^{2}=a(x^{2}-1)+2a$;
 \item $a^{3}x-a^{2}-4ax+4=0$;
 \item $bx\left(b^{2}+1\right)-(bx-1)\left(b^{2}-1\right)=2b^{2}$;
 \item $a(a-5)x+a(a+1)=-6(x-1)$:
 \item $(x+a)^{2}-(x-a)^{2}+(a-4)(a+4)=a^{2}$;
 \item $b(b+3)+x\left(6-b^{2}\right)=bx$.
\end{enumeratea}
\end{multicols}
\end{esercizio}
\pagebreak
\subsubsection*{19.2 - Equazioni letterali con due parametri}

\begin{esercizio}[\Ast]
\label{ese:19.8}
Risolvi e discuti le seguenti equazioni nell'incognita~$x$ con due parametri.
\begin{multicols}{2}
\begin{enumeratea}
 \item $(m+1)(n-2)x=0$;
 \item $m(x-1)=n$;
 \item $(a+1)(b+1)x=0$;
 \item $(m+n)(x-1)=m-n$.
\end{enumeratea}
\end{multicols}
\end{esercizio}

\begin{esercizio}[\Ast]
\label{ese:19.9}
Risolvi e discuti le seguenti equazioni nell'incognita~$x$ con due parametri.
\begin{multicols}{2}
\begin{enumeratea}
 \item $x(2a-1)+2b(x-2)=-4a-x$;
 \item $ax-3+b=2(x+b)$;
 \item $(a+1)x=b+1$;
 \item $(a+b)(x-2)+3a-2b=2b(x-1)$.
\end{enumeratea}
\end{multicols}
\end{esercizio}

\begin{esercizio}[\Ast]
\label{ese:19.10}
Risolvi e discuti le seguenti equazioni nell'incognita~$x$ con due parametri.
\begin{enumeratea}
 \item $x(x+2)+3ax=b+x^{2}$;
 \item $(x-a)^{2}+b(2b+1)=(x-2a)^{2}+b-3a^{2}$.
\end{enumeratea}
\end{esercizio}

\subsubsection*{19.3 - Equazioni con il parametro al denominatore}

\begin{esercizio}[\Ast]
\label{ese:19.11}
Risolvi e discuti le seguenti equazioni che presentano il parametro al denominatore.
\begin{multicols}{2}
\begin{enumeratea}
 \item $\dfrac{x+2}{6a}+\dfrac{x-1}{2a^{2}}=\dfrac{1}{3a}$;
 \item $\dfrac{x-1}{b}+\dfrac{2x+3}{4b}=\dfrac{x}{4}$;
 \item $\dfrac{2x-1}{3a}+\dfrac{x}{3}=\dfrac{2}{a}$;
 \item $\dfrac{x}{a}+\dfrac{2x}{2-a}=\dfrac{a-x+2}{2a-a^{2}}$.
\end{enumeratea}
\end{multicols}
\end{esercizio}

\begin{esercizio}[\Ast]
\label{ese:19.12}
Risolvi e discuti le seguenti equazioni che presentano il parametro al denominatore.
\begin{multicols}{2}
\begin{enumeratea}
 \item $\dfrac{x}{a-1}+8=4a-\dfrac{x}{a-3}$;
 \item $\dfrac{x-1}{a-1}+\dfrac{x+a}{a}=\dfrac{a-1}{a}$;
 \item $\dfrac{a^{2}-9}{a+2}x=a-3$;
 \item $\dfrac{x+2}{a^{2}-2a}+\dfrac{x}{a^{2}+2a}+\dfrac{1}{a}=\dfrac{2}{a^{2}-4}$.
\end{enumeratea}
\end{multicols}
\end{esercizio}

\begin{esercizio}[\Ast]
\label{ese:19.13}
Risolvi e discuti le seguenti equazioni che presentano il parametro al denominatore.
\begin{enumeratea}
 \item $\dfrac{x+1}{a^{2}+2a+1}+\dfrac{2x+1}{a^{2}-a-2}-\dfrac{2x}{(a+1)(a-2)}+\dfrac{1}{a-2}=0$;
 \item $\dfrac{x+1}{a-5}+\dfrac{2x-1}{a-2}=\dfrac{2}{a^{2}-7a+10}$;
 \item $\dfrac{x+2}{b-2}+\dfrac{2}{b^{2}-4b+4}+\left(\dfrac{1}{b-2}+\dfrac{x}{b-1}\right)\cdot (b-1)=0$;
 \item $\dfrac{3+b^{3}x}{7b^{2}-b^{3}}+\dfrac{(2b^{2}+b)x+1}{b(b-7)}=\dfrac{3b^{2}x+1}{b^{2}}-2x$;
 \item $\dfrac{x-2}{t^{2}+3t}+\dfrac{x-1}{t+3}=\dfrac{x-2}{t^{2}}+\dfrac{1}{t+3}$;
 \item $\dfrac{x}{2a}+\dfrac{x+1}{1-2a}=\dfrac{1}{a}$.
\end{enumeratea}
\end{esercizio}
\pagebreak
\subsubsection*{19.4 - Equazioni frazionarie letterali}

\begin{esercizio}[\Ast]
\label{ese:19.14}
Risolvi e discuti le seguenti equazioni parametriche frazionarie.
\begin{multicols}{2}
\begin{enumeratea}
 \item $\dfrac{t-1}{x-2}=2t$;
 \item $\dfrac{x+m}{x+1}=1$;
 \item $\dfrac{3}{x+1}=2a-1$;
 \item $\dfrac{2a-x}{x-3}-\dfrac{ax+2}{9-3x}=0$.
\end{enumeratea}
\end{multicols}
\end{esercizio}

\begin{esercizio}[\Ast]
\label{ese:19.15}
Risolvi e discuti le seguenti equazioni parametriche frazionarie.
\begin{multicols}{2}
\begin{enumeratea}
 \item $\dfrac{k-1}{x}=\dfrac{2}{k+1}$;
 \item $\dfrac{k}{x+1}=\dfrac{2k}{x-1}$;
 \item $\dfrac{a-1}{x+3}-\dfrac{a}{2-x}=\dfrac{ax+3a}{x^{2}+x-6}$;
 \item $\dfrac{a}{x}=\dfrac{1}{a}$.
\end{enumeratea}
\end{multicols}
\end{esercizio}

\begin{esercizio}[\Ast]
\label{ese:19.16}
Risolvi e discuti le seguenti equazioni parametriche frazionarie.
\begin{enumeratea}
 \item $\dfrac{x-a}{x^{2}-1}-\dfrac{x+3a}{2x-x^{2}-1}=\dfrac{x+5}{x+1}-2\dfrac{x}{(x-1)^{2}}-1$;
 \item $\dfrac{3}{1+3x}+\dfrac{a}{3x-1}=\dfrac{a-5x}{1-9x^{2}}$;
 \item $\dfrac{2a}{x^{2}-x-2}+\dfrac{1}{3x^{2}+2x-1}=\dfrac{6a^{2}-13a-4}{3x^{3}-4x^{2}-5x+2}$;
 \item $\dfrac{a+1}{x+1}-\dfrac{2a}{x-2}=\dfrac{3-5a}{x^{2}-x-2}$.
\end{enumeratea}
\end{esercizio}

\begin{esercizio}[\Ast]
\label{ese:19.17}
Risolvi e discuti le seguenti equazioni parametriche frazionarie.
\begin{multicols}{2}
\begin{enumeratea}
 \item $\dfrac{a}{x+a}=1+a$;
 \item $\dfrac{x}{x-a}+\dfrac{1}{x+a}=1$;
 \item $\dfrac{x+a}{x-a}=\dfrac{x-a}{x+a}$;
 \item $\dfrac{2}{1-ax}+\dfrac{1}{2+ax}=0$.
\end{enumeratea}
\end{multicols}
\end{esercizio}

\begin{esercizio}[\Ast]
\label{ese:19.18}
Risolvi e discuti le seguenti equazioni parametriche frazionarie.
\begin{multicols}{2}
\begin{enumeratea}
 \item $\dfrac{2}{x-2}+\dfrac{a+1}{a-1}=0$;
 \item $\dfrac{1}{x+t}-\dfrac{1}{t+1}=\dfrac{tx}{tx+x+t^{2}+t}$;
 \item $\dfrac{tx}{x-2}+\dfrac{t^{2}}{t+1}-\dfrac{t}{x-2}=0$;
 \item $\dfrac{2x+1}{2x-1}=\dfrac{2a-1}{a+1}$.
\end{enumeratea}
\end{multicols}
\end{esercizio}

\begin{esercizio}
\label{ese:19.19}
Risolvi e discuti le seguenti equazioni parametriche frazionarie.
\begin{multicols}{2}
\begin{enumeratea}
 \item $\dfrac{a}{x+1}=\dfrac{3}{x-2}$;
 \item $\dfrac{x}{x+1}+\dfrac{x}{x-1}=\dfrac{bx}{1-x^{2}}+\dfrac{a+2x^{2}}{x^{2}-1}$;
 \item $\dfrac{2x+1}{x}+\dfrac{2x^{2}-3b^{2}}{bx-x^{2}}=\dfrac{1}{x-b}$;
 \item $\dfrac{x-1}{x+a}=2+\dfrac{1-x}{x-a}$.
\end{enumeratea}
\end{multicols}
\end{esercizio}

\subsubsection*{19.5 - Equazioni letterali e formule inverse}

\begin{esercizio}
\label{ese:19.20}
L'interesse~$I$ maturato da un capitale~$C$, al tasso di interesse annuo~$i$, per un numero di anni~$t$ è
\begin{equation*}
  I=C\cdot i\cdot t.
\end{equation*}

Ricava le formule per calcolare:~$C=\ldots\ldots\ldots\ldots$, $\quad i=\ldots\ldots\ldots\ldots$, $\quad t =\ldots\ldots\ldots\ldots$.

Se il capitale è \officialeuro~$\np{12000}$, il tasso di interesse annuo~$\np{3,5}\%$, il periodo di tempo è di~$6$ anni, calcola~$I$.
\end{esercizio}

\begin{esercizio}
\label{ese:19.21}
Conversione da gradi Celsius $C$ a gradi Fahrenheit $F$:
\begin{equation*}
  C=\frac{5(F-32)}{9}.
\end{equation*}

Ricava la formula per calcolare $F=\ldots\ldots\ldots\ldots$.

Calcola il valore di~$C$ quando~$F$ vale~$106$ e il valore di~$F$ quando~$C$ vale~$12$.
\end{esercizio}

\begin{esercizio}
\label{ese:19.22}
Il valore attuale~$V_a$ di una rendita che vale~$V_n$ dopo~$n$ anni al tasso di interesse~$i$, anticipata di~$t$ anni è
\begin{equation*}
  V_{a}=V_{n}\cdot (1-i\cdot t).
\end{equation*}

Ricava le formule per calcolare:~$V_n=\ldots\ldots\ldots\ldots$, $\quad i=\ldots\ldots\ldots\ldots$, $\quad t =\ldots\ldots\ldots\ldots$.

Se il valore attuale è \officialeuro~$\np{120000}$, il tasso di interesse il~$2\%$, calcola il valore della rendita dopo~$20$ anni.
\end{esercizio}

\begin{esercizio}
\label{ese:19.23}
Lo sconto semplice~$S$, per un montante~$M$, al tasso di interesse~$i$, per un tempo di~$t$ anni è:
\begin{equation*}
  S=\frac{M\cdot i\cdot t}{1+i\cdot t}.
\end{equation*}

Ricava le formule per calcolare:~$M=\ldots\ldots\ldots\ldots$, $\quad i=\ldots\ldots\ldots\ldots$.

Se lo sconto semplice è \officialeuro~$\np{12000}$, il tempo è~$12$ anni, il tasso di interesse il~$\np{4,5}\%$, calcola il montante.
\end{esercizio}

\begin{esercizio}
\label{ese:19.24}
La superficie~$S$ di un trapezio con base maggiore~$B$, base minore~$b$ e altezza~$h$ è
\begin{equation*}
  S=\frac{1}{2}\cdot (B+b)\cdot h.
\end{equation*}

Ricava le formule per calcolare:~$B=\ldots\ldots\ldots\ldots$, $\quad b=\ldots\ldots\ldots\ldots$, $\quad h =\ldots\ldots\ldots\ldots$.

Se la base maggiore è~$12\unit{cm}$, la base minore~$8\unit{cm}$, la superficie~$12\unit{cm^2}$, calcola l'altezza del trapezio.
\end{esercizio}

\begin{esercizio}
\label{ese:19.25}
La superficie laterale~$S_l$ di un tronco di piramide con perimetro della base maggiore~$2p_B$, perimetro della base minore~$2p_b$ e apotema~$a$
($2p_B$ e~$2p_b$ sono da considerare come un'unica incognita):
\begin{equation*}
  S_{l}=\frac{(2p_B+2p_b)\cdot a}{2}.
\end{equation*}

Ricava le formule per calcolare:~$2p_B=\ldots\ldots\ldots\ldots$, $\quad~2p_b=\ldots\ldots\ldots\ldots$, $\quad a =\ldots\ldots\ldots$.

Se la superficie laterale vale~$144\unit{cm^2}$, il perimetro della base minore~$12\unit{cm}$ e il perimetro della base maggiore~$14\unit{cm}$, calcola l'apotema.
\end{esercizio}

\begin{esercizio}
\label{ese:19.26}
Il volume~$V$ del segmento sferico con una base di raggio~$r$ e altezza~$h$ è
\begin{equation*}
  V=\pi \cdot h^{2}\cdot \left(r-\frac{h}{3}\right).
\end{equation*}

Ricava la formula per calcolare~$r=\ldots\ldots\ldots\ldots$.

Se il volume misura~$200\unit{cm^3}$ e l'altezza~$10\unit{cm}$, calcola la misura del raggio.
\end{esercizio}

\begin{esercizio}
\label{ese:19.27}
La superficie totale~$S$ del cono con raggio di base~$r$ e apotema~$a$ è
\begin{equation*}
  S=\pi \cdot r\cdot (r+a).
\end{equation*}

Ricava la formula per calcolare~$a=\ldots\ldots\ldots\ldots$.

Se la superficie totale è~$98\unit{cm^2}$ e il raggio di base~$6\unit{cm}$, calcola la misura dell'apotema.
\end{esercizio}

\begin{esercizio}
\label{ese:19.28}
La velocità~$v$ di un corpo che si muove di moto rettilineo uniformemente accelerato con velocità iniziale~$v_0$ e accelerazione costante~$a$, dopo un tempo~$t$ è
\begin{equation*}
  v=v_{0}+a\cdot t.
\end{equation*}

Ricava le formule per calcolare:~$v_0=\ldots\ldots\ldots\ldots$, $\quad a=\ldots\ldots\ldots\ldots$, $\quad t =\ldots\ldots\ldots\ldots$.

Se un corpo è passato in~$10$ secondi dalla velocità (iniziale) di~$10\unit{m/s}$ alla velocità di~$24\unit{m/s}$, qual è stata la sua accelerazione?
\end{esercizio}

\begin{esercizio}
\label{ese:19.29}
Lo spazio~$s$ percorso da un corpo che si muove di moto rettilineo uniformemente accelerato con posizione iniziale~$s_0$,
 velocità iniziale~$v_0$ e accelerazione~$a$, dopo un intervallo di tempo~$t$ è
\begin{equation*}
  s=s_{0}+v_{0}\cdot t+\dfrac{1}{2}\cdot a\cdot t^{2}.
\end{equation*}

Ricava le formule per calcolare:~$v_0=\ldots\ldots\ldots\ldots$, $\quad a=\ldots\ldots\ldots\ldots$.

Se un corpo ha percorso~$100\unit{m}$, partendo dalla posizione iniziale~$0$, accelerazione~$3\unit{m/s^2}$, in~$10$ secondi, qual era la sua velocità iniziale?
\end{esercizio}

\begin{esercizio}
\label{ese:19.30}
La formula di Bernoulli relativa al moto di un fluido è
\begin{equation*}
  p+\rho \cdot g\cdot h+\dfrac{1}{2}\rho \cdot v^{2}=k.
\end{equation*}

Ricava le formule per calcolare:~$h=\ldots\ldots\ldots\ldots$, $\quad \rho=\ldots\ldots\ldots\ldots$.
\end{esercizio}

\begin{esercizio}
\label{ese:19.31}
La seconda legge di Gay-Lussac per i gas è
\begin{equation*}
  V=V_{0}\cdot (1+\alpha \cdot t).
\end{equation*}

Ricava le formule per calcolare:~$V_0=\ldots\ldots\ldots\ldots$, $\quad t=\ldots\ldots\ldots\ldots$.
\end{esercizio}

\begin{esercizio}
\label{ese:19.32}
L'equazione di stato dei gas perfetti è
\begin{equation*}
  pV=nRT.
\end{equation*}

Ricava le formule per calcolare:~$V=\ldots\ldots\ldots\ldots$, $\quad T=\ldots\ldots\ldots\ldots$.
\end{esercizio}

\begin{esercizio}
\label{ese:19.33}
Il rendimento del ciclo di Carnot è
\begin{equation*}
  \eta =1-\dfrac{T_{1}}{T_{2}}.
\end{equation*}

Ricava le formule per calcolare:~$T_1=\ldots\ldots\ldots\ldots$, $\quad T_2=\ldots\ldots\ldots\ldots$.
\end{esercizio}

\begin{esercizio}
\label{ese:19.34}
La legge di Stevino è
\begin{equation*}
  P_{B}=P_{A}+\rho \cdot g\cdot (z_{A}-z_{B}).
\end{equation*}

Ricava le formule per calcolare:~$\rho=\ldots\ldots\ldots\ldots$, $\quad z_A=\ldots\ldots\ldots\ldots$, $\quad z_B =\ldots\ldots\ldots\ldots$.
\end{esercizio}

\begin{esercizio}
\label{ese:19.35}
Risolvi le seguenti equazioni rispetto alla lettera richiesta.
\begin{multicols}{2}
\TabPositions{2.5cm}
\begin{enumeratea}
 \item $y=\dfrac{2-a}{x}$\tab$x=\ldots$, $a=\ldots$;
 \item $y=2-\dfrac{a}{x}$\tab$x=\ldots$, $a=\ldots$;
 \item $y=\dfrac{2}{x}-a$\tab$x=\ldots$, $a=\ldots$;
 \item $y=-{\dfrac{2-a}{x}}$\tab$x=\ldots$, $a=\ldots$.
\end{enumeratea}
\end{multicols}
\end{esercizio}

\begin{esercizio}
\label{ese:19.36}
Risolvi le seguenti equazioni rispetto alla lettera richiesta.
\begin{multicols}{2}
\TabPositions{3.5cm}
\begin{enumeratea}
 \item $\dfrac{2x+1}{2x-1}=\dfrac{2k-1}{k+1}$\tab$k=\ldots$;
 \item $(m-1)x=m-3$\tab$m=\ldots$;
 \item $\dfrac{2}{x+2}+\dfrac{a-1}{a+1}=0$\tab$a=\ldots$;
 \item $(a+1)(b-1)x=0$\tab$b=\ldots$.
\end{enumeratea}
\end{multicols}
\end{esercizio}

\begin{esercizio}[\Ast]
\label{ese:19.37}
Risolvi le seguenti equazioni rispetto alla lettera richiesta.
\TabPositions{5cm}
\begin{enumeratea}
 \item $\dfrac{x}{a+b}+\dfrac{x-b}{a-b}=\dfrac{b}{a^{2}-b^{2}}$\tab$a=\ldots$, $x=\ldots$;
 \item $\dfrac{2x}{a+b}+\dfrac{bx}{a^{2}-b^{2}}-\dfrac{1}{a-b}=0$\tab$a=\ldots$, $b=\ldots$.
\end{enumeratea}
\end{esercizio}

\subsection{Risposte}
\paragraph{19.1.}
a)~$\forall a\in \insR \rightarrow \left\{\frac{a}{4}\right\}$;
\quad b)~$a=2 \rightarrow \emptyset$, $a\neq~2 \rightarrow \left\{\frac{3}{2(a-2)}\right\}$;
\quad \protect\\
c)~$b=0 \rightarrow \insR$, $b=1\rightarrow\emptyset$, $b\neq~0\wedge b\neq~1\rightarrow \left\{\frac{2}{b-1}\right\}$;
\quad d)~$a=1\rightarrow \emptyset$, $a\neq~1\rightarrow \left\{\frac{1}{a-1}\right\}$.

\paragraph{19.2.}
a)~$k=0 \rightarrow \emptyset$, $k\neq~0 \rightarrow \left\{\frac{2-k}{k}\right\}$;
\quad b)~$b=-1 \rightarrow \insR$, $b\neq -1 \rightarrow \left\{-1\right\}$;
\quad \protect\\
c)~$k=1 \rightarrow \insR$, $k=-1 \rightarrow \emptyset$, $k\neq~1\wedge k\neq -1 \rightarrow \left\{-{\frac{2}{k+1}}\right\}$;
\quad d)~$a=2 \rightarrow \insR$, $a\neq~2 \rightarrow \left\{-1\right\}$.

\paragraph{19.4.}
a)~$a=1 \rightarrow \insR$, $a\neq~1 \rightarrow \{0\}$;
\quad b)~$k=0 \rightarrow \emptyset$, $k\neq~0 \rightarrow \left\{\frac{2}{k}\right\}$;
\quad \protect\\
c)~$a=0 \rightarrow \insR$, $a=3\rightarrow \emptyset$, $a\neq~0 \wedge a\neq~3 \rightarrow \left\{\frac{10}{3-a}\right\}$;
\quad d)~$a=0 \rightarrow \insR$, $a\neq~0 \rightarrow \left\{\frac{2}{3} (a-2)\right\}$.

\paragraph{19.5.}
a)~$a=3 \rightarrow \insR$, $a\neq~3 \rightarrow \{2\}$;
\quad b)~$a=2 \rightarrow \insR$, $a=1 \rightarrow \emptyset$, $a\neq~2\wedge a\neq~1 \rightarrow \left\{\frac{1}{a-1}\right\}$;
\quad c)~$a=2 \rightarrow \emptyset$, $a=-2 \rightarrow \insR$, $a\neq -2\wedge a\neq~2 \rightarrow \left\{\frac{1}{a-2}\right\}$;
\quad\protect\\
d)~$m=1\vee m=-1 \rightarrow \insR$, $m\neq~1\wedge m\neq -1 \rightarrow \emptyset$.

\paragraph{19.6.}
a)~$a=2 \rightarrow \insR$, $a=1 \rightarrow \emptyset$, $a\neq~1\wedge a\neq~2 \rightarrow \left\{\frac{1}{1-a}\right\}$;
\quad\protect\\
b)~$3a^{2}-2=0 \rightarrow \emptyset$, $3a^{2}-2\neq~0 \rightarrow \left\{\frac{2}{3(3a^{2}-2)}\right\}$;
\quad c)~$a=1 \rightarrow \insR$, $a\neq~1 \rightarrow \{a+1\}$;
\quad d)~$a=-2 \rightarrow \emptyset$, $a\neq -2 \rightarrow \left\{\frac{a^{2}+a-1}{a+2}\right\}$.

\paragraph{19.7.}
a)~$a=0 \rightarrow \insR$, $a\neq~0 \rightarrow \{0\}$;
\quad \protect\\
b)~$a=-2\vee a=2 \rightarrow \insR$, $a=0 \rightarrow \emptyset$, $a\neq -2\wedge a\neq~0\wedge a\neq~2 \rightarrow \left\{\frac{1}{a}\right\}$;
\quad\protect\\
c)~$b=0 \rightarrow \emptyset$, $b\neq~0 \rightarrow \left\{\frac{1+b^{2}}{2b}\right\}$;
\quad d)~$a=2 \rightarrow \insR$, $a=3 \rightarrow \emptyset$, $a\neq~2\wedge a\neq~3 \rightarrow \left\{\frac{a+3}{3-a}\right\}$;
\quad e)~$a=0 \rightarrow \emptyset$, $a\neq~0 \rightarrow \left\{\frac{4}{a}\right\}$;
\quad f)~$b=-3 \rightarrow \insR$, $b=2 \rightarrow \emptyset$, $b\neq -3\wedge b\neq~2 \rightarrow \left\{\frac{b}{b-2}\right\}$.

\paragraph{19.8.}
a)~$m=-1\vee n=2 \rightarrow \insR$, $m\neq -1\wedge n\neq~2 \rightarrow \{0\}$;
\protect\\
b)~$m=0\wedge n\neq~0 \rightarrow \emptyset$, $m=0\wedge n=0 \rightarrow \insR$, $m\neq~0 \rightarrow \left\{\frac{m+n}{m}\right\}$;
\protect\\
c)~$a=-1\vee b=-1 \rightarrow \insR$, $a\neq -1\wedge b\neq -1 \rightarrow \{0\}$;
\protect\\ d)~$m=n=0 \rightarrow \insR$, $m=-n\neq~0 \rightarrow \emptyset$, $m\neq -n \rightarrow \left\{\frac{2m}{m+n}\right\}$.

\paragraph{19.9.}
a)~$a=b=0 \rightarrow \insR$, $a=-b\neq~0 \rightarrow \emptyset$, $a\neq -b \rightarrow \left\{\frac{2(b-a)}{a+b}\right\}$;
\protect\\ b)~$a=2\wedge b=-3 \rightarrow \insR$, $a=2\wedge b\neq -3 \rightarrow \emptyset$, $a\neq~2 \rightarrow \left\{\frac{b+3}{a-2}\right\}$;
\protect\\ c)~$a=-1\wedge b=-1 \rightarrow \insR$, $a=-1\wedge b\neq -1 \rightarrow \emptyset$, $a\neq -1 \rightarrow \left\{\frac{b+1}{a+1}\right\}$;
\protect\\ d)~$a=b=0 \rightarrow \insR$, $a=b\neq~0 \rightarrow \emptyset$, $a\neq b \rightarrow \left\{\frac{2b-a}{a-b}\right\}$.

\paragraph{19.10.}
a)~$a=-{\frac{2}{3}}\wedge b=0 \rightarrow \insR$, $a=-{\frac{2}{3}}\wedge b\neq~0 \rightarrow \emptyset$, $a\neq -{\frac{2}{3}} \rightarrow \left\{\frac{b}{2+3a}\right\}$;
\protect\\ b)~$a=0\wedge b=0 \rightarrow \insR$, $a=0\wedge b\neq~0 \rightarrow \emptyset$, $a\neq~0 \rightarrow \left\{-{\frac{b^{2}}{a}}\right\}$.

\paragraph{19.11.}
a)~$a=0 \rightarrow$ assurdo, $a=-3 \rightarrow \emptyset$, $a\neq~0\wedge a\neq -3 \rightarrow \left\{\frac{3}{a+3}\right\}$;
\protect\\ b)~$b=0 \rightarrow$ assurdo, $b=6 \rightarrow \emptyset$, $b\neq~0\wedge b\neq~6 \rightarrow \left\{\frac{1}{6-b}\right\}$;
\protect\\ c)~$a=0 \rightarrow$ assurdo, $a=-2 \rightarrow \emptyset$, $a\neq~0\wedge a\neq -2 \rightarrow \left\{\frac{7}{2+a}\right\}$;
\protect\\ d)~$a=0\vee a=2 \rightarrow$ assurdo, $a=-3 \rightarrow \emptyset$, $a\neq~0\wedge a\neq~2\wedge a\neq -3 \rightarrow \left\{\frac{a+2}{a+3}\right\}$.

\paragraph{19.12.}
a)~$a=1\vee a=3 \rightarrow$ assurdo, $a\neq~1\wedge a\neq~3 \rightarrow \{2(a-1)(a-3)\}$;
\protect\\ b)~$a=0\vee a=1 \rightarrow$ assurdo, $a=\frac{1}{2} \rightarrow \emptyset$, $a\neq~0\wedge a\neq \frac{1}{2}\wedge a\neq~1 \rightarrow \left\{\frac{1}{2a-1}\right\}$;
\protect\\ c)~$a=-2 \rightarrow$ assurdo, $a=-3 \rightarrow \emptyset$, $a=3 \rightarrow \insR$, $a\neq -3\wedge a\neq -2\wedge a\neq~3 \rightarrow \left\{\frac{a+2}{a+3}\right\}$;
\protect\\ d)~$a=0\vee a=-2\vee a=2 \rightarrow$ assurdo, $a\neq~0\wedge a\neq -2\wedge a\neq~2 \rightarrow \left\{-{\frac{a}{2}}\right\}$.

\paragraph{19.13.}
a)~$a=2\vee a=-1 \rightarrow$ assurdo, $a\neq~2\wedge a\neq -1 \rightarrow \left\{\frac{a(a+4)}{2-a}\right\}$;
\protect\\ b)~$a=5\vee a=2 \rightarrow$ assurdo, $a=4 \rightarrow \emptyset$, $a\neq~5\wedge a\neq~2\wedge a\neq~4 \rightarrow \left\{\frac{1}{3(4-a)}\right\}$;
\protect\\ c)~$b=2\vee b=1 \rightarrow$ assurdo, $b\neq~2\wedge b\neq~1 \rightarrow \left\{\frac{b}{2-b}\right\}$;
\protect\\ d)~$b=0\vee b=7\rightarrow$ assurdo, $b\neq~0\wedge b\neq~7 \rightarrow \left\{-{\frac{1}{2b^{2}}}\right\}$;
\protect\\ e)~$t=0\vee t=-3 \rightarrow$ assurdo, $t^{2}=3 \rightarrow \insR$, $t\neq~0\wedge t\neq -3\wedge t^{2}\neq~3 \rightarrow \{2\}$;
\protect\\ f)~$a=0\vee a=\frac{1}{2} \rightarrow$ assurdo, $a\neq~0\wedge a\neq \frac{1}{2} \rightarrow \{2-6a\}$.

\paragraph{19.14.}
a)~$t=0\vee t=1 \rightarrow \emptyset$, $t\neq~0\wedge t\neq~1 \rightarrow \left\{\frac{5t-1}{2t}\right\}$;
\quad b)~$m=1 \rightarrow \insR-\{-1\}$, $m\neq~1 \rightarrow \emptyset$;
\quad c)~$a=\frac{1}{2} \rightarrow \emptyset$, $a\neq \frac{1}{2} \rightarrow \left\{-{\frac{2(a-2)}{2a-1}}\right\}$;
\protect\\ d)~$a=3\vee a=\frac{7}{9} \rightarrow \emptyset$, $a\neq~3\wedge a\neq \frac{7}{9} \rightarrow \left\{\frac{2(3a+1)}{3-a}\right\}$.

\paragraph{19.15.}
a)~$k=-1 \rightarrow$ assurdo, $k=1 \rightarrow \emptyset$, $k\neq~1\wedge k\neq -1 \rightarrow \left\{-{\frac{\left(k^2-1\right)}{2}}\right\}$;
\protect\\ b)~$k=0 \rightarrow \insR-\{1\text{,~}-1\}$ $k\neq~0 \rightarrow \{-3\}$; c)~$a=1 \rightarrow \insR-\{-3\text{,~}2\}$, $a\neq~1 \rightarrow \emptyset$;
\protect\\ d)~$a=0 \rightarrow$ assurdo, $a\neq~0 \rightarrow \left\{a^{2}\right\}$.

\paragraph{19.16.}
a)~$a=-5\vee a=-1\vee a=7 \rightarrow \emptyset$, $a\neq -5\wedge a\neq -1\wedge a\neq~7 \rightarrow \left\{\frac{-2(a-1)}{a+5}\right\}$;
\quad b)~$a=-{\frac{4}{3}}\vee a=\frac{5}{9}\vee a=\frac{13}{3} \rightarrow \emptyset$, $a\neq -{\frac{4}{3}}\wedge a\neq \frac{5}{9}\wedge a\neq \frac{13}{3} \rightarrow \left\{\frac{3-2a}{4+3a}\right\}$;
\protect\\ c)~$a=-{\frac{1}{6}} \rightarrow \insR-\left\{-1\text{,~}2\text{,~}\frac{1}{3}\right\}$, $a=\frac{7}{3}\vee a=4\vee a=1 \rightarrow \emptyset$, $a\neq -{\frac{1}{6}}\wedge a\neq \frac{7}{3}\wedge a\neq~4\wedge a\neq~1 \rightarrow \{a-2\}$;
\quad d)~$a=1\vee a=-3\vee a=3 \rightarrow \emptyset$, $a\neq -3\wedge a\neq~1\wedge a\neq~3 \rightarrow \left\{\frac{5-a}{1-a}\right\}$.

\paragraph{19.17.}
a)~$a=-1\vee a=0 \rightarrow \emptyset$, $a\neq -1\wedge a\neq~0 \rightarrow \left\{-{\frac{\ a^{2}}{1+a}}\right\}$;
\protect\\ b)~$a=-1\vee a=0 \rightarrow \emptyset$, $a\neq -1\wedge a\neq~0 \rightarrow \left\{-{\frac{a(a-1)}{a+1}}\right\}$;
\quad c)~$a=0 \rightarrow \insR-\{0\}$, $a\neq~0 \rightarrow \{0\}$;
\quad d)~$a=0 \rightarrow \emptyset$, $a\neq~0 \rightarrow \left\{-{\frac{5}{a}}\right\}$.

\paragraph{19.18.}
a)~$a=1 \rightarrow$ assurdo, $a=-1 \rightarrow \emptyset$, $a\neq~1\wedge a\neq -1 \rightarrow \left\{\frac{4}{a+1}\right\}$;
\protect\\ b)~$t=-1 \rightarrow$ assurdo, $t\neq -1 \rightarrow \left\{\frac{1}{t+1}\right\}$;
\protect\\ c)~$t=-1 \rightarrow$ assurdo, $t=0 \rightarrow \insR-\{2\}$, $t=-{\frac{1}{2}} \rightarrow \emptyset$, $t\neq -\frac{1}{2}\wedge t\neq -1\wedge t\neq~0 \rightarrow \left\{\frac{3t+1}{2t+1}\right\}$;
\quad d)~$a=-1 \rightarrow$ assurdo, $a=2 \rightarrow \emptyset$, $a\neq -1\wedge a\neq~2 \rightarrow \left\{\frac{3a}{2(a-2)}\right\}$.

\paragraph{19.37.}
a)~$a=\frac{b(b+1)}{2x-b}$, $x=\frac{b(a+b+1)}{2a}$;
\quad b)~$a=\frac{b(x+1)}{2x-1}$, $b=\frac{a(2x-1)}{x+1}$.
