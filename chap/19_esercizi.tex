% (c) 2012-2014 - Dimitrios Vrettos d.vrettos@gmail.com
\section{Esercizi}
\subsection{Esercizi dei singoli paragrafi}
\subsubsection*{19.1 - Equazione lineare in due incognite}

\begin{esercizio}
 \label{ese:19.1}
Completa la tabella delle coppie di soluzioni dell'equazione~$x+2y-1=0$.

\begin{tabular*}{.9\textwidth}{@{\extracolsep{\fill}}*{10}{lccccccccc}}
\toprule
$x$ & & $-1$ & 0 & &$\frac{1}{2}$ & & & $\np{2,25}$ &\\
$y$ & 0 & & & $-1$ & & $\frac{3}{4}$ & 2 & & $\np{1,5}$\\
\bottomrule
\end{tabular*}
\end{esercizio}

\begin{esercizio}
 \label{ese:19.2}
Completa la tabella delle coppie di soluzioni dell'equazione~$3x-2y=5$.

\begin{tabular*}{.9\textwidth}{@{\extracolsep{\fill}}*{10}{lccccccccc}}
\toprule
$x$ & & 0 & 1 & & $\frac{1}{6}$ & & & $-\sqrt{2}$ & $\np{0,25}$\\
$y$ & 0 & & &$-1$ & & $\frac{3}{4}$ & $\sqrt{2}$ & & \\
\bottomrule
\end{tabular*}
\end{esercizio}

\begin{esercizio}
 \label{ese:19.3}
 Completa la tabella delle coppie di soluzioni dell'equazione~$3x-2\sqrt{2}\,y=0$.

 \begin{tabular*}{.9\textwidth}{@{\extracolsep{\fill}}*{8}{lccccccc}}
\toprule
$x$ & & 0 & & & $\frac{1}{6}$ & & $\sqrt{2}$ \\
$y$ & 0 & & 1 &$-1$ & & $\sqrt{2}$ & \\
\bottomrule
\end{tabular*}
\end{esercizio}

%%%%%%%%%%%%%%%%%%%%%%%%%%%%%%%%%%%%%%%%%%%%%%%%%%%%%%%%%%
\begin{esercizio}
 \label{ese:19.4}
Risolvi graficamente le seguenti equazioni in due incognite.
\begin{multicols}{2}
 \begin{enumeratea}
\spazielenx
\item $2x-2y+3=0$;
\item $-{\dfrac{1}{5}}x-\dfrac{5}{2}y+1=0$;
\item $-2y+3=0$;
\item $x+2y+\dfrac{7}{4}=0$.
\end{enumeratea}
\end{multicols}
\end{esercizio}

\begin{esercizio}
 \label{ese:19.5}
Risolvi graficamente le seguenti equazioni in due incognite.
\begin{multicols}{2}
 \begin{enumeratea}
\spazielenx
\item $-2x+4y-1=0$;
\item $2y+\dfrac{2}{3}x+6=0$;
\item $\sqrt{2}\,x+\sqrt{6}\,y=0$;
\item $\sqrt{3}\,y+\sqrt{6}=-x$.
 \end{enumeratea}
\end{multicols}
\end{esercizio}

\begin{esercizio}
 \label{ese:19.6}
 Stabilisci quali coppie appartengono all'Insieme Soluzione dell'equazione.
\TabPositions{4cm}
\begin{enumeratea}
\item $5x+7y-1=0$\tab$\left(-\frac{7}{5};0\right)$, $\left(-\frac{1}{5};-1\right)$, $\left(0;\frac{1}{7}\right)$, $\left(\frac{2}{5};-\frac{1}{7}\right)$;
\item $-x+\dfrac{3}{4}y-\dfrac{4}{3}=0$\tab$(0;-1)$, $\left(\frac{1}{12};\frac{7}{9}\right)$, $\left(-\frac{4}{3};0\right)$, $(-3;4)$;
\item $-x-y+\sqrt{2}=0$\tab$\left(\sqrt{2};0\right)$, $\left(0;-\sqrt{2}\right)$, $\left(1+\sqrt{2};-1\right)$, $\left(1;-1-\sqrt{2}\right)$.
\end{enumeratea}
\end{esercizio}

\subsubsection*{19.2 - Risoluzione di sistemi di equazioni lineari}

\begin{esercizio}
 \label{ese:19.7}
Risolvi i seguenti sistemi con il metodo di sostituzione.
 \begin{multicols}{2}
 \begin{enumeratea}
  \item $\left\{\begin{array}{l}
     y=-2\\
     2x-y+2=0
    \end{array}\right.;$
  \item $\left\{\begin{array}{l}
	   y=-x+1\\
	   2x+3y+4=0
       \end{array}\right..$
 \end{enumeratea}
 \end{multicols}
\end{esercizio}

\begin{esercizio}[\Ast]
 \label{ese:19.8}
Risolvi i seguenti sistemi con il metodo di sostituzione.
 \begin{multicols}{2}
 \begin{enumeratea}
  \item $\left\{\begin{array}{l}
        x=1\\
        x+y=1
       \end{array}
\right.;$
\item $\left\{\begin{array}{l}
        y=x\\
        2x-y+2=0
       \end{array}\right.;$
\item $\left\{\begin{array}{l}
		2x+y=1\\
		2x-y=-1
	  \end{array}\right.;$
\item $\left\{\begin{array}{l}
	   2y=2\\
	   x+y=1
	\end{array}\right..$
 \end{enumeratea}
 \end{multicols}
\end{esercizio}

\begin{esercizio}[\Ast]
 \label{ese:19.9}
Risolvi i seguenti sistemi con il metodo di sostituzione.
 \begin{multicols}{2}
 \begin{enumeratea}
  \item $\left\{\begin{array}{l}3x-y=7\\x+2y=14\end{array}\right.;$
\item $\left\{\begin{array}{l}3x-2y=1\\4y-6x=-2\end{array}\right.;$
\item $\left\{\begin{array}{l}3x+y=2\\x+2y=-1\end{array}\right.;$
\item ${\longarray\left\{\begin{array}{l}x+4y-1=3\\
	\dfrac{x}{2}+\dfrac{y}{3}+1=-{\dfrac{x}{6}}-1
	\end{array}\right..}$
 \end{enumeratea}
 \end{multicols}
\end{esercizio}

\begin{esercizio}[\Ast]
 \label{ese:19.10}
Risolvi i seguenti sistemi con il metodo di sostituzione.
 \begin{multicols}{2}
 \begin{enumeratea}
  \item $\left\{\begin{array}{l}2x-3y=2\\6x-9y=6\end{array}\right.;$
\item $\left\{\begin{array}{l}x+2y=14\\3x-y=7\end{array}\right.;$
\item $\left\{\begin{array}{l}x+2y=1\\-2x-4y=2\end{array}\right.;$
\item $\left\{\genfrac{}{}{0pt}{0}{2x-y=3}{-6x+3y=-9}\right..$
 \end{enumeratea}
 \end{multicols}
\end{esercizio}

\begin{esercizio}[\Ast]
 \label{ese:19.11}
Risolvi i seguenti sistemi con il metodo di sostituzione.
 \begin{multicols}{2}
 \begin{enumeratea}
 \item $\longarray\left\{\begin{array}{l}\dfrac{x-4y}{3}=x-5y\\x-2=6y+4 \end{array}\right.;$
\item $\longarray\left\{\begin{array}{l}\dfrac{y^{2}-4x+2}{5}=\dfrac{2y^{2}-x}{10}-1\\x=-2y+8\end{array}\right.;$
\item $\longarray\left\{\begin{array}{l}3x-\dfrac{3}{4}(2y-1)=\dfrac{13}{4}(x+1)\\\dfrac{x+1}{4}-\dfrac{y}{2}=\dfrac{1+y}{2}-\dfrac{1}{4}\end{array}\right.;$
\item $\longarray\left\{\begin{array}{l}\dfrac{x}{3}-\dfrac{y}{2}=0\\\dfrac{y-x-1}{2}+x-y+1=\dfrac{1}{2}\end{array}\right..$
 \end{enumeratea}
 \end{multicols}
\end{esercizio}

\begin{esercizio}[\Ast]
 \label{ese:19.12}
Risolvi i seguenti sistemi con il metodo di sostituzione.
 \begin{multicols}{2}
 \begin{enumeratea}
  \item $\longarray\left\{\begin{array}{l}y-\dfrac{x}{3}+\dfrac{3}{4}=0\\\dfrac{2x+1}{1-x}+\dfrac{2+y}{y-1}=-1\end{array}\right.;$
\item $\longarray\left\{\begin{array}{l}x+y=2\\3\left(\dfrac{x}{6}+3y\right)=4\end{array}\right.;$
\item $\longarray\left\{\begin{array}{l}\dfrac{1}{2}y-\dfrac{1}{6}x=5-\dfrac{6x+10}{4}\\2(x-2)-3x=40-6\left(y-\dfrac{1}{3}\right)\end{array}\right.;$
\item $\longarray\left\{\begin{array}{l}2\dfrac{y}{3}+x+1=0\\\dfrac{y+1}{2}+\dfrac{x-1}{3}+1=0\end{array}\right..$
 \end{enumeratea}
 \end{multicols}
\end{esercizio}

\begin{esercizio}[\Ast]
 \label{ese:19.13}
Risolvi i seguenti sistemi con il metodo di sostituzione.
 \begin{multicols}{2}
 \begin{enumeratea}
  \item $\longarray\left\{\begin{array}{l}(x-2)^{2}+y=x^{2}-2x-3y+6\\\dfrac{x}{4}-2y=2 \end{array}\right.;$
  \item $\left\{\begin{array}{l}x+y+1=0 \\x-y+k=0 \end{array}\right.;$
\item $\longarray\left\{\begin{array}{l}y-\dfrac{3-2x}{3}=\dfrac{x-y}{3}+1\\\dfrac{x+1}{2}+\dfrac{5}{4}=y+\dfrac{2-3x}{4}\end{array}\right.;$
\item $\left\{\begin{array}{l}x-2y-3=0\\kx+(k+1)y+1=0 \end{array}\right..$
 \end{enumeratea}
 \end{multicols}
\end{esercizio}

\begin{esercizio}
 \label{ese:19.14}
 Risolvere il sistema che formalizza il problema~\ref{pr:19.1} a pagina \pageref{pr:19.1}:
\[\longarray\left\{\begin{array}{l}2x+\dfrac{1}{2}y=98
\\2x+3y=170 \end{array}\right.\]
e concludere il problema determinando l'area del rettangolo.
\end{esercizio}

\begin{esercizio}
 \label{ese:19.15}
Determinare due numeri reali~$x$ e~$y$ tali che il
triplo della loro somma sia uguale al doppio del primo aumentato di~10
e il doppio del primo sia uguale al prodotto del secondo con~5.
 \end{esercizio}

\begin{esercizio}[\Ast]
 \label{ese:19.16}
Applica il metodo del confronto per risolvere i seguenti sistemi.
 \begin{multicols}{2}
 \begin{enumeratea}
 \item $\left\{\begin{array}{l}x+y=0\\-x+y=0\end{array}\right.;$
\item $\left\{\begin{array}{l}3x+y=5\\x+2y=0\end{array}\right.;$
\item $\left\{\begin{array}{l}x=2\\x+y=3\end{array}\right.;$
\item $\left\{\begin{array}{l}x=-1\\2x-y=1\end{array}\right..$
 \end{enumeratea}
 \end{multicols}
\end{esercizio}

\begin{esercizio}[\Ast]
 \label{ese:19.17}
Applica il metodo del confronto per risolvere i seguenti sistemi.
 \begin{multicols}{2}
 \begin{enumeratea}
 \item $\left\{\begin{array}{l}y=2x-1\\y=2x\end{array}\right.;$
\item $\left\{\begin{array}{l}x-2y=1\\2x-y=7\end{array}\right.;$
\item $\left\{\begin{array}{l}x+y=2\\-x-y=2\end{array}\right.;$
\item $\left\{\begin{array}{l}2x-y=4\\x-\dfrac{1}{2}y=2\end{array}\right..$
 \end{enumeratea}
 \end{multicols}
\end{esercizio}

\begin{esercizio}[\Ast]
 \label{ese:19.18}
Applica il metodo del confronto per risolvere i seguenti sistemi.
 \begin{multicols}{2}
 \begin{enumeratea}
\item $\left\{\begin{array}{l}y-\dfrac{3x-4}{2}=1-\dfrac{y}{4}\\2y-2x=-{\dfrac{4}{3}}\end{array}\right.;$
\item $\longarray\left\{\begin{array}{l}\dfrac{2}{3}x-y+\dfrac{1}{3}=0\\x-\dfrac{2}{3}y+\dfrac{1}{3}=0\end{array}\right.;$
\item $\left\{\begin{array}{l}\dfrac{1}{2}y-\dfrac{1}{6}x=5-\dfrac{6x+10}{8}\\8(x-2)+3x=40-6\left(y-\dfrac{1}{6}\right)\end{array}\right.;$
\item $\left\{\begin{array}{l}x=\dfrac{y-4}{3}+1\\y=\dfrac{x+3}{3}\end{array}\right.;$
\item $\left\{\begin{array}{l}x-y+k=0\\x+y=k-1\end{array}\right..$
 \end{enumeratea}
 \end{multicols}
\end{esercizio}

\begin{esercizio}
 \label{ese:19.19}
In un triangolo isoscele la somma della base con il doppio del lato
è~$168\unit{m}$ e la differenza tra la metà della base e~$1/13$ del lato è~$28\unit{m}$.
Indicata con~$x$ la misura della base e con~$y$ quella del lato,
risolvete con il metodo del confronto il sistema lineare che formalizza
il problema. Determinate l'area del triangolo.
 \end{esercizio}
\pagebreak
%%%%%%%%%%%%%%%%%%%%%%%%%%%%%%%%%%%%%%%%%%%%%%%%%%%%%%%%%%
 \begin{esercizio}[\Ast]
 \label{ese:19.20}
Risolvere i seguenti sistemi con il metodo di riduzione.

\begin{multicols}{2}
 \begin{enumeratea}
 \item $\left\{\begin{array}{l}x+y=0\\-x+y=0\end{array}\right.;$
 \item $\left\{\begin{array}{l}2x+y=1 \\2x-y=-1\end{array}\right.;$
 \item $\left\{\begin{array}{l}2x+y=1+y\\4x+y=2\end{array}\right.;$
 \item $\left\{\begin{array}{l}x+y=0\\x-y=-1\end{array}\right..$
\end{enumeratea}
\end{multicols}
\end{esercizio}

\begin{esercizio}[\Ast]
 \label{ese:19.21}
Risolvere i seguenti sistemi con il metodo di riduzione.
\begin{multicols}{2}
 \begin{enumeratea}
 \item $\left\{\begin{array}{l}x-y=0\\-2x+3y=1\end{array}\right.;$
 \item $\left\{\begin{array}{l}2x=3-x\\2x+y=3\end{array}\right.;$
 \item $\longarray\left\{\begin{array}{l}\dfrac{2}{3}x+\dfrac{2}{3}y=3\\\dfrac{3}{2}x-\dfrac{3}{2}y=2\end{array}\right.;$
 \item $\left\{\begin{array}{l}5y+2x=1 \\3x+2y+2=0\end{array}\right..$
\end{enumeratea}
\end{multicols}

\end{esercizio}

\begin{esercizio}[\Ast]
 \label{ese:19.22}
Risolvere i seguenti sistemi con il metodo di riduzione.

 \begin{enumeratea}
 \item $\left\{\begin{array}{l}-3x+y=2\\5x-2y=7\end{array}\right.;$
 \item $\longarray\left\{\begin{array}{l}2x=3-x\\2x+y=3\end{array}\right.;$
 \item $\longarray\left\{\begin{array}{l}\dfrac{4}{3}x-\dfrac{4y-x}{2}+\dfrac{35}{12}-\dfrac{x+y}{4}=0\\\dfrac{3(x+y)}{2}-\dfrac{1}{2}(5x-y)=\dfrac{1}{3}(11-4x+y)\end{array}\right.;$
 \item $\left\{\begin{array}{l}x+ay+a=0\\2x-ay+a=0\end{array}\right.;$
 \item $\left\{\begin{array}{l}2ax+2y-1=0\\ax+y=3\end{array}\right..$
\end{enumeratea}
\end{esercizio}

\begin{esercizio}
 \label{ese:19.23}
Il segmento~$AB$ misura~$80\unit{cm}$; il punto~$P$ lo divide in due parti tali che il quadruplo della
parte minore uguagli il triplo della differenza fra la maggiore e il
triplo della minore. Determinare~$\overline{AP}$ e~$\overline{PB}$ formalizzando
il problema con un sistema lineare che risolverete con il metodo di
riduzione.
\begin{center}
 % (c) 2012 Dimitrios Vrettos - d.vrettos@gmail.com
\begin{tikzpicture}[font=\small,x=10mm, y=10mm]
\draw (0,0) --(8,0);
\foreach \x in {0,8}{
\draw (\x,3pt) -- (\x,-3pt);
}
\node [below left] at (0,0) {$A$};
\node[below right]  at (8,0) {$B$};
\draw[fill=blue] (1.5,0) circle (1.5pt) node [below]  {$P$};

\end{tikzpicture}

\end{center}
\end{esercizio}

\begin{esercizio}
 \label{ese:19.24}
Stabilire se il sistema è determinato:
\[\left\{\begin{array}{l}(x-1)(x+1)-3(x-2)=2(x-y+3)+x^{2}\\x(x+y-3)+y(4-x)=x^{2}-4x+y\end{array}\right..\]
\end{esercizio}

\begin{esercizio}
 \label{ese:19.25}
Verificare che il determinante della matrice del sistema è nullo:
\[\longarray\left\{\begin{array}{l}\dfrac{3}{2}x-\dfrac{7}{4}y=10^{5}\\6x-7y=5^{10}\end{array}\right..\]
\end{esercizio}

\begin{esercizio}[\Ast]
 \label{ese:19.26}
Risolvere con la regola di Cramer i seguenti sistemi.
\begin{multicols}{2}
\begin{enumeratea}
\item $\left\{\begin{array}{l}x-3y=2\\x-2y=2\end{array}\right.;$
\item $\left\{\begin{array}{l}2x+2y=3\\3x-3y=2\end{array}\right.;$
\item $\left\{\begin{array}{l}5y+2x=1 \\3x+2y+2=0\end{array}\right.;$
\item $\left\{\begin{array}{l}5x+2y=-1 \\3x-2y=1\end{array}\right..$
\end{enumeratea}
\end{multicols}
\end{esercizio}

\begin{esercizio}[\Ast]
 \label{ese:19.27}
Risolvere con la regola di Cramer i seguenti sistemi.
\begin{multicols}{2}
\begin{enumeratea}
\item $\longarray\left\{\begin{array}{l}\dfrac{1}{2}y-\dfrac{2}{3}y=2\\\dfrac{1}{3}x+\dfrac{1}{2}y=1 \end{array}\right.;$
\item $\longarray\left\{\begin{array}{l}\dfrac{y}{5}-\dfrac{x}{2}=10\\\dfrac{x}{3}+\dfrac{y}{2}=5 \end{array}\right.;$
\item $\left\{\begin{array}{l}2(x-2y)+3x-2(y+1)=0\\x-2(x-3y)-5y=6(x-1) \end{array}\right.;$
\item $\longarray\left\{\begin{array}{l}4-2x=\dfrac{3}{2}(y-1)\\\dfrac{2x+3y}{2}=\dfrac{7+2x}{2}\end{array}\right..$
\end{enumeratea}
\end{multicols}
\end{esercizio}

\begin{esercizio}[\Ast]
 \label{ese:19.28}
Risolvere con la regola di Cramer i seguenti sistemi.
\begin{multicols}{2}
\begin{enumeratea}
\item $\left\{\begin{array}{l}3x+y=-3\\-2x+3y=+2\end{array}\right.;$
\item $\left\{\begin{array}{l}6x-2y=5\\x+\frac{1}{2}y=0\end{array}\right.;$
\item $\left\{\begin{array}{l}10x-20y=-11\\x+y-1=0\end{array}\right.;$
\item $\left\{\begin{array}{l}2x-3y+1=0\\4x+6y=0\end{array}\right..$
\end{enumeratea}
\end{multicols}
\end{esercizio}

\begin{esercizio}[\Ast]
 \label{ese:19.29}
Risolvere con la regola di Cramer i seguenti sistemi.
\begin{multicols}{2}
\begin{enumeratea}
\item $\left\{\begin{array}{l}x+2y=1\\2x+4y=1\end{array}\right.;$
\item $\left\{\begin{array}{l}3x+2y=4\\\frac{3}{2}x+y=2\end{array}\right.;$
\item $\left\{\begin{array}{l}ax+ay=3a^{2}\\x-2y=-3a\end{array}\right.;$
\item $\left\{\begin{array}{l}3x-2y=8k\\x-y=3k\end{array}\right..$
\end{enumeratea}
\end{multicols}
\end{esercizio}

\begin{esercizio}
 \label{ese:19.30}
 Risolvi col metodo di Cramer il sistema
\[\longarray\left\{\begin{array}{l}25x-3y=18\\\dfrac{3(y+6)}{5}=5x\end{array}\right..\]

Cosa osservi?
\end{esercizio}

\begin{esercizio}[\Ast]
 \label{ese:19.31}
Risolvi i seguenti sistemi applicando tutti e quattro i metodi.
\begin{multicols}{2}
\begin{enumeratea}{\longarray
\item $\left\{\begin{array}{l}3-x=\dfrac{y-x}{4}-\dfrac{2+x}{7}\\\dfrac{17}{3}-y=\dfrac{y-3x}{6}-\dfrac{2y-x}{3}\end{array}\right.;$
\item $\left\{\begin{array}{l}\dfrac{4+x}{4}-\dfrac{x+y}{2}=\dfrac{5y-7x}{6}\\\dfrac{2y-x}{7}-4=\dfrac{6y-x}{2}\end{array}\right.;$
\item $\left\{\begin{array}{l}\dfrac{2x-y}{4}-4=\dfrac{1}{3}-\dfrac{x+y}{3}\\\dfrac{2x-3y}{4}-\dfrac{5}{6}=2-\dfrac{3x-2y}{6}\end{array}\right.;$
\item $\left\{\begin{array}{l}2x-\dfrac{4y}{7}=2+\dfrac{2x}{7}\\3x-\dfrac{25}{2}=4y\end{array}\right..$}
\end{enumeratea}
\end{multicols}
\end{esercizio}

\begin{esercizio}[\Ast] %pag181-182
 \label{ese:19.32}
Risolvi i seguenti sistemi applicando tutti e quattro i metodi.
\begin{multicols}{2}
\begin{enumeratea}{\longarray
\item $\left\{\begin{array}{l}\dfrac{12-19x}{6}=\dfrac{(x+1)(1-y)}{2}+\dfrac{xy}{2}\\\dfrac{5}{9}-\dfrac{3x-2y}{3}=\dfrac{\dfrac{1}{3}-y-2x}{2}\end{array}\right.;$
\item $\left\{\begin{array}{l}\dfrac{2}{5}x-\dfrac{3}{4}y=-\dfrac{5}{6}\\3x+\dfrac{1}{2}y-6=0\end{array}\right.;$
\item $\left\{\begin{array}{l}2x-5=\dfrac{3}{5}y\\2x-\dfrac{6}{5}(y-1)=5\end{array}\right.;$
\item $\left\{\begin{array}{l}\dfrac{5}{2}x-\dfrac{4}{3}=\dfrac{1}{4}y\\\dfrac{10}{3}x+\dfrac{3}{2}y-3=0\end{array}\right..$}
\end{enumeratea}
\end{multicols}
\end{esercizio}

\begin{esercizio}
 \label{ese:19.33}
Per ciascuno dei seguenti sistemi stabilisci se è determinato,
indeterminato o impossibile.
\begin{multicols}{2}
\begin{enumeratea}
\item $\left\{\begin{array}{l}x-2y=3 \\4x-8y=12\end{array}\right.;$
\item $\left\{\begin{array}{l}x-2y=3 \\2x-4y=5\end{array}\right.;$
\item $\left\{\begin{array}{l}x-2y=3 \\2x-6y=12\end{array}\right.;$
\item $\longarray\left\{\begin{array}{l}\dfrac{1}{2}x-\dfrac{3}{2}y=-2\\\dfrac{5}{4}x-\dfrac{15}{4}y=-{\dfrac{5}{2}}\end{array}\right..$
\end{enumeratea}
\end{multicols}
\end{esercizio}

\begin{esercizio}
 \label{ese:19.34}
Per ciascuno dei seguenti sistemi stabilisci se è determinato,
indeterminato o impossibile.
\begin{multicols}{2}
\begin{enumeratea}
\item $\longarray\left\{\begin{array}{l}\dfrac{1}{7}x-\dfrac{4}{5}y=0\\\dfrac{5}{4}x-7y=\dfrac{19}{2}\end{array}\right.;$
\item $\left\{\begin{array}{l}2x+y=1 \\2x-y=-1\end{array}\right.;$
\item $\left\{\begin{array}{l}-40x+12y=-3\\17x-2y=100\end{array}\right.;$
\item $\left\{\begin{array}{l}x-y=3 \\-x+y=1 \end{array}\right..$
\end{enumeratea}
\end{multicols}
\end{esercizio}

\begin{esercizio}
 \label{ese:19.35}
Per ciascuno dei seguenti sistemi stabilisci se è determinato,
indeterminato o impossibile.
\begin{multicols}{2}
\begin{enumeratea}
\item $\longarray\left\{\begin{array}{l}-x+3y=-{\dfrac{8}{15}}\\5x-15y=\dfrac{2^{3}}{3}\end{array}\right.;$
\item $\longarray\left\{\begin{array}{l}\dfrac{x}{2}=-{\dfrac{y}{2}}+1\\x+y=2\end{array}\right..$
\end{enumeratea}
\end{multicols}
\end{esercizio}

\begin{esercizio}
 \label{ese:19.36}
La somma di due numeri reali è~16 e il doppio del
primo aumentato di~4 uguaglia la differenza tra~5 e il doppio del
secondo. Stabilisci, dopo aver formalizzato il problema con un sistema
lineare, se è possibile determinare i due numeri.
\end{esercizio}

\begin{esercizio}
 \label{ese:19.37}
Stabilisci per quale valore di~$a$ il sistema
$\left\{\begin{array}{l}ax+y=-2\\-3x+2y=0\end{array}\right.$ è
determinato. Se~$a=-{\frac{3}{2}}$ il sistema è indeterminato o
impossibile?
\end{esercizio}

\begin{esercizio}
 \label{ese:19.38}
Perché se~$a=\frac{1}{3}$ il sistema
$\left\{\begin{array}{l}x+ay=2a\\3x+y=2\end{array}\right.$ è
indeterminato?
\end{esercizio}

\begin{esercizio}
 \label{ese:19.39}
Per quale valore di~$k$ il sistema risulta impossibile?
\[\left\{\begin{array}{l}2x-3ky=2k\\x-ky=2k \end{array}\right.. \]
\end{esercizio}

\begin{esercizio}
 \label{ese:19.40}
Per quale valore di~$k$ il sistema risulta indeterminato?
\[\left\{\begin{array}{l}(k-2)x+3y=6 \\(k-1)x+4y=8 \end{array}\right..\]
\end{esercizio}

\begin{esercizio}[\Ast]
 \label{ese:19.41}
Risolvi graficamente i sistemi; in base al disegno verifica se le rette
sono incidenti, parallele o coincidenti e quindi se il sistema è
determinato, impossibile o indeterminato.
\begin{multicols}{2}
\begin{enumeratea}
\item $\left\{\begin{array}{l}y=2x-1\\y=2x+1\end{array}\right.;$
\item $\left\{\begin{array}{l}y=2x-2\\y=3x+1\end{array}\right.;$
\item $\left\{\begin{array}{l}y=x-1 \\2y=2x-2 \end{array}\right.;$
\item $\left\{\begin{array}{l}2x-y=2 \\2y-x=2\end{array}\right..$
\end{enumeratea}
\end{multicols}
\end{esercizio}

\begin{esercizio}[\Ast]
 \label{ese:19.42}
Risolvi graficamente i sistemi; in base al disegno verifica se le rette
sono incidenti, parallele o coincidenti e quindi se il sistema è
determinato, impossibile o indeterminato.
\begin{multicols}{2}
\begin{enumeratea}
\item $\left\{\begin{array}{l}{3x+y=-3}\\{-2x+3y=-2}\end{array}\right.;$
\item $\left\{\begin{array}{l}x-3y=2\\x-2y=2\end{array}\right.;$
\item $\left\{\begin{array}{l}{3x+y=-3}\\{-2x+3y=+2}\end{array}\right.;$
\item $\left\{\begin{array}{l}2x=2-y\\2x-y=1\end{array}\right..$
\end{enumeratea}
\end{multicols}
\end{esercizio}

\begin{esercizio}[\Ast]
 \label{ese:19.43}
Risolvi graficamente i sistemi; in base al disegno verifica se le rette
sono incidenti, parallele o coincidenti e quindi se il sistema è
determinato, impossibile o indeterminato.
\begin{multicols}{3}
\begin{enumeratea}
\item $\left\{\begin{array}{l}5x+2y=-1 \\3x-2y=1 \end{array}\right.;$
\item $\left\{\begin{array}{l}2x=3-x\\2x+y=3\end{array}\right.;$
\item $\left\{\begin{array}{l}2x=2-y\\2x-y=1\end{array}\right..$
\end{enumeratea}
\end{multicols}
\end{esercizio}

\begin{esercizio}
 \label{ese:19.44}
Vero o falso?
\TabPositions{11cm}
 \begin{enumeratea}
\item Risolvere graficamente un sistema lineare significa trovare il punto di intersezione di due rette? \tab\boxV\quad\boxF
\item Un sistema lineare determinato ha una sola coppia soluzione?\tab\boxV\quad\boxF
\item Un sistema lineare è impossibile quando le rette che rappresentano le sue equazioni coincidono?\tab\boxV\quad\boxF
\end{enumeratea}
\end{esercizio}

\begin{esercizio}
 \label{ese:19.45}
Completa:

\begin{itemize*}
\item se~$r_{1}\cap r_{2}=r_{1}=r_{2}$ allora il sistema è\dotfill;
\item se~$r_{1}\cap r_{2}=P$ (punto singolo) allora il sistema è\dotfill;
\item se~$r_{1}\cap r_{2}=\emptyset $ allora il sistema è\dotfill
\end{itemize*}
\end{esercizio}
\pagebreak
\subsubsection*{19.3 - Sistemi frazionari o fratti}

\begin{esercizio}[\Ast]
 \label{ese:19.46}
Verifica l'insieme soluzione dei seguenti sistemi.
\begin{multicols}{2}
\begin{enumeratea}
\item $\longarray\left\{\begin{array}{l}\dfrac{4y+x}{5x}=1\\\dfrac{x+y}{2x-y}=2\end{array}\right.;$
\item $\longarray\left\{\begin{array}{l}y=\dfrac{4x-9}{12}\\{\dfrac{y+2}{y-1}+\dfrac{1+2x}{1-x}+1=0}\end{array}\right.;$
\item $\longarray\left\{\begin{array}{l}2+3\dfrac{y}{x}=\dfrac{1}{x}\\3\dfrac{x}{y}-1=\dfrac{-2}{y} \end{array}\right.;$
\item $\longarray\left\{\begin{array}{l}\dfrac{y}{2x-1}=-1\\2\dfrac{x}{y-1}=1\end{array}\right..$
\end{enumeratea}
\end{multicols}
\end{esercizio}

\begin{esercizio}[\Ast]
 \label{ese:19.47}
Verifica l'insieme soluzione dei seguenti sistemi.
\begin{multicols}{2}
\begin{enumeratea}
\item $\longarray\left\{\begin{array}{l}3\dfrac{x}{y}-\dfrac{7}{y}=1\\2\dfrac{y}{x}+\dfrac{5}{x}=1\end{array}\right.;$
\item $\longarray\left\{\begin{array}{l}2\dfrac{x}{3y}-\dfrac{1}{3y}=1\\\dfrac{3}{y+2x}=-1\end{array}\right.;$
\item $\longarray\left\{\begin{array}{l}\dfrac{x}{9y}=-{\dfrac{1}{2}}+\dfrac{1}{3y}\\9\dfrac{y}{2x}-1-\dfrac{3}{x}=0\end{array}\right.;$
\item $\longarray\left\{\begin{array}{l}\dfrac{x}{2-\dfrac{y}{2}-2}=1\\\dfrac{x-y}{x+\dfrac{3}{2}y-1}=1\end{array}\right..$
\end{enumeratea}
\end{multicols}
\end{esercizio}

\begin{esercizio}[\Ast]
 \label{ese:19.48}
Verifica l'insieme soluzione dei seguenti sistemi.
\begin{multicols}{2}
\begin{enumeratea}
\item $\longarray\left\{\begin{array}{l}\dfrac{\dfrac{x}{2}+\dfrac{2y}{3}-\dfrac{1}{6}}{x+y-2}=6\\x+y=1\end{array}\right.;$
\item $\longarray\left\{\begin{array}{l}\dfrac{x-2y}{4}=\dfrac{\dfrac{x-y}{2}+2x}{4}\\\dfrac{x}{\dfrac{y}{3}+1}=1\end{array}\right.;$
\item $\longarray\left\{\begin{array}{l}\dfrac{x+3y-1}{x-y}=\dfrac{1}{y-x} \\x=2y-10 \end{array}\right.;$
\item $\longarray\left\{\begin{array}{l}\dfrac{2}{x-2}-\dfrac{3}{y+3}=1\\\dfrac{5}{y+3}=\dfrac{6}{2-x}-4\end{array}\right..$
\end{enumeratea}
\end{multicols}
\end{esercizio}

\begin{esercizio}[\Ast]
 \label{ese:19.49}
Verifica l'insieme soluzione dei seguenti sistemi.
\begin{multicols}{2}
\begin{enumeratea}
\item $\longarray\left\{\begin{array}{l}y-\dfrac{x}{3}+\dfrac{3}{4}=0\\\dfrac{2x+1}{1-x}+\dfrac{2+y}{y-1}=-1\end{array}\right.;$
\item $\longarray\left\{\begin{array}{l}x+y=2\\y\left(\dfrac{x}{y}+3\right)=4\end{array}\right.;$
\item $\longarray\left\{\begin{array}{l}\dfrac{x}{3}-\dfrac{y}{2}=0\\\dfrac{y(y-x-1)}{y+1}+x-y+1=\dfrac{1}{2}\end{array}\right.;$
\item $\longarray\left\{\begin{array}{l}\dfrac{3x-7y+1}{4x^{2}-9y^{2}}=\dfrac{4}{18y^{2}-8x^{2}}\\\dfrac{4(1-3x)^{2}}{2}=\dfrac{72x^{2}-30x+9y}{4}+2\end{array}\right..$
\end{enumeratea}
\end{multicols}
\end{esercizio}

\begin{esercizio}[\Ast]
 \label{ese:19.50}
Verifica l'insieme soluzione dei seguenti sistemi.

\begin{enumeratea}
\item $\longarray\left\{\begin{array}{l}\dfrac{2x-3y}{x-2y}-\dfrac{3y-1}{x+5y}=\dfrac{2(x^{2}+2xy)-(3y-2)^{2}}{x^{2}+3xy-10y^{2}}\\x+y=-19\end{array}\right.;$
\item $\longarray\left\{\begin{array}{l}\dfrac{x-3}{x-3y+1}+\dfrac{xy-y}{x-3y-1}=\dfrac{x^{2}-3xy+x^{2}y-3xy^{2}+3y^{2}}{x^{2}+9y^{2}-6xy-1}\\\dfrac{x-3}{5y-1}-\dfrac{y-3}{1+5y}=\dfrac{x+5y^{2}-5xy+2}{1-25y^{2}}\end{array}\right.;$
\item $\longarray\left\{\begin{array}{l}\dfrac{x-2y}{x^{2}-xy-2y^{2}}-\dfrac{1}{y}=2\\\dfrac{4}{y}-\dfrac{5}{x+y}=-9\end{array}\right.;$
\item $\longarray\left\{\begin{array}{l}{2x-y-11=0}\\{\dfrac{y+1}{x-1}+\dfrac{3-y}{5x-5}-\dfrac{2}{3}=0}\end{array}\right..$
\end{enumeratea}
\end{esercizio}

\begin{esercizio}
 \label{ese:19.51}[\Ast]
Verifica l'insieme soluzione dei seguenti sistemi.
\begin{multicols}{2}
\begin{enumeratea}
\item $\longarray\left\{\begin{array}{l}{\dfrac{x+1}{x}=\dfrac{y+2}{y-2}}\\{\dfrac{3x-1}{3x-2}=\dfrac{1+y}{y-2}}\end{array}\right.;$
\item $\longarray\left\{\begin{array}{l}{\dfrac{2}{5x-y}=\dfrac{-3}{5y-x}}\\{\dfrac{1}{4x-3y}=\dfrac{2x+y-1}{3y-4x}}\end{array}\right.;$
%\item $\longarray\left\{\begin{array}{l}{\dfrac{\sqrt{3}}{x-\sqrt{2}}+\dfrac{2\sqrt{2}}{y-\sqrt{3}}=0}\\{\dfrac{1}{x-\sqrt{3}}-\dfrac{\sqrt{6}}{2\left(y+2\sqrt{2}\right)}=0}\end{array}\right.;$
\item $\longarray\left\{\begin{array}{l}{\dfrac{x-y+1}{x+y-1}=2}\\{\dfrac{x+y+1}{x-y-1}=-2}\end{array}\right.;$
\item $\longarray\left\{\begin{array}{l}{\dfrac{2}{x-2}=\dfrac{3}{y-3}}\\{\dfrac{1}{y+3}=\dfrac{-1}{2-x}}\end{array}\right..$
\end{enumeratea}
\end{multicols}
\end{esercizio}
%\pagebreak
\subsubsection*{19.4 - Sistemi letterali}

\begin{esercizio}[\Ast]
 \label{ese:19.52}
Risolvere e discutere il seguente sistema. Per quali valori di~$a$ la coppia soluzione è formata da
numeri reali positivi?
\[\left\{\begin{array}{l}{x+ay=2a}\\\dfrac{x}{2a}+y=\dfrac{3}{2}\end{array}\right..\]
\end{esercizio}


\begin{esercizio}
 \label{ese:19.53}
Perché se il seguente sistema è determinato la coppia soluzione è accettabile?
\[\left\{\begin{array}{l}3x-2y=0\\\dfrac{2x-y}{x+1}=\dfrac{1}{a}\end{array}\right..\]
\end{esercizio}


\begin{esercizio}
 \label{ese:19.54}
Nel seguente sistema è vero che la coppia soluzione è formata da numeri reali positivi se~$a>2$?
 \[\left\{\begin{array}{l}\dfrac{a-x}{a^{2}}+a+\dfrac{y-2a}{a+1}=-1\\2y=x\end{array}\right..\]
\end{esercizio}


\begin{esercizio}
 \label{ese:19.55}
Spiegate perché non esiste alcun valore di~$a$ per cui la
coppia~$(0;2)$ appartenga a~$\IS$ del sistema:
\[\left\{\begin{array}{l}3x-2y=0\\\dfrac{2x-y}{x+1}=\dfrac{1}{a}\end{array}\right..\]
\end{esercizio}

\begin{esercizio}[\Ast]
 \label{ese:19.56}
Nel seguente sistema determinate i valori da attribuire al
parametro~$a$ affinché la coppia soluzione accettabile sia formata da
numeri reali positivi.
\[\left\{\begin{array}{l}\dfrac{y}{x}-\dfrac{y-a}{3}=\dfrac{1-y}{3}\\a(x+2)+y=1\end{array}\right..\]
\end{esercizio}

\begin{esercizio}[\Ast]
 \label{ese:19.57}
Risolvere i seguenti sistemi.
\begin{multicols}{2}
 \begin{enumeratea}
 {\longarray
\item $\left\{\begin{array}{l}x+ay=2a\\\dfrac{x}{2a}+y=\dfrac{3}{2}\end{array}\right.;$
\item $\left\{\begin{array}{l}\dfrac{x^{3}-8}{x-2}=x^{2}-3x+y-2\\\dfrac{x^{2}-4xy+3y^{2}}{3y-x}=k\end{array}\right.;$
\item $\left\{\begin{array}{l}kx-y=2\\x+6ky=0\end{array}\right.;$
\item $\left\{\begin{array}{l}kx-8y=4\\2x-4ky=3\end{array}\right..$}
 \end{enumeratea}
\end{multicols}
\end{esercizio}

\begin{esercizio}[\Ast]
 \label{ese:19.58}
Risolvere i seguenti sistemi.
\begin{multicols}{2}
 \begin{enumeratea}
\item $\left\{\begin{array}{l}4x-k^{2}y=k\\kx-4ky=-3k\end{array}\right.;$
\item $\left\{\begin{array}{l}kx-4ky=-6\\kx-k^{2}y=0\end{array}\right.;$
\item $\left\{\begin{array}{l}(k-1)x+(1-k)y=0\\(2-2k)x+y=-1\end{array}\right..$
 \end{enumeratea}
\end{multicols}
\end{esercizio}

 \subsubsection*{19.5 - Sistemi lineari di tre equazioni in tre incognite}

\begin{esercizio}[\Ast]
 \label{ese:19.59}
 Determinare la terna di soluzione dei seguenti sistemi.
\begin{multicols}{2}
\begin{enumeratea}
\item $\left\{\begin{array}{l}x-2y+z=1 \\x-y=2 \\x+3y-2z=0 \end{array}\right.;$
\item $\left\{\begin{array}{l}x+y+z=4 \\x-3y+6z=1\\x-y-z=2 \end{array}\right.;$
\item $\left\{\begin{array}{l}x+2y-3z=6-3y\\2x-y+4z=x\\3x-z=y+2\end{array}\right.;$
\item $\left\{\begin{array}{l}2x-y+3z=1 \\x-2y+z=5\\x+2z=3 \end{array}\right.;$
\item $\left\{\begin{array}{l}x+2y-z=1 \\y-4z=0\\x-2y+z=2 \end{array}\right.;$
\item $\left\{\begin{array}{l}x-3y+6z=1 \\x+y+z=5 \\x+2z=3 \end{array}\right.;$
\end{enumeratea}
\end{multicols}
\end{esercizio}

\begin{esercizio}[\Ast]
 \label{ese:19.60}
 Determinare la terna di soluzione dei seguenti sistemi.
\begin{multicols}{3}
\begin{enumeratea}
\item $\left\{\begin{array}{l}x-4y+6z=2 \\x+4y-z=2\\x+3y-2z=2 \end{array}\right.;$
\item $\left\{\begin{array}{l}4x-y-2z=1 \\3x+2y-z=4\\x+y+2z=4 \end{array}\right.;$
\item $\left\{\begin{array}{l}x-3y=3 \\x+y+z=-1\\2x-z=0 \end{array}\right.;$
\item $\left\{\begin{array}{l}2x-y+3z=1 \\x-6y+8z=2\\3x-4y+8z=2 \end{array}\right.;$
\item $\left\{\begin{array}{l}4x-6y-7z=-1 \\x+y-z=1\\3x+2y+6z=1 \end{array}\right.;$
\item $\left\{\begin{array}{l}4x-3y+z=4\\x+4y-3z=2 \\y-7z=0 \end{array}\right..$
\end{enumeratea}
\end{multicols}
\end{esercizio}

\begin{esercizio}[\Ast]
 \label{ese:19.61}
  Determinare la terna di soluzione dei seguenti sistemi.
\begin{multicols}{3}
\begin{enumeratea}
\item $\left\{\begin{array}{l}3x-6y+2z=1 \\x-4y+6z=5\\x-y+4z=10 \end{array}\right.;$
\item $\left\{\begin{array}{l}4x-y-7z=-12 \\x+3y+z=-4\\2x-y+6z=5 \end{array}\right.;$
\item $\left\{\begin{array}{l}2x+y-5z=2 \\x+y-7z=-2\\x+y+2z=1 \end{array}\right.;$
\item $\left\{\begin{array}{l}3x-y+z=-1\\x-y-z=3 \\x+y+2z=1 \end{array}\right.;$
\item $\left\{\begin{array}{l}x-4y+2z=7 \\-3x-2y+3z=0 \\x-2y+z=1 \end{array}\right.;$
\item $\left\{\begin{array}{l}-2x-2y+3z=4 \\2x-y+3z=0\\2x+y=1 \end{array}\right..$
\end{enumeratea}
\end{multicols}
\end{esercizio}


\begin{esercizio}
 \label{ese:19.62}
Quale condizione deve soddisfare il parametro~$a$ affinché il sistema seguente non sia privo di
significato? Determina la terna soluzione assegnando ad~$a$ il valore~2.
 \[\left\{\begin{array}{l}x+y+z=\dfrac{a^{2}+1}{a}\\ay-z=a^{2} \\y+ax=a+1+a^{2}z\end{array}\right..\]
\end{esercizio}

\begin{esercizio}
 \label{ese:19.63}
Determina il dominio del sistema e stabilisci se la terna soluzione è accettabile:
\[\longarray\left\{\begin{array}{l}\dfrac{5}{1-x}+\dfrac{3}{y+2}=\dfrac{2x}{xy-2+2x-y}\\
\dfrac{x+1-3(y-1)}{xyz}=\dfrac{1}{xy}-\dfrac{2}{yz}-\dfrac{3}{xz}\\
x+2y+z=0\end{array}\right..\]
\end{esercizio}

\begin{esercizio}
 \label{ese:19.64}
Verifica se il sistema è indeterminato:
\[\left\{\begin{array}{l}x+y=1 \\y-z=5
\\x+z+2=0 \end{array}\right..\]
\end{esercizio}

\begin{esercizio}
 \label{ese:19.65}
Determina il volume del parallelepipedo retto avente
per base un rettangolo, sapendo che le dimensioni della base e
l'altezza hanno come misura (in~$\unit{cm}$) i valori
di~$x$, $y$, $z$ ottenuti risolvendo il sistema:
\[\left\{\begin{array}{l}3x+1=2y+3z \\6x+y+2z=7
\\9(x-1)+3y+4z=0 \end{array}\right..\]
\end{esercizio}

\subsubsection*{19.6 - Sistemi da risolvere con sostituzioni delle variabili}

\begin{esercizio}[\Ast]
 \label{ese:19.66}
 Risolvi i seguenti sistemi per mezzo di opportune sostituzioni delle variabili.

\begin{enumeratea}
\item $\longarray\left\{\begin{array}{l}\dfrac{1}{2x}+\dfrac{1}{y}=-4\\\dfrac{2}{3x}+\dfrac{2}{y}=1\end{array}\right.;$ \quad sostituire~$u=\dfrac{1}{x};v=\dfrac{1}{y}$.
\item $\left\{\begin{array}{l}x^{2}+y^{2}=13\\x^{2}-y^{2}=5 \end{array}\right.;$\quad sostituire~$u=x^{2};v=y^{2}$
\item $\longarray\left\{\begin{array}{l}\dfrac{1}{x+y}+\dfrac{2}{x-y}=1\\\dfrac{3}{x+y}-\dfrac{5}{x-y}=2\end{array}\right.;$\quad sostituire~$u=\dfrac{1}{x+y};v=\ldots$.
\end{enumeratea}
\end{esercizio}

\begin{esercizio}[\Ast]
 \label{ese:19.67}
 Risolvi i seguenti sistemi per mezzo di opportune sostituzioni delle variabili.
\begin{multicols}{2}
\begin{enumeratea}
{\longarray
\item $\left\{\begin{array}{l}\dfrac{5}{2x}-\dfrac{2}{y}=2\\\dfrac{1}{x}+\dfrac{2}{y}=1\end{array}\right.;$
\item $\left\{\begin{array}{l}\dfrac{1}{x}+\dfrac{2}{y}=3\\\dfrac{1}{x}+\dfrac{3}{y}=4\end{array}\right.;$
\item $\left\{\begin{array}{l}\dfrac{2}{x}+\dfrac{4}{y}=-3\\\dfrac{2}{x}-\dfrac{3}{y}=4 \end{array}\right.;$
\item $\left\{\begin{array}{l}\dfrac{1}{x+1}-\dfrac{2}{y-1}=2\\\dfrac{2}{x+1}-\dfrac{1}{y-1}=3\end{array}\right..$}
\end{enumeratea}
\end{multicols}
\end{esercizio}

\begin{esercizio}[\Ast]
 \label{ese:19.68}
 Risolvi i seguenti sistemi per mezzo di opportune sostituzioni delle variabili.
\begin{multicols}{2}
\begin{enumeratea}
\item $\longarray\left\{\begin{array}{l}\dfrac{1}{x}-\dfrac{3}{y}+\dfrac{2}{z}=3 \\\dfrac{2}{x}-\dfrac{3}{y}+\dfrac{2}{z}=4
\\\dfrac{2}{x}+\dfrac{4}{y}-\dfrac{1}{z}=-3\end{array}\right.;$
\item $\longarray\left\{\begin{array}{l}\dfrac{4}{x^{2}}-\dfrac{2}{y^{2}}-\dfrac{2}{z^{2}}=0\\\dfrac{1}{x^{2}}+\dfrac{1}{z^{2}}=2
\\\dfrac{2}{y^{2}}-\dfrac{2}{z^{2}}=0\end{array}\right.;$
\item $\left\{\begin{array}{l}x^{3}+y^{3}=9 \\2x^{3}-y^{3}=-6 \end{array}\right.;$
\item $\left\{\begin{array}{l}x^{2}+y^{2}=-1\\x^{2}-3y^{2}=12\end{array}\right..$
\end{enumeratea}
\end{multicols}
\end{esercizio}

\subsection{Esercizi riepilogativi}

Gli esercizi indicati con (\croce) sono tratti da \emph{Matematica }~2, Dipartimento di
Matematica, ITIS V. Volterra, San Donà di Piave, Versione [11-12]~[S-A11], pg.~53; licenza CC, BY-NC-BD, per gentile concessione dei
professori che hanno redatto il libro.% Il libro è scaricabile da \url{http://www.istitutovolterra.it/dipartimenti/matematica/dipmath/docs/M2_1112.pdf}
\pagebreak
% il link http://www.istitutovolterra.it/dipartimenti/matematica/dipmath/docs/M2_1112.pdf non risulta essere più raggiungibile

\begin{esercizio}[\Ast]
 \label{ese:19.69}
 Risolvi i seguenti sistemi con più metodi ed eventualmente controlla
la soluzione graficamente.
\begin{multicols}{2}
\begin{enumeratea}
\item $\left\{\begin{array}{l}2x+y=1 \\2x-y=-1\end{array}\right.;$
\item $\left\{\begin{array}{l}2x=1+3y\\-y-2x=3\end{array}\right.;$
\item $\left\{\begin{array}{l}-x+2y=1 \\3x-y=3\end{array}\right.;$
\item $\left\{\begin{array}{l}5x-y=2\\2x+3y=-1 \end{array}\right..$
\end{enumeratea}
\end{multicols}
\end{esercizio}
%\newpage
\begin{esercizio}[\Ast]
 \label{ese:19.70}
 Risolvi i seguenti sistemi con più metodi ed eventualmente controlla
la soluzione graficamente.
\begin{multicols}{2}
\begin{enumeratea}
\item $\left\{\begin{array}{l}x+2y=3 \\3x-y=2\end{array}\right.;$
\item $\left\{\begin{array}{l}2x-y=1 \\x+2y=2\end{array}\right.;$
\item $\left\{\begin{array}{l}5x+3y=2 \\3x-2y=1\end{array}\right.;$
\item $\left\{\begin{array}{l}7x-2y=4\\8x-6y=9 \end{array}\right..$
\end{enumeratea}
\end{multicols}
\end{esercizio}

\begin{esercizio}[\Ast]
 \label{ese:19.71}
 Risolvi i seguenti sistemi con più metodi ed eventualmente controlla
la soluzione graficamente.
\begin{multicols}{2}
\begin{enumeratea}
\item $\left\{\begin{array}{l}3x-2y=4 \\2x+3y=5\end{array}\right.;$
\item $\left\{\begin{array}{l}3x-y=7 \\x-2y=5 \end{array}\right.;$
\item $\left\{\begin{array}{l}3x-2y=2\\2y-2x=-{\frac{4}{3}} \end{array}\right.;$
\item $\left\{\begin{array}{l}5x-2x=7\\-x-2y=-{\frac{1}{2}} \end{array}\right..$
\end{enumeratea}
\end{multicols}
\end{esercizio}

\begin{esercizio}[\Ast]
 \label{ese:19.72}
 Risolvi i seguenti sistemi con più metodi ed eventualmente controlla
la soluzione graficamente.
\begin{multicols}{2}
\begin{enumeratea}
{\longarray
\item $\left\{\begin{array}{l}\dfrac{2}{3}x-2y=-{\dfrac{1}{6}}\\-y-\dfrac{2}{3}y=\dfrac{3}{2} \end{array}\right.;$
\item $\left\{\begin{array}{l}\dfrac{1}{3}x-\dfrac{3}{2}y+1=0\\9y-2x-6=0 \end{array}\right.;$
\item $\left\{\begin{array}{l}-{\dfrac{1}{3}}x+\dfrac{3}{2}y-1=0\\3x-\dfrac{1}{5}y+\dfrac{3}{2}=0 \end{array}\right.;$
\item $\left\{\begin{array}{l}-{\dfrac{2}{3}}y+3x=y\\x-\dfrac{1}{2}y+3=0 \end{array}\right..$}
\end{enumeratea}
\end{multicols}
\end{esercizio}

\begin{esercizio}[\Ast]
 \label{ese:19.73}
 Risolvi i seguenti sistemi con più metodi ed eventualmente controlla
la soluzione graficamente.
\begin{multicols}{2}
\begin{enumeratea}
\item $\longarray\left\{\begin{array}{l}5y+\dfrac{3}{2}x=-2\\3x+10y-3=0 \end{array}\right.;$
\item $\left\{\begin{array}{l}{2x+y=0}\\{x-2y=-5}\end{array}\right.;$
\item $\longarray\left\{\begin{array}{l}\dfrac{1}{2}x-3y=\dfrac{1}{2}\\3(y-2)+x=0 \end{array}\right.;$
\item $\left\{\begin{array}{l}{x+y=-1}\\{x-y=5}\end{array}\right..$
\end{enumeratea}
\end{multicols}
\end{esercizio}
\pagebreak
\begin{esercizio}[\Ast]
 \label{ese:19.74}
 Risolvi i seguenti sistemi con più metodi ed eventualmente controlla
la soluzione graficamente.
\begin{multicols}{2}
\begin{enumeratea}
\item $\left\{\begin{array}{l}{2x+2y=6}\\{x-2y=-3}\end{array}\right.;$
\item $\left\{\begin{array}{l}{\dfrac{1}{3}x+3y+2=0}\\{2x+\dfrac{1}{2}y=\dfrac{11}{2}}\end{array}\right.;$
\item $\left\{\begin{array}{l}{2x-y=3}\\{x-2y=0}\end{array}\right.;$
\item $\left\{\begin{array}{l}{\dfrac{1}{2}x+\dfrac{1}{2}y=1}\\{\dfrac{2}{3}x+\dfrac{1}{3}y=1}\end{array}\right..$
\end{enumeratea}
\end{multicols}
\end{esercizio}

\begin{esercizio}[\Ast]
 \label{ese:19.75}
 Risolvi i seguenti sistemi con più metodi ed eventualmente controlla
la soluzione graficamente.
\begin{multicols}{2}
\begin{enumeratea}
 {\longarray
\item $\left\{\begin{array}{l}{2x-y=0}\\{4x+\dfrac{1}{2}y=\dfrac{5}{2}}\end{array}\right.;$
\item $\left\{\begin{array}{l}{2x+\dfrac{1}{2}y=-\dfrac{3}{10}}\\{-25x+5y=6}\end{array}\right.;$
\item $\left\{\begin{array}{l}{2x+y-3=0}\\{4x+2y+6=0}\end{array}\right.;$
\item $\left\{\begin{array}{l}{2x-y=-1}\\{x+\dfrac{1}{2}y=-{\dfrac{1}{2}}}\end{array}\right..$}
\end{enumeratea}
\end{multicols}
\end{esercizio}

\begin{esercizio}[\Ast]
\label{ese:19.76}
Risolvi i seguenti sistemi con più metodi ed eventualmente controlla
la soluzione graficamente.
\begin{multicols}{2}
\begin{enumeratea}
 {\longarray
\item $\left\{\begin{array}{l}{\dfrac{1}{2}x-\dfrac{1}{3}y=1}\\{3x-2y=3}\end{array}\right.;$
\item $\left\{\begin{array}{l}{10x-5y=26}\\{x+5y=-\dfrac{42}{5}}\end{array}\right.;$
\item $\left\{\begin{array}{l}\dfrac{1}{2}(x-3)-y=\dfrac{3}{2}(y-1)\\\dfrac{3}{2}(y-2)+x=6\left(x+\dfrac{1}{3}\right)\end{array}\right.;$
\item $\left\{\begin{array}{l}\dfrac{x+4y}{6}-3=0\\\dfrac{x}{2}-\dfrac{y}{4}=0\end{array}\right..$}
\end{enumeratea}
\end{multicols}
\end{esercizio}
%\pagebreak
\begin{esercizio}[\Ast]
 \label{ese:19.77}
 Risolvi i seguenti sistemi con più metodi ed eventualmente controlla
la soluzione graficamente.

\begin{enumeratea}
 {\longarray
\item $\left\{\begin{array}{l}3(x-4)=-{\dfrac{4y}{5}}\\7(x+y)+8\left(x-\dfrac{3y}{8}-2\right)=0\end{array}\right.;$
\item $\left\{\begin{array}{l}\dfrac{2}{5}(y-x-1)=\dfrac{y-x}{3}-\dfrac{2}{5}\\(x-y)^{2}-x(x-2y)=x+y(y-1)\end{array}\right.;$
\item $\left\{\begin{array}{l}2x-3(x-y)=-1+3y\\\dfrac{1}{2}x+\dfrac{1}{3}y=-{\dfrac{1}{6}}\end{array}\right.;$
\item $\left\{\begin{array}{l}(y+2)(y-3)-(y-2)^{2}+(x+1)^{2}=(x+3)(x-3)-\dfrac{1}{2}\\\left(y-\dfrac{1}{2}\right)\left(y+\dfrac{1}{4}\right)-(y-1)^{2}+2x+3=\dfrac{3}{4}\end{array}\right..$}
\end{enumeratea}
\end{esercizio}
%\newpage
\begin{esercizio}[\Ast]
 \label{ese:19.78}
 Risolvi i seguenti sistemi con più metodi ed eventualmente controlla
la soluzione graficamente.
\begin{multicols}{2}
\begin{enumeratea}
 {\longarray
\item $\left\{\begin{array}{l}\dfrac{\dfrac{x}{2}-y+5}{\dfrac{4}{3}-\dfrac{5}{6}}=x-\dfrac{\dfrac{x}{2}-\dfrac{y}{3}}{2}\\-x-\dfrac{\dfrac{y}{3}-x}{2}=1\end{array}\right.;$
\item $\left\{\begin{array}{l}x^{2}+\dfrac{y}{4}-3x=\dfrac{(2x+1)^{2}}{4}-\dfrac{y}{2}\\(y-1)^{2}=-8x+y^{2}\end{array}\right.;$
\item $\left\{\begin{array}{l}\dfrac{\dfrac{x+1}{2}-y}{2}=y-20x\\x-\dfrac{y}{4}=\dfrac{x-y}{6}\end{array}\right.;$
\item $\left\{\begin{array}{l}\dfrac{4y-\dfrac{5}{2}x+\dfrac{3}{2}}{\dfrac{5}{6}}=x-2y\\x=3y\end{array}\right..$}
\end{enumeratea}
\end{multicols}
\end{esercizio}

\begin{esercizio}[\Ast] %pag 505
 \label{ese:19.79}
 Risolvi i seguenti sistemi letterali.
\begin{multicols}{2}
\begin{enumeratea}
\item $\left\{\begin{array}{l}x-a=by \\2x+ay-3a=2b\end{array}\right.;$
\item $\left\{\begin{array}{l}x+y=2a \\bx+ay=a^{2}+b^{2} \end{array}\right.;$
\item $\left\{\begin{array}{l}x+ay=3a\\2x-3ay=a \end{array}\right.;$
\item $\left\{\begin{array}{l}x+y=2a\\2x+ay=2a^{2} \end{array}\right..$
\end{enumeratea}
\end{multicols}
\end{esercizio}

\begin{esercizio}[\Ast] %pag 506
 \label{ese:19.80}
 Risolvi i seguenti sistemi letterali.
\begin{multicols}{2}
\begin{enumeratea}
\item $\longarray\left\{\begin{array}{l}\dfrac{x-2a}{a}+\dfrac{y-3b}{b}=0\\\dfrac{3x+2y}{a}+\dfrac{x-2y}{b}=
\dfrac{6b}{a}+\dfrac{2a}{b}\end{array}\right.;$
\item $\longarray\left\{\begin{array}{l}\dfrac{x-a}{a-1}+\dfrac{y-2}{a+1}=0\\\dfrac{x}{a}-\dfrac{y}{a+1}=
\dfrac{a-1}{a+1}\end{array}\right.;$
\item $\left\{\begin{array}{l}x+my=m^{2}+1\\2x-y=m+3 \end{array}\right.;$
\item $\longarray\left\{\begin{array}{l}\dfrac{x}{a}-\dfrac{y}{b}=0\\\dfrac{x}{b}+\dfrac{y}{a}=\dfrac{a^{2}+b^{2}}{ab} \end{array}\right..$
\end{enumeratea}
\end{multicols}
\end{esercizio}

\begin{esercizio}[\Ast] %pag 509
 \label{ese:19.81}
 Risolvi i seguenti sistemi letterali.
\begin{multicols}{2}
\begin{enumeratea}\longarray{
\item $\left\{\begin{array}{l}\dfrac{x+y}{4x-4y}-\dfrac{x-y}{4x+4y}=\dfrac{x^{2}-1}{x^{2}-y^{2}} \\\dfrac{x}{a+1}+\dfrac{y}{a}=2\end{array}\right.;$
\item $\left\{\begin{array}{l}\dfrac{2x}{a+b}-\dfrac{y}{a-b}=1 \\\dfrac{(a-b)x}{a+b}-\dfrac{(a+b)y}{a-b}=-2b\end{array}\right.;$
\item $\left\{\begin{array}{l}\dfrac{x}{3-a}+\dfrac{y}{a}=\dfrac{2}{a}\\\dfrac{x}{a-1}-\dfrac{y}{2-a}={\dfrac{4-2a}{a-1}} \end{array}\right.;$
\item $\left\{\begin{array}{l}\dfrac{x+y-2}{a-1}-\dfrac{x-y}{a+1}=0\\\dfrac{x+ay}{a^{2}-1}+\dfrac{x-ay}{a+1}={\dfrac{a^{2}-a+2}{a^{2}-1}} \end{array}\right..$}
\end{enumeratea}
\end{multicols}
\end{esercizio}

\begin{multicols}{2}
\begin{esercizio}
 \label{ese:19.82}
Determina due numeri sapendo che la loro somma è~37 e la loro
differenza è~5.
\end{esercizio}

\begin{esercizio}[\Ast]
 \label{ese:19.83}
Il doppio della somma di due numeri è uguale al secondo numero aumentato
del triplo del primo, inoltre aumentando il primo numero di~12 si
ottiene il doppio del secondo diminuito di~6.
\end{esercizio}

\begin{esercizio}[\Ast]
 \label{ese:19.84}
Determina tre numeri la cui somma è~81. Il secondo supera il primo
di~3. Il terzo numero è dato dalla somma dei primi due.
\end{esercizio}

\begin{esercizio}[\Ast]
 \label{ese:19.85}
Determina due numeri sapendo che la loro somma è pari al doppio del minore aumentato di~$1/4$
del maggiore, mentre la loro differenza è uguale a~$9$.
\end{esercizio}

\begin{esercizio}[\Ast]
 \label{ese:19.86}
Determina due numeri la cui somma è~57 e di cui si sa che il doppio del più grande
diminuito della metà del più piccolo è~49.
\end{esercizio}

\begin{esercizio}[\Ast]
 \label{ese:19.87}
Determina tre lati sapendo che il triplo del primo lato è
uguale al doppio del secondo aumentato di~$10\unit{m}$; la differenza tra il
doppio del terzo lato e il doppio del secondo lato è uguale al primo
lato aumentato di~12; la somma dei primi due lati è uguale al terzo
lato.
\end{esercizio}

\begin{esercizio}[\Ast]
 \label{ese:19.88}
Determina un numero di due cifre sapendo che la cifra delle decine
è il doppio di quella delle unità e scambiando le due cifre si
ottiene un numero più piccolo di~27 del precedente.
\end{esercizio}

\begin{esercizio}[\Ast]
 \label{ese:19.89}
Determina il numero intero di due
cifre di cui la cifra delle decine supera di~2 la cifra delle unità e
la somma delle cifre è~12.
\end{esercizio}

\begin{esercizio}[\croce]
 \label{ese:19.90}
Determina due numeri naturali il cui quoziente è~5 e la cui
differenza è~12.
\end{esercizio}

\begin{esercizio}[\Ast, \croce]
 \label{ese:19.91}
Determinare un numero naturale di due cifre sapendo che la loro somma
è~12 e che, invertendole, si ottiene un numero che supera di~6 la
metà di quello iniziale.
\end{esercizio}

\begin{esercizio}[\croce]
 \label{ese:19.92}
Determinare la frazione che diventa uguale a~$5/6$ aumentando i suoi
termini di~2 e diventa~$1/2$ se i suoi termini diminuiscono di~2.
\end{esercizio}

\begin{esercizio}[\Ast, \croce]
 \label{ese:19.93}
La somma delle età di due coniugi è~65 anni; un settimo
dell'età del marito è uguale ad un sesto
dell'età della moglie. Determinare le età dei
coniugi.
\end{esercizio}

\begin{esercizio}[\Ast, \croce]
 \label{ese:19.94}
Un numero naturale diviso per~3 dà un certo quoziente e resto~1. Un
altro numero naturale, diviso per~5, dà lo stesso quoziente e resto~3.
Sapendo che i due numeri hanno per somma~188, determinali e calcola
il quoziente.
\end{esercizio}

\begin{esercizio}[\Ast]
 \label{ese:19.95}
Giulio e Giulia hanno svuotato i loro
salvadanai per comparsi una bici. Nel negozio c'è
una bella bici che piace a entrambi e costa \officialeuro~$180$, ma nessuno dei
due ha i soldi sufficienti per comprarla. Giulio dice:
<<Se mi dai la metà dei tuoi soldi compro io la
bici>>. Giulia ribatte: <<se mi dai la terza
parte dei tuoi soldi la bici la compro io>>. Quanti soldi
hanno rispettivamente Giulio e Giulia?
\end{esercizio}

\begin{esercizio}
 \label{ese:19.96}
A una recita scolastica per beneficenza vengono incassati
\officialeuro~$216$ per un totale di~102 biglietti venduti. I ragazzi della
scuola pagano \officialeuro~$1$, i ragazzi che non sono di quella scuola
pagano \officialeuro~$\np{1,5}$ e gli adulti pagano \officialeuro~$3$. Quanti sono i
ragazzi della scuola che hanno assistito alla recita?
\end{esercizio}

\begin{esercizio}
 \label{ese:19.97}
Da un cartone quadrato di lato~$12\unit{cm}$, si taglia prima una scriscia
parallela a un lato e di spessore non noto, poi si taglia dal lato
adiacente una striscia parallela al lato spessa~$2\unit{cm}$ in più rispetto
alla striscia precedente. Sapendo che il perimetro del rettangolo
rimasto è~$\np[cm]{33,6}$, calcola l'area del rettangolo rimasto.
\end{esercizio}

\begin{esercizio}[\Ast]
 \label{ese:19.98}
Al bar per pagare~4 caffè e~2 cornetti si spendono \officialeuro~$\np{4,60}$, per pagare~6 caffè
e~3 cornetti si spendono \officialeuro~$\np{6,90}$. \`E possibile
determinare il prezzo del caffè e quello del cornetto?
\end{esercizio}

\begin{esercizio}[\Ast]
 \label{ese:19.99}
 Al bar, Mario offre la colazione agli amici perché è il suo
compleanno: per~4 caffè e~2 cornetti paga \officialeuro~$\np{4,60}$. Subito dopo
arrivano altri tre amici che prendono un caffè e un cornetto
ciascuno, e questa volta Mario paga \officialeuro~$\np{4,80}$. Quanto costa un caffè e
quanto un cornetto?
\end{esercizio}

\begin{esercizio}[\Ast]
 \label{ese:19.100}
Un cicloturista percorre~$218\unit{km}$ in tre giorni. Il secondo giorno percorre
il~$20\%$ in più del primo giorno. Il terzo giorno percorre~$14\unit{km}$ in
più del secondo giorno. Qual è stata la lunghezza delle tre tappe?
\end{esercizio}

\begin{esercizio}[\Ast]
 \label{ese:19.101}
In un parcheggio ci sono moto e auto. In tutto si contano~43 mezzi e~140
ruote. Quante sono le auto e quante le moto?
\end{esercizio}

\begin{esercizio}
 \label{ese:19.102}
Luisa e Marisa sono due sorelle. Marisa, la più grande, è nata~3 anni
prima della sorella e la somma delle loro età è~59. Qual è
l'età delle due sorelle?
\end{esercizio}

\begin{esercizio}
 \label{ese:19.103}
Mario e Lucia hanno messo da parte del denaro. Lucia ha \officialeuro~5
in più di Mario. Complessivamente potrebbero comprare
schede prepagate per i cellulari per \officialeuro~$45$. Quanto possiede Mario e quanto
Lucia?
\end{esercizio}

\begin{esercizio}
 \label{ese:19.104}
Una macchina per ghiaccio produce~10 cubetti di ghiaccio al minuto, mentre
una seconda macchina ne produce~7 al minuto. Sapendo
che in tutto sono stati prodotti~175 cubetti e che complessivamente le
macchine hanno lavorato per~22 minuti, quanti cubetti ha prodotto la
prima macchina e quanti ne ha prodotti la seconda?
\end{esercizio}

\begin{esercizio}[\Ast]
 \label{ese:19.105}
In un parcheggio ci sono automobili, camion e moto per un totale di~62 mezzi.
Le auto hanno~4 ruote, i camion ne hanno~6 e le moto ne hanno~2.
In totale le ruote sono~264. Il numero delle ruote delle auto è uguale
al numero delle ruote dei camion. Determina quante auto, quanti camion
e quante moto ci sono nel parcheggio.
\end{esercizio}

\begin{esercizio}
 \label{ese:19.106}
 Un vasetto di marmellata pesa~$780\unit{g}$. Quando nel vasetto rimane
metà marmellata, il vasetto pesa~$420\unit{g}$. Quanto pesa il vasetto vuoto?
\end{esercizio}


\begin{esercizio}[\Ast]
 \label{ese:19.107}
Una gelateria prepara per la giornata di Ferragosto~$30\unit{kg}$ di
gelato. Vende i coni da due palline a \officialeuro~$\np{1,50}$ e i coni da tre
palline a \officialeuro~$\np{2,00}$. Si sa che da~$2\unit{kg}$ di gelato si fanno~25
palline di gelato. A fine giornata ha venduto tutto il gelato e ha
incassato \officialeuro~$\np{272,50}$. Quanti coni da due palline ha venduto?
\end{esercizio}

\begin{esercizio}[Prove Invalsi~2004-2005]
 \label{ese:19.108}
Marco e Luca sono fratelli. La somma delle loro età è~23
anni. Il doppio dell'età di Luca è uguale alla
differenza tra l'età del loro padre e il triplo
dell'età di Marco. Quando Luca è nato, il padre
aveva~43 anni. Determina l'età di Marco e di Luca.
\end{esercizio}

\begin{esercizio}[Giochi d'autunno~2010, Centro Pristem]
 \label{ese:19.109}
Oggi Angelo ha un quarto dell'età di sua
madre. Quando avrà~18 anni, sua madre avrà il triplo della sua
età. Quanti anni hanno attualmente i due?
\end{esercizio}

\begin{esercizio}[Giochi di Archimede, 2008]
 \label{ese:19.110}
 Pietro e Paolo festeggiano il loro onomastico in pizzeria con i
loro amici. Alla fine della cena il conto viene diviso in parti uguali
tra tutti i presenti e ciascuno dovrebbe pagare \officialeuro~$12$. Con grande
generosità però gli amici decidono di offrire la cena a Pietro e
Paolo; il conto viene così nuovamente diviso in parti uguali tra gli amici
di Pietro e Paolo (cioè tutti i presenti esclusi Pietro e Paolo) e
ciascuno di loro paga \officialeuro~$16$. Quanti sono gli amici di Pietro e Paolo?
\end{esercizio}


\begin{esercizio}[\Ast]
 \label{ese:19.111}
 Al bar degli studenti caffè e cornetto costano
\officialeuro~$\np{1,50}$; cornetto e succo di frutta costano \officialeuro~$\np{1,80}$,
caffè e succo di frutta costano \officialeuro~$\np{1,70}$. Quanto costano in
tutto~7 caffè,~5 cornetti e~3 succhi di frutta?
\end{esercizio}

\begin{esercizio}[\croce]
 \label{ese:19.112}
Un negozio ha venduto scatole contenenti~6 fazzoletti ciascuna
ed altre contenenti~12 fazzoletti ciascuna, per un totale di~156
fazzoletti. Il numero delle confezioni da~12 ha superato di~1 la metà
di quello delle confezioni da~6. Quante confezioni di ogni tipo si sono
vendute?
\end{esercizio}

\begin{esercizio}[\Ast, \croce]
 \label{ese:19.113}
Nella città di Nonfumo gli unici negozi sono tabaccherie e
latterie. L'anno scorso le tabaccherie erano i~$2/3$
delle latterie; quest'anno due tabaccherie sono
diventate latterie cosicché ora le tabaccherie sono i~$9/16$ delle
latterie. Dall'anno scorso a
quest'anno il numero cimplessivo dei negozi di Nonfumo
è rimasto lo stesso. Quante latterie c'erano
l'anno scorso a Nonfumo?
\end{esercizio}

\begin{esercizio}
 \label{ese:19.114}
Un rettangolo di perimetro~$80\unit{cm}$ ha la base che è i~$2/3$
dell'altezza. Calcolare l'area del
rettangolo.
\end{esercizio}

\begin{esercizio}[\Ast]
 \label{ese:19.115}
 Un trapezio isoscele ha il perimetro di~$72\unit{cm}$. La base minore è
i~$3/4$ della base maggiore; il lato obliquo è pari alla somma dei~$2/3$
della base minore con i~$3/2$ della base maggiore. Determina le misure
delle basi del trapezio.
\end{esercizio}

\begin{esercizio}[\Ast]
 \label{ese:19.116}
 Calcola l'area di un rombo le cui diagonali
sono nel rapporto~$3/2$. Si sa che la differenza tra le due diagonali è~$16\unit{cm}$.
\end{esercizio}

\begin{esercizio}
 \label{ese:19.117}
In un triangolo rettangolo i~$3/4$ dell'angolo
acuto maggiore sono pari ai~$24/13$ dell'angolo acuto
minore. Determinare l'ampiezza degli angoli.
\end{esercizio}

\begin{esercizio}[\Ast]
 \label{ese:19.118}
 In un triangolo, un angolo supera di~$16\grado$ un secondo
angolo; il terzo angolo è pari ai~$29/16$ della somma dei primi due.
Determina le misure degli angoli del triangolo.
\end{esercizio}

\begin{esercizio}
 \label{ese:19.119}
In un rettangolo di
perimetro~$120\unit{cm}$, la base è~$2/3$ dell'altezza. Calcola
l'area del rettangolo.
\end{esercizio}

\begin{esercizio}
 \label{ese:19.120}
Determina le misure dei tre lati~$x$, $y$, $z$ di un triangolo sapendo che il perimetro
è~$53\unit{cm}$. Inoltre la misura~$z$ differisce di~$19\unit{cm}$ dalla somma delle
altre due misure e che la misura~$x$ differisce di~$11\unit{cm}$ dalla differenza
tra~$y$ e~$z$.
\end{esercizio}

\begin{esercizio}[\Ast]
 \label{ese:19.121}
Aumentando la base di un rettangolo di~$5\unit{cm}$ e
l'altezza di~$12\unit{cm}$, si ottiene un rettangolo di
perimetro~$120\unit{cm}$ che è più lungo di~$12\unit{cm}$ del perimetro del
rettangolo iniziale. Quanto misurano base e altezza del rettangolo?
\end{esercizio}

\begin{esercizio}[\Ast]
 \label{ese:19.122}
In un triangolo isoscele di perimetro~$64\unit{cm}$, la differenza tra la base e la metà del
lato obliquo è~$4\unit{cm}$. Determina la misura della base e del lato obliquo
del triangolo.
\end{esercizio}

\begin{esercizio}[\Ast]
 \label{ese:19.123}
Un segmento~$AB$ di~$23\unit{cm}$ viene diviso da un suo punto~$P$ in due parti tali che il triplo
della loro differenza è uguale al segmento minore aumentato di~$20\unit{cm}$.
Determina le misure dei due segmenti in cui viene diviso~$AB$ dal punto~$P$.
\end{esercizio}
\end{multicols}

\subsection{Risposte}

\paragraph{19.8.} a)~$(1;0)$, \quad b)~$(-2;-2)$, \quad c)~$(0;1)$, \quad d)~$(0;1)$.

\paragraph{19.9.} a)~$(4;5)$, \quad b)~indeterminato, \quad c)~$(1;-1)$, \quad d)~$(-4;2)$.

\paragraph{19.10.} a)~indeterminato, \quad b)~$(4;5)$, \quad c)~impossibile, \quad d)~indeterminato.

\paragraph{19.11.} a)~$(-66;-12)$, \quad b)~$(2;3)$, \quad d)~$(0;0)$.

\paragraph{19.12.} a)~$\left(-{\frac{9}{8}};-\frac{9}{8}\right)$, \quad b)~$\left(\frac{28}{17};\frac{6}{17}\right)$, \quad d)~$(1;-3)$.

\paragraph{19.13.} a)~$\left(-4;-{\frac{3}{2}}\right)$, \quad c)~$\left(\frac{1}{6};\frac{35}{24}\right)$.

\paragraph{19.16.} a)~$(0;0)$, \quad b)~$(2;-1)$, \quad c)~$(2;1)$, \quad d)~$(-1;-3)$.

\paragraph{19.17.} a)~impossibile, \quad c)~impossibile, \quad d)~indeterminato.

\paragraph{19.18.} a)~$\left(\frac{2}{3};0\right)$, \quad b)~$\left(-{\frac{1}{5}};\frac{1}{5}\right)$, \quad c)~$(3;4)$, \quad d)~$(0;1)$.

\paragraph{19.20.} a)~$(0;0)$, \quad b)~$(0;1)$, \quad c)~$\left(\frac{1}{2};0\right)$, \quad d)~$\left(-{\frac{1}{2}};\frac{1}{2}\right)$.

\paragraph{19.21.} a)~$(1;1)$, \quad b)~$(1;1)$, \quad c)~$\left(\frac{35}{12};\frac{19}{12}\right)$, \quad d)~$\left(-{\frac{12}{11}};\frac{7}{11}\right)$.

\paragraph{19.22.} a)~$(-11;-31)$, \quad b)~$(1;1)$, \quad c)~$(1;2)$.

\paragraph{19.26.} a)~$(2;0)$, \quad b)~$\left(\frac{13}{12};\frac{5}{12}\right)$, \quad c)~$\left(-{\frac{12}{11}};\frac{7}{11}\right)$, \quad d)~$\left(0;-\frac{1}{2}\right)$.

\paragraph{19.27.} a)~$(21,-12)$, \quad b)~$\left(-{\frac{240}{19}};\frac{350}{19}\right)$, \quad c)~$\left(\frac{34}{37};\frac{16}{37}\right)$, \quad d)~$\left(1;\frac{7}{3}\right)$.

\paragraph{19.28.} a)~$\left(-1;0\right)$, \quad b)~$\left(\frac{1}{2};-1\right)$, \quad c)~$\left(\frac{3}{10};\frac{7}{10}\right)$, \quad d)~$\left(-{\frac{1}{4}};\frac{1}{6}\right)$.

\paragraph{19.29.} a)~impossibile, \quad b)~indeterminato, \quad c)~$(a;2a)$, \quad d)~$(2k;-k)$.

\paragraph{19.31.} a)~$\left(5;9\right)$, \quad b)~$\left(-4;-2\right)$, \quad c)~$\left(5;2\right)$, \quad d)~$\left({\frac{1}{6}};-3\right)$.

\paragraph{19.32.} a)~$\left(\frac{4}{11};-\frac{1}{3}\right)$, \quad b)~$\left(\frac{5}{3};2\right)$, \quad c)~$\left(\frac{31}{10};2\right)$, \quad d)~$\left(\frac{3}{5};\frac{2}{3}\right)$.

\paragraph{19.41.} a)~rette parallele, sistema impossibile, \quad b)~$(-3;-8)$, \quad c)~rette identiche, indeterminato, \quad d)~$(2;2)$.

\paragraph{19.42.} a)~$\left(-1;0\right)$, \quad b)~$(2;0)$, \quad c)~$\left(-1;0\right)$, \quad d)~rette parallele, impossibile.

\paragraph{19.43.} a)~$\left(0;-\frac{1}{2}\right)$, \quad b)~$(1;1)$, \quad c)~$\left(\frac{3}{4};\frac{1}{2}\right)$.

\paragraph{19.46.} a)~indeterminato, \quad b)~$(3;3)$, \quad c)~$\left(-\frac{5}{11};\frac{7}{11}\right)$, \quad d)~impossibile.

\paragraph{19.47.} a)~$\left(\frac{9}{5};-\frac{8}{5}\right)$, \quad b)~$(-1;-1)$, \quad c)~impossibile, \quad d)~$\left(-\frac{1}{5};\frac{2}{5}\right)$.

\paragraph{19.48.} a)~$(39;-38)$, \quad b)~$\left(\frac{3}{4};-\frac{3}{4}\right)$, \quad c)~$(-6;2)$, \quad d)~$(-2;-5)$.

\paragraph{19.49.} a)~$\left(-\frac{9}{8};-\frac{9}{8}\right)$, \quad b)~$(1;1)$, \quad c)~impossibile, \quad d)~$\left(-\frac{3}{17};\frac{6}{17}\right)$.

\paragraph{19.50} a)~$(-18;-1)$, \quad b)~$\left(\frac{7}{4};\frac{1}{2}\right)$, \quad c)~$(2;-1)$.

\paragraph{19.51} a)~$\left(\frac{6}{5};\frac{34}{5}\right)$.

\paragraph{19.52.} $a>0$.

\paragraph{19.56.} $-\frac{1}{2}<a<\frac{1}{2}$.

\paragraph{19.57.} a)~$a\neq~0\rightarrow (a;1)$,
\quad b)~determinato per~$k\neq~14$, $k\neq \frac{6}{7}$ con soluzioni~$\left(\frac{k-6}{4}; \frac{5k-6}{4}\right)$;
se~$k=14\vee k=\frac{6}{7}$ impossibile,
\quad c)~determinato~$\forall k$ con soluzioni~$\left(\frac{12k}{6k^{2}+1};\frac{2}{6k^{2}+1}\right)$,
\protect\\ d)~determinato per~$k\neq -2$, $k\neq~2$ con soluzioni~$\left(\frac{4k-6}{k^{2}-4}; \frac{8-3k}{4(k^{2}-4)}\right)$; se~$k=-2 \vee k=2$ impossibile.

\paragraph{19.58.} a)~Determinato per~$k\neq -4$, $k\neq~4$, $k\neq~0$ con soluzioni~$\left(\frac{3k^{2}+4k}{16-k^{2}}; \frac{k+12}{16-k^{2}}\right)$;\protect\\
se~$k=-4\vee k=4$ impossibile; se~$k=0$ indeterminato con soluzioni tipo~$(0;t)$ con~$t\in\insR$,
\quad b)~determinato per~$k\neq~0, k\neq~4$ con soluzioni~$\left(\frac{6}{4-k}; \frac{6}{k(4-k)}\right)$;
se~$k=0\vee k=4$ impossibile,
\quad c)~determinato per~$k\neq~1, k\neq \frac{3}{2}$ con soluzioni~$\left(\frac{1}{2k-3}; \frac{1}{2k-3}\right)$;
se~$k=\frac{3}{2}$ impossibile; se~$k=1$ indeterminato con soluzioni del tipo~$(t;-1)$.

\paragraph{19.59.} a)~$(0; -2; 3)$,\quad b)~$\left(3;\frac{8}{9};\frac{1}{9}\right)$,\quad c)~$(1; 1;0)$,\quad d)~$(-21, -7, 12)$, \quad e)~$\left(\frac{3}{2};-\frac{2}{7};-\frac{1}{14}\right)$,\protect\\ f)~$(-5; 6; 4)$.

\paragraph{19.60.} a)~$(2; 0; 0)$, \quad b)~$(1; 1; 1)$, \quad c)~$(0; -1; 0)$, \quad d)~$\left(\frac{2}{3};-\frac{2}{3};-\frac{1}{3}\right)$, \quad e)~$\left(\frac{9}{31};\frac{17}{31};-\frac{5}{31}\right)$, \protect\\ f)~$\left(\frac{7}{6};\frac{7}{30};\frac{1}{30}\right)$.

\paragraph{19.61.} a)~$(5; 3; 2)$, \quad b)~$\left(-{\frac{60}{43}};-\frac{53}{43};\frac{47}{43}\right)$, \quad  c)~$\left(\frac{10}{3};-3;\frac{1}{3}\right)$,
\quad d)~$(6; 11; -8)$, e)~$\left(-5;-\frac{33}{4};-\frac{21}{2}\right)$, f)~$\left(-{\frac{5}{2}};6;\frac{11}{3}\right)$.

\paragraph{19.66.} a)~$\left(-{\frac{1}{27}};\frac{2}{19}\right)$, \quad b)~$(3;2)$, $(-3;2)$, $(3;-2)$, $(-3;-2)$,
\quad c)~$\left(\frac{55}{9};-\frac{44}{9}\right)$.

\paragraph{19.67.} a)~$\left(\frac{7}{6};14\right)$, \quad b)~$\left(1;1\right)$, \quad c)~$\left(2;-1\right)$, \quad d)~$\left(-{\frac{1}{4}};-2\right)$.

\paragraph{19.68.} a)~$\left(1;-\frac{5}{8};-\frac{5}{7}\right)$, \quad b)~$(1;1;1)$, $(-1;1;1)$, $(1;-1;1)$, $(1;1;-1)$, $(-1;-1;1)$, $(-1;1;-1)$, $(1;-1;-1)$, $(-1;-1;-1)$, \quad c)~$(1;2)$, \quad d)~$\emptyset $.


\paragraph{19.69.} a)~$(0;1)$, \quad b)~$(-1;-1)$, \quad c)~$\left(\frac{7}{5};\frac{6}{5}\right)$, \quad d)~$\left(\frac{5}{17};-\frac{9}{17}\right)$.

\paragraph{19.70.} a)~$\left(1;1\right)$, \quad b)~$\left(\frac{4}{5};\frac{3}{5}\right)$, \quad c)~$\left(\frac{7}{19};\frac{1}{19}\right)$,
\quad d)~$\left(\frac{3}{13};-\frac{31}{26}\right)$.

\paragraph{19.71.} a)~$\left(\frac{22}{13};\frac{7}{13}\right)$, \quad b)~$\left(\frac{9}{5};-\frac{8}{5}\right)$,
\quad c)~$(\frac{2}{3};0)$, \quad d)~$\left(\frac{7}{3};-\frac{11}{12}\right)$.

\paragraph{19.72.} a)~$\left(-{\frac{59}{20}};-\frac{9}{10}\right)$, \quad b)~indeterminato, \quad c)~$(-{\frac{123}{266}};\frac{75}{133})$, \quad d)~$(-30;-54)$.

\paragraph{19.73.} a)~Impossibile, \quad b)~$(-1;2)$, \quad c)~$\left(\frac{13}{3};\frac{5}{9}\right)$, \quad d)~$(2;-3)$.

\paragraph{19.74.} a)~$(1;2)$, \quad b)~$(3;-1)$, \quad c)~$(2;1)$, \quad d)~$(1;1)$.

\paragraph{19.75.} a)~$\left(\frac{1}{2};1\right)$, \quad b)~$\left(-{\frac{1}{5}};\frac{1}{5}\right)$,
\quad c)~impossibile, \quad d)~$\left(-{\frac{1}{2}};0\right)$.

\paragraph{19.76.} a)~$\emptyset $, \quad b)~$\left(\frac{8}{5};-2\right)$, \quad c)~$(-\frac{50}{47};-\frac{10}{47})$,
\quad d)~$(2;4)$.

\paragraph{19.77.} a)~impossibile, \quad c)~$(1;-2)$, \quad d)~$(-1;\frac{1}{2})$.

\paragraph{19.78.} a)~$(-{\frac{92}{27}};\frac{38}{9})$, \quad b)~$(\frac{1}{8};1)$, \quad c)~$(-{\frac{1}{21}};-{\frac{10}{21}})$, \quad d)~$(\frac{27}{26};\frac{9}{26}$.

\paragraph{19.79.} a)~$(a+b;1)$, \quad b)~$(a+b;a-b)$, \quad c)~$(2a;1)$, \quad d)~$(0;2a)$.

\paragraph{19.80.} a)~$(2a;3b)$, \quad b)~$(a;2)$, \quad c)~$(m+1;m-1)$, \quad d)~$(a;b)$.

\paragraph{19.81.} a)~$(a+1;a)$, \quad b)~$(a+b;a-b)$, \quad c)~$(3-a;2-a)$, \quad d)~$\left(a;\frac{1}{a}\right)$.

\begin{multicols}{2}

\paragraph{19.83.} $(18;18)$.

\paragraph{19.84.} $\np{18,75}; \np{21,75}; \np{40,5}$.

\paragraph{19.85.} $(27;36)$.

\paragraph{19.86.} $(26;31).$

\paragraph{19.87.} $(12\unit{m}\text{,~}13\unit{m}\text{,~}25\unit{m})$.

\paragraph{19.88.} 63.

\paragraph{19.89.} 75.

\paragraph{19.91.} 84.

\paragraph{19.93.} $(35;30)$.

\paragraph{19.94.} $(70;118;23)$.

\paragraph{19.95.} $(108;144)$.

\paragraph{19.98.} Indeterminato.

\paragraph{19.99.} \officialeuro~$\np{0,7}$ e \officialeuro~$\np{0,9}$.

\paragraph{19.100.} $60\unit{km}$; $72\unit{km}$; $86\unit{km}$.

\paragraph{19.101.} 27; 16.

\paragraph{19.105.} 30 auto; 20 camion; 12 moto.

\paragraph{19.107.} 135.

\paragraph{19.111.} \officialeuro~$\np{11,90}$.

\paragraph{19.113.} 30.

\paragraph{19.115.} $\frac{288}{23}\unit{cm};\frac{216}{23}\unit{cm}$.

\paragraph{19.116.} $\np[cm^2]{1536}$.

\paragraph{19.118.} $24\grado$; $40\grado$; $116\grado$.

\paragraph{19.121.} Impossibile.

\paragraph{19.122.} $16\unit{cm}$; $24\unit{cm}$.

\paragraph{19.123.} $7\unit{cm}$; $16\unit{cm}$.
\end{multicols}
