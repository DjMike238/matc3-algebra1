% (c) 2012-2013 Claudio Carboncini - claudio.carboncini@gmail.com
% (c) 2012-2014 Dimitrios Vrettos - d.vrettos@gmail.com

\chapter{Equazioni numeriche frazionarie}

Affrontiamo ora le equazioni in cui l'incognita compare anche al denominatore.

\begin{definizione} Un'equazione in cui l'incognita compare al denominatore si chiama \emph{frazionaria} o \emph{fratta}.
\end{definizione}

\begin{exrig}
 \begin{esempio}
Risolvere~$\dfrac{3x-2}{1+x}=\dfrac{3x}{x-2}$.
 \end{esempio}
Questa equazione si differenzia da quelle affrontate in precedenza per il fatto che l'incognita compare anche al denominatore.
Riflettendo sulla richiesta del problema, possiamo senz'altro affermare che, se esiste il valore che rende
la frazione al primo membro uguale alla frazione al secondo membro, esso non deve annullare nessuno dei due denominatori,
poiché in questo caso renderebbe priva di significato la scrittura, in quanto frazioni con denominatore~$0$ sono prive di significato.

Per risolvere un'equazione frazionaria, prima di tutto dobbiamo renderla nella forma
\begin{equation*}
\frac{F(x)}{G(x)}=0.
\end{equation*}

\begin{enumeratea}
 \item Determiniamo il~$\mcm$ dei denominatori, $\mcm=(1+x)\cdot (x-2)$.
    Osserviamo che per~$x = -1$ oppure per~$x = 2$ le frazioni perdono di significato, in quanto si annulla il denominatore;
 \item imponiamo le condizioni di esistenza:~$1+x\neq~0$ e~$x-2\neq~0$ cioè~$\CE x\neq -1\wedge x\neq~2$. La ricerca dei valori
    che risolvono l'equazione viene ristretta all'insieme~$\Dom=\insR-\{-1\text{,~}2\}$, detto \emph{dominio} dell'equazione o
    \emph{insieme di definizione};
 \item applichiamo il primo principio d'equivalenza trasportando al primo membro la frazione che si trova al secondo membro
    e riduciamo allo stesso denominatore ($\mcm$)
    \begin{equation*}
      \frac{(3x-2)\cdot (x-2)-3x\cdot (1+x)}{(1+x)\cdot (x-2)}=0;
    \end{equation*}
 \item applichiamo il secondo principio di equivalenza moltiplicando ambo i membri per il~$\mcm$,
    certamente diverso da zero per le condizioni poste precedentemente. L'equazione diventa:~$(3x-2)\cdot (x-2)-3x\cdot (1+x)=0$;
 \item eseguiamo le moltiplicazioni e sommiamo i monomi simili per portare l'equazione alla forma canonica:
    $3x^{2}-6x-2x+4-3x-3x^{2}=0\: \Rightarrow\: -11x=-4$;
 \item dividiamo ambo i membri per~$-11$, per il secondo principio di equivalenza si ha:~$x=\frac{4}{11}$;
 \item confrontiamo il valore trovato con le~$\CE$: in questo caso la soluzione appartiene al dominio~$\Dom$, quindi possiamo concludere
    che è accettabile. L'insieme soluzione è:~$\IS=\left\{\frac{4}{11}\right\}$.
\end{enumeratea}

%\newpage
 \begin{esempio}
Risolvere~$\dfrac{x^{2}+x-3}{x^{2}-x}=1-\dfrac{5}{2x}$.
\end{esempio}

\begin{enumeratea}
 \item Determiniamo il~$\mcm$ dei denominatori. Per fare questo dobbiamo prima scomporli in fattori.
    Riscriviamo:~$\dfrac{x^{2}+x-3}{x\cdot (x-1)}=1-\dfrac{5}{2x}$ con~$\mcm=2x\cdot (x-1)$;
 \item condizioni di esistenza: \[x-1\neq~0\wedge~2x\neq~0\text{,}\] cioè~$x\neq~1\wedge x\neq~0$. Il dominio è~$\Dom=\insR-\{1\text{,~}0\}$;
 \item trasportiamo al primo membro ed uguagliamo a zero \[\frac{x^{2}+x-3}{x\cdot (x-1)}-1+\frac{5}{2x}=0\]
    e riduciamo allo stesso denominatore ($\mcm$) ambo i membri \[\frac{2x^{2}+2x-6-2x^{2}+2x+5x-5}{2x\cdot (x-1)}=0;\]
 \item applichiamo il secondo principio di equivalenza moltiplicando ambo i membri per il~$\mcm$,
    certamente diverso da zero per le condizioni poste in precedenza. L'equazione diventa:~$2x^{2}+2x-6-2x^{2}+2x+5x-5=0$;
 \item riduciamo i monomi simili per portare l'equazione alla forma canonica:~$9x=11$;
 \item dividiamo ambo i membri per~$9$, otteniamo:~$x=\frac{11}{9}$;
 \item confrontando con le~$\CE$, la soluzione appartiene all'insieme~$\Dom$, dunque è accettabile e l'insieme soluzione è:
    $\IS=\left\{\frac{11}{9}\right\}$.
\end{enumeratea}

\end{exrig}

\ovalbox{\risolvii \ref{ese:18.1}, \ref{ese:18.2}, \ref{ese:18.3}, \ref{ese:18.4}, \ref{ese:18.5}, \ref{ese:18.6}, \ref{ese:18.7},
\ref{ese:18.8}, \ref{ese:18.9}, \ref{ese:18.10}, \ref{ese:18.11}}

\vspazio\ovalbox{\ref{ese:18.12}, \ref{ese:18.13}, \ref{ese:18.14}, \ref{ese:18.15}, \ref{ese:18.16}, \ref{ese:18.17}, \ref{ese:18.18}, \ref{ese:18.19}}

\newpage
% (c) 2012 Claudio Carboncini - claudio.carboncini@gmail.com
\section{Esercizi}
\subsection{Esercizi dei singoli paragrafi}
\subsubsection*{18.1 - MCD e mcm tra polinomi}

\begin{esercizio}[\Ast]
\label{ese:18.1}
Calcola il~$\mcd$ e il~$\mcm$ dei seguenti gruppi di polinomi.
\begin{multicols}{2}
\begin{enumeratea}
 \item $a+3$, $5a+15$, $a^{2}+6a+9$;
 \item $a^{2}-b^{2}$, $ab-b^{2}$, $a^{2}b-2ab^{2}+b^{3}$.
\end{enumeratea}
\end{multicols}
\end{esercizio}

\begin{esercizio}[\Ast]
\label{ese:18.2}
Calcola il~$\mcd$ e il~$\mcm$ dei seguenti gruppi di polinomi.
\begin{multicols}{2}
\begin{enumeratea}
 \item $x^{2}-5x+4$, $x^{2}-3x+2$, $x^{2}-4x+3$;
 \item $x^{2}+2x-2$, $x^{2}-4x+4$, $x^{2}-4$.
\end{enumeratea}
\end{multicols}
\end{esercizio}

\begin{esercizio}[\Ast]
\label{ese:18.3}
Calcola il~$\mcd$ e il~$\mcm$ dei seguenti gruppi di polinomi.
\begin{enumeratea}
 \item $a^{3}b^{2}-2a^{2}b^{3}$, $a^{3}b-4a^{2}b^{2}+4ab^{3}$, $a^{3}b^{2}-4ab^{4}$;
 \item $x^{3}+2x^{2}-3x$, $x^{3}-x$, $x^{2}-2x+1$.
\end{enumeratea}
\end{esercizio}

\begin{esercizio}[\Ast]
\label{ese:18.4}
Calcola il~$\mcd$ e il~$\mcm$ dei seguenti gruppi di polinomi.
\begin{multicols}{2}
\begin{enumeratea}
 \item $a-b$, $ab-a^{2}$, $a^{2}-b^{2}$;
 \item $b+2a$, $b-2a$, $b^{2}-4a^{2}$, $b^{2}-4a+4a^{2}$.
\end{enumeratea}
\end{multicols}
\end{esercizio}

\begin{esercizio}[\Ast]
\label{ese:18.5}
Calcola il~$\mcd$ e il~$\mcm$ dei seguenti gruppi di polinomi.
\begin{multicols}{2}
\begin{enumeratea}
 \item $a^{2}-9$, $3a-a^{2}$, $3a+a^{2}$;
 \item $a+1$, $a^{2}-1$, $a^{3}+1$.
\end{enumeratea}
\end{multicols}
\end{esercizio}

\begin{esercizio}[\Ast]
\label{ese:18.6}
Calcola il~$\mcd$ e il~$\mcm$ dei seguenti gruppi di polinomi.
\begin{multicols}{2}
\begin{enumeratea}
 \item $x^{2}+2xy+y^{2}$, $x^{2}-y^{2}$, $(x+y)^{2}(x-y)$;
 \item $b^{3}+b^{2}-4b-4$, $b^{2}-a$, $b^{2}-1$.
\end{enumeratea}
\end{multicols}
\end{esercizio}

\begin{esercizio}[\Ast]
\label{ese:18.7}
Calcola il~$\mcd$ e il~$\mcm$ dei seguenti gruppi di polinomi.
\begin{enumeratea}
 \item $a-2$, $a^{2}-9$, $a^{2}+a-6$;
 \item $3x+y+3x^{2}+xy$, $9x^{2}-1$, $9x^{2}+6xy+y^{2}$.
\end{enumeratea}
\end{esercizio}

\begin{esercizio}[\Ast]
\label{ese:18.8}
Calcola il~$\mcd$ e il~$\mcm$ dei seguenti gruppi di polinomi.
\begin{enumeratea}
 \item $2x^{3}-12x^{2}y+24xy^{2}-16y^{3}$, $6x^{2}-12xy$, $4x^{3}-16x^{2}y+16xy^{2}$;
 \item $x-1$, $x^{2}-2x+1$, $x^{2}-1$.%trovato risultato
\end{enumeratea}
\end{esercizio}

\begin{esercizio}
\label{ese:18.9}
Calcola il~$\mcd$ e il~$\mcm$ dei seguenti gruppi di polinomi.
\begin{multicols}{2}
\begin{enumeratea}
 \item $x^{3}-9x+x^{2}$, $4-(x-1)^{2}$, $x^{2}+4x+3$;
 \item $x-2$, $x-1$, $x^{2}-3x+2$;
 \item $a^{2}-1$, $b+1$, $a+ab-b-1$;
 \item $x$, $2x^{2}-3x$, $4x^{2}-9$.
\end{enumeratea}
\end{multicols}
\end{esercizio}

\begin{esercizio}
\label{ese:18.10}
Calcola il~$\mcd$ e il~$\mcm$ dei seguenti gruppi di polinomi.
\begin{multicols}{2}
\begin{enumeratea}
 \item $x-1$, $x^{2}-1$, $x^{3}-1$;
 \item $y^{3}+8a^{3}$, $y+2a$, $y^{2}-2ay+4a^{2}$;
 \item $z-5$, $2z-10$, $z^{2}-25$, $z^{2}+25+10z$;
 \item $a^{2}-2a+1$, $a^{2}-3a+2$, $1-a$.
\end{enumeratea}
\end{multicols}
\end{esercizio}

\begin{esercizio}
\label{ese:18.11}
Calcola il~$\mcd$ e il~$\mcm$ dei seguenti gruppi di polinomi.
\begin{enumeratea}
 \item $2x$, $3x-2$, $3x^{2}-2x$, $10x^{2}$;
 \item $a^{2}-a$, $a^{2}+a$, $a-a^{2}$, $2a^{2}-2$;
 \item $x-2$, $x^{2}-4$, $ax+2a-3x-6$, $a^{2}-6a+9$;
 \item $x^{2}-a^{2}$, $x+a$, $x^{2}+ax$, $ax+a^{2}$;
 \item $x^{2}-4x+4$, $2x-x^{2}$, $x^{2}-2x$, $x^{3}$, $x^{3}-2x^{2}$.
\end{enumeratea}
\end{esercizio}

\subsection{Risposte}

\paragraph{18.1.}
a)~$(a+3)$, $5(a+3)^2$;\quad b)~$(a-b)$, $b(a+b)(a-b)^2$.

\paragraph{18.2.}
a)~$(x-1)$, $(x-1)(x-2)(x-3)(x-4)$;\quad b)~$1$, $(x-2)^2(x+2)\left(x^2+2x-2\right)$.

\paragraph{18.3.}
a)~$ab(a-2b)$, $a^2 b^2(a-2b)^2(a+2b)$;\quad b)~$(x-1)$, $x(x-1)^2(x+1)(x+3)$.

\paragraph{18.4.}
a)~$(a-b)$, $a(a-b)(a+b)$;\quad b)~$1$, $(b-2a)(b+2a)\left(b^2-4a+4a^2\right)$.

\paragraph{18.5.}
a)~$1$, $a(a-3)(a+3)$;\quad b)~$(a+1)$, $(a+1)(a-1)\left(a^2-a+1\right)$.

\paragraph{18.6.}
a)~$(x+y)$, $(x+y)^2(x-y)$;\quad b)~$1$, $(b-1)(b+1)(b-2)(b+2)\left(b^2-a\right)$.

\paragraph{18.7.}
a)~$1$, $(a-2)(a-3)(a+3)$;\quad b)~$1$, $(x+1)(3x-1)(3x+1)(3x+y)^2$.

\paragraph{18.8.}
a)~$2(x-2y)$, $12x(x-2y)^3$;\quad b)~$(x-1)$, $(x-1)^2(x+1)$.

\cleardoublepage

