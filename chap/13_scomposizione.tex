% (c) 2012 Claudio Carboncini - claudio.carboncini@gmail.com
% (c) 2012-2013 Dimitrios Vrettos - d.vrettos@gmail.com

\chapter{Scomposizione in fattori}

%\section{Cosa vuol dire scomporre in fattori}
Scomporre un polinomio in fattori significa scrivere il polinomio come il prodotto di polinomi e monomi che
moltiplicati tra loro danno come risultato il polinomio stesso. Si può paragonare la scomposizione in fattori
di un polinomio alla scomposizione in fattori dei numeri naturali.

\begin{wrapfloat}{figure}{r}{0pt}
 % (c) 2012 Dimitrios Vrettos - d.vrettos@gmail.com
\begin{tikzpicture}[font=\small]
\matrix(a) [matrix of nodes, anchor=west, minimum width=4.5mm]{
36&2\\
18&2\\
9&3\\
3&3\\
1&{}\\};

\draw (a-1-2.north west)--(a-5-2.south west);
\end{tikzpicture}

\end{wrapfloat}

Per esempio, scomporre il numero~$36$ significa scriverlo come~$2^{2}\cdot 3^{2}$ dove~$2$ e~$3$ sono i suoi fattori primi.
Anche~$36 = 9 \cdot 4$ è una scomposizione, ma non è in fattori primi. Allo stesso modo un polinomio va scomposto in fattori non ulteriormente
scomponibili che si chiamano irriducibili. %(questa dovrebbe andare a sinistra della tabella)

Il polinomio~$3a^{3}b^{2}-3ab^{4}$ si può scomporre in fattori in questo modo \[3ab^{2}(a-b)(a+b)\text{,}\] infatti eseguendo i prodotti si
ottiene \[3ab^{2}(a-b)(a+b)=3ab^{2}(a^{2}+ab-ba-b^{2})=3ab^{2}\left(a^{2}-b^{2}\right)=3a^{3}b^{2}-3ab^{4}.\]
La scomposizione termina quando non è possibile scomporre ulteriormente i fattori individuati.
Come per i numeri la scomposizione in fattori dei polinomi identifica il polinomio in maniera univoca (a meno di multipli).

\begin{definizione}
Un polinomio si dice \emph{riducibile} (scomponibile) se può essere scritto come prodotto di due o più polinomi (detti fattori) di grado maggiore di zero.
In caso contrario esso si dirà \emph{irriducibile}.
\end{definizione}

La caratteristica di un polinomio di essere irriducibile dipende dall'insieme numerico al quale appartengono i coefficienti del polinomio;
uno stesso polinomio può essere irriducibile nell'insieme dei numeri razionali, ma riducibile in quello dei numeri reali o
ancora in quello dei complessi.
Dalla definizione consegue che un polinomio di primo grado è irriducibile.

\begin{definizione}
La \emph{scomposizione in fattori di un polinomio} è la sua scrittura come prodotto di fattori irriducibili.
\end{definizione}
\ovalbox{\risolvi \ref{ese:13.1}}

\section{Raccoglimento totale a fattore comune}
Questo è il primo metodo che si deve cercare di utilizzare per scomporre un polinomio.
Il metodo si basa sulla proprietà distributiva della moltiplicazione rispetto all'addizione.

Prendiamo in considerazione il seguente prodotto:\[a(x+y+z)=ax+ay+az\]
Il nostro obiettivo è ora quello di procedere
da destra verso sinistra, cioè avendo il polinomio~$ax+ay+az$ come possiamo fare per individuare il prodotto che lo ha generato?
In questo caso semplice possiamo osservare che i tre monomi contengono tutti la lettera~$a$, che quindi si può mettere in comune,
o come anche si dice ``in evidenza''.
Perciò scriviamo \[ax+ay+az=a(x+y+z).\]

\begin{exrig}
 \begin{esempio}
Analizziamo la scomposizione in fattori~$3a^{2}b\left(2a^{3}-5b^{2}-7c\right)$.
 \begin{equation*}
   \begin{split}
    3a^{2}b\left(2a^{3}-5b^{2}-7c\right) &=3a^{2}b(2a^{3})+3a^{2}b(-5b^{2})+3a^{2}b(-7c)\\
    &=6a^{5}b-15a^{2}b^{3}-21a^{2}bc.
   \end{split}
 \end{equation*}
L'ultima uguaglianza, letta da destra verso sinistra, è il raccoglimento totale a fattore comune.
Partendo da~$6a^{5}b-15a^{2}b^{3}-21a^{2}bc$ possiamo notare che i coefficienti numerici~$6$, $15$ e~$21$ hanno il~$3$ come fattore in comune.
Notiamo anche che la lettera~$a$ è in comune, come la lettera~$b$. Raccogliendo tutti i fattori comuni si avrà il prodotto
$3a^{2}b\left(2a^{3}-5b^{2}-7c\right)$ di partenza.
 \end{esempio}
\end{exrig}

\begin{procedura}
Mettere in evidenza il fattore comune:
\begin{enumeratea}
\item trovare il~$\mcd$ di tutti i termini che formano il polinomio: tutti i fattori in comune con l'esponente minimo con cui compaiono;
\item scrivere il polinomio come prodotto del~$\mcd$ per il polinomio ottenuto dividendo ciascun monomio del polinomio di partenza per il~$\mcd$;
\item verificare la scomposizione eseguendo la moltiplicazione per vedere se il prodotto dà come risultato il polinomio da scomporre.
\end{enumeratea}
\end{procedura}

\begin{exrig}
 \begin{esempio}
Scomporre in fattori~$5a^{2}x^{2}-10ax^{5}$.
  \begin{enumeratea}
  \item Tra i coefficienti numerici il fattore comune è~$5$,\,tra la parte letterale sono in comune le lettere~$a$ e~$x$, la~$a$ con esponente~$1$, la~$x$ con esponente~$2$,\, pertanto il~$\mcd$ è~$5ax^{2}$;
  \item passiamo quindi a scrivere~$5a^{2}x^{2}-10ax^{5}=5ax^{2}(\ldots \ldots \ldots)$,\, nella parentesi vanno i monomi che si ottengono dalle divisioni~$5a^{2}x^{2}:5ax^{2}=a$ e~$-10ax^{5}:5ax^{2}=-2x^{3}$. Quindi: $5a^{2}x^{2}-10ax^{5}=5ax^{2}(a-2x^{3})$;
  \item verifica: $5ax^{2}(a-2x^{3})=5a^{2}x^{2}-10ax^{5}$.
  \end{enumeratea}
 \end{esempio}

 \begin{esempio}
Scomporre in fattori~$10x^{5}y^{3}z-15x^{3}y^{5}z-20x^{2}y^{3}z^{2}$.
 \begin{enumeratea}
 \item Trovo tutti i fattori comuni con l'esponente minore per formare il~$\mcd$.\, $\mcd=5x^{2}y^{3}z$;
 \item divido ciascun termine del polinomio per~$5x^{2}y^{3}z$:
   \begin{description}
   \item $10x^{5}y^{3}z:5x^{2}y^{3}z=2x^{3}$,\quad$-15x^{3}y^{5}z:5x^{2}y^{3}z=-3xy^{2}$,\quad$-20x^{2}y^{3}z^{2}:5x^{2}y^{3}z=-4z$,
   \item il polinomio si può allora scrivere come~$5x^{2}y^{3}z (2x^{3}-3xy^{2}-4z)$.
   \end{description}
 \end{enumeratea}

Il fattore da raccogliere a fattore comune può essere scelto con il segno~$+$ (positivo) o con il segno~$−$ (negativo).
Nell'esempio precedente è valida anche la seguente 
scomposizione:~$10x^{5}y^{3}z-15x^{3}y^{5}z-20x^{2}y^{3}z^{2}=-5x^{2}y^{3}z (-2x^{3}+3xy^{2}+4z)$.
 \end{esempio}

 \begin{esempio}
Scomporre in fattori~$-8x^{2}y^{3}+10x^{3}y^{2}$.
 \begin{enumeratea}
  \item Poiché il primo termine è negativo possiamo mettere a fattore comune un 
  numero negativo. Tra~8 e~10 il~$\mcd$ è~2. Tra~$x^{2}y^{3}$ e~$x^{3}y^{2}$ mettiamo 
  a fattore comune le lettere~$x$ e~$y$, entrambe con esponente~$2$, perché è
     il minimo esponente con cui compaiono. In definitiva il monomio da mettere a fattore comune è~$-2x^{2}y^{2}$;
  \item pertanto possiamo cominciare a scrivere~$-2x^{2}y^{2}(\ldots \ldots \ldots)$;
  eseguiamo le divisioni $-8x^{2}y^{3}:(-2x^{2}y^{2})=+4y$ e  $10x^{3}y^{2}:(-2x^{2}y^{2})=-5x$.
I quozienti trovati~$+4y$ e~$-5x$ vanno nelle parentesi.
 \end{enumeratea}
In definitiva~$-8x^{2}y^{3}+10x^{3}y^{2}=-2x^{2}y^{2}(4y-5x)$.
 \end{esempio}

 \begin{esempio}
Scomporre in fattori~$6a(x-1)+7b(x-1)$.
  \begin{enumeratea}
  \item Il fattore comune è~$(x-1)$, quindi il polinomio si può scrivere come~$(x-1)\cdot [\ldots \ldots \ldots]$;
  \item nella parentesi quadra scriviamo i termini che si ottengono dalle divisioni:
   \begin{description}
    \item $6a(x-1):(x-1)=6a$,\quad $7b(x-1):(x-1)=7b$.
   \end{description}
  \end{enumeratea}
  In definitiva~$6a(x-1)+7b(x-1)=(x-1)(6a+7b)$.
 \end{esempio}

 \begin{esempio}
Scomporre in fattori~$10(x+1)^{2}-5a(x+1)$.
  \begin{enumeratea}
  \item Il fattore comune è~$5(x+1)$, quindi possiamo cominciare a scrivere~$5(x+1)\cdot [\ldots \ldots \ldots]$;
  \item nella parentesi quadra scriviamo i termini che si ottengono dalle divisioni:
   \begin{description}
    \item $10(x+1)^{2}:5(x+1)=2(x+1)$,\quad $-5a(x+1):5(x+1)=a$.
   \end{description}
  \end{enumeratea}
  In definitiva~$10(x+1)^{2}-5a(x+1)=5(x+1)\bigl[2(x+1)-a \bigr]$.
 \end{esempio}

\end{exrig}

\ovalbox{\risolvii \ref{ese:13.2}, \ref{ese:13.3}, \ref{ese:13.4}, \ref{ese:13.5}, \ref{ese:13.6}, \ref{ese:13.7}, \ref{ese:13.8}, \ref{ese:13.9}, \ref{ese:13.10}, \ref{ese:13.11}}

\section{Raccoglimento parziale a fattore comune}

Quando un polinomio non ha alcun fattore comune a tutti i suoi termini, possiamo provare a mettere in evidenza tra gruppi di monomi
e successivamente individuare il polinomio in comune.

Osserviamo il prodotto~$(a+b)(x+y+z)=ax+ay+az+bx+by+bz$. Supponiamo ora di avere il polinomio
$ax+ay+az+bx+by+bz$ come possiamo fare a tornare indietro per scriverlo come prodotto di polinomi?

%\newpage
\begin{exrig}
 \begin{esempio}
Scomponiamo in fattori~$ax+ay+az+bx+by+bz$. Non c'è nessun fattore comune a tutto il polinomio.

Proviamo a mettere in evidenza per gruppi di termini. Evidenziamo~$a$ tra i primi tre termini e~$b$ tra gli ultimi tre, 
avremo:~$a(x+y+z)+b(x+y+z)$. Ora risulta semplice vedere che il trinomio~$(x+y+z)$ è in comune e quindi lo possiamo mettere 
in evidenza~$ax+ay+az+bx+by+bz=a(x+y+z)+b(x+y+z)=(x+y+z)(a+b)$.
 \end{esempio}
\end{exrig}

\begin{procedura}
Eseguire il raccoglimento parziale.
\begin{enumeratea}
\item Dopo aver verificato che non è possibile effettuare un raccoglimento a fattore comune totale raggruppo i monomi
   in modo che in ogni gruppo sia possibile mettere in comune qualche fattore;
\item verifico se la nuova scrittura del polinomio ha un polinomio (binomio, trinomio, \ldots) comune a tutti i termini;
\item se è presente il fattore comune a tutti i termini lo metto in evidenza;
\item se il fattore comune non è presente la scomposizione è fallita, allora posso provare a raggruppare
   diversamente i monomi o abbandonare questo metodo.
\end{enumeratea}
\end{procedura}

\begin{exrig}
 \begin{esempio}
Scomporre in fattori~$ax+ay+bx+ab$.
  \begin{enumeratea}
  \item Provo a mettere in evidenza la~$a$ nel primo e secondo termine e la~$b$ nel terzo e quarto termine:~$ax+ay+bx+ab=a(x+y)+b(x+a)$;
  \item in questo caso non c'è nessun fattore comune: il metodo è fallito. In effetti il polinomio non si può scomporre in fattori.
  \end{enumeratea}
 \end{esempio}

 \begin{esempio}
Scomporre in fattori~$bx-2ab+2ax-4a^{2}$.
 \begin{enumeratea}
 \item Non vi sono fattori da mettere a fattore comune totale, proviamo con il raccoglimento parziale:~$b$ nei primi due monomi e~$2a$ negli altri due;
 \item %$\mmevid{ev_rosso}{bx}-\mmevid{ev_rosso}{2ab}+\mmevid{ev_verde}{2ax}-\mmevid{ev_verde}{4a^{2}}=b(\mmevid{ev_blu}{x-2a})+2a(\mmevid{ev_blu}{x-2a})=(x-2a)(b+2a)$.
$\underline{bx} -\underline{2ab}+\underline{\underline{2ax}}-\underline{\underline{4a^{2}}}=b(\underline{x-2a})+2a(\underline{x-2a})=(x-2a)(b+2a)$.
 \end{enumeratea}
 \end{esempio}

 \begin{esempio}
Scomporre in fattori~$bx^{3}+2x^{2}-bx-2+abx+2a$.
 \begin{enumeratea}

% \item Raggruppiamo nel seguente modo:~$\mmevid{ev_rosso}{bx^{3}}+\mmevid{ev_verde}{2x^{2}}-\mmevid{ev_rosso}{bx}-\mmevid{ev_verde}{2}+\mmevid{ev_rosso}{abx}+\mmevid{ev_verde}{2a}$ tra quelli evidenziati in \evid{ev_rosso}{\phantom{I}} mettiamo a fattore comune~$bx$ e tra quelli evidenziati in \evid{ev_verde}{\phantom{I}} mettiamo a fattore comune~$2$;

\item Raggruppiamo nel seguente modo:~$\underline{bx^{3}}+\underline{\underline {2x^{2}}}-\underline{bx}-\underline{\underline~2}
     +\underline{abx}+\underline{\underline{2a}}$ tra quelli con sottolineatura semplice metto a fattore comune~$bx$, tra quelli
     con doppia sottolineatura metto a fattore comune~$2$;

% \item $\mmevid{ev_rosso}{bx^{3}}+\mmevid{ev_verde}{2x^{2}}-\mmevid{ev_rosso}{bx}-\mmevid{ev_verde}{2}+\mmevid{ev_rosso}{abx}+\mmevid{ev_verde}{2a}=bx\bigl(\mmevid{ev_blu}{x^{2}-1+a}\bigr)+2\bigl(\mmevid{ev_blu}{x^{2}-1+a}\bigr)=\bigl(x^{2}-1+a\bigr)\bigl(bx+2\bigr)$.

 \item $\underline{bx^{3}}+\underline{\underline {2x^{2}}}-\underline{bx}-\underline{\underline{2}}+\underline{abx}+\underline{\underline{2a}}
     =bx\bigl(\underline{x^{2}-1+a}\bigr)+2\bigl(\underline{x^{2}-1+a}\bigr)=\bigl(x^{2}-1+a\bigr)\bigl(bx+2\bigr)$.
 \end{enumeratea}
 \end{esempio}

 \begin{esempio}
Scomporre in fattori~$5ab^{2}-10abc-25abx+50acx$.
 \begin{enumeratea}
  \item Il fattore comune è~$5a$, quindi:
    \begin{itemize*}
    \item $5ab^{2}-10abc-25abx+50acx=5a\bigl(b^{2}-2bc-5bx+10cx\bigr)$;
    \end{itemize*}
%  \item vediamo se è possibile scomporre il polinomio in parentesi con un raccoglimento parziale~$5a(\mmevid{ev_rosso}{b^{2}}-\mmevid{ev_rosso}{2bc}-\mmevid{ev_verde}{5bx}+\mmevid{ev_verde}{10cx})=5a\bigl[b(\mmevid{ev_blu}{b-2c})-5x(\mmevid{ev_blu}{b-2c})\bigr]=5a(b-2c)(b-5x)$.

 \item vediamo se è possibile scomporre il polinomio in parentesi con un raccoglimento parziale~$5a(\underline{b^{2}}-\underline{2bc}-\underline{\underline{5bx}}+\underline{\underline{10cx}})=5a\bigl[b(\underline{b-2c})-5x(\underline{b-2c})\bigr]=5a(b-2c)(b-5x)$.
 \end{enumeratea}
 \end{esempio}
\end{exrig}

\ovalbox{\risolvii \ref{ese:13.12}, \ref{ese:13.13}, \ref{ese:13.14}, \ref{ese:13.15}, \ref{ese:13.16}, \ref{ese:13.17}, \ref{ese:13.18}, \ref{ese:13.19}, \ref{ese:13.20},\ref{ese:13.21}, \ref{ese:13.22}}

\pagebreak
\section{Riconoscimento di prodotti notevoli}

\subsection{Quadrato di un binomio}

Uno dei metodi più usati per la scomposizione di polinomi è legato al saper riconoscere i prodotti notevoli.
Se abbiamo un trinomio costituito da due termini che sono quadrati di due monomi ed il terzo termine è uguale al doppio prodotto
degli stessi due monomi, allora il trinomio può essere scritto sotto forma di quadrato di un binomio, secondo la regola che segue: %(sezione \ref{sect:quadrato_di_un_binomio} a pagina \pageref{sect:quadrato_di_un_binomio})
\begin{equation*}
(A+B)^{2}=A^{2}+2AB+B^{2}\quad \Rightarrow \quad A^{2}+2AB+B^{2}=(A+B)^{2}.
\end{equation*}
Analogamente nel caso in cui il monomio che costituisce il doppio prodotto sia negativo:
\begin{equation*}
(A-B)^{2}=A^{2}-2AB+B^{2}\quad \Rightarrow \quad A^{2}-2AB+B^{2}=(A-B)^{2}.
\end{equation*}
Poiché il quadrato di un numero è sempre positivo, valgono anche le seguenti uguaglianze
\begin{equation*}
(A+B)^{2}=(-A-B)^{2}\quad\Rightarrow\quad A^{2}+2AB+B^{2}=(A+B)^{2}=(-A-B)^{2}\phantom{.}
\end{equation*}
\begin{equation*}
(A-B)^{2}=(-A+B)^{2}\quad \Rightarrow \quad A^{2}-2AB+B^{2}=(A-B)^{2}=(-A+B)^{2}.
\end{equation*}

\begin{exrig}
 \begin{esempio}
Scomporre in fattori~$4a^{2}+12ab^{2}+9b^{4}$.

Notiamo che il primo ed il terzo termine sono quadrati, rispettivamente di~$2a$ e di~$3b^{2}$,
ed il secondo termine è il doppio prodotto degli stessi monomi, pertanto possiamo
scrivere:
\[4a^{2}+12{ab}^{2}+9b^{4}=(2a)^{2}+2\cdot (2a)\cdot (3b^{2})+\left(3b^{2}\right)^{2}=\left(2a+3b^{2}\right)^{2}.\]
 \end{esempio}

 \begin{esempio}
Scomporre in fattori~$x^{2}-6x+9$.

Il primo ed il terzo termine sono quadrati, il secondo termine compare con il segno ``meno''.
Dunque:~$x^{2}-6x+9=x^{2}-2\cdot 3\cdot x+3^{2}=(x-3)^{2}$, ma anche~$x^{2}-6x+9=(-x+3)^{2}$.
 \end{esempio}

 \begin{esempio}
Scomporre in fattori~$x^{4}+4x^{2}+4$.

Può accadere che tutti e tre i termini siano tutti quadrati. $x^{4}+4x^{2}+4$ è formato da tre quadrati, ma il secondo termine, quello di grado intermedio,
è anche il doppio prodotto dei due monomi di cui il primo ed il terzo termine sono i rispettivi quadrati.
Si ha dunque:
\[x^{4}+4x^{2}+4=\left(x^{2}\right)^{2}+2\cdot (2)\cdot (x^{2})+(2)^{2}=\left(x^{2}+2\right)^{2}.\]
 \end{esempio}
\end{exrig}

\begin{procedura}
Individuare il quadrato di un binomio:
\begin{enumeratea}
\item individuare le basi dei due quadrati;
\item verificare se il terzo termine è il doppio prodotto delle due basi;
\item scrivere tra parentesi le basi dei due quadrati e il quadrato fuori dalla parentesi;
\item mettere il segno ``più'' o ``meno'' in accordo al segno del termine che non è un quadrato.
\end{enumeratea}
\end{procedura}
\pagebreak
Può capitare che i quadrati compaiano con il coefficiente negativo, ma si può rimediare mettendo in evidenza il segno ``meno''.

\begin{exrig}
 \begin{esempio}
Scomporre in fattori~$-9a^{2}+12{ab}-4b^{2}$.

Mettiamo~$-1$ a fattore comune~$-9a^{2}+12ab-4b^{2}=-(9a^{2}-12{ab}+4b^{2})=-(3a-2b)^{2}$.
 \end{esempio}

 \begin{esempio}
Scomporre in fattori~$-x^{4}-x^{2}-\frac{1}{4}$.
\[-x^{4}-x^{2}-\frac{1}{4}=-\left(x^{4}+x^{2}+\frac{1}{4}\right)=-\left(x^{2}+\frac{1}{2}\right)^{2}.\]
 \end{esempio}

 \begin{esempio}
Scomporre in fattori~$-x^{2}+6xy^{2}-9y^{4}$.
\[x^{2}+6xy^{2}-9y^{4}=-\left(x^{2}-6xy^{2}+9y^{4}\right)=-\left(x-3y^{2}\right)^{2}.\]
 \end{esempio}
\end{exrig}

Possiamo avere un trinomio che ``diventa'' quadrato di binomio dopo aver messo qualche fattore comune in evidenza.

\begin{exrig}
 \begin{esempio}
Scomporre in fattori~$2a^{3}+20a^{2}+50a$.

Mettiamo a fattore comune~$2a$, allora~$2a^{3}+20a^{2}+50a=2a(a^{2}+10a+25)=2a(a+5)^{2}$.
 \end{esempio}

 \begin{esempio}
Scomporre in fattori~$2a^{2}+4a+2$.
\[2a^{2}+4a+2=2\left(a^{2}+2a+1\right)=2(a+1)^{2}.\]
 \end{esempio}

 \begin{esempio}
Scomporre in fattori~$-12a^{3}+12a^{2}-3a$.
\[-12a^{3}+12a^{2}-3a=-3a\left(4a^{2}-4a+1\right)=-3a(2a-1)^{2}.\]
 \end{esempio}

 \begin{esempio}
Scomporre in fattori~$\dfrac{3}{8}a^{2}+3ab+6b^{2}$.

\[\frac{3}{8}a^{2}+3ab+6b^{2}=\frac{3}{2}\left(\frac{1}{4}a^{2}+2ab+4b^{2}\right)=\frac{3}{2}\left(\frac{1}{2}a+2b\right)^{2}\text{,}\]
o anche
\[\frac{3}{8}a^{2}+3ab+6b^{2}=\frac{3}{8}\left(a^{2}+8ab+16b^{2} \right)=\frac{3}{8}\left(a+4b\right)^{2}.\]
 \end{esempio}
\end{exrig}
\ovalbox{\risolvii \ref{ese:13.23}, \ref{ese:13.24}, \ref{ese:13.25}, \ref{ese:13.26}, \ref{ese:13.27}, \ref{ese:13.28}, \ref{ese:13.29}, \ref{ese:13.30}, \ref{ese:13.31}, \ref{ese:13.32}}

\subsection{Quadrato di un polinomio}

Se siamo in presenza di sei termini, tre dei quali sono quadrati, verifichiamo se il polinomio è il quadrato di un
trinomio secondo le seguenti regole (sezione \ref{sect:quadrato_di_un_polinomio} a pagina \pageref{sect:quadrato_di_un_polinomio})
\begin{equation*}
(A+B+C)^{2}=A^{2}+B^{2}+C^{2}+2AB+2AC+2BC
\end{equation*}
\begin{equation*}
A^{2}+B^{2}+C^{2}+2AB+2AC+2BC=(A+B+C)^{2}=(-A-B-C)^{2}.
\end{equation*}
Notiamo che i doppi prodotti possono essere tutti e tre positivi, oppure uno positivo e due negativi:
indicano se i rispettivi monomi sono concordi o discordi.
\pagebreak
\begin{exrig}
 \begin{esempio}
Scomporre in fattori~$16a^{4}+b^{2}+1+8a^{2}b+8a^{2}+2b$.

I primi tre termini sono quadrati rispettivamente di~$4a^{2}$, $b$ e~$1$ e si può verificare poi che gli altri tre termini sono
i doppi prodotti:~$16a^{4}+b^{2}+1+8a^{2}b+8a^{2}+2b=\left(4a^{2}+b+1\right)^{2}$.
 \end{esempio}

 \begin{esempio}
Scomporre in fattori~$x^{4}+y^{2}+z^{2}-2x^{2}y-2x^{2}z+2yz$.
\[x^{4}+y^{2}+z^{2}-2x^{2}y-2x^{2}z+2yz=\left(x^{2}-y-z\right)^{2}=\left(-x^{2}+y+z\right)^{2}.\]
 \end{esempio}

 \begin{esempio}
Scomporre in fattori~$x^{4}-2x^{3}+3x^{2}-2x+1$.

In alcuni casi anche un polinomio di cinque termini può essere il quadrato di un trinomio.
Per far venire fuori il quadrato del trinomio si può scindere il termine~$3x^{2}$ come somma:
\[3x^{2}=x^{2}+2x^{2}.\]
In questo modo si ha:
\[x^{4}-2x^{3}+3x^{2}-2x+1=x^{4}-2x^{3}+x^{2}+2x^{2}-2x+1=(x^{2}-x+1)^{2}.\]
 \end{esempio}
\end{exrig}

Nel caso di un quadrato di un polinomio la regola è sostanzialmente la stessa:
\begin{equation*}
(A+B+C+D)^{2}=A^{2}+B^{2}+C^{2}+D^{2}+2AB+2AC+2AD+2BC+2BD+2CD.
\end{equation*}
\ovalbox{\risolvii \ref{ese:13.33},\ref{ese:13.34},\ref{ese:13.35}, \ref{ese:13.36}, \ref{ese:13.37}}

\subsection{Cubo di un binomio}
I cubi di binomi sono di solito facilmente riconoscibili. Un quadrinomio è lo sviluppo del cubo di un binomio se due suoi termini sono i cubi
di due monomi e gli altri due termini sono i tripli prodotti tra uno dei due monomi ed il quadrato dell’altro, secondo le seguenti formule: % (sezione \ref{sect:cubo_di_un_binomio} a pagina \pageref{sect:cubo_di_un_binomio})
\begin{equation*}
(A+B)^{3}=A^{3}+3A^{2}B+3AB^{2}+B^{3}\quad \Rightarrow \quad A^{3}+3A^{2}B+3AB^{2}+B^{3}=(A+B)^{3}\phantom{.}
\end{equation*}
\begin{equation*}
(A-B)^{3}=A^{3}-3A^{2}B+3AB^{2}-B^{3}\quad \Rightarrow \quad A^{3}-3A^{2}B+3AB^{2}-B^{3}=(A-B)^{3}.
\end{equation*}
Per il cubo non si pone il problema, come per il quadrato, del segno della base, perché un numero elevato ad esponente dispari,
se è positivo rimane positivo e se è negativo rimane negativo.

\begin{exrig}
 \begin{esempio}
Scomporre in fattori~$8a^{3}+12a^{2}b+6{ab}^{2}+b^{3}$.

Notiamo che il primo ed il quarto termine sono cubi, rispettivamente di~$2a$ e di~$b$, il secondo termine è il triplo
prodotto tra il quadrato di~$2a$ e~$b$, mentre il terzo termine è il triplo prodotto tra~$2a$ e il quadrato di~$b$.
Abbiamo dunque:
\[8a^{3}+12a^{2}b+6ab^{2}+b^{3}=(2a)^{3}+3\cdot (2a)^{2}\cdot (b)+3\cdot (2a)\cdot (b)^{2}=(2a+b)^{3}.\]
 \end{esempio}

 \begin{esempio}
Scomporre in fattori~$-27x^{3}+27x^{2}-9x+1$.

Le basi del cubo sono il primo e il quarto termine, rispettivamente cubi di~$-3x$ e di~$1$.
Dunque:
\[-27x^{3}+27x^{2}-9x+1=(-3x)^{3}+3\cdot (-3x)^{2}\cdot 1+3\cdot (-3x)\cdot 1^{2}+1=(-3x+1)^{3}.\]
 \end{esempio}

 \begin{esempio}
Scomporre in fattori~$x^{6}-x^{4}+\frac{1}{3}x^{2}-\frac{1}{27}$.

Le basi del cubo sono~$x^{2}$ e~$-\frac{1}{3}$ i termini centrali sono i tripli prodotti,
quindi~$\left(x^{2}-\frac{1}{3}\right)^{3}$.
\end{esempio}
\end{exrig}
\ovalbox{\risolvii \ref{ese:13.38}, \ref{ese:13.39}, \ref{ese:13.40}, \ref{ese:13.41}, \ref{ese:13.42}, \ref{ese:13.43}, \ref{ese:13.44}, \ref{ese:13.45},\ref{ese:13.46}}

\subsection{Differenza di due quadrati}
Un binomio che sia la differenza dei quadrati di due monomi può essere scomposto come prodotto tra la somma dei due monomi
(basi dei quadrati) e la loro differenza. % (sezione \ref{sect:diffrenza_di_quadrati} a pagina \pageref{sect:diffrenza_di_quadrati})
\begin{equation*}
(A+B)\cdot (A-B)=A^{2}-B^{2}\quad \Rightarrow \quad A^{2}-B^{2}=(A+B)\cdot (A-B).
\end{equation*}

\begin{exrig}
 \begin{esempio}
Scomporre in fattori~$\frac{4}{9}a^{4}-25b^{2}$.
\[\frac{4}{9}a^{4}-25b^{2}=\left(\frac{2}{3}a^{2}\right)^{2}-\left(5b\right)^{2}=\left(\frac{2}{3}a^{2}+5b\right)
\cdot \left(\frac{2}{3}a^{{2}}-5b\right).\]
 \end{esempio}

 \begin{esempio}
Scomporre in fattori~$-x^{6}+16y^{2}$.
\[-x^{6}+16y^{2}=-\left(x^{3}\right)^{2}+\left(4y\right)^{2}=\left(x^{3}+4y\right)\cdot \left(-x^{3}+4y\right).\]
 \end{esempio}

 \begin{esempio}
Scomporre in fattori~$a^{2}-\left(x+1\right)^{2}$.
La formula precedente vale anche se~$A$ e~$B$ sono polinomi. Quindi $a^{2}-\left(x+1\right)^{2}=\left[a+(x+1)\right]\cdot \left[a-(x+1)\right]=(a+x+1)(a-x-1)$.
\end{esempio}

 \begin{esempio}
Scomporre in fattori~$\left(2a-b^{2}\right)^{2}-(4x)^{2}$.
\[\left(2a-b^{2}\right)^{2}-(4x)^{2}=\left(2a-b^{2}+4x\right)\cdot \left(2a-b^{2}-4x\right).\]
 \end{esempio}

 \begin{esempio}
Scomporre in fattori~$(a+3b)^{2}-(2x-5)^{2}$.
\[(a+3b)^{2}-(2x-5)^{2}=(a+3b+2x-5)\cdot (a+3b-2x+5).\]
 \end{esempio}
\end{exrig}

Per questo tipo di scomposizioni, la cosa più difficile è riuscire a riconoscere un quadrinomio o un polinomio
di sei termini come differenza di quadrati. Riportiamo i casi principali:
\begin{itemize*}
 \item $(A+B)^{2}-C^{2}=A^{{2}}+2AB+B^{2}-C^{2}$;
 \item $A^{2}-(B+C)^{2}=A^{2}-B^{2}-2BC-C^{2}$;
 \item $(A+B)^{2}-(C+D)^{2}=A^{2}+2AB+B^{2}-C^{2}-2CD-D^{2}$.
\end{itemize*}

\begin{exrig}
 \begin{esempio}
Scomporre in fattori~$4a^{2}-4b^{2}-c^{2}+4bc$.

Gli ultimi tre termini possono essere raggruppati per formare il quadrati di un binomio.
 \begin{equation*}
   \begin{split}
     4a^{2}-4b^{2}-c^{2}+4bc &=4a^{2}-\left(4b^{2}+c^{2}-4bc\right) \\
                 &= (2a)^{2}-(2b-c)^{2}=(2a+2b-c)\cdot (2a-2b+c).
   \end{split}
  \end{equation*}
 \end{esempio}

 \begin{esempio}
Scomporre in fattori~$4x^{4}-4x^{2}-y^{2}+1$.
\[4x^{4}-4x^{2}-y^{2}+1=\left(2x^{2}-1\right)^{2}-(y)^{2}=(2x^{2}-1+y)\cdot (2x^{2}-1-y).\]
 \end{esempio}

 \begin{esempio}
Scomporre in fattori~$a^{2}+1+2a+6bc-b^{2}-9c^{2}$.
 \begin{equation*}
   \begin{split}
     a^{2}+1+2a+6bc-b^{2}-9c^{2} &=\left(a^{2}+1+2a\right)-\left(b^{2}+9c^{2}-6{bc}\right) \\
                 &= (a+1)^{2}-(b-3c)^{2}=(a+1+b-3c)\cdot (a+1-b+3c).
   \end{split}
  \end{equation*}
 \end{esempio}
\end{exrig}
\ovalbox{\risolvii  \ref{ese:13.47}, \ref{ese:13.48}, \ref{ese:13.49}, \ref{ese:13.50}, \ref{ese:13.51}, \ref{ese:13.52}, \ref{ese:13.53}, \ref{ese:13.54}, \ref{ese:13.55}, \ref{ese:13.56}}

\section{Altre tecniche di scomposizione}

\subsection{Trinomi particolari}

Consideriamo il seguente prodotto:

\[(x+3)(x+2)=x^{2}+3x+2x+6=x^{2}+5x+6.\]

Poniamoci ora l'obiettivo opposto: se abbiamo il
polinomio~$x^{2}+5x+6$ come facciamo a trovare ritrovare il prodotto
che lo ha originato? Possiamo notare che il~5 deriva dalla somma tra
il~3 e il~2, mentre il~6 deriva dal prodotto tra~3 e~2. Generalizzando:

\[\left(x+a\right)\cdot \left(x+b\right)=x^{{2}}+ax+bx+ab=x^{2}+\left(a+b\right)x+a\cdot b.\]
Leggendo la formula precedente da destra verso sinistra:

\[x^{{2}}+\left(a+b\right)x+a\cdot b=\left(x+a\right)\cdot\left(x+b\right).\]

Possiamo allora concludere che se abbiamo un trinomio di secondo grado
in una sola lettera, a coefficienti interi, avente il termine di
secondo grado con coefficiente~1, se riusciamo a trovare due numeri~$a$ e~$b$
tali che la loro somma è uguale al
coefficiente del termine di primo grado ed il loro prodotto è uguale
al termine noto, allora il polinomio è scomponibile nel prodotto~$(x+a)(x+b)$.

Osserva che il termine noto, poiché è dato dal prodotto dei numeri
che cerchiamo, ci dice se i due numeri sono concordi o discordi.
Inoltre, se il numero non è particolarmente grande è sempre
possibile scrivere facilmente tutte le coppie di numeri che danno come
prodotto il numero cercato, tra tutte queste coppie dobbiamo poi
individuare quella che ha per somma il coefficiente del termine di
primo grado.

\begin{exrig}
 \begin{esempio}
 $x^{2}+7x+12.$

 I coefficienti sono positivi e quindi i due numeri da trovare sono
entrambi positivi.
Il termine noto~12 può essere scritto sotto forma di prodotto di due
numeri naturali solo come:

\[12\cdot 1\text{,}\quad~6\cdot 2\text{,}\quad~3\cdot 4.\]

Le loro somme sono rispettivamente~13, 8, 7. La coppia di numeri che
dà per somma $+7$ e prodotto $+12$ è pertanto~$+3$ e~$+4$. Dunque il
trinomio si scompone come:

\[x^{2}+7x+12=\left(x+4\right)\cdot \left(x+3\right).\]
 \end{esempio}

 \begin{esempio}
 $x^{2}-\overset{S}{8}x+\overset{P}{15}.$

I segni dei coefficienti ci dicono che i due numeri, dovendo avere somma (S)
negativa e prodotto (P) positivo, sono entrambi negativi. Dobbiamo cercare
due numeri negativi la cui somma (S) sia~$-8$ e il cui prodotto (P) sia~15. Le
coppie di numeri negativi che danno~15 come prodotto (P) sono $(-15\text{,~}-1)$ e~$(-5\text{,~}-3)$.
Allora i due numeri cercati sono~$-5$ e~$-3$. Il trinomio si scompone
come:
\[x^{2}-8x+15=\left(x-5\right)\cdot \left(x-3\right).\]
 \end{esempio}

\begin{esempio}
 $x^{2}+\overset{S}{4}x-\overset{P}{5}.$

I due numeri sono discordi, il maggiore in valore assoluto è quello
positivo. C'è una sola coppia di numeri che dà~$-5$
come prodotto, precisamente~$+5$ e~$-1$. Il polinomio si scompone:
\[x^{2}+4x-5=\left(x+5\right)\cdot \left(x-1\right).\]
\end{esempio}

\begin{esempio}
 $x^{2}-\overset{S}{3}x-\overset{P}{10}.$

I due numeri sono discordi, in valore assoluto il più grande è quello
negativo. Le coppie di numeri che danno~$-10$ come prodotto sono~$(-10\text{,~}+1)$ e
$(-5\text{,~}+2)$. Quella che dà~$-3$ come somma è~$(-5\text{,~}+2)$. Quindi
\[x^{2}-3x-10=\left(x-5\right)\cdot \left(x+2\right).\]
\end{esempio}

\begin{esempio}
 In alcuni casi si può applicare questa regola anche quando il trinomio
non è di secondo grado, è necessario però che il termine di grado
intermedio sia esattamente di grado pari alla metà di quello di grado
maggiore.

\begin{itemize*}
\item $x^{4}+5x^{2}+6=\left(x^{2}+3\right)\cdot \left(x^{2}+2\right)$;
\item $x^{6}+x^{3}-12=\left(x^{3}+4\right)\cdot \left(x^{3}-3\right)$;
\item $a^{4}-10a^{2}+9=\underbrace{\left(a^{2}-9\right)\cdot\left(a^{2}-1\right)}_{\text{differenze di quadrati}}=\left(a+3\right)\cdot \left(a-3\right)\cdot \left(a+1\right)\cdot\left(a-1\right)$;
\item $-x^{4}-x^{2}+20=-\left(x^{4}+x^{2}-20\right)=-\left(x^{2}+5\right)\cdot\left(x^{2}-4\right)=-\left(x^{2}+5\right)\cdot%
\left(x+2\right)\cdot \left(x-2\right)$;
\item $2x^{5}-12x^{3}-14x=2x\cdot \left(x^{4}-6x^{2}-7\right)=2x\cdot%
\left(x^{2}-7\right)\cdot \left(x^{2}+1\right)$;
\item $-2a^{7}+34a^{5}-32a^{3}=-2a^{3}\left(a^{4}-17a^{2}+16\right)
    =-2a^{3}\left(a^{2}-1\right)\left(a^{2}-16\right)\\
    =-2a^{3}\left(a-1\right)\left(a+1\right)\left(a-4\right)\left(a+4\right).$
\end{itemize*}
\end{esempio}

\end{exrig}

È possibile applicare questo metodo anche quando il
polinomio è in due variabili.

\begin{exrig}
 \begin{esempio}
 $x^{2}+5xy+6y^{2}$.

 Per capire come applicare la regola precedente, possiamo scrivere il
trinomio in questo modo:~$x^{2}+\overset{S}{5}xy+\overset{P}{6}y{^{2}}$.

Bisogna cercare due monomi~$A$ e~$B$ tali che~$A+B=5y$
e~$A\cdot B=6y^{2}$. Partendo dal fatto che i due numeri che danno~5
come somma e~6 come prodotto sono~$+3$ e~$+2$, i monomi cercati sono~$+3y$ e~$+2y$,
infatti~$+3y+3y=+5y$ e~$+3y\cdot (+2y)=+6y^{2}$. Pertanto si
può scomporre come segue:~$x^{2}+5xy+6y^{2}=(x+3y)(x+2y)$.
 \end{esempio}
\end{exrig}

La regola, opportunamente modificata, vale anche se il primo
coefficiente non è~1. Vediamo un esempio.
\pagebreak
\begin{exrig}
 \begin{esempio}
 $2x^{2}-x-1$.

Non possiamo applicare la regola del trinomio caratteristico, con somma
e prodotto, ma con un accorgimento, possiamo riscrivere il polinomio in un
altro modo. Cerchiamo due numeri la cui somma sia~$-1$ e il prodotto sia
pari al prodotto tra il primo e l'ultimo coefficiente,
o meglio tra il coefficiente del termine di secondo grado e il termine
noto, in questo caso~$2\cdot (-1)=-2$. I numeri sono~$-2$ e~$+1$.
Spezziamo il monomio centrale in somma di due monomi in questo modo
\[2x^{2}-x-1=2x^{2}-2x+x-1.\]

Ora possiamo applicare il raccoglimento a fattore comune parziale
\[2x^{2}-x-1={2x^{2}}\underbrace{{-2x}+x}_{-x}-1=2x\cdot
\underline{(x-1)}+1\cdot
\underline{(x-1)}=\left(x-1\right)\cdot \left(2x+1\right).\]
 \end{esempio}
\end{exrig}

\begin{procedura}
Sia da scomporre un trinomio di secondo grado a coefficienti interi~$ax^{2}+bx+c$
con~$a\neq 1$, cerchiamo due numeri~$m$ ed~$n$ tali che $m+n=b$ e~$m\cdot n=a\cdot c$;
se riusciamo a trovarli, li useremo per dissociare
il coefficiente~$b$ e riscrivere il polinomio nella forma~$ax^{2}+\left(m+n\right)\cdot x+c$
su cui poi eseguire un raccoglimento parziale.
\end{procedura}

\ovalbox{\risolvii \ref{ese:13.57}, \ref{ese:13.58}, \ref{ese:13.59}, \ref{ese:13.60}, \ref{ese:13.61}, \ref{ese:13.62}, \ref{ese:13.63}, \ref{ese:13.64}, \ref{ese:13.65}, \ref{ese:13.66}}

\subsection{Scomposizione con la regola Ruffini}
Anche il teorema di Ruffini permette di scomporre in fattori i polinomi.
Dato il polinomio~$P(x)$, se riusciamo a trovare un numero~$k$
per il quale~$P(k)=0$, allora~$P(x)$ è
divisibile per il binomio~$x-k$, allora possiamo scomporre~$P(x)=(x-k)\cdot Q(x)$, dove~$Q(x)$
è il quoziente della divisione
tra~$P(x)$ e~$(x-k)$.

Il problema di scomporre un polinomio~$P(x)$ si riconduce quindi a quello
della ricerca del numero~$k$ che sostituito alla~$x$ renda nullo il
polinomio. Un numero di questo tipo si dice anche \emph{radice del polinomio}.

Il numero~$k$ non va cercato del tutto a caso, abbiamo degli elementi per
restringere il campo di ricerca di questo numero quando il polinomio
è a coefficienti interi.

\osservazione Le radici intere del polinomio vanno cercate tra i divisori del termine
noto.

\begin{exrig}
 \begin{esempio}
 $P(x)=x^{3}+x^{2}-10x+8$.
 \end{esempio}
Le radici intere del polinomio sono da ricercare
nell'insieme dei divisori di~8, precisamente in~$\{\pm 1$, $\pm 2$, $\pm 4$, $\pm 8\}$.
Sostituiamo questi numeri nel polinomio,
finché non troviamo quello che lo annulla.

Per~$x=1$ si ha~$p(1)=(1)^{3}+(1)^{2}-10\cdot (1)+8=1+1-10+8=0$,
pertanto il polinomio è divisibile per~$x-1$.

Utilizziamo la regola di Ruffini per dividere~$P(x)$ per~$x-1$.

\begin{wrapfloat}{figure}{l}{0pt}
 % (c) 2012 Dimitrios Vrettos - d.vrettos@gmail.com

\begin{tikzpicture}[x=5mm,y=5mm]
\matrix (a)[matrix of nodes, nodes in empty cells,nodes={ text width=8mm, text depth=1mm, text centered}]{
&1&1&$-10$&8\\
1&&1&2&$-8$\\
&1&2&$-8$&0\\
};  
\begin{scope}[blue]
\draw(a-1-2.north west)--(a-3-1.south east);
\draw(a-2-1.south west)--(a-2-5.south east);
\draw(a-1-4.north east)--(a-3-4.south east);
      \end{scope}
\end{tikzpicture}
\end{wrapfloat}

Predisponiamo una griglia come quella a fianco, nella prima riga mettiamo i
coefficienti di~$P(x)$, nella seconda riga mettiamo come primo numero la
radice che abbiamo trovato, cioè~1. Poi procediamo come abbiamo già
indicato per la regola di Ruffini (sezione \ref{sect:regola_di_Ruffini} a pagina \pageref{sect:regola_di_Ruffini}).

I numeri che abbiamo ottenuto nell'ultima riga sono i
coefficienti del polinomio quoziente:~$q(x)=x^{2}+2x-8$.

Possiamo allora scrivere:
\[x^{3}+x^{2}-10x+8=(x-1)\cdot (x^{2}+2x-8).\]

Per fattorizzare il polinomio di secondo grado~$x^{2}+2x-8$ possiamo
ricorrere al metodo del trinomio notevole. Cerchiamo due numeri la sui
somma sia~$+2$ e il cui prodotto sia~$-8$. Questi numeri vanno cercati tra
le coppie che danno per prodotto~$-8$ e precisamente tra le seguenti
coppie~$(+8;-1)$, $(-8;+1)$, $(+4;-2)$, $(-4;+2)$. La coppia che dà per
somma~$+2$ è~$(+4;-2)$. In definitiva si ha:
\[x^{3}+x^{2}-10x+8=(x-1)\cdot (x^{2}+2x-8)=(x-1)(x-2)(x+4).\]


 \begin{esempio}
$x^{4}-5x^{3}-7x^{2}+29x+30$.
\end{esempio}

Le radici intere vanno cercate tra i divisori di~30, precisamente in~$\{\pm 1$, $\pm 2$, $\pm 3$, $\pm 5$, $\pm 6$, $\pm 10$, $\pm 15$, $\pm 30\}$.
Sostituiamo questi numeri al posto della~$x$, finché non troviamo la
radice.

Per~$x=1$ si ha~$P(1)=1-5-7+29+30$ senza effettuare il calcolo si
nota che i numeri positivi superano quelli negativi, quindi~1 non è
una radice.

Per~$x=-1$ si ha
\begin{align*}
P(-1)&=(-1)^{4}-5\cdot (-1)^{3}-7\cdot(-1)^{2}+29\cdot (-1)+30\\
&=+1+5-7-29+30\\
&=0.
\end{align*}

Una radice del polinomio è quindi~$-1$; utilizzando la regola di Ruffini
abbiamo:
\begin{center}
 % (c) 2012 Dimitrios Vrettos - d.vrettos@gmail.com

\begin{tikzpicture}[x=5mm,y=5mm]
\matrix (a)[matrix of nodes, nodes in empty cells,nodes={ text width=8mm, text depth=1mm, text centered}]{
&1&$-5$&$-7$&29&30\\
$-1$&&$-1$&6&1&$-30$\\
&1&$-6$&$-1$&30&0\\
};  
\begin{scope}[blue]
\draw(a-1-2.north west)--(a-3-1.south east);
\draw(a-2-1.south west)--(a-2-6.south east);
\draw(a-1-5.north east)--(a-3-5.south east);
      \end{scope}
\end{tikzpicture}
\end{center}
Con i numeri che abbiamo ottenuto nell'ultima riga
costruiamo il polinomio quoziente:~$x^{3}-6x^{2}-1x+30$. Possiamo allora
scrivere:
\[x^{4}-5x^{3}-7x^{2}+29x+30=(x+1)(x^{3}-6x^{2}-x+30).\]

Con lo stesso metodo scomponiamo il polinomio~$x^{3}-6x^{2}-1x+30$.
Cerchiamone le radici tra i divisori di~30, precisamente
nell'insieme~$\{\pm 1$, $\pm 2$, $\pm 3$, $\pm 5$, $\pm 6$, $\pm 10$, $\pm 15$, $\pm 30\}$. Bisogna ripartire dall'ultima
radice trovata, cioè da~$-1$.

Per~$x=-1$ si ha~$P(-1)=(-1)^{3}-6\cdot (-1)^{2}-1\cdot (-1)+30=-1-6+1+30\neq~0$.

Per~$x=+2$ si ha~$P(+2)=(+2)^{3}-6\cdot (+2)^{2}-1\cdot (+2)+30=+8-24-2+30\neq~0$.

Per~$x=-2$ si ha~$P(-2)=(-2)^{3}-6\cdot (-2)^{2}-1\cdot (-2)+30=-8-24+2+30=0$.

Quindi~$-2$ è una radice del polinomio. Applichiamo la regola di
Ruffini, ricordando che nella prima riga dobbiamo mettere i coefficienti
del polinomio da scomporre, cioè~$x^{3}-6x^{2}-1x+30$.
\begin{center}
 % (c) 2012 Dimitrios Vrettos - d.vrettos@gmail.com

\begin{tikzpicture}[x=5mm,y=5mm]
\matrix (a)[matrix of nodes, nodes in empty cells,nodes={ text width=8mm, text depth=1mm, text centered}]{
&1&$-6$&$-1$&30\\
$-2$&&$-2$&16&$-30$\\
&1&$-8$&$15$&0\\
};  
\begin{scope}[blue]
\draw(a-1-2.north west)--(a-3-1.south east);
\draw(a-2-1.south west)--(a-2-5.south east);
\draw(a-1-4.north east)--(a-3-4.south east);
      \end{scope}
\end{tikzpicture}
\end{center}
Il polinomio~$q(x)$ si scompone nel
prodotto~$x^{3}-6x^{2}-x+30=(x+2)\cdot (x^{2}-8x+15)$.

Infine possiamo scomporre~$x^{2}-8x+15$ come trinomio notevole: i due
numeri che hanno per somma~$-8$ e prodotto~$+15$ sono~$-3$ e~$-5$. In
conclusione possiamo scrivere la scomposizione:
\[x^{4}-5x^{3}-7x^{2}+29x+30=(x+1)\cdot (x+2)\cdot (x-3)\cdot (x-5).\]

Non sempre è possibile scomporre un polinomio utilizzando solo numeri
interi. In alcuni casi possiamo provare con le frazioni, in particolare
quando il coefficiente del termine di grado maggiore non è~1. In
questi casi possiamo cercare la radice del polinomio tra le frazioni
del tipo~$\frac{p}{q}$, dove~$p$ è un divisore del termine noto e~$q$ è
un divisore del coefficiente del termine di grado maggiore.


\begin{esempio}
$6x^{2}-x-2.$
\end{esempio}

Determiniamo prima di tutto l'insieme nel quale
possiamo cercare le radici del polinomio. Costruiamo tutte le frazioni
del tipo~$\frac{p}{q}$, con~$p$ divisore di~$-2$ e~$q$ divisore di~$6$. I
divisori di~2 sono~$\{\pm 1$, $\pm 2\}$ mentre i divisori di~6 sono~$\{\pm 1$, $\pm 2$, $\pm 3$, $\pm 6\}$.
Le frazioni tra cui cercare sono
\[\left\{\pm {\frac{1}{1}}\text{, }\pm \frac{1}{2}\text{, }\pm \frac{2}{1}\text{, }\pm
\frac{2}{3}\text{, }\pm \frac{2}{6}\right\}\]
cioè \[\left\{\pm 1\text{, }\pm\frac{1}{2}\text{, }\pm 2\text{, }\pm \frac{2}{3}\text{, }\pm \frac{1}{3}\right\}.\]

Si ha~$\quad A(1)=-3;\quad A(-1)=5;\quad A\left(\frac{1}{2}\right)=-1;\quad A\left(-{\frac{1}{2}}\right)=0$.

\begin{wrapfloat}{figure}{l}{0pt}
 % (c) 2012 Dimitrios Vrettos - d.vrettos@gmail.com

\begin{tikzpicture}[x=5mm,y=5mm]
\matrix (a)[matrix of nodes, nodes in empty cells,nodes={ text width=8mm, text depth=1mm, text centered}]{
&6&$-1$&$-2$\\
$-\frac{1}{2}$&&$-3$&2\\
&6&$-4$&0\\
};  
\begin{scope}[blue]
\draw(a-1-2.north west)--(a-3-1.south east);
\draw(a-2-1.south west)--(a-2-4.south east);
\draw(a-1-3.north east)--(a-3-3.south east);
      \end{scope}
\end{tikzpicture}
\end{wrapfloat}
Sappiamo dal teorema di Ruffini che il polinomio~$A(x)=6x^{2}-x-2$ è
divisibile per~$\left(x+\frac{1}{2}\right)$ dobbiamo quindi trovare il
polinomio~$Q(x)$ per scomporre~$6x^{2}-x-2$ come~$Q(x)\cdot \left(x+\frac{1}{2}\right)$.

Applichiamo la regola di Ruffini per trovare il quoziente. Il quoziente è~$Q(x)=6x-4$.
Il polinomio sarà scomposto in~$(6x-4)\cdot\left(x+\frac{1}{2}\right)$.
Mettendo a fattore comune~2 nel primo binomio si ha:
\[6x^{2}-x-2\ =\ (6x-4)%
\left(x+\frac{1}{2}\right)\ =\ 2(3x-2)\left(x+\frac{1}{2}\right)\ =(3x-2)(2x+1).\]
\end{exrig}

\ovalbox{\risolvii \ref{ese:13.67}, \ref{ese:13.68}, \ref{ese:13.69}, \ref{ese:13.70}, \ref{ese:13.71}}

\subsection{Somma e differenza di due cubi}

Per scomporre i polinomi del tipo~$A^{3}+B^{3}$ e~$A^{3}-B^{3}$
possiamo utilizzare il metodo di Ruffini.

\begin{exrig}
 \begin{esempio}
 $x^{3}-8$.
\end{esempio}
Il polinomio si annulla per~$x=2$, che è la radice cubica di~8.
Calcoliamo il quoziente.
\begin{wrapfloat}{figure}{l}{0pt}
 % (c) 2012 Dimitrios Vrettos - d.vrettos@gmail.com

\begin{tikzpicture}[x=5mm,y=5mm]
\matrix (a)[matrix of nodes, nodes in empty cells,nodes={ text width=8mm, text depth=1mm, text centered}]{
&1&0&0&$-8$\\
2&&2&4&8\\
&1&2&4&0\\
};  
\begin{scope}[blue]
\draw(a-1-2.north west)--(a-3-1.south east);
\draw(a-2-1.south west)--(a-2-5.south east);
\draw(a-1-4.north east)--(a-3-4.south east);
      \end{scope}
\end{tikzpicture}
\end{wrapfloat}
Il polinomio quoziente è~$Q(x)=x^{2}+2x+4$ e la scomposizione risulta
\[x^{3}-8\ =\ (x-2)(x^{2}+2x+4).\]

Notiamo che il quoziente somiglia al quadrato di un binomio, ma non lo
è in quanto il termine intermedio è il prodotto e non il doppio
prodotto dei due termini, si usa anche dire che è un ``falso quadrato''.
Un trinomio di questo tipo non è ulteriormente scomponibile.


 \begin{esempio}
 $x^{3}+27$.
 \end{esempio}
 \begin{wrapfloat}{figure}{l}{0pt}
 % (c) 2012 Dimitrios Vrettos - d.vrettos@gmail.com

\begin{tikzpicture}[x=5mm,y=5mm]
\matrix (a)[matrix of nodes, nodes in empty cells,nodes={ text width=8mm, text depth=1mm, text centered}]{
&1&0&0&27\\
$-3$&&$-3$&9&$-27$\\
&1&$-3$&9&0\\
};  
\begin{scope}[blue]
\draw(a-1-2.north west)--(a-3-1.south east);
\draw(a-2-1.south west)--(a-2-5.south east);
\draw(a-1-4.north east)--(a-3-4.south east);
      \end{scope}
\end{tikzpicture}
\end{wrapfloat}
Il polinomio si annulla per~$x=-3$, cioè~$P(-3)=(-3)^{3}+27=-27+27=0$.
Il polinomio quindi è divisibile per~$x+3$. Calcoliamo il quoziente
attraverso la regola di Ruffini.

Il polinomio quoziente è~$Q(x)=x^{2}-3x+9$ e la scomposizione risulta
\[x^{3}+27=(x+3)(x^{2}-3x+9).\]

\end{exrig}

In generale possiamo applicare le seguenti regole per la scomposizione
di somma e differenza di due cubi:

\[A^{3}+B^{3}=(A+B)(A^{2}-AB+B^{2})\text{,}\]
\[A^{3}-B^{3}=(A-B)(A^{2}+AB+B^{2}).\]

\ovalbox{\risolvii \ref{ese:13.72}, \ref{ese:13.73}, \ref{ese:13.74}}

\subsection{Scomposizione mediante metodi combinati}

Nei paragrafi precedenti abbiamo analizzato alcuni metodi per ottenere
la scomposizione in fattori di un polinomio e talvolta abbiamo mostrato
che la scomposizione si ottiene combinando metodi diversi.
Sostanzialmente non esiste una regola generale per la scomposizione di
polinomi, cioè non esistono criteri di divisibilità semplici come
quelli per scomporre un numero nei suoi fattori primi. In questo
paragrafo vediamo alcuni casi in cui si applicano vari metodi combinati
tra di loro.

Un buon metodo per ottenere la scomposizione è procedere tenendo conto
di questi suggerimenti:


\begin{enumerate}
\item analizzare se si può effettuare \emph{un raccoglimento totale};
\item \emph{contare il numero di termini} di cui si compone il polinomio:
 \begin{enumerate}
  \item \emph{due} termini. Analizzare se il binomio è
   \begin{enumerate}
	\item una \emph{differenza di quadrati} $A^{2}-B^{2}=(A-B)(A+B)$;
	\item una \emph{differenza di cubi} $A^{3}-B^{3}=(A-B)\left(A^{2}+AB+B^{2}\right)$;
	\item una \emph{somma di cubi} $A^{3}+B^{3}=(A+B)\left(A^{2}-AB+B^{2}\right)$;
	\item una \emph{somma di quadrati} $A^{2}+B^{2}$, nel qual caso è \emph{irriducibile}.
   \end{enumerate}
  \item \emph{tre} termini. Analizzare se è
   \begin{enumerate}
	\item un \emph{quadrato di un binomio} $A^{2}\pm~2AB+B^{2}=\left(A\pm B\right)^{2}$;
	\item un \emph{trinomio particolare} del tipo~$x^{2}+Sx+P=(x+a)(x+b)$ con~$a+b=S$ e~$a\cdot b=P$;
	\item un \emph{falso quadrato}~$A^{2}\pm AB+B^{2}$, che è irriducibile.
   \end{enumerate}
  \item \emph{quattro} termini. Analizzare se è
   \begin{enumerate}
	\item un \emph{cubo di un binomio} $A^{3}\pm~3A^{2}B+3AB^{2}\pm B^{3}=\left(A\pm B\right)^{3}$;
	\item una \emph{particolare differenza di quadrati}
	 \subitem $A^{2}\pm~2AB+B^{2}-C^{2}=(A\pm B+C)(A\pm B-C)$;
	\item un \emph{raccoglimento parziale}, tipo $ax+bx+ay+by=(a+b)(x+y)$.
   \end{enumerate}
  \item \emph{sei} termini. Analizzare se è
   \begin{enumerate}
	\item un \emph{quadrato di un trinomio} $A^{2}+B^{2}+C^{2}+2AB+2{AC}+2{BC}=\left(A+B+C\right)^{2}$;
	\item un \emph{raccoglimento parziale}, tipo
	 \subitem $ax+{bx}+{cx}+{ay}+{by}+{cy}=(a+b+c)(x+y)$.
   \end{enumerate}
  \end{enumerate}
 \item se non riuscite ad individuare nessuno dei casi precedenti, provate ad applicare la \emph{regola di Ruffini}.
\end{enumerate}


Ricordiamo infine alcune formule per somma e differenza di potenze
dispari.

\[A^5+B^5=(A+B)\left(A^4-A^3B+A^2B^2-AB^3+B^4\right)\text{,}\]
\[A^5-B^5=(A-B)\left(A^4+A^3B+A^2B^2+AB^3+B^4\right)\text{,}\]
\[A^{7}\pm B^{7}=(A\pm B)\left(A^{6}\mp A^{5}B+A^{4}B^{2}\mp A^{3}B^{3}+A^{2}B^{4}\mp AB^{5}+B^{6}\right)\text{,}\]
\begin{equation*}
\begin{split}
 (A^{11}-B^{11})=(A-B)(A^{10}+A^{9}B+A^{8}B^{2}&+A^{7}B^{3}+A^{6}B^{4}+\\
 &+A^{5}B^{5}+A^{4}B^{6}+A^{3}B^{7}+A^{2}B^{8}+AB^{9}+B^{10}).\\
\end{split}
\end{equation*}

La differenza di due potenze ad esponente pari (uguale o differente tra le basi dei due addendi)
rientra nel caso della differenza di quadrati:

\[A^{8}-B^{10}=\left(A^{4}-B^{5}\right)\left(A^{4}+B^{5}\right).\]

In alcuni casi si può scomporre anche la somma di potenze pari:

\[A^{6}+B^{6}=\left(A^{2}\right)^{3}+\left(B^{2}\right)^{3}=\left(A^{2}+B^{2}\right)\left(A^{4}-A^{2}B^{2}+B^{4}\right)\text{,}\]
\[A^{10}+B^{10}=\left(A^{2}\right)^{5}+\left(B^{2}\right)^{5}=\left(A^{2}+B^{2}\right)\left(A^{8}-A^{6}B^{2}+A^{4}B^{4}-A^2B^6+B^8\right).\]
\pagebreak

Proponiamo di seguito alcuni esercizi svolti in modo che
possiate acquisire una certa abilità nella scomposizione di polinomi.

\begin{exrig}
 \begin{esempio}
 $a^{2}x+5abx-36b^{2}x$.

Il polinomio ha~3 termini, è di terzo grado in~2 variabili, è
omogeneo;
tra i suoi monomi si ha~$\mcd= x$; effettuiamo il raccoglimento
totale:~$x\cdot\left(a^{2}+5ab-36b^{2}\right)$.
Il trinomio ottenuto come secondo fattore è di grado~2 in~2 variabili,
omogeneo e può essere riscritto
\[a^{2}+\left(5b\right)\cdot a-36b^{2}.\]
Proviamo a scomporlo come trinomio particolare:
cerchiamo due monomi~$m$ ed~$n$ tali che~$m+n=5b$
e~$m\cdot n=-36b^{2}$; i due monomi sono~$m=9b$
ed~$n=-4b$;

\[a^{2}x+5abx-36b^{2}x=x\cdot\left(a+9b\right)\cdot \left(a-4b\right).\]
 \end{esempio}

 \begin{esempio}
 $x^{2}+y^{2}+2xy-2x-2y$.

Facendo un raccoglimento parziale del coefficiente~2 tra gli ultimi tre
monomi otterremmo~$x^{2}+y^{2}+2\cdot(xy-x-y)$ su cui non possiamo
fare alcun ulteriore raccoglimento.

I primi tre termini formano però il quadrato di un binomio e tra gli
altri due possiamo raccogliere~$-2$, quindi~$(\underline{x+y})^{2}-2\cdot(\underline{x+y})$,
raccogliendo~$(x + y)$ tra i due termini si ottiene

\begin{equation*}
x^{2}+y^{2}+2xy-2x-2y=\left(x+y\right)\cdot \left(x+y-2\right).
\end{equation*}
 \end{esempio}

 \begin{esempio}
 $8a+10b+\left(1-4a-5b\right)^{2}-2$.

Tra i monomi sparsi possiamo raccogliere~2 a fattore comune
\[2\cdot \left(4a+5b-1\right)+\left(1-4a-5b\right)^{2}.\]

Osserviamo che la base del quadrato è l'opposto del polinomio contenuto
nel primo termine. Ma poiché numeri opposti hanno
lo stesso quadrato, possiamo cambiare il segno alla base del quadrato riscrivendo:
\[2\cdot\left(4a+5b-1\right)+\left(-1+4a+5b\right)^{2}.\]
Quindi si può mettere a fattore comune il termine $(4a+5b-1)$ ottenendo
\begin{align*}
8a+10b+(1-4a-5b)^{2}-2&=\left(4a+5b-1\right)\cdot\left(2-1+4a+5b\right)\\
&=\left(4a+5b-1\right)\cdot\left(1+4a+5b\right).
\end{align*}
 \end{esempio}

 \begin{esempio}
 $t^{{3}}-z^{{3}}+t^{2}-z^{2}$.

Il polinomio ha~4 termini, è di terzo grado in due variabili.
Poiché due monomi sono nella variabile $t$ e gli altri due nella
variabile $z$ potremmo subito effettuare un raccoglimento
parziale:~$t^{{3}}-z^{{3}}+t^{2}-z^{2}=t^{2}\cdot\left(t+1\right)-z^{2}\cdot \left(z+1\right)$,
che non permette un ulteriore passo. Occorre quindi un'altra
idea.

Notiamo che i primi due termini costituiscono una differenza di cubi e
gli altri due una differenza di quadrati; applichiamo le regole:

\begin{equation*}
t^{{3}}-z^{{3}}+t^{2}-z^{2}=\left(t-z\right)\cdot
\left(t^{2}+tz+z^{2}\right)+\left(t-z\right)\cdot
\left(t+z\right).
\end{equation*}
Ora effettuiamo il raccoglimento totale del fattore comune~$(t-z)$

\begin{equation*}
t^{3}-z^{3}+t^{2}-z^{2} = \left(t-z\right)\cdot
\left(t^{2}+tz+z^{2}+t+z\right).
\end{equation*}
 \end{esempio}

 \begin{esempio}
 $P(x)=x^{{3}}-7x-6$.
 \end{esempio}

Il polinomio ha~3 termini, è di~3{\textdegree} grado in una variabile.
Non possiamo utilizzare la regola del trinomio particolare poiché il
grado è~3. Procediamo con la regola di Ruffini: cerchiamo il numero che annulla 
il polinomio nell'insieme dei divisori del termine
noto~$D=\{\pm~1\text{,~}\pm~2\text{,~}\pm~3\text{,~}\pm~6\}$.

Per~$x=+1$ si ha~$P(+1)=(+1)^{3}-7\cdot (+1)-6=1-7-6\neq~0$.

Per~$x=-1$ si ha~$P(-1)=(-1)^{3}-7\cdot (-1)-6=-1+7-6=0$.

Quindi~$P(x)=\left(x+1\right)\cdot Q(x)$ con~$Q(x)$ polinomio di
secondo grado che determiniamo con la regola di Ruffini:

\begin{wrapfloat}{figure}{r}{0pt}
 % (c) 2012 Dimitrios Vrettos - d.vrettos@gmail.com

\begin{tikzpicture}[x=5mm,y=5mm]
\matrix (a)[matrix of nodes, nodes in empty cells,nodes={ text width=8mm, text depth=1mm, text centered}]{
&1&0&$-7$&$-6$\\
$-1$&&$-1$&1&6\\
&1&$-1$&$-6$&0\\
};  
\begin{scope}[blue]
\draw(a-1-2.north west)--(a-3-1.south east);
\draw(a-2-1.south west)--(a-2-5.south east);
\draw(a-1-4.north east)--(a-3-4.south east);
      \end{scope}
\end{tikzpicture}

\end{wrapfloat}
Pertanto:~$P(x)=x^{3}-7x-6=\left(x+1\right)\cdot \left(x^{2}-x-6\right)$.

Il polinomio quoziente è un trinomio di secondo grado; proviamo a
scomporlo come trinomio notevole.
Cerchiamo due numeri~$a$ e~$b$ tali che~$a+b=-1$ e~$a\cdot b=-6$.
I due numeri vanno cercati tra le coppie che hanno~$-6$ come prodotto,
precisamente~$(-6\text{,~}+1)$, $(-3\text{,~}+2)$, $(+6\text{,~}-1)$, $(+3\text{,~}-2)$. La coppia che fa al
caso nostro è~$(-3\text{,~}+2)$ quindi si
scompone~$Q(x)=x^{2}-x-6=\left(x-3\right)\cdot \left(x+2\right)$.

In definitiva~$x^{{3}}-7x-6=\left(x+1\right)\cdot (x-3)\cdot (x+2)$.

 \begin{esempio}
 $\left(m^{2}-4\right)^{2}-m^{2}-4m-4$.

Il polinomio ha~4 termini di cui il primo è un quadrato di un binomio;
negli altri tre possiamo raccogliere~$-1$;

\begin{equation*}
\left(m^{2}-4\right)^{2}-m^{2}-4m-4=\left(m^{2}-4\right)^{2}-\left(m^{2}+4m+4\right)
\end{equation*}

Notiamo che anche il secondo termine è un quadrato di un binomio, quindi:
\begin{equation}\label{eq:es_diff_quad}
\left(m^{2}-4\right)^{2}-\left(m+2\right)^{2}
\end{equation}
che si presenta come differenza di quadrati, allora diviene:
\[\left[\left(m^{2}-4\right)+\left(m+2\right)\right]\cdot
\left[\left(m^{2}-4\right)-\left(m+2\right)\right].\]

Eliminando le parentesi tonde~$\left(m^{2}+m-2\right)\cdot
\left(m^{2}-m-6\right)$.

I due fattori ottenuti si scompongono con la regola del trinomio. In
definitiva si ottiene:

\begin{align*}
(m^{2}+m-2)\cdot (m^{2}-m-6)&=\left(m+2\right)\cdot\left(m-1\right)\cdot \left(m-3\right)\cdot
\left(m+2\right)\\
&=\left(m+2\right)^{2}\cdot \left(m-1\right)\cdot
\left(m-3\right).
\end{align*}

Allo stesso risultato si poteva arrivare anche considerando che la \ref{eq:es_diff_quad} è sì una differenza di quadrati, ma a sua volta il termine $m^2-4$ è anch'esso una differenza di quadrati. Quindi si ha
\[\left(m^{2}-4\right)^{2}-\left(m+2\right)^{2} = (m+2)^2\cdot (m-2)^2-(m+2)^{2}\]
e mettendo in evidenza il fattore $\left(m+2\right)^{2}$ si può scrivere

\[\left(m+2\right)^{2}\cdot \left[ \left( m-2 \right)^2 -1 \right]. \]

Svolgendo le operazioni all'interno delle parentesi quadre si ottiene

\[\left(m+2\right)^{2}\cdot \left[ m^2-4m+4 -1 \right] = \left(m+2\right)^{2}\cdot \left[ m^2-4m+3 \right]. \]

A questo punto, per scomporre il fattore $m^2-4m+3$ si deve cercare una coppia di numeri interi, tali che la loro somma sia $-4$ ed il loro prodotto sia 3. La coppia di valori è $(-3\text{,~}-1)$ e quindi di può scrivere

\begin{align*}
\left(m^{2}-4\right)^2-m^2-4m-4 &=\left(m+2\right)^2\cdot \left( m^2-4m+3 \right)\\
&=\left(m+2\right)^{2}\cdot \left(m-1\right)\cdot \left(m-3\right).
\end{align*}


 \end{esempio}

 \begin{esempio}
 $\left(a-3\right)^{2}+\left(3a-9\right)\cdot\left(a+1\right)-\left(a^{2}-9\right)$.


\begin{equation*}
\left(a-3\right)^{2}+\left(3a-9\right)\cdot\left(a+1\right)-\left(a^{2}-9\right)%
=\left(a-3\right)^{2}+3\cdot \left(a-3\right)\cdot%
\left(a+1\right)-\left(a-3\right)\cdot \left(a+3\right).
\end{equation*}
Mettiamo a fattore comune~$(a-3)$:
\begin{equation*}
(a-3)\cdot \left[\left(a-3\right)+3\cdot
\left(a+1\right)-\left(a+3\right)\right].
\end{equation*}
Svolgiamo i calcoli nel secondo fattore e otteniamo:

\begin{equation*}
(a-3)(a-3+3a+3-a-3)=(a-3)(3a-3)=3(a-3)(a-1).
\end{equation*}
 \end{esempio}

 \begin{esempio}
 $a^4+a^{2}b^{2}+b^4$.

Osserva che per avere il quadrato del binomio occorre il doppio
prodotto, aggiungendo e togliendo~$a^{2}b^{2}$ otteniamo il doppio
prodotto cercato e al passaggio seguente ci troviamo con la differenza
di quadrati:

\[a^4+2a^2b^2+b^4 -a^2b^2=\left(a^2+b^2\right)^2-\left(ab\right)^2%
 =\left(a^2+b^2+ab\right)\left(a^2+b^2-ab\right).\]

 \end{esempio}

 \begin{esempio}
 $a^{5}+2a^{4}b+a^{3}b^{2}+a^{2}b^{3}+2ab^{4}+b^{5}$.

 \begin{align*}
  a^{5}+2a^{4}b+a^{3}b^{2}+a^{2}b^{3}+2ab^{4}+b^{5}&=a^{3}\left(a^{2}+2ab+b^{2}\right)+b^{3}\left(a^{2}+2ab+b^{2}\right)\\
  &=\left(a^{3}+b^{3}\right)\left(a^{2}+2ab+b^{2}\right)\\
  &=\left(a+b\right)\left(a^{2}-ab+b^{2}\right)\left(a+b\right)^{2}\\
  &=\left(a+b\right)^{3}\left(a^{2}-ab+b^{2}\right).
 \end{align*}
 \end{esempio}

 \begin{esempio}
 $a^{2}x^{2}+2ax^{2}-3x^{2}-4a^{2}-8a+12$.

 \begin{align*}
  a^{2}x^{2}+2ax^{2}-3x^{2}-4a^{2}-8a+12&=x^{2}\left(a^{2}+2a-3\right)-4\left(a^{2}+2a-3\right)\\
  &=\left(x^{2}-4\right)\left(a^{2}+2a-3\right)\\
  &=(x+2)(x-2)(a-1)(a+3).\\
 \end{align*}
 \end{esempio}
\end{exrig}

\newpage
% (c) 2012 Claudio Carboncini - claudio.carboncini@gmail.com
% (c) 2012 Dimitrios Vrettos - d.vrettos@gmail.com
\section{Esercizi}
\subsection{Esercizi dei singoli paragrafi}
%\subsubsection*{13.1 - Cosa vuol dire scomporre in fattori}
\subsubsection*{13.1 - Raccoglimento totale a fattore comune}

\begin{esercizio}
\label{ese:13.1}
Associa le espressioni a sinistra con i polinomi a destra.
  \begin{multicols}{2}
\begin{enumeratea}
\item $(a+2b)^{2}$;
\item $3ab^{2}(a^{2}-b)$;
\item $(2a+3b)(a-2b)$;
\item $(3a-b)(3a+b)$;
\item $(a+b)^{3}$;
\item $(a+b+c)^{2}$;
\item $2a^{2}-4ab+3ab-6b^{2}$;
\item $a^{2}+4ab+4b^{2}$;
\item $9a^{2}-b^{2}$;
\item $3a^{3}b^{2}-3ab^{3}$;
\item $a^{2}+b^{2}+c^{2}+2ab+2bc+2ac$;
\item $a^{3}+3a^{2}b+3ab^{2}+b^{3}$.
\end{enumeratea}
  \end{multicols}
\end{esercizio}

\begin{esercizio}[\Ast]
\label{ese:13.2}
Scomponi in fattori raccogliendo a fattore comune.
\begin{multicols}{2}
\begin{enumeratea}
 \item $ax+3a^{2}x-abx$;
 \item $15b^{2}+12bc+21abx+6ab^{2}$;
 \item $15x^{2}y-10xy+25x^{2}y^{2}$;
 \item $-12a^{8}b^{9}-6a^{3}b^{3}-15a^{4}b^{3}$;
 \item $2ab^{2}+2b^{2}c-2a^{2}b^{2}-2b^{2}c^{2}$;
 \item $2m^{7}+8m^{6}+8m^{5}$.
\end{enumeratea}
\end{multicols}
\end{esercizio}

\begin{esercizio}[\Ast]
\label{ese:13.3}
Scomponi in fattori raccogliendo a fattore comune.
\begin{multicols}{3}
\begin{enumeratea}
 \item $9x^{2}b+6xb+18xb^{2}$;
 \item $20a^{5}+15a^{7}+10a^{4}$;
 \item $x^{2}b-x^{5}-4x^{3}b^{2}$.
 \item $3xy+6x^{2}$;
 \item $b^{3}+\dfrac{1}{3}b$;
 \item $3xy-12y^{2}$;
\end{enumeratea}
\end{multicols}
\end{esercizio}

\begin{esercizio}
\label{ese:13.4}
Scomponi in fattori raccogliendo a fattore comune.
\begin{multicols}{3}
\begin{enumeratea}
 \item $x^{3}-ax^{2}$;
 \item $9a^{3}-6a^{2}$;
 \item $5x^{2}-15x$;
 \item $18x^{2}y-12y^{2}$;
 \item $4x^{2}y-x^{2}$;
 \item $5x^{3}-2x^{2}$.
\end{enumeratea}
\end{multicols}
\end{esercizio}

\begin{esercizio}
\label{ese:13.5}
Scomponi in fattori raccogliendo a fattore comune.
\begin{multicols}{2}
\begin{enumeratea}
 \item $-2x^{3}+2x$;
 \item $3a+3$;
 \item $-8x^{2}y^{3}-10x^{3}y^{2}$;
 \item $\dfrac{2}{3}a^{2}b-\dfrac{4}{3}a^{4}b^{3}-\dfrac{5}{9}a^{2}b^{2}$;
 \item $12a^{3}x^{5}-18ax^{6}-6a^{3}x^{4}+3a^{2}x^{4}$;
 \item $\dfrac{2}{3}a^{4}bc^{2}-4ab^{3}c^{2}+\dfrac{10}{3}abc^{2}$.
\end{enumeratea}
\end{multicols}
\end{esercizio}

\begin{esercizio}
\label{ese:13.6}
Scomponi in fattori raccogliendo a fattore comune.
\begin{multicols}{2}
\begin{enumeratea}
 \item $-{\dfrac{3}{5}}a^{4}bx+\dfrac{3}{2}ab^{4}x-2a^{3}b^{2}x$.
 \item $-{\dfrac{5}{2}}a^{3}b^{3}-\dfrac{5}{3}a^{4}b^{2}+\dfrac{5}{6}a^{3}b^{4}$;
 \item $91m^{5}n^{3}+117m^{3}n^{4}$;
 \item $\dfrac{2}{3}a^{2}x+\dfrac{5}{4}ax^{2}-\dfrac{5}{4}ax$;
 \item $-5a^{2}+10ab^{2}-15a$;
 \item $ab^{2}-a+a^{2}$.
\end{enumeratea}
\end{multicols}
\end{esercizio}
\pagebreak
\begin{esercizio}
\label{ese:13.7}
Scomponi in fattori raccogliendo a fattore comune.
\begin{multicols}{3}
\begin{enumeratea}
 \item $2b^{6}+4b^{4}-b^{9}$;
 \item $2a^{2}b^{2}x-4a^{2}b$;
 \item $-a^{4}-a^{3}-a^{5}$;
 \item $-3a^{2}b^{2}+6ab^{2}-15b$;
 \item $a^{2}b-b+b^{2}$;
 \item $3b^{5}-3b^{3}-6b^{2}$.
\end{enumeratea}
\end{multicols}
\end{esercizio}

\begin{esercizio}
\label{ese:13.8}
Scomponi in fattori raccogliendo a fattore comune.
\begin{multicols}{3}
\begin{enumeratea}
 \item $-{\dfrac{4}{9}}x+\dfrac{2}{3}x^{2}-\dfrac{1}{3}x^{3}$;
 \item $-a^{2}b^{2}-a^{3}b^{5}+b^{3}$;
 \item $-2x^{6}+4x^{5}-6x^{3}y^{9}$;
 \item $-2x^{2}z^{3}+4z^{5}-6x^{3}z^{3}$;
 \item $-5a^{4}-10a^{2}-30a$;
 \item $\dfrac{1}{2}a^{2}+\dfrac{1}{2}a$.
\end{enumeratea}
\end{multicols}
\end{esercizio}

\begin{esercizio}[\Ast]
\label{ese:13.9}
Scomponi in fattori raccogliendo a fattore comune.
\begin{multicols}{2}
\begin{enumeratea}
 \item $a^{n}+a^{n-1}+a^{n-2}$;
 \item $\dfrac{1}{3}ab^{3}+\dfrac{1}{6}a^{3}b^{2}$;
 \item $a^{n}+a^{2n}+a^{3n}$;
 \item $2x^{2n}-6x^{(n-1)}+4x^{(3n+1)}$;
 \item $a^{2}x^{n-1}-2a^{3}x^{n+1}+a^{4}x^{2n}$;
 \item $a(x+y)-b(x+y)$.
\end{enumeratea}
\end{multicols}
\end{esercizio}

\begin{esercizio}[\Ast]
\label{ese:13.10}
Scomponi in fattori raccogliendo a fattore comune.
\begin{multicols}{2}
\begin{enumeratea}
 \item $(x+y)^{3}-(x+y)^{2}$;
 \item $a^{n}+a^{n+1}+a^{n+2}$;
 \item $(a+2)^{3}-(a+2)^{2}-a-2$;
 \item $2a(x-2)+3x(x-2)^{2}-(x-2)^{2}$;
 \item $3(x+y)^{2}-6(x+y)+2x(x+y)$;
 \item $x^{2}(a+b)^{3}+x^{3}(a+b)+x^{5}(a+b)^{2}$.
\end{enumeratea}
\end{multicols}
\end{esercizio}

\begin{esercizio}[\Ast]
Scomponi in fattori raccogliendo a fattore comune.
\label{ese:13.11}
 \begin{multicols}{2}
 \begin{enumeratea}
 \item $5y^{3}(x-y)^{3}-3y^{2}(x-y)$;
 \item $5a(x+3y)-3(x+3y)$;
 \item $2x(x-1)-3a^{2}(x-1)$;
 \item $2(x-3y)-y(3y-x)$;
 \item $3x^{2}(a+b)-2x^{3}(a+b)+5x^{5}(a+b)$;
 \item $(2x-y)^{2}-5x^{3}(2x-y)-3y(2x-y)^{3}$.
\end{enumeratea}
 \end{multicols}
\end{esercizio}

\subsubsection*{13.2 - Raccoglimento parziale a fattore comune}

\begin{esercizio}[\Ast]
\label{ese:13.12}
Scomponi in fattori con il raccoglimento parziale a fattore comune, se possibile.
\begin{multicols}{3}
 \begin{enumeratea}
 \item $2x-2y+ax-ay$;
 \item $3ax-6a+x-2$;
 \item $ax+bx-ay-by$;
 \item $3ax-9a-x+3$;
 \item $ax^{3}+ax^{2}+bx+b$;
 \item $2ax-4a-x+2$.
\end{enumeratea}
\end{multicols}
\end{esercizio}

\begin{esercizio}[\Ast]
\label{ese:13.13}
Scomponi in fattori con il raccoglimento parziale a fattore comune, se possibile.
\begin{multicols}{2}
\begin{enumeratea}
 \item $b^{2}x+b^{2}y+2ax+2ay$.
 \item $3x^{3}-3x^{2}+3x-3$;
 \item $x^{3}-x^{2}+x-1$;
 \item $ay+2x^{3}-2ax^{3}-y$.
 \item $-x^{3}+x^{2}+x-1$;
 \item $x^{3}+x^{2}-x-1$;
\end{enumeratea}
\end{multicols}
\end{esercizio}

\begin{esercizio}[\Ast]
\label{ese:13.14}
Scomponi in fattori con il raccoglimento parziale a fattore comune, se possibile.
\begin{multicols}{2}
\begin{enumeratea}
 \item $x^{3}-1-x+x^{2}$;
 \item $-x^{3}-x-1-x^{2}$;
 \item $x^{3}+x^{2}+x+1$;
 \item $b^{2}x-b^{2}y+2x-2y$;
 \item $b^{2}x-b^{2}y-2ax-2ay$;
 \item $xy+x+ay+a+by+b$.
\end{enumeratea}
\end{multicols}
\end{esercizio}

\begin{esercizio}
\label{ese:13.15}
Scomponi in fattori con il raccoglimento parziale a fattore comune, se possibile.
\begin{multicols}{2}
\begin{enumeratea}
 \item $3x+6+ax+2a+bx+2b$;
 \item $2x-2+bx-b+ax-a$;
 \item $2x-2+bx-b-ax+a$;
 \item $2x+2+bx-b-ax+a$;
 \item $2x-b+ax-a-2+bx$;
 \item $a^{3}+2a^{2}+a+2$.
\end{enumeratea}
\end{multicols}
\end{esercizio}

\begin{esercizio}
\label{ese:13.16}
Scomponi in fattori con il raccoglimento parziale a fattore comune, se possibile.
\begin{multicols}{2}
\begin{enumeratea}
 \item $a^{2}x+ax-a-1$;
 \item $3xy^{3}-6xy-ay^{2}+2a$;
 \item $a^{2}x^{3}+a^{2}x^{2}+a^{2}x-2x^{2}-2x-2$;
 \item $3x^{4}-3x^{3}+3x^{2}-3x$;
 \item $2ax-2a+abx-ab+a^{2}x-a^{2}$;
 \item $3x^{4}y^{4}-6x^{4}y^{2}-ax^{3}y^{3}+2ax^{3}y$.
\end{enumeratea}
\end{multicols}
\end{esercizio}

\begin{esercizio}[\Ast]
\label{ese:13.17}
Scomponi in fattori con il raccoglimento parziale a fattore comune, se possibile.
\begin{multicols}{2}
\begin{enumeratea}
 \item $b^{2}x-2bx+by-2y$;
 \item $\dfrac{2}{3}x^{3}-\dfrac{1}{3}x^{2}+2x-1$;
 \item $ax+bx+2x-a-b-2$;
 \item $3(x+y)^{2}+5x+5y$;
 \item $bx^{2}-bx+b+x^{2}-x+1$;
 \item $a^{3}-a^{2}b^{2}-ab+b^{3}$.
\end{enumeratea}
\end{multicols}
\end{esercizio}

\begin{esercizio}[\Ast]
\label{ese:13.18}
Scomponi in fattori con il raccoglimento parziale a fattore comune, se possibile.
\begin{multicols}{2}
\begin{enumeratea}
 \item $\dfrac{1}{5}a^{2}b+3ab^{2}-\dfrac{1}{3}a-5b$;
 \item $3x^{4}+9x^{2}-6x^{3}-18x$;
 \item $2a-a^{2}+8b-4ab$;
 \item $4x^{2}+3a+4xy-4ax-3y-3x$;
 \item $3x^{4}-3x^{3}+2x-2$;
 \item $(a-2)(a-3)+ab-2b$;
 \item $\dfrac{1}{8}x^{3}-2xy^{2}+\dfrac{1}{2}yx^{2}-8y^{3}$;
 \item $ab-bx^{2}-\dfrac{2}{3}ax+\dfrac{2}{3}x^{3}$.
\end{enumeratea}
\end{multicols}
\end{esercizio}

\begin{esercizio}[\Ast]
\label{ese:13.19}
Scomponi in fattori con il raccoglimento parziale a fattore comune, se possibile.
\begin{multicols}{2}
\begin{enumeratea}
 \item $10x^3-12x^2-5xy+6y$;
 \item $6a^3+3a^2b-2ab^3-b^4$;
 \item $2^{11}x^{2}+2^{12}x+2^{15}x+2^{16}$;
 \item $6x^{2}+6xy-3x(x+y)-9x^{2}(x+y)^{2}$.
\end{enumeratea}
\end{multicols}
\end{esercizio}

\begin{esercizio}[\Ast]
\label{ese:13.20}
Scomponi in fattori raccogliendo prima a fattore comune totale e poi parziale.
\begin{multicols}{2}
 \begin{enumeratea}
 \item $a^{14}+4a^{10}-2a^{12}-8a^{8}$;
 \item $3x^{2}(x+y)^{2}+5x^{3}+5x^{2}y$;
 \item $ax^{3}y+ax^{2}y+axy+ay$;
 \item $b^{2}x+b^{2}y-2bx-2by$;
 \item $b^{2}x-2bx-2by+b^{2}y$;
 \item $2ab^{2}+2b^{2}c-2a^{2}b^{2}-2ab^{2}c$.
\end{enumeratea}
\end{multicols}
\end{esercizio}

\begin{esercizio}
\label{ese:13.21}
Scomponi in fattori raccogliendo prima a fattore comune totale e poi parziale.
\begin{multicols}{2}
\begin{enumeratea}
 \item $3ax+6a+a^{2}x+2a^{2}+abx+2ab$.
 \item $2bx^{2}+4bx-2x^{2}-4ax$;
 \item $x^{4}+x^{3}-x^{2}-x$;
 \item $15x(x+y)^{2}+5x^{2}+5xy$;
 \item $2a^{2}mx-2ma^{2}-2a^{2}x+2a^{2}$;
 \item $2x^{3}+2x^{2}-2ax^{2}-2ax$.
\end{enumeratea}
\end{multicols}
\end{esercizio}

\begin{esercizio}[\Ast]
\label{ese:13.22}
Scomponi in fattori raccogliendo prima a fattore comune totale e poi parziale.
\begin{enumeratea}
 \item $45x^{3}+15xy+75x^{2}y+21x^{2}y^{2}+7y^{3}+35xy^{3}$;
 \item $\dfrac{2}{3}ax^{3}-\dfrac{1}{3}ax^{2}+\dfrac{2}{3}ax-\dfrac{1}{3}a$;
 \item $\dfrac{7}{3}x^{2}-\dfrac{7}{3}xy+\dfrac{1}{9}x^{3}-\dfrac{1}{9}x^{2}y-\dfrac{5}{9}(x^{2}-xy)$;
 \item $2b(x+1)^{2}-2bax-2ba+4bx+4b$.
\end{enumeratea}
\end{esercizio}

\subsubsection*{13.3 - Riconoscimento di prodotti notevoli}
\begin{esercizio}
\label{ese:13.23}
Quando è possibile, scomponi in fattori, riconoscendo il quadrato di un binomio.
\begin{multicols}{3}
\begin{enumeratea}
 \item $a^{2}-2a+1$;
 \item $x^{2}+4x+4$;
 \item $y^{2}-6y+9$;
 \item $16t^{2}+8t+1$;
 \item $4x^{2}+1+4x$;
 \item $9a^{2}-6a+1$.
\end{enumeratea}
\end{multicols}
\end{esercizio}

\begin{esercizio}
\label{ese:13.24}
Quando è possibile, scomponi in fattori, riconoscendo il quadrato di un binomio.
\begin{multicols}{3}
\begin{enumeratea}
 \item $4x^{2}-12x+9$;
 \item $\dfrac{1}{4}a^{2}+ab+b^{2}$;
 \item $9x^{2}+4+12x$;
 \item $\dfrac{4}{9}a^{{4}}-4a^{2}+9$;
 \item $\dfrac{1}{4}x^{2}-\dfrac{1}{3}x+\dfrac{1}{9}$;
 \item $16a^{2}+\dfrac{1}{4}b^{2}-4ab$.
\end{enumeratea}
\end{multicols}
\end{esercizio}

\begin{esercizio}
\label{ese:13.25}
Quando è possibile, scomponi in fattori, riconoscendo il quadrato di un binomio.
\begin{multicols}{3}
\begin{enumeratea}
 \item $-9x^{2}-\dfrac{1}{4}+3x$;
 \item $4x^{2}+4xy+y^{2}$;
 \item $a^{4}+36a^{2}+12a^{3}$;
 \item $144x^{2}-6xa^{2}+\dfrac{1}{16}a^{4}$;
 \item $x^{2}-6xy+9y^{2}$;
 \item $-x^{2}-6xy-9y^{2}$.
\end{enumeratea}
\end{multicols}
\end{esercizio}

\begin{esercizio}
\label{ese:13.26}
Quando è possibile, scomponi in fattori, riconoscendo il quadrato di un binomio.
\begin{multicols}{3}
\begin{enumeratea}
 \item $25+10x+x^{2}$;
 \item $\dfrac{1}{4}x^{2}+\dfrac{1}{3}xy+\dfrac{1}{9}y^{2}$;
 \item $25-10x+x^{2}$;
 \item $\dfrac{9}{25}a^{4}-6a^{2}+25$;
 \item $4x^{2}+2x^{4}+1$;
 \item $4x^{2}-4x^{4}-1$.
\end{enumeratea}
\end{multicols}
\end{esercizio}

\begin{esercizio}
\label{ese:13.27}
Quando è possibile, scomponi in fattori, riconoscendo il quadrato di un binomio.
\begin{multicols}{3}
\begin{enumeratea}
 \item $-a^{3}-2a^{2}-a$;
 \item $3a^{7}b-6a^{5}b^{2}+3a^{3}b^{3}$;
 \item $100+a^{2}b^{4}+20ab^{2}$;
 \item $2x^{13}-8x^{8}y+8x^{3}y^{2}$;
 \item $x^{8}+8x^{4}y^{2}+16y^{4}$;
 \item $-x^{2}+6{xy}+9y^{2}$.
\end{enumeratea}
\end{multicols}
\end{esercizio}

\begin{esercizio}
\label{ese:13.28}
Quando è possibile, scomponi in fattori, riconoscendo il quadrato di un binomio.
\begin{multicols}{2}
\begin{enumeratea}
 \item $4a^{2}b^{4}-12ab^{3}+9b^{6}$;
 \item $a^{2}+a+1$;
 \item $36a^{6}b^{3}+27a^{5}b^{4}+12a^{7}b^{2}$;
 \item $25x^{14}+9y^{6}+30x^{7}y^{3}$;
 \item $-a^{7}-25a^{5}+10a^{6}$;
 \item $25a^{2}+49b^{2}+35ab$.
\end{enumeratea}
\end{multicols}
\end{esercizio}

\begin{esercizio}
\label{ese:13.29}
Quando è possibile, scomponi in fattori, riconoscendo il quadrato di un binomio.
\begin{multicols}{2}
\begin{enumeratea}
 \item $4y^{6}+4-4y^{2}$;
 \item $\dfrac{1}{4}a^{2}+2ab+b^{2}$;
 \item $25a^{2}-10{ax}-x^{2}$;
 \item $9x^{2}+4y^{2}-6{xy}$.
\end{enumeratea}
\end{multicols}
\end{esercizio}

\begin{esercizio}
\label{ese:13.30}
Individua perché i seguenti polinomi non sono quadrati di un binomio.
\begin{multicols}{3}
\begin{enumeratea}
 \item $4x^{2}+4xy-y^{2}$; %\, non è un quadrato di binomio perché\,\dotfill;
 \item $x^{2}-6xy+9y$; %\, non è un quadrato di binomio perché\dotfill;
 \item $25+100x+x^{2}$; %\, non è un quadrato di binomio perché\dotfill;
 \item $25t^{2}+4-10t$; %\, non è un quadrato di binomio perché\dotfill
 \item $\dfrac{1}{4}x^{2}+\dfrac{2}{3}xy+\dfrac{1}{9}$. %\, non è un quadrato di binomio perché\dotfill;
\end{enumeratea}
\end{multicols}
\end{esercizio}
\pagebreak
\begin{esercizio}[\Ast]
\label{ese:13.31}
Quando è possibile, scomponi in fattori, riconoscendo il quadrato di un binomio.
\begin{multicols}{3}
\begin{enumeratea}
 \item $24a^{3}+6a+24a^{2}$;
 \item $3a^{2}x-12axb+12b^{2}x$;
 \item $5a^{2}+2ax+\frac{1}{5}x^{2}$;
 \item $x^{6}y+x^{2}y+2x^{4}y$;
 \item $x^{5}+4x^{4}+4x^{3}$;
 \item $2y^{3}-12y^{2}x+18x^{2}y$.
\end{enumeratea}
\end{multicols}
\end{esercizio}

\begin{esercizio}[\Ast]
\label{ese:13.32}
Quando è possibile, scomponi in fattori, riconoscendo il quadrato di un binomio.
\begin{multicols}{2}
\begin{enumeratea}
 \item $-50t^{3}-8t+40t^{2}$;
 \item $2^{10}x^{2}+2^{6}\cdot 3^{20}+3^{40}$;
 \item $2^{20}x^{40}-2^{26}\cdot x^{50}+2^{30}\cdot x^{60}$;
 \item $10^{100}x^{50}-2\cdot 10^{75}x^{25}+10^{50}$;
 \item $10^{11}x^{10}-2\cdot 10^{9}x^{5}+10^{6}$;
 \item $x^{2n}+2x^{n}+1$.
\end{enumeratea}
\end{multicols}
\end{esercizio}

\begin{esercizio}
\label{ese:13.33}
Quando è possibile, scomponi in fattori, riconoscendo il quadrato di un polinomio.
\begin{multicols}{2}
\begin{enumeratea}
 \item $a^{2}+b^{2}+c^{2}+2ab+2ac+2bc$;
 \item $x^{2}+y^{2}+z^{2}+2xy-2xz-2yz$;
 \item $x^{2}+y^{2}+4+4x+2xy+4y$;
 \item $4a^{4}-6{ab}-4a^{2}b+12a^{3}+b^{2}+9a^{2}$;
 \item $9x^{6}+2y^{2}z+y^{4}-6x^{3}z-6x^{3}y^{2}+z^{2}$;
 \item $\frac{1}{4}a^{2}+b^{4}+c^{6}+ab^{2}+{ac}^{3}+2b^{2}c^{3}$.
\end{enumeratea}
\end{multicols}
\end{esercizio}

\begin{esercizio}
\label{ese:13.34}
Quando è possibile, scomponi in fattori, riconoscendo il quadrato di un polinomio.
\begin{multicols}{2}
\begin{enumeratea}
 \item $a^{2}+2ab+b^{2}-2a+1-2b$;
 \item $x^{2}+\frac{1}{4}y^{2}+4-xy+4x-2y$;
 \item $a^{2}+b^{2}+c^{2}-2ac-2bc+2ab$;
 \item $-x^{2}-2xy-9-y^{2}+6x+6y$;
 \item $4a^{2}+4ab-8a+b^{2}-4b+4$;
 \item $a^{2}b^{2}+2a^{2}b+a^{2}-2ab^{2}-2ab+b^{2}$.
\end{enumeratea}
\end{multicols}
\end{esercizio}

\begin{esercizio}
\label{ese:13.35}
Individua perché i seguenti polinomi non sono quadrati.
\begin{multicols}{2}
\begin{enumeratea}
 \item $a^{2}+b^{2}+c^{2}$; %\, non è un quadrato perché\dotfill;
 \item $x^{2}+y^{2}+4+4x+4xy+4y$; %\, non è un quadrato perché\dotfill;
 \item $a^{2}+b^{2}+c^{2}-2ac-2bc-2ab$; %\, non è un quadrato perché\dotfill;
 \item $a^{2}+b^{2}-1-2a-2b+2ab$. %\, non è un quadrato perché\dotfill
\end{enumeratea}
\end{multicols}
\end{esercizio}

\begin{esercizio}[\Ast]
\label{ese:13.36}
Quando è possibile, scomponi in fattori, riconoscendo il quadrato di un polinomio.
\begin{enumeratea}
 \item $a^{2}+4ab-2a+4b^{2}-4b+1$;
 \item $a^{2}b^{2}+2a^{2}b+a^{2}+4ab^{2}+4ab+4b^{2}$;
 \item $x^{2}-6xy+6x+9y^{2}-18y+9$.
 \item $x^{4}+2x^{3}+3x^{2}+2x+1$\quad  \emph{suggerimento}:~$3x^{2}=x^{2}+2x^{2}$;
 \item $4a^{4}+8a^{2}+1+8a^{3}+4a$\quad \emph{suggerimento}:~$8a^{2}=4a^{2}+4a^{2}$.
\end{enumeratea}
\end{esercizio}

\begin{esercizio}
\label{ese:13.37}
Quando è possibile, scomponi in fattori, riconoscendo il quadrato di un polinomio.
\begin{enumeratea}
 \item $9x^{4}+6x^{3}-11x^{2}-4x+4$ \quad \emph{suggerimento}:~$-11x^{2}=-12x^{2}+x^{2}$;
 \item $25x^{2}-20ax-30bx+4a^{2}+12ab+9b^{2}$;
 \item $2a^{10}x+4a^{8}x+2a^{6}x+4a^{5}x+4a^{3}x+2x$;
 \item $a^{2}+b^{2}+c^{2}+d^{2}-2ab+2ac-2ad-2bc+2bd-2cd$;
 \item $x^{6}+x^{4}+x^{2}+1+2x^{5}+2x^{4}+2x^{3}+2x^{3}+2x^{2}+2x$.
\end{enumeratea}
\end{esercizio}
%\pagebreak
%\subsubsection*{13.3 - Cubo di un binomio}
\begin{esercizio}
\label{ese:13.38}
Quando è possibile, scomponi in fattori, riconoscendo il cubo di un binomio.
\begin{multicols}{2}
\begin{enumeratea}
 \item $8a^{3}+b^{3}+12a^{2}b+6ab^{2}$;
 \item $b^{3}+12a^{2}b-6ab^{2}-8a^{3}$;
 \item $-12a^{2}+8a^{3}-b^{3}+6ab$;
 \item $-12a^{2}b+6ab+8a^{3}-b^{3}$.
\end{enumeratea}
\end{multicols}
\end{esercizio}

\begin{esercizio}
\label{ese:13.39}
Quando è possibile, scomponi in fattori, riconoscendo il cubo di un binomio.
\begin{multicols}{2}
\begin{enumeratea}
 \item $-x^{3}+6x^{2}-12x+8$;
 \item $-x^{9}-3x^{6}+3x^{3}+8$;
 \item $x^{3}y^{6}+1+3x^{2}y^{2}+3xy^{2}$;
 \item $x^{3}+3x-3x^{2}-1$.
\end{enumeratea}
\end{multicols}
\end{esercizio}

\begin{esercizio}
\label{ese:13.40}
Quando è possibile, scomponi in fattori, riconoscendo il cubo di un binomio.
\begin{multicols}{2}
\begin{enumeratea}
 \item $-5x^{5}y^{3}-5x^{2}-15x^{4}y^{2}-15x^{3}y$;
 \item $-a^{6}+27a^{3}+9a^{5}-27a^{4}$;
 \item $64a^{3}-48a^{2}+12a-1$;
 \item $a^{6}+9a^{4}+27a^{2}+27$.
\end{enumeratea}
\end{multicols}
\end{esercizio}

\begin{esercizio}
\label{ese:13.41}
Quando è possibile, scomponi in fattori, riconoscendo il cubo di un binomio.
\begin{multicols}{2}
\begin{enumeratea}
 \item $x^{3}-x^{2}+\dfrac{1}{3}x-\dfrac{1}{27}$;
 \item $0,001x^{6}+0,015x^{4}+0,075x^{2}+0,125$;
 \item $\dfrac{27}{8}a^{3}-\dfrac{27}{2}a^{2}x+18ax^{2}-8x^{3}$;
 \item $x^{3}-x^{2}+\dfrac{1}{3}x-\dfrac{1}{27}$.
\end{enumeratea}
\end{multicols}
\end{esercizio}

\begin{esercizio}
\label{ese:13.42}
Individua perché i seguenti polinomi non sono cubi.
\begin{multicols}{2}
\begin{enumeratea}
 \item $a^{10}-8a-6a^{7}+12a^{4}$; %\, non è un cubo perché\dotfill;
 \item $27a^{3}-b^{3}+9a^{2}b-9ab^{2}$; %\, non è un cubo perché\dotfill;
 \item $8x^{3}+b^{3}+6x^{2}b+6{xb}^{2}$; %\, non è un cubo perché\dotfill;
 \item $x^{3}+6ax^{2}-6a^{2}x+8a^{3}$. %\, non è un cubo perché\dotfill
\end{enumeratea}
\end{multicols}
\end{esercizio}

\begin{esercizio}
\label{ese:13.43}
Quando è possibile, scomponi in fattori, riconoscendo il cubo di un binomio.
\begin{multicols}{2}
\begin{enumeratea}
 \item $x^{3}-6x^{2}+12x-8$;
 \item $a^{3}b^{3}+12ab+48ab+64$;
 \item $216x^{3}-540ax^{2}+450a^{2}x-125a^{3}$;
 \item $8x^{3}+12x^{2}+6x+2$.
\end{enumeratea}
\end{multicols}
\end{esercizio}

\begin{esercizio}[\Ast]
\label{ese:13.44}
Quando è possibile, scomponi in fattori, riconoscendo il cubo di un binomio.
\begin{multicols}{2}
\begin{enumeratea}
 \item $a^{6}+3a^{4}b^{2}+3a^{2}b^{4}+b^{6}$;
 \item $8a^{3}-36a^{2}b+54ab^{2}-27b^{3}$;
 \item $a^{6}+3a^{5}+3a^{4}+a^{3}$;
 \item $a^{10}-8a-6a^{7}+12a^{4}$.%ex154b trovato risultato: a\left(a^3-2\right)^3
\end{enumeratea}
\end{multicols}
\end{esercizio}

\begin{esercizio}
\label{ese:13.45}
Quando è possibile, scomponi in fattori, riconoscendo il cubo di un binomio.
\begin{multicols}{2}
\begin{enumeratea}
 \item $8x^{3}-36x^{2}+54x-27$;
 \item $x^{6}+12ax^{4}+12a^{2}x^{2}+8a^{3}$;
 \item $x^{300}-10^{15}-3\cdot 10^{5}x^{200}+3\cdot 10^{10}x^{100}$;
 \item $a^{6n}+3a^{4n}x^{n}+3a^{2n}x^{2n}+x^{3n}$.
\end{enumeratea}
\end{multicols}
\end{esercizio}

\begin{esercizio}
\label{ese:13.46}
Quando è possibile, scomponi in fattori, riconoscendo il cubo di un binomio.
\begin{enumeratea}
 \item $10^{15}a^{60}+3\cdot 10^{30}a^{45}+3\cdot 10^{45}a^{30}+10^{60}a^{15}$;
 \item $10^{-33}x^{3}-3\cdot 10^{-22}x^{2}+3\cdot 10^{-11}x-1$.
\end{enumeratea}
\end{esercizio}

%\subsubsection*{13.4 - Differenza di due quadrati}
\begin{esercizio}
\label{ese:13.47}
Scomponi i seguenti polinomi come differenza di quadrati.
\begin{multicols}{3}
\begin{enumeratea}
 \item $a^{2}-25b^{2}$;
 \item $16-x^{2}y^{2}$;
 \item $25-9x^{2}$;
 \item $4a^{4}-9b^{2}$;
 \item $x^{2}-16y^{2}$;
 \item $144x^{2}-9y^{2}$.
\end{enumeratea}
\end{multicols}
\end{esercizio}
\pagebreak
\begin{esercizio}
\label{ese:13.48}
Scomponi i seguenti polinomi come differenza di quadrati.
\begin{multicols}{3}
\begin{enumeratea}
 \item $16x^{4}-81z^{2}$;
 \item $a^{2}b^{4}-c^{2}$;
 \item $4x^{6}-9y^{4}$;
 \item $-36x^{8}+25b^{2}$;
 \item $-1+a^{2}$;
 \item $\dfrac{1}{4}x^{4}-\dfrac{1}{9}y^{4}$.
\end{enumeratea}
\end{multicols}
\end{esercizio}

\begin{esercizio}
\label{ese:13.49}
Scomponi i seguenti polinomi come differenza di quadrati.
\begin{multicols}{3}
\begin{enumeratea}
 \item $\dfrac{a^{2}}{4}-\dfrac{y^{2}}{9}$;
 \item $2a^{2}-50$;
 \item $a^{3}-16{ab}^{6}$;
 \item $-4x^{2}y^{2}+y^{2}$;
 \item $-4a^{2}+b^{2}$;
 \item $25x^{2}y^{2}-\dfrac{1}{4}z^{6}$.
\end{enumeratea}
\end{multicols}
\end{esercizio}

\begin{esercizio}
\label{ese:13.50}
Scomponi i seguenti polinomi come differenza di quadrati.
\begin{multicols}{3}
\begin{enumeratea}
 \item $-a^{2}b^{4}+49$;
 \item $16y^{4}-z^{4}$;
 \item $a^{8}-b^{8}$;
 \item $a^{4}-16$;
 \item $16a^{2}-9b^{2}$;
 \item $9-4x^{2}$.
\end{enumeratea}
\end{multicols}
\end{esercizio}

\begin{esercizio}
\label{ese:13.51}
Scomponi i seguenti polinomi come differenza di quadrati.
\begin{multicols}{3}
\begin{enumeratea}
 \item $\dfrac{1}{4}x^{2}-1$;
 \item $a^{2}-9b^{2}$;
 \item $\dfrac{25}{16}a^{2}-1$;
 \item $-16+25x^{2}$;
 \item $25a^{2}b^{2}-\dfrac{9}{16}y^{6}$;
 \item $-4x^{8}+y^{12}$.
\end{enumeratea}
\end{multicols}
\end{esercizio}

\begin{esercizio}[\Ast]
\label{ese:13.52}
Scomponi i seguenti polinomi come differenza di quadrati.
\begin{multicols}{3}
\begin{enumeratea}
 \item $\dfrac{1}{4}x^{2}-0,01y^{4}$;
 \item $x^{6}-y^{8}$;
 \item $x^{4}-y^{8}$;
 \item $(b+3)^{2}-x^{2}$;
 \item $a^{8}-(b-1)^{2}$;
 \item $(x-1)^{2}-a^{2}$.
\end{enumeratea}
\end{multicols}
\end{esercizio}

\begin{esercizio}[\Ast]
\label{ese:13.53}
Quando è possibile, scomponi in fattori, riconoscendo la differenza di due quadrati.
\begin{multicols}{3}
\begin{enumeratea}
 \item $(x-y)^{2}-(y+z)^{2}$;
 \item $-(2a-1)^{2}+(3b+3)^{2}$;
 \item $x^{2}-b^{2}-9-6b$;
 \item $(2x-3)^{2}-9y^{2}$;
 \item $(x+1)^{2}-(y-1)^{2}$;
 \item $x^{2}+2x+1-y^{2}$.
\end{enumeratea}
\end{multicols}
\end{esercizio}

\begin{esercizio}[\Ast]
\label{ese:13.54}
Quando è possibile, scomponi in fattori, riconoscendo la differenza di due quadrati.
\begin{multicols}{3}
\begin{enumeratea}
 \item $b^{2}-x^{4}+1+2b$;
 \item $a^{4}+4a^{2}+4-y^{2}$;
 \item $x^{2}-y^{2}-1+2y$;
 \item $(2x+3)^{2}-(2y+1)^{2}$;
 \item $a^{2}-2{ab}+b^{2}-4$;
 \item $(2x-3a)^{2}-(x-a)^{2}$.
\end{enumeratea}
\end{multicols}
\end{esercizio}

\begin{esercizio}[\Ast]
\label{ese:13.55}
Quando è possibile, scomponi in fattori, riconoscendo la differenza di due quadrati.
\begin{multicols}{2}
\begin{enumeratea}
 \item $-(a+1)^{2}+9$;
 \item $16x^{2}y^{6}-(xy^{3}+1)^{2}$;
 \item $a^{2}+1+2a-9$;
 \item $x^{2}y^{4}-z^{2}+9+6xy^{2}$;
 \item $a^{2}-6a+9-x^{2}-16-8x$;
 \item $x^{2}+25+10x-y^{2}+10y-25$.
\end{enumeratea}
\end{multicols}
\end{esercizio}
\pagebreak
\begin{esercizio}[\Ast]
\label{ese:13.56}
Quando è possibile, scomponi in fattori, riconoscendo la differenza di due quadrati.
\begin{multicols}{2}
\begin{enumeratea}
 \item $(a-1)^{2}-(a+1)^{2}$;
 \item $a^{2n}-4$;
 \item $a^{2m}-b^{2n}$;
 \item $x^{2n}-y^{4}$.
\end{enumeratea}
\end{multicols}
\end{esercizio}

\subsubsection*{13.4 - Altre tecniche di scomposizione}

\begin{esercizio}
 \label{ese:13.57}
 Scomponi in fattori i seguenti trinomi particolari.
 \begin{multicols}{3}
 \begin{enumeratea}
 \item $x^{2}-5x-36$;
 \item $x^{2}-17x+16$;
 \item $x^{2}-13x+12$;
 \item $x^{2}+6x+8$;
 \item $x^{2}+7x+12$;
 \item $x^{2}-2x-3$.
 \end{enumeratea}
\end{multicols}
\end{esercizio}

\begin{esercizio}
 \label{ese:13.58}
 Scomponi in fattori i seguenti trinomi particolari.
 \begin{multicols}{3}
 \begin{enumeratea}
 \item $x^{2}+9x+18$;
 \item $x^{2}-5x+6$;
 \item $x^{2}-8x-9$;
 \item $x^{2}-7x+12$;
 \item $x^{2}-6x+8$;
 \item $x^{2}-51x+50$.
 \end{enumeratea}
\end{multicols}
\end{esercizio}

\begin{esercizio}
 \label{ese:13.59}
 Scomponi in fattori i seguenti trinomi particolari.
 \begin{multicols}{3}
 \begin{enumeratea}
 \item $x^{2}-3x-4$;
 \item $x^{2}+5x-14$;
 \item $x^{4}+8x^{2}+12$;
 \item $x^{2}+4x-12$;
 \item $x^{2}-3x+2$;
 \item $x^{4}-5x^{2}+4$.
 \end{enumeratea}
\end{multicols}
\end{esercizio}

\begin{esercizio}
 \label{ese:13.60}
 Scomponi in fattori i seguenti trinomi particolari.
 \begin{multicols}{3}
 \begin{enumeratea}
 \item $x^{2}+3x-10$;
 \item $x^{2}+13x+12$;
 \item $x^{2}+2x-35$;
 \item $x^{6}-5x^{3}+4$;
 \item $x^{2}+5x-36$;
 \item $x^{2}+8x+7$.
 \end{enumeratea}
\end{multicols}
\end{esercizio}

\begin{esercizio}
 \label{ese:13.61}
 Scomponi in fattori i seguenti trinomi particolari.
 \begin{multicols}{3}
 \begin{enumeratea}
 \item $x^{2}-10x+24$;
 \item $y^{2}+y-20$;
 \item $x^{2}+4x-45$;
 \item $x^{2}-4x-21$;
 \item $x^{2}+4x-21$;
 \item $x^{2}-10x+21$.
 \end{enumeratea}
\end{multicols}
\end{esercizio}

\begin{esercizio}
 \label{ese:13.62}
 Scomponi in fattori i seguenti trinomi particolari.
 \begin{multicols}{3}
 \begin{enumeratea}
 \item $x^{4}+9x^{2}-10$;
 \item $x^{6}-x^{3}-30$;
 \item $-x^{6}+7x^{3}-10$;
 \item $2x^{3}+14x^{2}+20x$;
 \item $-3x^{6}+15x^{4}-12x^{2}$;
 \item $x^{4}-37x^{2}+36$.
 \end{enumeratea}
\end{multicols}
\end{esercizio}

\begin{esercizio}
 \label{ese:13.63}
 Scomponi in fattori i seguenti trinomi particolari.
 \begin{multicols}{3}
 \begin{enumeratea}
 \item $x^{20}+4x^{12}-32x^{4}$;
 \item $x^{40}-x^{20}-20$;
 \item $x^{14}-37x^{7}+36$;
 \item $x^{2}+4xy-32y^{2}$;
 \item $a^{2}-ax-20x^{2}$;
 \item $a^{2}-12xa-64x^{2}$.
 \end{enumeratea}
\end{multicols}
\end{esercizio}

\begin{esercizio}
 \label{ese:13.64}
 Scomponi in fattori i seguenti trinomi particolari.
 \begin{multicols}{3}
 \begin{enumeratea}
 \item $m^{2}+20mn+36n^{2}$;
 \item $x^{4}-8x^{2}a+12a^{2}$;
 \item $x^{6}+9x^{3}y^{2}-36y^{4}$;
 \item $x^{2}y^{2}-2xy-35$;
 \item $a^{4}b^{2}-a^{2}b-72$;
 \item $x^{4}+11x^{2}+24$.
 \end{enumeratea}
\end{multicols}
\end{esercizio}
\pagebreak
\begin{esercizio}[\Ast]
 \label{ese:13.65}
 Scomponi i seguenti polinomi seguendo la
traccia.
 \begin{enumeratea}
 \item $2x^{{2}}-3x-5=2x^{2}+2x-5x-5=\dotfill~$;
 \item $3y^{{2}}+y-10 = 3y^{2}+6y-5y-10 =\dotfill$;
 \item $5t^{{2}}-11t+2 = 5t^{2}-10t-t+2 =\dotfill$;
 \item $-3t^{{2}}+4t-1= -3t^{2}+3t+t-1 = \dotfill$;
 \item $2x^{2}-3x-9= 2x^{2}-6x+3x-9 = \dotfill$
 \end{enumeratea}
\end{esercizio}

\begin{esercizio}
 \label{ese:13.66}
 Scomponi i seguenti polinomi.
\begin{multicols}{3}
 \begin{enumeratea}
 \item $3a^{{2}}-4a+1$;
 \item $11k-6k^{2}+7$;
 \item $4b^{{2}}-4b-3$;
 \item $6x^{2}-13x-15$;
 \item $x^{2}+10ax+16a^{2}$;
 \item $2x^{{4}}+x^{{2}}-3$.
 \end{enumeratea}
\end{multicols}
 \end{esercizio}

%\subsubsection*{13..2 - Scomposizione con la regola Ruffini}

\begin{esercizio}
\label{ese:13.67}
Scomponi in fattori i seguenti polinomi utilizzando il teorema di
Ruffini.
\begin{multicols}{3}
 \begin{enumeratea}
 \item $2x^{2}-5x+2$;
\item $3x^{2}-5x-2$;
\item $x^{3}-4x^{2}+x+6$;
\item $x^{3}+2x^{2}-9x-18$;
\item $2x^{3}-3x^{2}-8x+12$;
\item $x^{4}-x^{3}-5x^{2}-x-6$.
 \end{enumeratea}
\end{multicols}
\end{esercizio}

\begin{esercizio}[\Ast]
\label{ese:13.68}
Scomponi in fattori i seguenti polinomi utilizzando il teorema di
Ruffini.
\begin{multicols}{2}
 \begin{enumeratea}
\item $x^{3}+2x^{2}-2x+3$;
\item $x^{3}+x^{2}-5x+3$;
\item $2x^{3}-9x^{2}+7x+6$;
\item $3x^{3}+5x^{2}-16x-12$;
\item $2x^{3}+5x^{2}+5x+3$;
\item $2x^{3}-13x^{2}+24x-9$;
\item $6x^{3}-11x^{2}-3x+2$;
\item $4x^{4}-4x^{3}-25x^{2}+x+6$.
 \end{enumeratea}
\end{multicols}
\end{esercizio}

\begin{esercizio}[\Ast]
\label{ese:13.69}
Scomponi in fattori i seguenti polinomi utilizzando il teorema di
Ruffini.
\begin{multicols}{2}
 \begin{enumeratea}
 \item $x^{3}-9x-9+x^{2}$;
\item $m^{3}+2m^{2}-m-2$;
\item $a^{3}+a^{2}-4a-4$;
\item $3a^{2}+a-2$;
\item $6a^{3}-a^{2}-19a-6$;
\item $x^{3}-5x^{2}+8x-4$;
\item $3t^{3}-t^{2}-12t+4$;
\item $3x^{4}+x^{3}-29x^{2}-17x+42$;
\item $y^{4}+y^{3}-3y^{2}-4y-4$;
\item $t^{4}-8t^{2}-24t-32$.
 \end{enumeratea}
\end{multicols}
\end{esercizio}

\begin{esercizio}[\Ast]
\label{ese:13.70}
Scomponi in fattori i seguenti polinomi utilizzando il teorema di
Ruffini.
\begin{multicols}{2}
 \begin{enumeratea}
 \item $2x^{5}+16x^{4}+25x^{3}-34x^{2}-27x+90$;
\item $x^{5}-x^{4}-4x^{3}-5x^{2}-9x+18$;
\item $x^{4}+2x^{3}-3x^{2}-4x+4$;
\item $a^{5}+3a^{4}-2a^{3}-9a^{2}-11a-6$;
\item $2x^{5}+16x^{4}+19x^{3}-94x^{2}-213x-90$;
\item $6x^{2}-7x+2$;
\item $3x^{3}+x^{2}+x-2$;
\item $2x^{3}+x^{2}+2x+1$;
\item $3x^{3}+9x-x^{2}-3$;
\item $1+5x+6x^{2}+4x^{3}+8x^{4}$.
 \end{enumeratea}
\end{multicols}
\end{esercizio}

\begin{esercizio}[\Ast]
\label{ese:13.71}
Scomponi in fattori i seguenti polinomi utilizzando il teorema di
Ruffini.

 \begin{enumeratea}
 \item $a^{6}+6a^{4}+11a^{2}+6$. \emph{Suggerimento}: sostituisci~$a^{2}=x$;
\item $2x^{2n}+x^{n}-3$. \emph{Suggerimento}:~$x^{n}=a$;
\item $x^{3}-ax^{2}-2ax+2a^{2}$ \emph{Suggerimento}: cerca le radici tra i
monomi divisori di~$2a^{2}$.
 \end{enumeratea}
\end{esercizio}
\pagebreak
%\subsubsection*{13..3 - Somma e differenza di due cubi}
\begin{esercizio}
\label{ese:13.72}
 Scomponi in fattori tenendo presente la somma e la differenza di cubi.
 \begin{multicols}{3}
 \begin{enumeratea}
  \item $x^{3}-1$;
\item $27-x^{3}$;
\item $x^{3}+1$;
\item $x^{3}+8$;
\item $64a^{3}-8b^{3}$;
\item $8x^{3}-27y^{3}$;
\item $\np{0,001}^{3}-x^{3}$;
\item $10^{-3}x^{3}-10^{3}y^{3}$;
\item $x^{6}-y^{6}$;
\item $\dfrac{1}{8}a^{3}-\dfrac{1}{27}b^{3}$.
 \end{enumeratea}
 \end{multicols}
\end{esercizio}

\begin{esercizio}
\label{ese:13.73}
 Scomponi in fattori tenendo presente la somma e la differenza di cubi.
 \begin{multicols}{3}
 \begin{enumeratea}
  \item $27x^{3}-8y^{3}$;
 \item $a^{3}b^{3}-1$;
 \item $a^{9}-1$;
 \item $a^{6}-1$;
 \item $\dfrac{27}{8}x^{3}-8$;
 \item $a^{3}-125$;
 \item $\np{0,064}x^{3}+\dfrac{1}{27}y^{3}$;
 \item $\dfrac{1}{8}a^{3}-\dfrac{1}{27}t^{3}$;
 \item $x^{6}-y^{3}$;
 \item $x^{9}+27y^{3}$.
 \end{enumeratea}
 \end{multicols}
\end{esercizio}

\begin{esercizio}
\label{ese:13.74}
 Scomponi in fattori tenendo presente la somma e la differenza di cubi.
 \begin{multicols}{3}
 \begin{enumeratea}
 \item $8x^{12}-1$;
 \item $a^{300}+1$;
\item $5x^{4}y^{3}+\dfrac{625}{8}x$;
 \item $a^{3n}-8b^{3}$;
 \item $a^{3n+3}+1$;
 \item $\dfrac{5}{8}a^{4}-\dfrac{5}{27}ab^{3}$.
 \end{enumeratea}
 \end{multicols}
\end{esercizio}

\begin{esercizio}[\Ast]
\label{ese:13.75}
Scomponi in fattori riconoscendo i prodotti notevoli.
\begin{enumeratea}
 \item $\dfrac{16}{9}-\dfrac{48}{3}a+36a^{2}$;
 \item $9a^{2}-4ab+\dfrac{4}{9}b^{2}$;
 \item $a^{2}b^{4}-\dfrac{2}{3}a^{2}b^{2}c+\dfrac{1}{9}a^{2}c^{2}$;
 \item $\dfrac{4}{9}x^{2}y^{4}+25x^{6}+\dfrac{20}{3}x^{4}y^{2}$;
 \item $27a^{3}-18a^{2}b+4ab^{2}-\dfrac{8}{27}b^{3}$.
\end{enumeratea}
\end{esercizio}

\begin{esercizio}[\Ast]
\label{ese:13.76}
Scomponi in fattori riconoscendo i prodotti notevoli.
\begin{multicols}{2}
\begin{enumeratea}
 \item $\dfrac{64}{27}-216a^{3}-32a+144a^{2}$;
 \item $a^{3}b^{6}-a^{3}b^{4}c-\dfrac{1}{27}a^{3}c^{3}+\dfrac{1}{3}a^{3}b^{2}c^{2}$;
 \item $\dfrac{x^{6}}{9}-\dfrac{4}{25}x^{4}y^{2}$;
 \item $(3a+b)^2-(a+2b)^2$.
\end{enumeratea}
\end{multicols}
\end{esercizio}

\begin{esercizio}[\Ast]
\label{ese:13.77}
Scomponi in fattori riconoscendo i prodotti notevoli.
\begin{multicols}{2}
\begin{enumeratea}
 \item $4x^{2}y^{4}-\left(x+2xy^{2}\right)^2$;
 \item $a^{2}+2ab+b^{2}-m^{2}$;
 \item $4+9a^{2}-4b^{2}-12a-c^{2}-4bc$;
 \item $27a^{3}-8b^{6}$.
\end{enumeratea}
\end{multicols}
\end{esercizio}

\begin{esercizio}[\Ast]
 \label{ese:13.78}
 Scomponi in fattori.
 \begin{multicols}{2}
 \begin{enumeratea}
 \item $(x+1)^{2}-(y-1)^{2}$;
\item $5x^{4}y^{2}+5x^{2}y+\dfrac{5}{4}$;
\item $(y-1)^{2}-2y+2$;
\item $4-(y-1)^{2}$;
\item $4x^{2}-xy-4x+y$;
\item $\np{0,}\overline{{3}}a^{2}-\dfrac{1}{3}b^{2}$;
\item $3x+k+3x^{2}+kx$;
\item $x^{3}+3x-4x^{2}$;
\item $4x^{2}-7x-2$;
\item $6x^{2}-24xy+24y^{2}$.
 \end{enumeratea}
 \end{multicols}
\end{esercizio}

\begin{esercizio}[\Ast]
 \label{ese:13.79}
 Scomponi in fattori.
 \begin{multicols}{2}
 \begin{enumeratea}
\item $x^{2}-(2+a)x+2a$;
\item $2x^{2}+5x-12$;
\item $\dfrac{1}{16}a^{2}+4b^{4}-ab^{2}$;
\item $81a-16a^{3}b^{2}$;
\item $a^{2}-10a-75$;
\item $ax+bx-3ay-3by$;
\item $x^{5}+x^{3}+x^{2}+1$;
\item $\np{0,09}x^{4}y^{5}-\np{0,04}y$;
\item $-a^{2}x-2{abx}-b^{2}x+5a^{2}+10{ab}+5b^{2}$;
\item $\dfrac{1}{9}x^{2}-\np{0,25}b^{2}$.
 \end{enumeratea}
 \end{multicols}
\end{esercizio}

\begin{esercizio}[\Ast]
 \label{ese:13.80}
 Scomponi in fattori.
 \begin{multicols}{2}
 \begin{enumeratea}
\item $8a^{3}-\frac{1}{8}b^{3}$;
\item $4a^{3}+8a^{2}-a-2$;
\item $x^{3}-x^{4}+8-8x$;
\item $4xy+4{xz}-3{ya}-3{za}-{yh}-{zh}$;
\item $x^{6}-81x^{2}$;
\item $54a^{3}b-2b^{4}$;
\item $-12{xyz}+9{ya}+6x^{3}a-8x^{4}z$;
\item $y^{2}+{ay}-6a^{2}$;
\item $2x^{3}+4x-3x^{2}-6$;
\item $(x^{2}-7x+10)^{2}-x^{2}+10x-25$.
 \end{enumeratea}
 \end{multicols}
\end{esercizio}

\begin{esercizio}[\Ast]
 \label{ese:13.81}
 Scomponi in fattori.
 \begin{multicols}{2}
 \begin{enumeratea}
  \item $\dfrac{4}{9}a^{2}-b^{2}+\dfrac{2}{3}a+b$;
\item $x^{2}-6x+9-(y^{2}-2y+1)$;
\item $16a^{4}x^{2}-8a^{2}b^{2}x^{2}+b^{4}x^{2}$;
\item $4(x-1)^{2}-4y(x-1)+y^{2}$;
\item $4a^{4}b-4a^{3}b^{2}+6a^{3}b^{3}-6a^{2}b^{4}$;
\item $8x^{3}-14x^{2}+7x-1$;
\item $x^{4}-3x^{3}-10x^{2}+24x$;
\item $81a^{4}-64a^{2}b^{2}$;
\item $4x^{3}+8x^{2}+x-3$;
 \item $2a^{4}b^{3}c-8a^{2}bc^{5}$.
 \end{enumeratea}
 \end{multicols}
\end{esercizio}

\begin{esercizio}[\Ast]
 \label{ese:13.82}
 Scomponi in fattori.
 \begin{multicols}{2}
 \begin{enumeratea}
  \item $x^{3}+2x^{2}-x-2$;
\item $20x^{3}-45x$;
\item $18p^{3}q^{2}x-2pq^{4}x+18p^{3}q^{2}y-2pq^{4}y$;
\item $20a^{6}-16a^{3}c-25a^{4}b+20abc$;
\item $2a^{7}-6a^{4}x^{2}+6a^{4}b^{2}-18ab^{2}x^{2}$;
\item $x^{3}-6x^{2}y+12xy^{2}-8y^{3}$;
\item $3x^{5}+12x^{4}-21x^{3}-66x^{2}+72x$;
\item $32a^{3}x^{2}y-48a^{3}xy^{2}+4b^{3}x^{2}y-6b^{3}xy^{2}$;
\item $x^{5}+3x^{4}-xy^{4}-3y^{4}$;
\item $48a^{5}bx+16a^{5}by-6a^{2}b^{4}x-2a^{2}b^{4}y$.
 \end{enumeratea}
 \end{multicols}
\end{esercizio}


\begin{esercizio}[\Ast]
 \label{ese:13.83}
 Scomponi in fattori.
 \begin{multicols}{2}
 \begin{enumeratea}
  \item $x^{2}(x^{4}-18x^{2}+81)-x^{6}+729$;
\item $x^{5}-2x^{2}-x+2$;
\item $x^{8}-y^{8}-2x^{6}y^{2}+2x^{2}y^{6}$;
\item $16ab-81a^{5}b^{9}$;
\item $6x^{7}+2x^{6}-16x^{5}+8x^{4}$;
\item $x^{4}-4x^{2}-45$;
\item $-3a^{7}x^{2}+9a^{5}x^{4}-9a^{3}x^{6}+3ax^{8}$;
\item $x^{3}-13x^{2}+35x+49$;
\item $4ab^{3}c^{2}+20ab^{3}-3abc^{2}-15ab$;
\item $6a^{6}b^{3}-12a^{4}b^{5}+6a^{2}b^{7}$.
 \end{enumeratea}
 \end{multicols}
\end{esercizio}
\pagebreak
\begin{esercizio}[\Ast]
 \label{ese:13.84}
 Scomponi in fattori.
 \begin{multicols}{2}
 \begin{enumeratea}
  \item $y^{3}-5y^{2}-24y$;
\item $x^{2}+4xy-6x+4y^{2}-12y+9$;
\item $2x^{4}-4x^{3}+4x^{2}-4x+2$;
\item $x^{2}-y^{2}+2{ay}-a^{2}$;
\item $\left(3-a\right)^{2}+\left(5+a\right)\cdot \left(a-3\right)$;
\item $3x^{{3}}-x-1+3x^{2}$;
\item $x^{3}y^{2}-x^{2}y^{3}+\dfrac{1}{4}xy^{4}$;
\item $-27x^{6}+9x^{5}-x^{4}+\dfrac{x^{3}}{27}$;
\item $4x^{2}-9y^{2}-6{yz}^{2}-z^{4}$;
\item $\dfrac{1}{8}a^{4}b^{2}-\dfrac{3}{4}a^{3}b^{3}+\frac{3}{2}a^{2}b^{4}-{ab}^{5}$.
 \end{enumeratea}
 \end{multicols}
\end{esercizio}

\begin{esercizio}[\Ast]
 \label{ese:13.85}
 Scomponi in fattori.
\begin{multicols}{2}
 \begin{enumeratea}
\item $a^{2}+4{ab}+4b^{2}-x^{2}+2xy-y^{2}$;
\item $\frac{a^{2}}{4}+2{ab}-16b^{4}+4b^{2}$;
\item $3a^{4}-3a^{3}x+a^{2}x^{2}-\frac{1}{9}ax^{3}$;
\item $a^{3}x+4a^{2}x+4ax$;
\item $a^{3}b^{5}-\frac{2}{3}a^{2}b^{6}+\frac{1}{9}ab^{7}$;
\item $a^{2}-{ab}-9a+3b+18$;
\item $8{ab}^{2}-2a^{3}$;
\item $a^{4}-6a^{3}+3a^{2}+18a+9-1$;
\item $a^{3}+3a^{2}b+a^{2}+3{ab}^{2}+2{ab}+b^{3}+b^{2}$;
\item $\frac{x^{7}}{3}+x^{5}+x^{3}+\frac{x}{3}$.
 \end{enumeratea}
\end{multicols}
\end{esercizio}

\subsubsection*{13.5 - MCD e mcm tra polinomi}

\begin{esercizio}[\Ast]
\label{ese:13.86}
Calcola il~$\mcd$ e il~$\mcm$ dei seguenti gruppi di polinomi.
\begin{multicols}{2}
\begin{enumeratea}
 \item $a+3$, $5a+15$, $a^{2}+6a+9$;
 \item $a^{2}-b^{2}$, $ab-b^{2}$, $a^{2}b-2ab^{2}+b^{3}$.
\end{enumeratea}
\end{multicols}
\end{esercizio}

\begin{esercizio}[\Ast]
\label{ese:13.87}
Calcola il~$\mcd$ e il~$\mcm$ dei seguenti gruppi di polinomi.
\begin{multicols}{2}
\begin{enumeratea}
 \item $x^{2}-5x+4$, $x^{2}-3x+2$, $x^{2}-4x+3$;
 \item $x^{2}+2x-2$, $x^{2}-4x+4$, $x^{2}-4$.
\end{enumeratea}
\end{multicols}
\end{esercizio}

\begin{esercizio}[\Ast]
\label{ese:13.88}
Calcola il~$\mcd$ e il~$\mcm$ dei seguenti gruppi di polinomi.
\begin{enumeratea}
 \item $a^{3}b^{2}-2a^{2}b^{3}$, $a^{3}b-4a^{2}b^{2}+4ab^{3}$, $a^{3}b^{2}-4ab^{4}$;
 \item $x^{3}+2x^{2}-3x$, $x^{3}-x$, $x^{2}-2x+1$.
\end{enumeratea}
\end{esercizio}

\begin{esercizio}[\Ast]
\label{ese:13.89}
Calcola il~$\mcd$ e il~$\mcm$ dei seguenti gruppi di polinomi.
\begin{multicols}{2}
\begin{enumeratea}
 \item $a-b$, $ab-a^{2}$, $a^{2}-b^{2}$;
 \item $b+2a$, $b-2a$, $b^{2}-4a^{2}$, $b^{2}-4a+4a^{2}$.
\end{enumeratea}
\end{multicols}
\end{esercizio}

\begin{esercizio}[\Ast]
\label{ese:13.90}
Calcola il~$\mcd$ e il~$\mcm$ dei seguenti gruppi di polinomi.
\begin{multicols}{2}
\begin{enumeratea}
 \item $a^{2}-9$, $3a-a^{2}$, $3a+a^{2}$;
 \item $a+1$, $a^{2}-1$, $a^{3}+1$.
\end{enumeratea}
\end{multicols}
\end{esercizio}

\begin{esercizio}[\Ast]
\label{ese:13.91}
Calcola il~$\mcd$ e il~$\mcm$ dei seguenti gruppi di polinomi.
\begin{multicols}{2}
\begin{enumeratea}
 \item $x^{2}+2xy+y^{2}$, $x^{2}-y^{2}$, $(x+y)^{2}(x-y)$;
 \item $b^{3}+b^{2}-4b-4$, $b^{2}-a$, $b^{2}-1$.
\end{enumeratea}
\end{multicols}
\end{esercizio}

\begin{esercizio}[\Ast]
\label{ese:13.92}
Calcola il~$\mcd$ e il~$\mcm$ dei seguenti gruppi di polinomi.
\begin{enumeratea}
 \item $a-2$, $a^{2}-9$, $a^{2}+a-6$;
 \item $3x+y+3x^{2}+xy$, $9x^{2}-1$, $9x^{2}+6xy+y^{2}$.
\end{enumeratea}
\end{esercizio}
\pagebreak
\begin{esercizio}[\Ast]
\label{ese:13.93}
Calcola il~$\mcd$ e il~$\mcm$ dei seguenti gruppi di polinomi.
\begin{enumeratea}
 \item $2x^{3}-12x^{2}y+24xy^{2}-16y^{3}$, $6x^{2}-12xy$, $4x^{3}-16x^{2}y+16xy^{2}$;
 \item $x-1$, $x^{2}-2x+1$, $x^{2}-1$.%trovato risultato
\end{enumeratea}
\end{esercizio}

\begin{esercizio}
\label{ese:13.94}
Calcola il~$\mcd$ e il~$\mcm$ dei seguenti gruppi di polinomi.
\begin{multicols}{2}
\begin{enumeratea}
 \item $x^{3}-9x+x^{2}$, $4-(x-1)^{2}$, $x^{2}+4x+3$;
 \item $x-2$, $x-1$, $x^{2}-3x+2$;
 \item $a^{2}-1$, $b+1$, $a+ab-b-1$;
 \item $x$, $2x^{2}-3x$, $4x^{2}-9$.
\end{enumeratea}
\end{multicols}
\end{esercizio}

\begin{esercizio}
\label{ese:13.95}
Calcola il~$\mcd$ e il~$\mcm$ dei seguenti gruppi di polinomi.
\begin{multicols}{2}
\begin{enumeratea}
 \item $x-1$, $x^{2}-1$, $x^{3}-1$;
 \item $y^{3}+8a^{3}$, $y+2a$, $y^{2}-2ay+4a^{2}$;
 \item $z-5$, $2z-10$, $z^{2}-25$, $z^{2}+25+10z$;
 \item $a^{2}-2a+1$, $a^{2}-3a+2$, $1-a$.
\end{enumeratea}
\end{multicols}
\end{esercizio}

\begin{esercizio}
\label{ese:13.96}
Calcola il~$\mcd$ e il~$\mcm$ dei seguenti gruppi di polinomi.
\begin{enumeratea}
 \item $2x$, $3x-2$, $3x^{2}-2x$, $10x^{2}$;
 \item $a^{2}-a$, $a^{2}+a$, $a-a^{2}$, $2a^{2}-2$;
 \item $x-2$, $x^{2}-4$, $ax+2a-3x-6$, $a^{2}-6a+9$;
 \item $x^{2}-a^{2}$, $x+a$, $x^{2}+ax$, $ax+a^{2}$;
 \item $x^{2}-4x+4$, $2x-x^{2}$, $x^{2}-2x$, $x^{3}$, $x^{3}-2x^{2}$.
\end{enumeratea}
\end{esercizio}

\subsection{Esercizi riepilogativi}

\begin{esercizio}[\Ast]
 \label{ese:13.97}
 Scomponi in fattori.
 \begin{enumeratea}
\item $a^{4}b-2a^{3}b^{2}+4a^{3}{bc}+a^{2}b^{3}-4a^{2}b^{2}c+4a^{2}bc^{2}$;
\item $5a^{4}x^{3}-40a^{4}y^{3}-45a^{2}b^{2}x^{3}+360a^{2}b^{2}y^{3}$;
\item $-24a^{4}b^{2}x^{2}-72a^{4}b^{2}y^{2}-3ab^{5}x^{2}-9ab^{5}y^{2}$;
\item $20ab^{2}c+8abc+2abc^{2}+2a^{2}bc^{2}+2a^{2}b^{2}c$;
\item $640a^{3}x^{2}y-960a^{3}xy^{2}+10b^{3}x^{2}y-15b^{3}xy^{2}$;
\item $-4x-3-2(x+1)(16x^{2}+9+24x)$;
\item $(x-2)+3(x^{2}-4x+4)-(x+1)(x-2)^{2}$;
\item $(x-1)^{2}-(x+2)(x^{2}-2x+1)-2(x^{3}-3x^{2}+3x-1)$;
\item $x+1-2(x^{2}+2x+1)+(3x^{2}+x^{3}+3x+1)(x-2)$;
\item $(y-x)^{2}(3x+2)-2(x-y)^{3}-2x^{2}+2y^{2}$.
 \end{enumeratea}
\end{esercizio}

\begin{esercizio}[\Ast]
 \label{ese:13.98}
 Scomponi in fattori.
\begin{multicols}{2}
 \begin{enumeratea}
\item $(-x^{2}+6x-9)^{2}-(4x-12)(x+1)$;
\item $(3x+6)-5(x^{2}+4x+4)^{2}$;
\item $36x^{2}+24xy-48x+4y^{2}-16y+15$;
\item $x^{5}-2-x+2x^{4}$;
\item $6a^{3}+11a^{2}+3a$;
\item $3a^{4}-24ax^{3}$;
\item $x^{2}-2x+1$;
\item $x^{2}+y^{2}+z^{4}-2xy+2{xz}^{2}-2{yz}^{2}$;
\item $a^{6}+b^{9}+3a^{4}b^{3}+3a^{2}b^{6}$;
\item $a^{3}-6a^{2}+12a-8$.
 \end{enumeratea}
\end{multicols}
\end{esercizio}
\pagebreak
\begin{esercizio}
 \label{ese:13.99}
 Scomponi in fattori.
 \begin{multicols}{2}
 \begin{enumeratea}
 \item $a^{2}+b^{2}-1-2{ab}$;
\item $a^{4}+2b-1-b^{2}$;
\item $-8a^{2}b+24{ab}^{2}-18b^{3}$;
\item $6a^{5}-24{ab}^{4}$;
\item $a^{4}+b^{4}-2a^{2}b^{2}$;
\item $x^{6}-9x^{4}y+27x^{2}y^{2}-27y^{3}$;
\item $x^{2}-12x+32$;
\item $x^{2}-8x+15$;
\item $x^{4}-7x^{2}-60$;
\item $x^{3}-5x^{2}+6x$.
 \end{enumeratea}
 \end{multicols}
\end{esercizio}

\begin{esercizio}
 \label{ese:13.100}
 Scomponi in fattori.
 \begin{multicols}{3}
 \begin{enumeratea}
  \item $4a^{2}-9-4b^{2}+12b$;
\item $x^{5}-13x^{3}+36x$;
\item $4a^{2}+4a+1$;
\item $4x^{2}y^{2}-4xy+1$;
\item $x^{3}+1$;
\item $a^{2}+6a+9$;
\item $12xy-16y^{2}$;
\item $2x^{3}-16$;
\item $2x^{2}+4x+8$;
\item $ax^{2}-{ay}^{2}$.
 \end{enumeratea}
 \end{multicols}
\end{esercizio}

\begin{esercizio}
 \label{ese:13.101}
 Scomponi in fattori.
 \begin{multicols}{2}
 \begin{enumeratea}
  \item $a^{3}-8+12a-6a^{2}$;
\item $7t^{2}-28$;
\item $2x^{2}+8+8x$;
\item $25+9x^{2}+30x$;
\item $z^{{8}}-2z^{{4}}+1$;
\item $3k^{{4}}+k^{{6}}+1+3k^{2}$;
\item $3x^{5}-27xy^{4}$;
\item $25y^{4}-10y^{2}+1$;
\item $8a^{4}b-8a^{3}b^{2}+12a^{3}b^{3}-12a^{2}b^{4}$;
\item $3a^{3}x+3a^{3}y-3abx-3aby$.
 \end{enumeratea}
 \end{multicols}
\end{esercizio}

\begin{esercizio}
 \label{ese:13.102}
 Scomponi in fattori.
 \begin{multicols}{2}
 \begin{enumeratea}
  \item $81a^{6}b^{3}-a^{2}b^{3}$;
\item $6{abx}-3x+2{aby}-y$;
\item $x^{3}+6x^{2}y+12xy^{2}+8y^{3}$;
\item $8a^{7}b-8a^{3}b^{3}+12a^{6}b-12a^{2}b^{3}$;
\item $4a^{2}x-4a^{2}y^{2}-4ab^{2}x+4ab^{2}y^{2}$;
\item $a^{2}+12a+36$;
\item $x^{8}-y^{8}-2x^{6}y^{2}+2x^{2}y^{6}$;
\item $5x^{4}-5x^{2}y^{4}$;
\item $(2x-1)^{3}-(3-6x)^{2}$;
\item $x^{4}-2x^{3}+6x^{2}y+x^{2}-6xy+9y^{2}$.
 \end{enumeratea}
 \end{multicols}
\end{esercizio}

\begin{esercizio}
 \label{ese:13.103}
 Scomponi in fattori.
 \begin{multicols}{2}
 \begin{enumeratea}
\item $x^{2}+10xy+25y^{2}$;
\item $27a^{6}-54a^{4}b+36a^{2}b^{2}-8b^{3}$;
\item $64a^{9}-48a^{6}b^{2}+12a^{3}b^{4}-b^{6}$;
\item $4a^{2}x^{2}-4b^{2}x^{2}-9a^{2}y^{2}+9b^{2}y^{2}$;
\item $x^{6}-6x^{4}+12x^{2}-8$;
\item $a^{7}-a^{4}b^{2}-4a^{3}b^{2}+4b^{4}$;
\item $x^{4}+6x^{2}-40$;
\item $x^{5}-13x^{3}+12x^{2}$;
\item $32ab-2a^{5}b^{5}$;
\item $24x^{4}y+36x^{3}y^{3}+18x^{2}y^{5}+3xy^{7}$.
 \end{enumeratea}
 \end{multicols}
\end{esercizio}

\begin{esercizio}
 \label{ese:13.104}
 Scomponi in fattori.
 \begin{multicols}{2}
 \begin{enumeratea}
  \item $\dfrac{4}{9}a^{4}+\dfrac{4}{9}a^{2}b+\dfrac{b^{2}}{9}$;
\item $-2a^{10}+12a^{7}b-24a^{4}b^{2}+16{ab}^{3}$;
\item $x^{3}-7x^{2}-25x+175$;
\item $2ab^{6}+54a^{4}+18a^{2}b^{4}+54a^{3}b^{2}$;
\item $128a^{3}-200a$;
\item $\dfrac{4}{25}+\dfrac{4}{5}xy+x^{2}y^{2}$;
\item $x{4}-6x^{2}-27$;
\item $x^{4}+4x^{3}+x^{2}-6x$;
\item $8a^{5}b^{2}-64a^{2}b^{5}$;
\item $4a^{2}b^{5}-81b$.
 \end{enumeratea}
 \end{multicols}
\end{esercizio}
%\pagebreak
\begin{esercizio}
 \label{ese:13.105}
 Scomponi in fattori.
 \begin{multicols}{2}
 \begin{enumeratea}
  \item $ax + bx - 3ay - 3by $;
\item $2ax^{2} + 8ay^{2} + 8axy$;
\item $81a^{4} - b^{4}$;
\item $3a^{5}b^{3} + 24a^{2}b^{9}$;
\item $4x^{2} + 2xy +\dfrac{1}{4}y^{2}$;
\item $x^{2} - 3a^{3} + ax - 3a^{2}x $;
\item $x^{2}-12x+133$;
\item $3x^{5} - 27xy^{4}$;
\item $25y^{4} - 10y^{2}+1$;
\item $\dfrac{16}{27}x^{3}+\dfrac{8}{3}x^{2}y+4xy^{2}+2y^{3}$.
 \end{enumeratea}
 \end{multicols}
\end{esercizio}

\begin{esercizio}
 \label{ese:13.106}
 Scomponi in fattori.
 \begin{multicols}{2}
 \begin{enumeratea}
  \item $1 - 9x + 27x^{2} - 27x^{3}$;
\item $6x^{3}y-12x^{2}y^{2}+6xy^{3}$;
\item $x^{4} + 3x^{2} - 28 $;
\item $2x^{3} - 3x^{2} - 5x + 6$;
\item $3x^{4}y^{3} + 9x^{4} - 9xy^{3} - 27x$;
\item $81a^{6} - 18a^{4}b^{2} + a^{2}b^{2}$;
\item $125 + 75y + 15y^{2} + y^{3}$;
\item $4a^{2}x^{2} - 16a^{2}y^{2} - b^{2}x^{2} + 4b^{2}y^{2}$;
\item $x^{4} + 2x^{2} - 24$;
\item $5x^{3} - 17x^{2} + 16x - 4$.
 \end{enumeratea}
 \end{multicols}
\end{esercizio}

\begin{esercizio}
 \label{ese:13.107}
 Scomponi in fattori.
 \begin{multicols}{2}
 \begin{enumeratea}
\item $27a^{6} - 54a^{4}b + 36a^{2}b^{2} - 8b^{3}$;
\item $18a^{4}b - 2b^{3}$;
\item $x^{4} - 9x^{2} + 20$;
\item $3a^{4}b^{3} - 6a^{3}b^{3} - 9a^{2}b^{3}$;
\item $\dfrac{1}{8}x^{6}-\dfrac{1}{4}x^{4}+\dfrac{1}{6}x^{2}-\dfrac{1}{27}$;
\item $4a^{5}b^{2} + 32a^{2}b^{5}$;
\item $32a - 50ab^{2}$;
\item $5x^{4} y^{2} + 5x^{4} - 5xy^{4} - 5xy^{2}$;
\item $4y^{2} - 12y + 9$;
\item $\dfrac{1}{4}x^{2}+\dfrac{1}{3}ax+\dfrac{1}{9}a^{2}$.
 \end{enumeratea}
 \end{multicols}
\end{esercizio}

\begin{esercizio}
 \label{ese:13.108}
 Scomponi in fattori.
 \begin{multicols}{2}
 \begin{enumeratea}
\item $\dfrac{8}{27}x^{3}-2x^{2}+\dfrac{9}{2}x-\dfrac{27}{8}$;
\item $\dfrac{1}{9}a^{6} + 9a^{2} - 2a^{4}$;
\item $5x^{4} - 5x^{3}y^{2} - 5x^{2}y + 5xy^{3}$;
\item $-8a^{3} + 12a^{2}x^{2} - 6ax^{4} + x^{6}$;
\item $x^{2}+14x-32$;
\item $\dfrac{4}{49}x^{2}y^{2}-\dfrac{4}{7}xyz+z^{2}$;
\item $1-\dfrac{3}{2}x^{3}+\dfrac{9}{16}x^{6}$;
\item $2b^{6}c - 8c^{3}$;
\item $16a^{4}x^{2} - 8a^{2}b^{2}x^{2} + b^{4}x^{2}$;
\item $4x^{3} + 7x^{2} - 14x + 3$.
 \end{enumeratea}
 \end{multicols}
\end{esercizio}

\begin{esercizio}
 \label{ese:13.109}
 Scomponi in fattori.
 \begin{multicols}{2}
 \begin{enumeratea}
\item $x^{4} - 4x^{2} - 45$;
\item $3x^{3} + x^{2} - 8x + 4$;
\item $4a^{2} - 9 - 4b^{2} + 12b $;
\item $x^{3} + 3x^{2} - 6x - 8$;
\item $2ax^{2} + 8ay^{2} + 8axy$;
\item $x^{6} - 81x^{2} + x - 3$;
\item $x^{6} - y^{6} + x^{3} + y^{3}$;
\item $x^{2} - 3a^{3} + ax - 3a^{2}x$;
\item $50a^{4}b^{3} - 2b^{3}$;
\item $16x^{3}-72x^{2}+108x-54$.
 \end{enumeratea}
 \end{multicols}
\end{esercizio}

\begin{esercizio}
 \label{ese:13.110}
 Scomponi in fattori.
 \begin{multicols}{2}
 \begin{enumeratea}
\item $625a^{4} - b^{4}$;
\item $12ax^{2}+12{axy}+3{ay}^{2}$;
\item $x^{4} + 5x^{2} - 36$;
\item $-4x^{7} + 16x^{6} + 28x^{5} - 88x^{4} - 96x^{3}$;
\item $\dfrac{1}{9}x^6 - 2x^{4} + 9x^{2}$;
\item $a^{4} + 4a^{2} - 32$;
\item $4x^{3} + 7x^{2} - 14x + 3$;
\item $2ax^{4}y - 8bx^{4}y - 2axy^{4} + 8bxy^{4}$;
\item $36ab - 49a^{3}b^{3}$;
\item $\dfrac{4}{25}a^4+\dfrac{25}{9}b^2-\dfrac{4}{3}a^{2}b$.
 \end{enumeratea}
 \end{multicols}
\end{esercizio}

\begin{esercizio}[\Ast]
 \label{ese:13.111}
 Scomponi in fattori.
 \begin{multicols}{2}
 \begin{enumeratea}
\item $x^{4}+x^{3}-x^{2}-x$;
\item $20a^{3}b+45ab^{3}-60a^{2}b^{2}$;
\item $2a^{4}-3a^{3}b-2a+3b$;
\item $a^{5}+a^{4}-a-1$;
\item $x^{4}y-x^{2}y-a^{2}x^{2}y+a^{2}y$;
\item $6x^{5}y^{3}-12x^{3}y^{5}+6xy^{7}$.
 \end{enumeratea}
 \end{multicols}
\end{esercizio}

\begin{esercizio}
 \label{ese:13.112}
 Scomponi in fattori.
 \begin{multicols}{2}
 \begin{enumeratea}
\item $t^{5}-z^{5}$;
\item $3x^{2}+6x+6$;
\item $t^{{6}}-2t^{{3}}+1$;
\item $tx+x^{2}+y^{2}+ty+2xy$;
\item $12m^{3}+9m^{5}-3m^{7}$;
\item $a^{2}b-25b+a^{2}-25$;
\item $2ab-b^{2}+3\cdot \left(b-2a\right)^{2}$;
\item $x^{{6}}-y^{{6}}$;
\item $3k^{{3}}-k^{2}+k+5$;
\item $y^{{6}}+y^{{3}}-2$.
 \end{enumeratea}
 \end{multicols}
\end{esercizio}

\begin{esercizio}
 \label{ese:13.113}
 Scomponi in fattori.
\begin{multicols}{2}
 \begin{enumeratea}
\item $a^{{8}}-1$;
\item $32a^{4}b^{3} - 2b^{3}$;
\item $x^{6} - 8a^{3} + 12a^{2}x^{2} - 6ax^{4}$;
\item $x^{2} - 3a^{3} + ax - 3a^{2}x$;
\item $9y^{2}+6y+1$;
\item $9a^{3}-9$;
\item $a^{3}+4a-2a^{2}-3$;
\item $3a+2a^{3}-7a^{2}$;
\item $50a^{3}b^{2}-8a^{5}$;
\item $2ax^{4}y-6bx^{4}y-2axy^{4}+6bxy^{4}$.
 \end{enumeratea}
\end{multicols}
\end{esercizio}

\begin{esercizio}[\Ast]
 \label{ese:13.114}
 Scomponi in fattori.
 \begin{multicols}{2}
 \begin{enumeratea}
\item $4y^{5}-5y^{3}-6y^{2}-2y+3$;
\item $12a^{3}-3a+1-4a^{2}$;
\item $x^{6}y+y^{7}$;
\item $11ab^{2}x^{3}-11ab^{2}$;
\item $\dfrac{16}{9}a^{8}b^{8}-\dfrac{1}{9}a^{4}c^{8}$;
\item $(2x+5)^{2}-16$;
\item $\left(x^{2}-3y^{2}\right)^{2}-\left(2x^{2}-y^{2}\right)^{2}$;
\item $(a+3)(2a-10)-\left(a^{2}-25\right)+(a-5)^2$;
\item $18x^{3}-\dfrac{4}{3}x^{2}y+\dfrac{2}{81}xy^{2}$;
\item $a^{6}+4a^{3}-5$.
 \end{enumeratea}
 \end{multicols}
\end{esercizio}

\begin{esercizio}[\Ast]
 \label{ese:13.115}
 Scomponi in fattori.
 \begin{multicols}{2}
 \begin{enumeratea}
\item $a^{4}-5a^{2}+4$;
\item $5x^{4}y^{4}-10x^{2}y^{2}+5$;
\item $x^{4}+3x^{2}y^{2}+4y^{4}$;
\item $8x^{3}+2x^{2}-3x$;
\item $a^{4}-a^{3}+a^{2}-4a+3$;
\item $z^{5}-6z^{3}+6z^{2}-7z+6$;
\item $4a^{2}b+\frac{1}{3}ab^{2}-2b^{3}$;
\item $(x+2y)^{2}-5(x+2y)-24$;
\item $y^{2}-(2a-b)y-2ab$;
\item $y^{4}+(a-4)y^{2}-4a$.
 \end{enumeratea}
 \end{multicols}
\end{esercizio}
\pagebreak
\begin{esercizio}[\Ast]
 \label{ese:13.116}
 Scomponi in fattori.
 \begin{multicols}{2}
 \begin{enumeratea}
\item $3+3a^{2}-3y^{2}-6a$;
\item $a^{3}-a^{2}-a+1$;
\item $8x^{3}-y^{3}-4x+2y$;
\item $8x^{5}-8x^{3}y^{2}-8x^{3}y-2x^{3}$;
\item $x+xy-z+y^{2}z$;
\item $2xy+16-x^{2}-y^{2}$;
\item $a^{2}-b^{2}+a^{2}b^{2}-1$;
\item $2ax+3by-2bx-3ay$;
\item $4x^{2}+y^{2}+4x-2y$;
\item $ax^{3}+a+x^{2}-1$.
 \end{enumeratea}
 \end{multicols}
\end{esercizio}

\begin{esercizio}[\Ast]
 \label{ese:13.117}
 Scomponi in fattori.
 \begin{multicols}{2}
 \begin{enumeratea}
\item $8x^{3}-y^{3}+2xy^{2}-4x^{2}y$;
\item $x^{4}-4x^{2}-a^{2}x^{2}+4a^{2}$;
\item $3m^{3}-3m+2m^{2}-2$;
\item $x^{3}-4x^{2}+x+6$;
\item $ax+ay+x^{2}+2xy+y^{2}$;
\item $x^{2}+2ax^{2}-3ax-6a^{2}x+2a^{3}+a^{2}$;
\item $x^{6}-27+26x^{3}$;
\item $4a^{2}+9b^{2}+c^{2}+12ab-4ac-6bc$;
\item $a^{2}-b^{2}-x^{2}+1-2a+2bx$;
\item $2x^{4}+5x^{3}-5x^{2}-5x+3$.
 \end{enumeratea}
 \end{multicols}
\end{esercizio}

\begin{esercizio}[\Ast]
 \label{ese:13.118}
 Scomponi in fattori.
 \begin{multicols}{2}
 \begin{enumeratea}
\item $a^{4}-2a^{3}b+3a^{2}b^{2}-2ab^{3}+b^{4}$;
\item $2a^{4}+16a^{2}+15a^{3}-15a-18$;
\item $x^{4}-x^{3}+x^{2}-3x-6$;
\item $x^{3}y+7x^{2}y+12xy-2x^{2}-14x-24$;
\item $x^{3}-3x-2$;
\item $x^{3}-(2a-b)x^{2}-2abx$.
 \end{enumeratea}
 \end{multicols}
\end{esercizio}

\begin{esercizio}
 \label{ese:13.119}
 Scomponi in fattori.
 \begin{enumeratea}
\item $ab^{4}-\dfrac{1}{3}a^{2}b^{2}-b^{6}+\dfrac{1}{27}a^{3}$;
\item $-a^{4}b^{4}-3a^{3}b^{3}xy-3a^{3}b^{2}x^{2}y^{2}-abx^{3}y^{3}$;
\item $(a+2)\left(a^{3}-8\right)+\left(a^{3}+8\right)(a-2)$;
\item $(x-y)^{2}+2(x-y)(3a+b)+(3a+b)^{2}$;
\item $a^{8}b^{8}-2a^{6}b^{6}-2a^{5}b^{5}+a^{4}b^{4}+2a^{3}b^{3}+a^{2}b^{2}$;
\item $4y^{2}-12x^{2}y+25x^{2}y^{2}-20xy^{2}+9x^{4}+30x^{3}y$;
\item $\frac{1}{8}-8x^{3}y^{3}+6x^{2}y^{2}+\frac{3}{2}xy$;
\item $4xy(a-3b)+2xy^{2}a-6xy^{2}b-2x^{2}y(3b-a)$;
\item $x^{2}-4x-5xy+x^{2}y+6y+4$;
\item $x^{6}-8-7x^{3}$.
 \end{enumeratea}
\end{esercizio}

\begin{esercizio}[\Ast]
 \label{ese:13.120}
 Scomponi in fattori.
 \begin{multicols}{2}
 \begin{enumeratea}
  \item $x^{a+1}-5x^{a}-4x^{a-2}$;
\item $x^{n^{2}-1}+2x^{n^{2}+2}+x^{n^{2}}(x-3)$;
\item $x^{4n+1}-x^{3n+1}y^{n}+2x^{n}y^{4n}-2y^{5n}$;
\item $x^{n+2}+3x^{n}y^{2n}-x^{2}y^{3}-3y^{3+2n}$;
\item $x^{a}y^{b}+x^{a}-y^{b}-1$;
\item $x^{2n+1}y^{h+1}-2x^{2n+1}-y^{h+1}+2$;
\item $x^{a+4}-3x^{a+2}y^{a}+x^{2}y^{2}-3y^{2+a}$.
 \end{enumeratea}
 \end{multicols}
\end{esercizio}
\newpage
\subsection{Risposte}

\paragraph{13.2.}
a)~$ax(3a-b+1)$,\quad b)~$3b(7ax+2ab+5b+4c)$, \quad c)~$5xy(5xy+3x-2)$,\protect\\ d)~$-3a^{3}b^{3}\left(4a^{5}b^{6}+5a+2\right)$,\quad e)~$2b^{2}(a+c-a^{2}-c^{2})$, \quad f)~$2m^{5}\left(m+2\right)^{2}$.

\paragraph{13.3.}
a)~$3bx(3x+6b+2)$,\quad b)~$5a^{4}\left(3a^{3}+4a+2\right)$, \quad c)~$-x^{2}\left(x^{3}+4b^{2}x-b\right)$.

\paragraph{13.9.}
a)~$a^{n-2}\left(a^{2}+a+1\right)$,\quad c)~$a^{n}\left(1+a^{n}+a^{2n}\right)$.

\paragraph{13.10.}
a)~$(x+y)^{2}(x+y-1)$,\quad b)~$a^{n}(1+a+a^{2})$,\quad c)~$(a+2)\left(a^{2}+3a+1\right)$, \protect\\ d)~$(x-2)\left(3x^2-7x+2a+2\right)$,\quad e)~$x^{2}(a+b)(ax^{3}+bx^{3}+x+a^{2}+2ab+b^{2})$,\protect\\ f)~$(x+y)\left(5x+3y-6\right)$.

\paragraph{13.11.}
e)~$x^{2}(a+b)(5x^{3}-2x+3)$,\quad f)~$(2x-y)\left(2x-y-5x^3-12x^2y+12xy^2-3y^3\right)$.

\paragraph{13.12.}
a)~$(x-y)(2+a)$,\quad b)~$(x-2)(3a+1)$, \quad c)~$(a+b)(x-y)$.

\paragraph{13.13.}
b)~$(3x-3)\left(x^2+1\right)$,\quad c)~$(x-1)\left(x^{2}+1\right)$, \quad d)~$(a-1)\left(y-2x^{3}\right)$.

\paragraph{13.14.}
f)~$(y+1)(x+a+b)$.

\paragraph{13.17.}
a)~$(b-2)(bx+y)$,\quad b)~$\frac{1}{3}\left(x^{2}+3\right)(2x-1)$,\quad c)~$(a+b+2)(x-1)$,\protect\\ d)~$(x+y)(3x+3y+5)$,\quad e)~$(b+1)(x^{2}-x+1)$,\quad f)~$\left(a^{2}-b\right)\left(a-b^{2}\right)$.

\paragraph{13.18.}
a)~$\left(\frac{3}{5}ab-1\right)\left(\frac{1}{3}a+5b\right)$,\quad f)~$(a-2)(a-3+b)$,\quad g)~$(x+4y)\left(\frac{1}{8}x^2-2y^2\right)$,\protect\\ h)~$\left(a-x^2\right)\left(b-\frac{2}{3}x\right)$.

\paragraph{13.19.}
a)~$\left(2x^2-y\right)(5x-6)$, \quad b)~$(3a^2-b^3)(2a+b)$, \quad c)~$2^{11}(x+2)(x+16)$,\quad d)~$-3x(x+y)\left(3x^2+3xy-1\right)$.

\paragraph{13.20.}
a)~$a^{8}\left(a^{2}-2\right)\left(a^{4}+4\right)$,\quad b)~$x^{2}(x+y)(3x+3y+5)$, \quad c)~$ay(x+1)(x^{2}+1)$.

\paragraph{13.22.}
a)~$\left(15x+7y^{2}\right)\left(3x^{2}+y+5xy\right)$,\quad b)~$\frac{1}{3}a(x^{2}+1)(2x-1)$,\quad c)~$\frac{1}{9}x(x-y)(16+x)$, \protect\\ d)~$2b(x+1)(x-a+3)$.

\paragraph{13.31.}
a)~$6a(2a+1)^{2}$,\quad b)~$3x(a-2b)^{2}$, \quad c)~$\dfrac{1}{5}(x+5a)^{2}$, \quad d)~$x^{2}y\left(x^{2}+1\right)^{2}$, \quad e)~$x^{3}(x+2)^{2}$,\quad f)~$2y(3x-y)^{2}$.

\paragraph{13.32.}
a)~$-2t(5t-2)^{2}$, \quad b)~$\left(2^{5}x+3^{20}\right)^{2}$, \quad c)~$2^{20} x^{40}\left(1-2^{5}x^{10} \right)^2$, \quad d)~$10^{50}\left(10^{25} x^{25}-1 \right)^2$ \quad e)~$10^{6} \left(10^{5} x^{10}-2 \cdot 10^{3}x^{5}+1\right)$,\quad f)~$\left(x^{n}+1\right)^2$.

\paragraph{13.36.}
a)~$(a+2b-1)^{2}$,\quad b)~$(ab+a+2b)^{2}$, \quad c)~$(x-3y+3)^{2}$.

\paragraph{13.44.}
a)~$\left(a^{2}+b^{2}\right)^{3}$,\quad b)~$(2a-3b)^{3}$,\quad c)~$a^{3}(a+1)^{3}$,\quad d)~$a\left(a^3-2\right)^3$.

\paragraph{13.52.}
d)~$(b+3-x)(b+3+x)$,\quad e)~$(a^{4}-b+1)(a^{4}+b-1)$, \quad f)~$(x+a-1)(x-a-1)$.

\paragraph{13.53.}
d)~$(2x+3y-3)(2x-3y-3)$,\quad e)~$(x+y)(x-y+2)$, \quad f)~$(x+y+1)(x-y+1)$.

\paragraph{13.54.}
d)~$4(x+y+2)(x-y+1)$,\quad e)~$(a-b-2)(a-b+2)$, \quad f)~$(3x-4a)(x-2a)$.

\paragraph{13.55.}
e)~$-(x+a+1)(x-a+7)$,\quad f)~$(x+y)(x-y+10)$.

\paragraph{13.56.}
b)~$\left(a^{n}-2\right)\left(a^{n}+2\right)$,\quad d)~$\left(x^{n}-y^{2}\right)\left(x^{n}+y^{2}\right)$.

\paragraph{13.65.} a)~$(x+1)(2x-5)$,\quad b)~$(y+z)(3y-5)$,\quad e)~$(x-3)\left(2x+3\right)$.

\paragraph{13.68.} a)~$(x+3)\left(x^{2}-x+1\right)$,\quad b)~$(x-1)^{2}(x+3)$,\quad
c)~$(x-2)(x-3)(2x+1)$,\protect\\
d)~$(x-2)(x+3)(3x+2)$,\quad
e)~$(2x+3)\left(x^{2}+x+1\right)$,\quad f)~$(x-1)(x-2)^{2}$,\quad
f)~$(x-3)^{2}(2x-1)$,\protect\\
g)~$(x-2)(2x+1)(3x-1)$,\quad
h)~$(x+2)(x-3)(2x+1)(2x-1)$.

\paragraph{13.69.} a)~$(x+1)(x+3)\left(x-3\right)$,\quad b)~$(m-1)(m+1)\left(m+2\right)$,\quad
c)~$(a+1)(a-2)\left(a+2\right)$,\protect\\
d)~$(a+1)\left(3a-2\right)$,\quad
e)~$(a-2)(3a+1)\left(2a+3\right)$,\quad f)~$(x-1)(x-2)^{2}$,\quad
g)~$(t+2)(t-2)\left(3t-1\right)$,\quad h)~$(x-3)(x-1)(x+2)(3x+7)$,\quad
i)~$(y+2)(y-2)\left(y^{2}+y+1\right)$,\quad
\protect\\j)~$(t+2)(t-4)\left(t^{2}+2t+4\right)$.

\paragraph{13.70.} a)~$(x+2)(x+3)(x+5)\left(2x^{2}-4x+3\right)$,\quad b)~$(x+2)(x-3)(x-1)\left(x^{2}+x+3\right)$,\quad
\protect\\ c)~$(x-1)^{2}\left(x+2\right)^{2}$,\quad d)~$(a+1)(a-2)(a+3)(a^{2}+a+1)$,\quad
e)~$(x+2)(x+3)(x+5)(2x^{2}-4x-3)$,\quad f)~$(2x-1)(3x-2)$,\quad
g)~$(3x-2)\left(x^{2}+x+1\right)$,\quad h)~$(2x+1)\left(x^{2}+1\right)$,\quad
i)~$(3x-1)\left(x^{2}+3\right)$.

\paragraph{13.71.} a)~$(a^{2}+1)(a^{2}+2)(a^{2}+3)$,\quad b)~$(x^{n}-1)(2x^{n}+3)$,\quad c)~$(x-a)\left(x^{2}-2a\right)$.

\paragraph{13.75.} a)~$\left(\frac{4}{3}-6a\right)^{2}$,\quad
b)~$\left(3a-\frac{2}{3}b\right)^{2}$,\quad
c)~$\left(ab-\frac{1}{3}ac\right)^{2}$,\quad
d)~$\left(\frac{2}{3}xy^{2}+5x^{3}\right)^{2}$,\quad
e)~$\left(3a-\frac{2}{3}b\right)^{3}$.

\paragraph{13.76.} a)~$\left(\frac{4}{3}-6a\right)^{3}$,\quad
b)~$\left(ab^{2}-\frac{1}{3}ac\right)^{3}$,\quad
c)~$\left(\frac{x^{3}}{3}+\frac{2}{5}x^{2}y\right)\left(\frac{x^{3}}{3}-\frac{2}{5}x^{2}y\right)$,\quad
d)~$(4a+3b)(2a-b)$.

\paragraph{13.77.} a)~$-x\left(4xy^{2}+x\right)$,\quad
b)~$(a+b+m)(a+b-m)$,\quad
c)~$(2-3a+2b+c)(2-3a-2b-c)$,\quad
d)~$\left(3a-2b^{2}\right)\left(9a^{2}+6ab^{2}+4b^{4}\right)$.

\paragraph{13.78.} a)~$(x+y)\left(x-y+2\right)$,\quad
b)~$5\left(\frac{1}{2}+x^{2}y\right)^{2}$,\quad
c)~$(y-1)\left(y-3\right)$,\quad
d)~$(y+1)\left(3-y\right)$,\protect\\
e)~$(x-1)\left(4x-y\right)$,\quad
f)~$\frac{1}{3}(a+b)\left(a-b\right)$,\quad
g)~$(x+1)\left(3x+k\right)$,\quad h)~$x(x-1)\left(x-3\right)$,\protect\\
i)~$(x-2)\left(4x+1\right)$,\quad j)~$6\left(x-2y\right)^{2}$.

\paragraph{13.79.} a)~$(x-2)\left(x-a\right)$,\quad b)~$(x+4)\left(2x-3\right)$,\quad
c)~$\left(\frac{1}{4}a-2b^{2}\right)^{2}$,\quad d)~$a(9-4{ab})(9+4{ab})$,\quad \protect\\
e)~$(a-15)(a+5)$,\quad f)~$(a+b)(x-3y)$,\quad
g)~$(x+1)\left(x^{2}+1\right)\left(x^{2}-x+1\right)$,\protect\\
h)~$\frac{1}{100}y\left(3x^{2}y^{2}+2\right)\left(3x^{2}y^{2}-2\right)$,\quad
i)~$(a+b)^{2}\left(5-x\right)$,\quad j)~$\frac{1}{36}(2x+3b)\left(2x-3b\right)$.

\paragraph{13.80.} a)~$\left(2a-\frac{1}{2}b\right)\left(4a^{2}+{ab}+\frac{1}{4}b^{2}\right)$,\quad b)~$(a+2)\left(2a+1\right)\left(2a-1\right)$,\quad \protect\\
c)~$(1-x)\left(x+2\right)\left(x^{2}-2x+4\right)$,\quad d)~$(y+z)(4x-3a-h)$,\quad
e)~$x^{2}(x+3)(x-3)\left(x^{2}+9\right)$,\protect\\
f)~$2b(3a-b)\left(9a^{2}+3{ab}+b^{2}\right)$,\quad
g)~$(3a-4{xz})\left(2x^{3}+3y\right)$,\quad h)~$(y-2a)\left(y+3a\right)$,\quad \protect\\
i)~$\left(x^{2}+2\right)(2x-3)$,\quad j)~$(x-5)^{2}(x-1)(x-3)$.

\paragraph{13.81.} a)~$\left(\frac{2}{3}a+b\right)\left(\frac{2}{3}a-b+1\right)$,\quad b)~$(x-4+y)(x-2-y)$,\quad
c)~$x^{2}(2a-b)^{2}(2a+b)^{2}$,\protect\\
d)~$(2x-2-y)^{2}$,\quad
e)~$2a^{2}b(2a+3b^{2})(a-b)$,\quad f)~$(x-1)(2x-1)(4x-1)$,\quad \protect\\
g)~$x(x-2)(x+3)(x-4)$,\quad h)~$a^{2}(9a-8b)(9a+8b)$,\quad
i)~$(2x+3)(2x-1)(x+1)$,\protect\\
j)~$2a^{2}{bc}({ab}-2c^{2})({ab}+2c^{2})$.

\paragraph{13.82.} a)~ $(x-1)(x+2)(x+1)$,\quad b)~$5x(2x-3)(2x+3)$,\quad
c)~$2{pq}^{2}(3p-q)(3p+q)(x+y)$,\protect\\
d)~$a(4a^{2}-5b)(5a^{3}-4c)$,\quad e)~$2a(a^{3}+3b^{2})(a^{3}-3x^{2})$,\quad f)~$(x-2y)^{3}$,\quad \protect\\
g)~$3x(x-1)(x-2)(x+3)(x+4)$,\quad h)~$2xy(2a+b)(2x-3y)(4a^{2}-2{ab}+b^{2})$,\quad \protect\\
i)~$(x+3)(x-y)(x+y)(x^{2}+y^{2})$,\quad j)~$2a^{2}b(2a-b)(3x+y)(4a^{2}+2{ab}+b^{2})$.

\paragraph{13.83.} a)~$-9(x+3)(x-3)(2x^{2}+9)$,\quad b)~$(x+1)(x-1)^{2}(x^{2}+x+2)$,\quad \protect\\
c)~$(x-y)^{3}(x+y)^{3}(x^{2}+y^{2})$,\quad d)~${ab}(2-3{ab}^{2})(2+3{ab}^{2})(4+9a^{2}b^{4})$,\quad \protect\\
e)~$2x^{4}(x-1)(x+2)(3x-2)$,\quad f)~$(x-3)(x+3)(x^{2}+5)$,\quad
g)~$3ax^{2}(x-a)^{3}(x+a)^{3}$,\protect\\
h)~$(x+1)(x-7)^{2}$,\quad
i)~${ab}(4b^{2}-3)(c^{2}+5)$,\quad j)~$6a^{2}b^{3}(a-b)^{2}(a+b)^{2}$.

\paragraph{13.84.} a)~$y(y+3)(y-8)$,\quad b)~ $(x+2y-3)^{2}$,\quad
c)~$2(x^{2}+1)(x-1)^{2}$,\quad d)~$(x-a+y)(x+a-y)$,\quad
e)~$2(a-3)(a+1)$,\quad f)~$(3x^{2}-1)(x+1)$,\quad
g)~$xy^{2}(x-\frac{1}{2}y)^{2}$,\quad h)~$x^{3}\left(\frac{1}{3}-3x\right)^{3}$,\protect\\
i)~$(2x+3y+z^{2})(2x-3y-z^{2})$,\quad j)~$\frac{1}{8}{ab}^{2}(a-2b)^{3}$.

\paragraph{13.85.} a)~$(a+2b+x-y)(a+2b-x+y)$,\quad 
b)~$\left(\frac{1}{2}a+2b-4b^{2}\right)\left(\frac{1}{2}a+2b+4b^{2}\right)$,\protect\\
c)~$3a\left(a-\frac{1}{3}x\right)^{3}$,\quad
d)~$ax(a+2)^{2}$,\quad
e)~${ab}^{5}\left({ab}-\frac{1}{3}b^{2}\right)^{2}$,\quad f)~$(a-3)(a-b-6)$,\protect\\
g)~$-2a(a+2b)(a-2b)$,\,
h)~$(a-4)(a+1)(a^{2}-3a-2)$,\,
i)~$(a+b)^{2}(a+b+1)$,\, j)~$\frac{1}{3}x(x^{2}+1)^{3}$.

\paragraph{13.86.}
a)~$(a+3)$, $5(a+3)^2$;\quad b)~$(a-b)$, $b(a+b)(a-b)^2$.

\paragraph{13.87.}
a)~$(x-1)$, $(x-1)(x-2)(x-3)(x-4)$;\quad b)~$1$, $(x-2)^2(x+2)\left(x^2+2x-2\right)$.

\paragraph{13.88.}
a)~$ab(a-2b)$, $a^2 b^2(a-2b)^2(a+2b)$;\quad b)~$(x-1)$, $x(x-1)^2(x+1)(x+3)$.

\paragraph{13.89.}
a)~$(a-b)$, $a(a-b)(a+b)$;\quad b)~$1$, $(b-2a)(b+2a)\left(b^2-4a+4a^2\right)$.

\paragraph{13.90.}
a)~$1$, $a(a-3)(a+3)$;\quad b)~$(a+1)$, $(a+1)(a-1)\left(a^2-a+1\right)$.

\paragraph{13.91.}
a)~$(x+y)$, $(x+y)^2(x-y)$;\quad b)~$1$, $(b-1)(b+1)(b-2)(b+2)\left(b^2-a\right)$.

\paragraph{13.92.}
a)~$1$, $(a-2)(a-3)(a+3)$;\quad b)~$1$, $(x+1)(3x-1)(3x+1)(3x+y)^2$.

\paragraph{13.93.}
a)~$2(x-2y)$, $12x(x-2y)^3$;\quad b)~$(x-1)$, $(x-1)^2(x+1)$.

\paragraph{13.97.} a)~$a^{2}b(a-b+2c)^{2}$,\quad 
b)~$5a^{2}(a-3b)(a+3b)(x-2y)(x^{2}+2xy+4y^{2})$,\protect\\
c)~$-3{ab}^{2}(2a+b)(x^{2}+3y^{2})(4a^{2}-2{ab}+b^{2})$,\quad %d)~$2xy(a-3b)(x-y)(x^{2}+xy+y^{2})$,\protect\\
e)~$5xy(4a+b)(2x-3y)(16a^{2}-4{ab}+b^{2})$,\protect\\
f)~$-(4x+3)(8x^{2}+14x+7)$,\quad
g)~$(x-1)(x-2)(3-x)$,\quad h)~$(x-1)^{2}(1-3x)$,\protect\\
i)~$(x+1)(x^{3}-5x-3)$,\quad
j)~$(x-y)(x^{2}+xy-4y-2y^{2})$.

\paragraph{13.98.} a)~$(x-3)(x^{3}-9x^{2}+23x-31)$,\quad b)~$-(2+x)(5x^{3}+30x^{2}+60x+37)$,\protect\\
c)~$(6x+2y-3)(6x+2y-5)$,\quad d)~$(x+2)\left(x^{2}+1\right)(x+1)(x-1)$,\quad
e)~$a(3a+1)(2a+3)$,\protect\\ f)~$3a(a-2x)\left(a^{2}+2ax+4x^{2}\right)$.

\paragraph{13.111.}
a)~$x(x+1)(x+1)^{2}$,\quad 
b)~$5ab(2a-3b)^{2}$,\quad 
c)~$(a-1)\left(a^{2}+a+1\right)(2a-3b)$,\protect\\
d)~$(a+1)^{2}(a-1)\left(a^{2}+1\right)$,\quad 
e)~$y(x+1)(x-1)(x+a)(x-a)$,\quad 
f)~$6xy^{3}(x+y)^{2}(x-y)^{2}$.

\paragraph{13.114.}
a)~$(y+1)(2y-1)(2y-3)\left(y^{2}+y+1\right)$,\quad 
b)~$(2a+1)(2a-1)(3a-1)$,\protect\\
c)~$y\left(x^{2}+y^{2}\right)\left(x^{4}-x^{2}y^{2}+y^{4}\right)$,\quad 
d)~$11ab^{2}(x-1)\left(x^{2}+x+1\right)$,\protect\\
e)~$\frac{1}{9}a^{4}\left(2ab^{2}-c^{2}\right)\left(2ab^{2}+c^{2}\right)\left(4a^{2}b^{4}+c^{4}\right)$,\quad 
f)~$(2x+9)(2x+1)$,\protect\\
g)~$-\left(x^{2}+2y^{2}\right)\left(3x^{2}-4y^{2}\right)$,\quad 
h)~$2(a-2)(a-5)$,\quad 
i)~$2x\left(3x-\frac{1}{9}y\right)^{2}$,\protect\\
j)~$(a-1)\left(a^{2}+a+1\right)\left(a^{3}+5\right)$.

\paragraph{13.115.}
a)~$(a-1)(a+1)(a-2)(a+2)$,\quad 
b)~$5(xy+1)^{2}(xy-1)^{2}$,\protect\\ 
c)~$\left(x^{2}+2y^{2}+xy\right)\left(x^{2}+2y^{2}-xy\right)$,\quad 
d)~$x(2x-1)(4x+3)$,\quad 
e)~$(a-1)\left(a^{3}+a-3\right)$,\protect\\
f)~$(z-1)(z-2)(z+3)\left(z^{2}+1\right)$,\quad 
g)~$\frac{1}{3}b(3a-2b)(4a+3b)$,\quad 
h)~$(x+2y-8)(x+2y+3)$,\protect\\
i)~$(y-2a)(y+b)$,\quad 
j)~$(y+2)(y-2)\left(y^{2}+a\right)$.

\paragraph{13.116.}
a)~$3(1-a-y)(1-a+y)$,\quad 
b)~$(a-1)^{2}(a+1)$,\quad 
c)~$(2x-y)\left(4x^{2}+2xy+y^{2}-2\right)$,\protect\\
d)~$2x^{3}(2x+2y+1)(2x-2y-1)$,\quad 
e)~$(y+1)(zy-z+a)$,\quad 
g)~$(a-1)(a+1)\left(b^{2}+1\right)$,\protect\\
h)~$(a-b)(2x-3y)$,\quad
i)~$(2x-y)(2x+y+2)$,\quad 
j)~$(x+1)\left(ax^{2}-ax+a+x-1\right)$.

\paragraph{13.117.}
a)~$(2x-y)\left(x^{2}+y^{2}\right)$,\quad 
b)~$(x+a)(x-a)(x+2)(x-2)$,\quad 
c)~$(m-1)(m+1)(3m+2)$,\quad 
d)~$(x+1)(x-2)(x-3)$,\quad 
e)~$(x+y)(x+y+a)$,\quad 
f)~$(2a+1)\left(x^{2}-3ax+a^{2}\right)$,\quad 
h)~$(2a+3b-c)^{2}$,\quad 
i)~$(a-1+b-x)(a-1-b+x)$,\quad 
j)~$(x-1)(x+1)(x+3)(2x-1)$.

\paragraph{13.118.}
a)~$\left(a^{2}+b^{2}-ab\right)^{2}$,\quad
b)~$(a+1)(a-1)(a+6)(2a+3)$,\quad
c)~$(x+1)(x-2)\left(x^{2}+3\right)$,\protect\\
d)~$(x+3)(x+4)(xy-2)$,\quad 
e)~$(x+1)^{2}(x-2)$,\quad
f)~$x(x-2a)(x+b)$.

\paragraph{13.120.} a)~$x^{a-2}(x^{3}-5x^{2}-4)$,\quad b)~$x^{n^{2}-1}(2x-1)(x^{2}+x-1)$,\quad
c)~$(x^{n}-y^{n})(x^{3n+1}+2y^{4n})$,\quad d)~$(x^{n}-y^{3})(x^{2}+3y^{2n})$,\quad
e)~$(x^{a}-1)(y^{b}+1)$,\quad f)~$(x^{2n+1}-1)(y^{1+h}-2)$,\protect\\
g)~$(x^{2+a}+y^{2})(x^{2}-3y^{a})$.


\cleardoublepage
