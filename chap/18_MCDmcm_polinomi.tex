% (c) 2012 Claudio Carboncini - claudio.carboncini@gmail.com
% (c) 2012-2014 Dimitrios Vrettos - d.vrettos@gmail.com

\chapter{MCD e mcm tra polinomi}

\section{Divisore comune e multiplo comune}

Il calcolo del \emph{minimo comune multiplo}~($\mcm$) e del \emph{massimo comune divisore}~($\mcd$) si estende anche ai polinomi.
Per determinare~$\mcd$ e~$\mcm$ di due o più polinomi occorre prima di tutto scomporli in fattori irriducibili.
La cosa non è semplice poiché non si può essere sicuri di aver trovato il massimo comune divisore o il minimo comune multiplo
per la difficoltà di decidere se un polinomio è irriducibile: prudentemente si dovrebbe parlare di divisore comune e di multiplo comune.

Un polinomio~$A$ si dice \emph{multiplo} di un polinomio~$B$ se esiste un polinomio~$C$ per il quale si ha~$A=B\cdot C$; in questo caso diremo
anche che~$B$ è \emph{divisore} del polinomio~$A$.

\section{Massimo Comune Divisore}
Dopo aver scomposto ciascun polinomio in fattori, il massimo comune divisore tra due o più polinomi è il prodotto di tutti i
fattori comuni ai polinomi, presi ciascuno una sola volta, con il minimo esponente.
Sia i coefficienti numerici, sia i monomi possono essere considerati polinomi.
\begin{procedura}
Calcolare il~$\mcd$ tra polinomi:
\begin{enumeratea}
\item scomponiamo in fattori ogni polinomio;
\item prendiamo i fattori comuni a tutti i polinomi una sola volta con l'esponente più piccolo;
\item se non ci sono fattori comuni a tutti i polinomi il~$\mcd$ è~$1$.
\end{enumeratea}
\end{procedura}

\begin{exrig}
 \begin{esempio}
Determinare il~$\mcd\left(3a^{2}b^{3}-3b^{3}\text{,~}6a^{3}b^{2}-6b^{2}\text{,~}2a^{2}b^{2}-24ab^{2}+12b^{2}\right)$.
 \begin{itemize*}
 \item Scomponiamo in fattori i singoli polinomi;
  \begin{itemize*}
  \item $3a^{2}b^{3}-3b^{3}=3b^{3}\left(a^{2}-1\right)=3b^{3}(a-1)(a+1)$;
  \item $6a^{3}b^{2}-6b^{2}=6b^{2}\left(a^{3}-1\right)=6b^{2}(a-1)\left(a^{2}+a+1\right)$;
  \item $12a^{2}b^{2}-24ab^{2}+12b^{2}=12b^{2}\left(a^{2}-2a+1\right)=12b^{2}(a-1)^{2}$.
  \end{itemize*}
 \item i fattori comuni a tutti i polinomi presi con l'esponente più piccolo sono:
  \begin{itemize*}
  \item tra i numeri il~$3$;
  \item tra i monomi~$b^{2}$;
  \item tra i polinomi~$a-1$.
  \end{itemize*}
 \item quindi il~$\mcd=3b^{2}(a-1)$.
 \end{itemize*}
 \end{esempio}
\end{exrig}

\section{Minimo comune multiplo}
Dopo aver scomposto ciascun polinomio in fattori, il minimo comune multiplo tra due o più polinomi è il prodotto dei fattori comuni
e non comuni di tutti i polinomi, quelli comuni presi una sola volta, con il massimo esponente.

\begin{procedura}
Calcolare il~$\mcm$ tra polinomi:
\begin{enumeratea}
\item scomponiamo in fattori ogni polinomio;
\item prendiamo tutti i fattori comuni e non comuni dei polinomi, i fattori comuni presi una sola
   volta con il massimo esponente.
\end{enumeratea}
\end{procedura}

\begin{exrig}
 \begin{esempio}
Determinare il~$\mcm\left(3a^{2}b^{3}-3b^{3}\text{,~}6a^{3}b^{2}-6b^{2}\text{,~}2a^{2}b^{2}-24ab^{2}+12b^{2}\right)$.
 \begin{itemize*}
 \item Scomponiamo in fattori i singoli polinomi;
  \begin{itemize*}
  \item $3a^{2}b^{3}-3b^{3}=3b^{3}\left(a^{2}-1\right)=3b^{3}(a-1)(a+1)$;
  \item $6a^{3}b^{2}-6b^{2}=6b^{2}\left(a^{3}-1\right)=6b^{2}(a-1)\left(a^{2}+a+1\right)$;
  \item $12a^{2}b^{2}-24ab^{2}+12b^{2}=12b^{2}\left(a^{2}-2a+1\right)=12b^{2}(a-1)^{2}$.
  \end{itemize*}
 \item i fattori comuni presi con il massimo esponente e quelli non comuni sono:
  \begin{itemize*}
  \item tra i coefficienti numerici il~$12$;
  \item tra i monomi~$b^{3}$;
  \item tra i polinomi~$(a-1)^{2}\cdot (a+1)\cdot \left(a^{2}+a+1\right)$.
  \end{itemize*}
 \item quindi il~$\mcm=12b^{3}(a-1)^{2}(a+1)\left(a^{2}+a+1\right)$.
 \end{itemize*}
 \end{esempio}
\end{exrig}

\ovalbox{\risolvii \ref{ese:18.1}, \ref{ese:18.2}, \ref{ese:18.3}, \ref{ese:18.4}, \ref{ese:18.5}, \ref{ese:18.6}, \ref{ese:18.7}, \ref{ese:18.8}, \ref{ese:18.9}, \ref{ese:18.10}, \ref{ese:18.11}}

\newpage
% (c) 2012 Claudio Carboncini - claudio.carboncini@gmail.com
\section{Esercizi}
\subsection{Esercizi dei singoli paragrafi}
\subsubsection*{18.1 - MCD e mcm tra polinomi}

\begin{esercizio}[\Ast]
\label{ese:18.1}
Calcola il~$\mcd$ e il~$\mcm$ dei seguenti gruppi di polinomi.
\begin{multicols}{2}
\begin{enumeratea}
 \item $a+3$, $5a+15$, $a^{2}+6a+9$;
 \item $a^{2}-b^{2}$, $ab-b^{2}$, $a^{2}b-2ab^{2}+b^{3}$.
\end{enumeratea}
\end{multicols}
\end{esercizio}

\begin{esercizio}[\Ast]
\label{ese:18.2}
Calcola il~$\mcd$ e il~$\mcm$ dei seguenti gruppi di polinomi.
\begin{multicols}{2}
\begin{enumeratea}
 \item $x^{2}-5x+4$, $x^{2}-3x+2$, $x^{2}-4x+3$;
 \item $x^{2}+2x-2$, $x^{2}-4x+4$, $x^{2}-4$.
\end{enumeratea}
\end{multicols}
\end{esercizio}

\begin{esercizio}[\Ast]
\label{ese:18.3}
Calcola il~$\mcd$ e il~$\mcm$ dei seguenti gruppi di polinomi.
\begin{enumeratea}
 \item $a^{3}b^{2}-2a^{2}b^{3}$, $a^{3}b-4a^{2}b^{2}+4ab^{3}$, $a^{3}b^{2}-4ab^{4}$;
 \item $x^{3}+2x^{2}-3x$, $x^{3}-x$, $x^{2}-2x+1$.
\end{enumeratea}
\end{esercizio}

\begin{esercizio}[\Ast]
\label{ese:18.4}
Calcola il~$\mcd$ e il~$\mcm$ dei seguenti gruppi di polinomi.
\begin{multicols}{2}
\begin{enumeratea}
 \item $a-b$, $ab-a^{2}$, $a^{2}-b^{2}$;
 \item $b+2a$, $b-2a$, $b^{2}-4a^{2}$, $b^{2}-4a+4a^{2}$.
\end{enumeratea}
\end{multicols}
\end{esercizio}

\begin{esercizio}[\Ast]
\label{ese:18.5}
Calcola il~$\mcd$ e il~$\mcm$ dei seguenti gruppi di polinomi.
\begin{multicols}{2}
\begin{enumeratea}
 \item $a^{2}-9$, $3a-a^{2}$, $3a+a^{2}$;
 \item $a+1$, $a^{2}-1$, $a^{3}+1$.
\end{enumeratea}
\end{multicols}
\end{esercizio}

\begin{esercizio}[\Ast]
\label{ese:18.6}
Calcola il~$\mcd$ e il~$\mcm$ dei seguenti gruppi di polinomi.
\begin{multicols}{2}
\begin{enumeratea}
 \item $x^{2}+2xy+y^{2}$, $x^{2}-y^{2}$, $(x+y)^{2}(x-y)$;
 \item $b^{3}+b^{2}-4b-4$, $b^{2}-a$, $b^{2}-1$.
\end{enumeratea}
\end{multicols}
\end{esercizio}

\begin{esercizio}[\Ast]
\label{ese:18.7}
Calcola il~$\mcd$ e il~$\mcm$ dei seguenti gruppi di polinomi.
\begin{enumeratea}
 \item $a-2$, $a^{2}-9$, $a^{2}+a-6$;
 \item $3x+y+3x^{2}+xy$, $9x^{2}-1$, $9x^{2}+6xy+y^{2}$.
\end{enumeratea}
\end{esercizio}

\begin{esercizio}[\Ast]
\label{ese:18.8}
Calcola il~$\mcd$ e il~$\mcm$ dei seguenti gruppi di polinomi.
\begin{enumeratea}
 \item $2x^{3}-12x^{2}y+24xy^{2}-16y^{3}$, $6x^{2}-12xy$, $4x^{3}-16x^{2}y+16xy^{2}$;
 \item $x-1$, $x^{2}-2x+1$, $x^{2}-1$.%trovato risultato
\end{enumeratea}
\end{esercizio}

\begin{esercizio}
\label{ese:18.9}
Calcola il~$\mcd$ e il~$\mcm$ dei seguenti gruppi di polinomi.
\begin{multicols}{2}
\begin{enumeratea}
 \item $x^{3}-9x+x^{2}$, $4-(x-1)^{2}$, $x^{2}+4x+3$;
 \item $x-2$, $x-1$, $x^{2}-3x+2$;
 \item $a^{2}-1$, $b+1$, $a+ab-b-1$;
 \item $x$, $2x^{2}-3x$, $4x^{2}-9$.
\end{enumeratea}
\end{multicols}
\end{esercizio}

\begin{esercizio}
\label{ese:18.10}
Calcola il~$\mcd$ e il~$\mcm$ dei seguenti gruppi di polinomi.
\begin{multicols}{2}
\begin{enumeratea}
 \item $x-1$, $x^{2}-1$, $x^{3}-1$;
 \item $y^{3}+8a^{3}$, $y+2a$, $y^{2}-2ay+4a^{2}$;
 \item $z-5$, $2z-10$, $z^{2}-25$, $z^{2}+25+10z$;
 \item $a^{2}-2a+1$, $a^{2}-3a+2$, $1-a$.
\end{enumeratea}
\end{multicols}
\end{esercizio}

\begin{esercizio}
\label{ese:18.11}
Calcola il~$\mcd$ e il~$\mcm$ dei seguenti gruppi di polinomi.
\begin{enumeratea}
 \item $2x$, $3x-2$, $3x^{2}-2x$, $10x^{2}$;
 \item $a^{2}-a$, $a^{2}+a$, $a-a^{2}$, $2a^{2}-2$;
 \item $x-2$, $x^{2}-4$, $ax+2a-3x-6$, $a^{2}-6a+9$;
 \item $x^{2}-a^{2}$, $x+a$, $x^{2}+ax$, $ax+a^{2}$;
 \item $x^{2}-4x+4$, $2x-x^{2}$, $x^{2}-2x$, $x^{3}$, $x^{3}-2x^{2}$.
\end{enumeratea}
\end{esercizio}

\subsection{Risposte}

\paragraph{18.1.}
a)~$(a+3)$, $5(a+3)^2$;\quad b)~$(a-b)$, $b(a+b)(a-b)^2$.

\paragraph{18.2.}
a)~$(x-1)$, $(x-1)(x-2)(x-3)(x-4)$;\quad b)~$1$, $(x-2)^2(x+2)\left(x^2+2x-2\right)$.

\paragraph{18.3.}
a)~$ab(a-2b)$, $a^2 b^2(a-2b)^2(a+2b)$;\quad b)~$(x-1)$, $x(x-1)^2(x+1)(x+3)$.

\paragraph{18.4.}
a)~$(a-b)$, $a(a-b)(a+b)$;\quad b)~$1$, $(b-2a)(b+2a)\left(b^2-4a+4a^2\right)$.

\paragraph{18.5.}
a)~$1$, $a(a-3)(a+3)$;\quad b)~$(a+1)$, $(a+1)(a-1)\left(a^2-a+1\right)$.

\paragraph{18.6.}
a)~$(x+y)$, $(x+y)^2(x-y)$;\quad b)~$1$, $(b-1)(b+1)(b-2)(b+2)\left(b^2-a\right)$.

\paragraph{18.7.}
a)~$1$, $(a-2)(a-3)(a+3)$;\quad b)~$1$, $(x+1)(3x-1)(3x+1)(3x+y)^2$.

\paragraph{18.8.}
a)~$2(x-2y)$, $12x(x-2y)^3$;\quad b)~$(x-1)$, $(x-1)^2(x+1)$.

\cleardoublepage
