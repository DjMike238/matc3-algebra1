% (c) 2012 -2014 Dimitrios Vrettos - d.vrettos@gmail.com
% (c) 2014 Claudio Carboncini - claudio.carboncini@gmail.com
\section{Esercizi}
\subsection{Esercizi dei singoli paragrafi}
\subsubsection*{1.4 - Operazioni con i numeri naturali}

\begin{esercizio}
\label{ese:1.1}
Rispondi alle seguenti domande:
 \begin{enumeratea}
 \item Esiste il numero naturale che aggiunto a~3 dà come somma~6?
 \item Esiste il numero naturale che aggiunto a~12 dà come somma~7?
 \item Esiste il numero naturale che moltiplicato per~4 dà come prodotto~12?
 \item Esiste il numero naturale che moltiplicato per~5 dà come prodotto~11?
 \end{enumeratea}
\end{esercizio}

\begin{esercizio}
\label{ese:1.2}
 Inserisci il numero naturale mancante, se esiste:
\begin{multicols}{3}
\begin{enumeratea}
 \item $7-\ldots =1$;
 \item$3-3=\ldots~$;
 \item$5-6=\ldots~$;
 \item $3-\ldots =9$;
 \item$15:5=\ldots~$;
 \item$18:\ldots =3$;
 \item $\ldots:4=5$;
 \item$12:9=\ldots~$;
 \item$36\cdot\ldots =9$.
\end{enumeratea}
\end{multicols}
\end{esercizio}

\begin{esercizio}
\label{ese:1.3}
 Vero o falso?
\begin{multicols}{2}
\TabPositions{3.2cm}
\begin{enumeratea}
 \item $5:0=0$	\tab\qquad\boxV\qquad\boxF
 \item $0:5=0$	\tab\qquad\boxV\qquad\boxF
 \item $5:5=0$	\tab\qquad\boxV\qquad\boxF
 \item $1:0=1$	\tab\qquad\boxV\qquad\boxF
 \item $0:1=0$	\tab\qquad\boxV\qquad\boxF
 \item $0:0=0$	\tab\qquad\boxV\qquad\boxF
 \item $1:1=1$	\tab\qquad\boxV\qquad\boxF
 \item $1:5=1$	\tab\qquad\boxV\qquad\boxF
\end{enumeratea}
\end{multicols}
\end{esercizio}

\begin{esercizio}
\label{ese:1.4}
 Se è vero che~$p=n\cdot m$, quali affermazioni sono vere?
\begin{multicols}{2}
\TabPositions{3.2cm}
\begin{enumeratea}
 \item $p$ è multiplo di~$n$	\tab\qquad\boxV\qquad\boxF
 \item $p$ è multiplo di~$m$	\tab\qquad\boxV\qquad\boxF
 \item $m$ è multiplo di~$p$	\tab\qquad\boxV\qquad\boxF
 \item $m$ è multiplo di~$n$	\tab\qquad\boxV\qquad\boxF
 \item $p$ è divisibile per~$m$	\tab\qquad\boxV\qquad\boxF
 \item $m$ è divisibile per~$n$	\tab\qquad\boxV\qquad\boxF
 \item $p$ è divisore di~$m$	\tab\qquad\boxV\qquad\boxF
 \item $n$ è multiplo di~$m$	\tab\qquad\boxV\qquad\boxF
\end{enumeratea}
\end{multicols}
\end{esercizio}

\begin{esercizio}
\label{ese:1.5}
 Quali delle seguenti affermazioni sono vere?

\begin{multicols}{2}
\TabPositions{3.2cm}
 \begin{enumeratea}
 \item 6 è un divisore di~3 \tab\qquad\boxV\qquad\boxF
 \item 3 è un divisore di~6 \tab\qquad\boxV\qquad\boxF
 \item 8 è un multiplo di~2 \tab\qquad\boxV\qquad\boxF
 \item 5 è divisibile per~10 \tab\qquad\boxV\qquad\boxF
 \end{enumeratea}
\end{multicols}
\end{esercizio}

\begin{esercizio}
\label{ese:1.6}
 Esegui le seguenti operazioni:
\begin{multicols}{2}
 \begin{enumeratea}
 \item $18\divint 3=\ldots$;
 \item $18\bmod 3=\ldots$;
 \item $20\divint 3=\ldots$;
 \item $20\bmod 3=\ldots$;
 \item $185\divint 7=\ldots$;
 \item $185\bmod 7=\ldots$;
 \item $97\divint 5=\ldots$;
 \item $97\bmod 5=\ldots$;
 \item $240\divint 12=\ldots$;
 \item $240\bmod 12=\ldots$.
 \end{enumeratea}
\end{multicols}
\end{esercizio}

\pagebreak
\begin{esercizio}
\label{ese:1.7}
 Esegui le seguenti divisioni con numeri a più cifre, senza usare la calcolatrice.
\begin{multicols}{4}
 \begin{enumeratea}
 \item $311:22$;
 \item $429:37$;
 \item $512:31$;
 \item $629:43$;
 \item $755:53$;
 \item $894:61$;
 \item $968:45$;
 \item $991:13$;
 \item $\np{1232}:123$;
 \item $\np{2324}:107$;
 \item $\np{3435}:201$;
 \item $\np{4457}:96$;
 \item $\np{5567}:297$;
 \item $\np{6743}:311$;
 \item $\np{7879}:201$;
 \item $\np{8967}:44$;
 \item $\np{13455}:198$;
 \item $\np{22334}:212$;
 \item $\np{45647}:721$;
 \item $\np{67649}:128$.
 \end{enumeratea}
\end{multicols}
\end{esercizio}

\subsubsection*{1.5 - Proprietà delle operazioni}
\begin{esercizio}
\label{ese:1.8}
 Stabilisci se le seguenti uguaglianze sono vere o false indicando la proprietà utilizzata:
\TabPositions{5cm}
 \begin{enumeratea}
 \item $33:11=11:33$			\tab proprietà\dotfill\:\boxV\qquad\boxF
 \item $108-72:9=(108-72):9$		\tab proprietà\dotfill\:\boxV\qquad\boxF
 \item $8-4=4-8$			\tab proprietà\dotfill\:\boxV\qquad\boxF
 \item $35\cdot 10=10\cdot 35$	\tab proprietà\dotfill\:\boxV\qquad\boxF
 \item $9\cdot(2+3)=9\cdot3+9\cdot2$ \tab proprietà\dotfill\:\boxV\qquad\boxF
 \item $80-52+36=(20-13-9)\cdot 4$ 	\tab proprietà\dotfill\:\boxV\qquad\boxF
 \item $(28-7):7=28:7-7:7$		\tab proprietà\dotfill\:\boxV\qquad\boxF
 \item $(8\cdot 1):2=8:2$		\tab proprietà\dotfill\:\boxV\qquad\boxF
% \item $(8-2)+3=8-(2+3)$		\tab proprietà\dotfill\:\boxV\qquad\boxF
 \item $(13+11)+4=13+(11+4)$		\tab proprietà\dotfill\:\boxV\qquad\boxF
 \end{enumeratea}
\end{esercizio}

\begin{esercizio}
\label{ese:1.9}
Data la seguente operazione tra i numeri naturali~$a\circ b=2\cdot a +3\cdot b$, verifica se è:
 \begin{enumeratea}
 \item commutativa, cioè se~$a\circ b=b\circ a$;
 \item associativa, cioè se~$a\circ (b\circ c)=(a\circ b)\circ c$;
 \item 0 è elemento neutro.
 \end{enumeratea}
\end{esercizio}

\subsubsection*{1.6 - Potenza}
\begin{esercizio}
\label{ese:1.10}
Inserisci i numeri mancanti:
 \begin{multicols}{2}
 \begin{enumeratea}
 \item $3^1\cdot3^2\cdot3^3=3^{\ldots+\ldots+\ldots}=3^{\ldots}$;
 \item $3^4:3^2=3^{\ldots-\ldots}=3^{\ldots}$;
 \item $(3:7)^5=3^{\ldots}:7^{\ldots}$;
 \item $6^3:5^3=(6:5)^{\ldots}$;
 \item $7^3\cdot5^3\cdot2^3=(7\cdot 5 \cdot 2)^{\ldots}$;
 \item $\left(2^6\right)^2=2^{\ldots\cdot\ldots}=2^{\ldots}$;
 \item $\left(18^6\right):\left(9^6\right)=(\ldots\ldots)^{\ldots}=2^{\ldots}$;
 \item $\left(5^6\cdot5^4\right)^4:\left[\left(5^2\right)^3\right]^6=\ldots\ldots\ldots=5^{\ldots}$.
 \end{enumeratea}

 \end{multicols}
\end{esercizio}

\begin{esercizio}[\Ast]
\label{ese:1.11}
Calcola applicando le proprietà delle potenze:
 \begin{multicols}{2}
 \begin{enumeratea}
 \item $2^5\cdot2^3:2^2\cdot3^6$;
 \item $\left(5^2\right)^3:5^3\cdot5$;
 \item $\left\{\left[\left(2^3\right)^2:2^3\right]^3:2^5\right\}:\left(2^8:2^6\right)^2$;
 \item $\left[\left(2^1\right)^4\cdot 3^4\right]^2:6^5\cdot6^0$.
 \item $2^2\cdot\left(2^3+5^2\right)$;
 \item $\left[\left(3^6:3^4\right)^2\cdot3^2\right]^1$;
 \item $4^4\cdot\left(3^4+4^2\right)$;
 \item $3^4\cdot\left(3^4+4^2-2^2\right)^0:3^3+0\cdot 100$.
 \end{enumeratea}
 \end{multicols}
\end{esercizio}

\pagebreak
\begin{esercizio}
\label{ese:1.12}
 Completa, applicando le proprietà delle potenze:
\begin{multicols}{2}
 \begin{enumeratea}
 \item $7^4\cdot7^{\ldots}=7^5$;
 \item $3^9\cdot5^9=(\ldots\ldots)^9$;
 \item $5^{15}:5^{\ldots}=55$;
 \item $(\ldots\ldots)^6\cdot5^6=15^6$;
 \item $8^4:2^4=2^{\ldots}$;
 \item $\left(18^5:6^5\right)^2=3^{\ldots}$;
 \item $20^7:20^0=20^{\ldots}$;
 \item $\left(\ldots^3\right)^4=1$;
 \end{enumeratea}
\end{multicols}
\end{esercizio}

\begin{esercizio}
\label{ese:1.13}
 Il risultato di~$3^5+5^3$ è:
 \begin{center}
 \boxA\:~368 \quad\boxB\:~$(3+5)^5$ \quad\boxC\:~$15+15$ \quad\boxD\:~$8^8$.
 \end{center}
\end{esercizio}

\begin{esercizio}
\label{ese:1.14}
 Il risultato di~$(73+27)^2$ è:
 \begin{center}
 \boxA\:~200 \quad\boxB\:~$73^2+27^2$ \quad\boxC\:~$10^4$ \quad\boxD\:~$1\,000$.
 \end{center}
\end{esercizio}

\subsubsection*{1.7 - Numeri Primi}
\begin{esercizio}
\label{ese:1.15}
 Per ognuno dei seguenti numeri indica i divisori propri:
  \begin{multicols}{2}
 \begin{enumeratea}
 \item 15 ha divisori propri \ldots, \ldots, \ldots, \ldots;
 \item 19 ha divisori propri \ldots, \ldots, \ldots, \ldots;
 \item 24 ha divisori propri \ldots, \ldots, \ldots, \ldots;
 \item 30 ha divisori propri \ldots, \ldots, \ldots, \ldots.
 \end{enumeratea}
  \end{multicols}
\end{esercizio}

\begin{esercizio}[Crivello di Eratostene]
\label{ese:1.16}
\begin{multicols}{2}
Nella tabella che segue sono rappresentati i numeri naturali fino a~100. Per trovare i
numeri primi, seleziona~1 e~2, poi cancella tutti i multipli di~2. Seleziona il~3 e cancella i multipli di~3. Seleziona il
primo dei numeri che non è stato cancellato, il~5, e cancella
tutti i multipli di~5. Procedi in questo modo fino alla fine
della tabella. Quali sono i numeri primi minori di~100?

\columnbreak\vfil
% (c) 2012 Dimitrios Vrettos - d.vrettos@gmail.com
% \usetikzlibrary{matrix,fit}
\tikzset{%
	table nodes/.style={%
		rectangle,
		draw=black,
 		align=center,
   		minimum height=5mm,
     	text depth=0.5ex,
     	text height=1.5ex,
     	inner xsep=-1pt,
     	outer sep=0pt
	},
	table/.style={%
        matrix of nodes,
        row sep=-\pgflinewidth,
        column sep=-\pgflinewidth,
        nodes={%
            table nodes
        },
        execute at empty cell={\node[draw=none]{};}
    }
}

\begin{center}
\begin{tikzpicture}

\matrix (first) [table,text width=7mm,name=table]
{
1 & 2 & 3 &4 & 5 & 6 & 7 & 8 & 9 & 10\\
11 & 12 & 13 &14 & 15 & 16 & 17 & 18 & 19 & 20\\
21 & 22 & 23 &24 & 25 & 26 & 27 & 28 & 29 & 30\\
31 & 32 & 33 &34 & 35 & 36 & 37 & 38 & 39 & 40\\
41 & 42 & 43 &44 & 45 & 46 & 47 & 48 & 49 & 50\\
51 & 52 & 53 &54 & 55 & 56 &57 & 58 & 59 & 60\\
61 & 62 & 63 &64 & 65 & 66 & 67 & 68 & 69 & 70\\
71 & 72 & 73 &74 & 75 & 76 & 77 & 78 & 79 & 80\\
81 & 82 & 83 &84 & 85 & 86 & 87 & 88 & 89 & 90\\
91 & 92 & 93 &94 & 95 & 96 & 97 & 98 & 99 & 100\\
};

\end{tikzpicture}
\end{center}

\end{multicols}
\end{esercizio}

\subsubsection*{1.8 - Criteri di divisibilità}
\begin{esercizio}
\label{ese:1.17}
 Per quali numeri sono divisibili i valori seguenti? Segna i divisori con una crocetta.
\TabPositions{3.5cm}
 \begin{enumeratea}
 \item 1\,320 è divisibile per \tab\fbox{2}\:\fbox{3}\:\fbox{4}\:\fbox{5}\:\fbox{6}\:\fbox{7}\:\fbox{8}\:\fbox{9}\:\fbox{10}\:\fbox{11}
 \item 2\,344 è divisibile per \tab\fbox{2}\:\fbox{3}\:\fbox{4}\:\fbox{5}\:\fbox{6}\:\fbox{7}\:\fbox{8}\:\fbox{9}\:\fbox{10}\:\fbox{11}
 \item 84 è divisibile per \tab\fbox{2}\:\fbox{3}\:\fbox{4}\:\fbox{5}\:\fbox{6}\:\fbox{7}\:\fbox{8}\:\fbox{9}\:\fbox{10}\:\fbox{11}
 \item 1\,255 è divisibile per \tab\fbox{2}\:\fbox{3}\:\fbox{4}\:\fbox{5}\:\fbox{6}\:\fbox{7}\:\fbox{8}\:\fbox{9}\:\fbox{10}\:\fbox{11}
 \item 165 è divisibile per \tab\fbox{2}\:\fbox{3}\:\fbox{4}\:\fbox{5}\:\fbox{6}\:\fbox{7}\:\fbox{8}\:\fbox{9}\:\fbox{10}\:\fbox{11}
 \item 720 è divisibile per \tab\fbox{2}\:\fbox{3}\:\fbox{4}\:\fbox{5}\:\fbox{6}\:\fbox{7}\:\fbox{8}\:\fbox{9}\:\fbox{10}\:\fbox{11}
 \item 792 è divisibile per \tab\fbox{2}\:\fbox{3}\:\fbox{4}\:\fbox{5}\:\fbox{6}\:\fbox{7}\:\fbox{8}\:\fbox{9}\:\fbox{10}\:\fbox{11}
 \item 462 è divisibile per \tab\fbox{2}\:\fbox{3}\:\fbox{4}\:\fbox{5}\:\fbox{6}\:\fbox{7}\:\fbox{8}\:\fbox{9}\:\fbox{10}\:\fbox{11}
 \end{enumeratea}
\end{esercizio}
\pagebreak
\begin{esercizio}
\label{ese:1.18}
 Determina tutti i divisori di 32, 18, 24, 36.
% \begin{multicols}{2}
% \begin{enumeratea}
% \item 32\quad\ldots, \ldots, \ldots, \ldots, \ldots, \ldots, \ldots
% \item 18\quad\ldots, \ldots, \ldots, \ldots, \ldots, \ldots, \ldots
% \item 24\quad\ldots, \ldots, \ldots, \ldots, \ldots, \ldots, \ldots
% \item 36\quad\ldots, \ldots, \ldots, \ldots, \ldots, \ldots, \ldots
% \end{enumeratea}
% \end{multicols}
\end{esercizio}

\subsubsection*{1.9 - Scomposizione in fattori primi}

\begin{esercizio}[\Ast]
\label{ese:1.19}
Scomponi i seguenti numeri in fattori primi:
 \begin{multicols}{5}
 \begin{enumeratea}
 \item 16;
 \item 18;
 \item 24;
 \item 30;
 \item 32;
 \item 36;
 \item 40;
 \item 42;
 \item 48;
 \item 52;
 \item 60;
 \item 72;
 \item 81;
 \item 105;
 \item 120;
 \item 135;
 \item 180;
 \item 225;
 \item 525;
 \item 360.
 \end{enumeratea}
 \end{multicols}
\end{esercizio}


\begin{esercizio}[\Ast]
\label{ese:1.20}
Scomponi i seguenti numeri in fattori primi:
 \begin{enumeratea}
 \begin{multicols}{5}
 \item 675;
 \item 715;
 \item \np{1900};
 \item \np{1078};
 \item \np{4050};
 \item \np{4536};
 \item \np{12150};
 \item \np{15246};
 \item \np{85050};
 \item \np{138600};
 \item \np{234000};
 \item \np{255000};
 \item \np{293760};
 \item \np{550800};
 \item \np{663552}.
 \end{multicols}
 \end{enumeratea}
\end{esercizio}

\subsubsection*{1.10 - Massimo Comune Divisore e minimo comune multiplo}

\begin{esercizio}[\Ast]
\label{ese:1.21}
Calcola~$\mcm$ e~$\mcd$ tra i seguenti gruppi di numeri:
\begin{multicols}{3}
 \begin{enumeratea}
 \item 6,~15
 \item 12,~50
 \item 1,~6,~10,~14
 \item 15,~5,~10
 \item 2,~4,~8
 \item 2,~1,~4
 \item 5,~6,~8
 \item 24,~12,~16
 \item 6,~16,~26
 \item 6,~8,~12
 \item 50,~120,~180
 \item 20,~40,~60
 \item 16,~18,~32
 \item 30,~60,~27
 \item 45,~15,~35
 \end{enumeratea}
\end{multicols}
\end{esercizio}

\begin{esercizio}[\Ast]
\label{ese:1.22}
Calcola~$\mcm$ e~$\mcd$ tra i seguenti gruppi di numeri:
\begin{multicols}{3}
 \begin{enumeratea}
 \item 6,~8,~10,~12
 \item 30,~27,~45
 \item 126,~180
 \item 24,~12,~16
 \item 6,~4,~10
 \item 5,~4,~10
 \item 12,~14,~15
 \item 3,~4,~5
 \item 6,~8,~12
 \item 15,~18,~21
 \item 12,~14,~15
 \item 15,~18,~24
 \item 100,~120,~150
 \item 44,~66,~12
 \item 24,~14,~40
 \end{enumeratea}
\end{multicols}
\end{esercizio}

\begin{multicols}{2}
\begin{esercizio}[\Ast]
\label{ese:1.23}
 Tre funivie partono contemporaneamente da una stessa stazione sciistica. La prima compie il tragitto di
andata e ritorno in~15 minuti, la seconda in~18 minuti, la terza in~20. Dopo quanti minuti partiranno di nuovo
insieme?
\end{esercizio}

\begin{esercizio}[\Ast]
\label{ese:1.24}
 Due aerei partono contemporaneamente dall'aeroporto di Milano e vi ritorneranno dopo aver
percorso le loro rotte: il primo ogni~15 giorni e il secondo ogni~18 giorni. Dopo quanti giorni i due
aerei si troveranno di nuovo insieme a Milano?
\end{esercizio}

\begin{esercizio}[\Ast]
\label{ese:1.25}
 Una cometa passa in prossimità della Terra ogni~360 anni, una seconda ogni~240 anni e una terza ogni~750 anni.
 Se quest'anno sono state avvistate tutte e tre, fra quanti anni sarà possibile vederle di nuovo tutte e
tre nello stesso anno?
\end{esercizio}

\begin{esercizio}[\Ast]
\label{ese:1.26}
 Disponendo di~56 penne,~70 matite e~63 gomme, quante confezioni uguali si possono fare? Come sarà
composta ciascuna confezione?
\end{esercizio}
\end{multicols}

\subsubsection*{1.11 - Espressioni numeriche}

\begin{esercizio}[\Ast]
\label{ese:1.27}
Esegui le seguenti operazioni rispettando l'ordine.
 \begin{multicols}{4}
 \begin{enumeratea}
 \item $15+7-2$;
 \item $16-4+2$;
 \item $18-8-4$;
 \item $16\cdot 2-2$;
 \item $12-2\cdot 2$;
 \item $10-5\cdot 2$;
 \item $20\cdot 4:5$;
 \item $16:4\cdot 2$;
 \item $2+2^2+3$;
 \item $4\cdot 2^3+1$;
 \item $2^4:2-4$;
 \item $(1+2)^3-2^3$;
 \item $\left(3^2\right)^3-3^2$;
 \item $2^4+2^3$;
 \item $2^3\cdot 3^2$;
 \item $3^3:3^2\cdot 3^2$.
 \end{enumeratea}
 \end{multicols}
\end{esercizio}

\subsection{Esercizi riepilogativi}
\begin{esercizio}[\Ast]
Quali delle seguenti scritture rappresentano numeri naturali?
 \begin{multicols}{4}
 \begin{enumeratea}
 \item $5+3-1$;
 \item $6+4-10$;
 \item $5-6+1$;
 \item $7+2-10$;
 \item $2\cdot 5:5$;
 \item $2\cdot 3:4$;
 \item $3\cdot 4-12$;
 \item $12:4-4$;
 \item $11:3+2$;
 \item $27:9:3$;
 \item $18:2-9$;
 \item $10-1:3$.
 \end{enumeratea}
 \end{multicols}
\end{esercizio}


\begin{esercizio}
Calcola il risultato delle seguenti operazioni nei numeri naturali; alcune operazioni non sono
possibili, individuale.
 \begin{multicols}{4}
 \begin{enumeratea}
 \item $5:5=\ldots$;
 \item $5:0=\ldots$;
 \item $1\cdot 5 =\ldots$;
 \item $1-1=\ldots$;
 \item $10:2=\ldots$;
 \item $0:5=\ldots$;
 \item $5\cdot1=\ldots$;
 \item $0:0=\ldots$;
 \item $10:5=\ldots$;
 \item $1:5=\ldots$;
 \item $0\cdot5=\ldots$;
 \item $5:1=\ldots$;
 \item $0\cdot0=\ldots$;
 \item $1\cdot0=\ldots$;
 \item $1:0=\ldots$;
 \item $1:1=\ldots$
 \end{enumeratea}
 \end{multicols}
\end{esercizio}

\begin{esercizio}[\Ast]
Aggiungi le parentesi in modo che l'espressione abbia il risultato indicato.
 \begin{multicols}{2}
 \begin{center}
 ~a)~~$2+5\cdot3+2=35$

 ~b)~~$2+5\cdot3+2=27$
 \end{center}
 \end{multicols}
\end{esercizio}

\begin{esercizio}[\Ast]
Traduci in espressioni aritmetiche le seguenti frasi e calcola il risultato:
 \begin{enumeratea}
 \item aggiungi~12 al prodotto tra~6 e~4;
 \item sottrai il prodotto tra~12 e~2 alla somma tra~15 e~27;
 \item moltiplica la differenza tra~16 e~7 con la somma tra~6 e~8;
 \item al doppio di~15 sottrai la somma dei prodotti di~3 con~6 e di~2 con~5;
 \item sottrai il prodotto di~6 per~4 al quoziente tra~100 e~2;
 \item moltiplica la differenza di~15 con~9 per la somma di~3 e~2;
 \item sottrai al triplo del prodotto di~6 e~2 il doppio del quoziente tra~16 e~4.
 \item il quadrato della somma tra il quoziente intero di~25 e~7 e il cubo di~2;
 \item la somma tra il quadrato del quoziente intero di~25 e~7 e il quadrato del cubo di~2;
 \item la differenza tra il triplo del cubo di~5 e il doppio del quadrato di~5.
 \end{enumeratea}
\end{esercizio}

\begin{esercizio}[\Ast]
 Calcola il valore delle seguenti espressioni:
 \begin{enumeratea}
 \item $(1+2\cdot3):(5-2\cdot2)+1+2\cdot4$;
 \item $ (18-3\cdot2):(16-3\cdot4)\cdot(2:2+2)$;
 \item $2+2\cdot6-[21-(3+4\cdot 3:2)]:2$;
 \item $\lbrace[15-(5\cdot2-4)]\cdot2\rbrace:(30:15+1)-\lbrace[25\cdot4]:10-(11-2)\rbrace$.
 \end{enumeratea}
\end{esercizio}
\pagebreak
\begin{esercizio}[\Ast]
 Calcola il valore delle seguenti espressioni:
 \begin{enumeratea}
 \item $[6\cdot(2\cdot4-2\cdot3)-6]+\lbrace3\cdot(21:7-2)\cdot[(6\cdot5):10]-3\cdot2\rbrace$;
 \item $100:2+3^2-2^2\cdot6$;
 \item $2^7:2^3-2^2$;
 \item $30-5\cdot3+7\cdot2^2-2$.
 \end{enumeratea}
\end{esercizio}

\begin{esercizio}[\Ast]
 Calcola il valore delle seguenti espressioni:
 \begin{enumeratea}
 \item $(3+4)^2-\left(3^2+4^2\right)$;
 \item $5\cdot5^3\cdot5^4:\left(5^2\right)^3+5$;
 \item $32^5:16^4-2^9$;
 \item $\left[3^0+\left(2^4-2^3\right)^2:\left(4^3:4^2\right)+3 \right]:\left(2^6:2^4\right)$.
 \end{enumeratea}
\end{esercizio}


\begin{esercizio}[\Ast]
 Calcola il valore delle seguenti espressioni:
 \begin{enumeratea}
 \item $\left[\left(4^5:4^3\right)-2^3\right]\cdot\left[\left(3^4\cdot 3^3\right):\left(3^2\cdot3\right)\right]:\left(2^2+2^0+3^1\right)$;
 \item $\left(12-5^2:5\right)\cdot4^2:2^3+2^2-1+\left[\left(2^4:2^3\right)^3+4^3:4+2^5\right]:7$;
 \item $\left(5^2\cdot2^2-\left(2^5-2^5:\left(2^2\cdot 3 +4^2:4\right)+2^3\cdot\left(3^2-2^2\right)\right)\right):\left(3\cdot2 \right)\cdot5$;
 \item $\left(3^4\cdot3^3:3^6\right)^2+\left(7^2-5^2\right):2^2~$.
 \end{enumeratea}
\end{esercizio}

\begin{esercizio}[\Ast]
 Calcola il valore delle seguenti espressioni:
 \begin{enumeratea}
 \item $\left(3\cdot2^2-10\right)^4\cdot\left(3^3+2^3\right):7-10\cdot2^3$;
 \item $(195:15)\cdot\left\lbrace \left[3^2\cdot6+3^2\cdot4^2-5\cdot(6-1)^2 \right]\right\rbrace:\left(4^2-3\right)$;
 \item $5+[(16:8)\cdot3+(10:5)\cdot3 ]\cdot\left(2^3\cdot5-1\right)^2-[(3\cdot10):6-1]$;
 \item $\left[4\cdot\left(3\cdot2-3\cdot1^2\right)-5\right]-\left\lbrace 2\cdot(14:7+4):\left[2\cdot(3+2)^2:10+1-4^2:8\right]\right\rbrace~$.
 \end{enumeratea}
\end{esercizio}
\begin{multicols}{2}
 \begin{esercizio}[\Ast]
 Un'automobile percorre~$18\;\unit{km}$ con~1 litro di benzina. Quanta benzina deve aggiungere il proprietario dell'auto
sapendo che l'auto ha già~12~litri di benzina nel serbatoio, che deve intraprendere un viaggio di~$432\;\unit{km}$ e che deve
arrivare a destinazione con almeno~4~litri di benzina nel serbatoio?
\end{esercizio}

\begin{esercizio}[\Ast]
Alla cartoleria presso la scuola una penna costa~3~euro più di una matita. Gianni ha comprato~2~penne e~3~matite e ha speso
16~euro. Quanto spenderà Marco che ha comprato~1~penna e~2~matite?
\end{esercizio}

\begin{esercizio}[\Ast]
 In una città tutte le linee della metropolitana iniziano il loro servizio alla stessa ora. La linea rossa fa una corsa ogni
15~minuti, la linea gialla ogni~20~minuti e la linea blu ogni~30~minuti. Salvo ritardi, ogni quanti minuti le tre linee
partono allo stesso momento?
\end{esercizio}

\begin{esercizio}
 Tre negozi si trovano sotto lo stesso porticato, ciascuno ha un'insegna luminosa intermittente: la prima si spegne ogni
6~secondi, la seconda ogni~5~secondi, la terza ogni~7~secondi. Se le insegne vengono accese contemporaneamente
alle~19:00 e spente contemporaneamente alle~21:00, quante volte durante la serata le tre insegne
si spegneranno contemporaneamente?
\end{esercizio}

\begin{esercizio}
In una gita scolastica ogni insegnante accompagna un gruppo di~12~studenti. Se alla gita partecipano~132~studenti,
quanti insegnanti occorrono?
\end{esercizio}

\begin{esercizio}
Un palazzo è costituito da~4~piani con~2~appartamenti per ogni piano. Se ogni appartamento ha~6~finestre con~4~vetri
ciascuna, quanti vetri ha il palazzo?
\end{esercizio}

\begin{esercizio}
Spiega brevemente il significato delle seguenti parole:

a)~numero~primo,\quad b)~numero~dispari,

c)~multiplo,\quad d)~cifra.
% \begin{enumeratea}
% \item numero primo;
% \item numero dispari;
% \item multiplo;
% \item cifra.
% \end{enumeratea}
\end{esercizio}

\begin{esercizio}
Rispondi brevemente alle seguenti domande:
 \begin{enumeratea}
 \item cosa vuol dire scomporre in fattori un numero?
 \item ci può essere più di una scomposizione in fattori di un numero?
 \item cosa vuol dire scomporre in fattori primi un numero?
 \item che differenza c'è tra la frase ``$a$ e $b$ sono due numeri primi'' e la frase ``$a$ e $b$ sono primi tra di loro''?
 \end{enumeratea}
\end{esercizio}
\end{multicols}

\subsection{Risposte}

\paragraph{1.11.}
a)~$6^6$,\quad b)~$5^4$,\quad c)~1,\quad d)~$6^3$.

\paragraph{1.19.}
a)~$2^4$,\quad b)~$ 2\cdot 3^2 $,\quad c)~$ 2^3 \cdot 3 $,\quad d)~$ 2\cdot 3\cdot 5 $,\quad e)~$ 2^5 $,\quad f)~$ 2^2 \cdot 3^2 $,\quad g)~$ 2^3 \cdot 5 $,\quad h)~$ 2\cdot 3\cdot 7 $,\quad i)~$ 2^4 \cdot 3 $,\quad j)~$ 2^2 \cdot 13 $,\quad k)~$ 2^2 \cdot 3 \cdot5 $,\quad l)~$ 2^3 \cdot 3^2 $,\quad m)~$ 3^4 $,\quad n)~$ 3\cdot 5\cdot 7 $,\quad o)~$ 2^3 \cdot 3\cdot 5 $,\quad p)~$ 3^3 \cdot 5 $,\quad q)~$ 2^2 \cdot 3^2 \cdot 5 $,\quad r)~$ 3^2 \cdot 5^2 $, s)~$3\cdot5^2\cdot7$,\quad t)~$ 2^3 \cdot 3^2 \cdot5$.

\paragraph{1.20.}
d)~$2\cdot7^2\cdot11$,\quad e)~$2\cdot3^4\cdot5^2$,\quad f)~$2^3\cdot3^4\cdot7$,\quad g)~$2\cdot3^5\cdot5^2$,\quad h)~$2\cdot3^2\cdot7\cdot11^2$,\quad i)~$2\cdot3^5\cdot5^2\cdot7$,\quad
j)~$2^3\cdot3^2\cdot5^2\cdot7\cdot11$,\quad k)~$2^4\cdot3^2\cdot5^3\cdot13$,\quad l)~$2^3\cdot3\cdot5^4\cdot17$,\quad m)~$2^7\cdot3^3\cdot5\cdot17$,\quad n)~$2^4\cdot3^4\cdot5^2\cdot17$,\quad o)~$2^{13}\cdot3^4$.

\paragraph{1.21.}
a)~30;~3,\quad b)~300;~2,\quad c)~210;~1,\quad d)~30;~5,\quad e)~8;~2,\quad f)~4;~1,\quad g)~120;~1,\quad k)~1800;~10,\quad l)~120;~20.

\paragraph{1.22.}
m)~600;~10,\quad n)~132;~2,\quad o)~840;~2.

\begin{multicols}{3}
\paragraph{1.23.}
3 ore.

\paragraph{1.24.}
90 giorni.

\paragraph{1.25.}
$ \np{18000}$~anni.
\end{multicols}

\paragraph{1.26.}
7 confezioni, ognuna conterrà 8~penne, 10~matite, e 9~gomme.

\paragraph{1.27.}
a)~20,\quad e)~8,\quad i)~9,\quad m)~720.

\paragraph{1.28.}
a, b, e, g, j, k.

\paragraph{1.30.}
a)~$ (2+5)\cdot(3+2) $,\quad b)~$ 2+5\cdot(3+2) $.

\paragraph{1.31.}
a)~36,\quad b)~18,\quad c)~126,\quad d)~2,\quad e)~26,\quad f)~30.

\begin{multicols}{2}
\paragraph{1.32.}
a)~16,\quad b)~9,\quad c)~8,\quad d)~5.

\paragraph{1.33.}
a)~9,\quad b)~35,\quad c)~12,\quad d)~41.

\paragraph{1.34.}
a)~24,\quad b)~30,\quad c)~0,\quad d)~5.

\paragraph{1.35.}
a)~81,\quad b)~25,\quad c)~25,\quad d)~15.

\paragraph{1.36.}
a)~0,\quad b)~73,\quad c)~$18\,253$,\quad d)~4.

\paragraph{1.37.}
Almeno~16.

\paragraph{1.38.}~9~euro.

\paragraph{1.39.}~60~minuti.
\end{multicols}
% EOF
