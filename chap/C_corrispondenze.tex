% (c) 2012 Silvia Cibola - silvia.cibola@gmail.com
% (c) 2012-2014 Dimitrios Vrettos - d.vrettos@gmail.com

\chapter{Corrispondenze fra insiemi}

\section{Prime definizioni}

Ti proponiamo un semplice esercizio per introdurre l'argomento che qui vogliamo trattare.

Quando camminiamo per la strada della nostra città, vediamo tanti segnali lungo il percorso che, attraverso simboli,
ci danno informazioni sul comportamento corretto che dobbiamo tenere. Sia~$A= \{\text{segnali stradali della figura sotto}\}$ e~$B= \{\text{descrizione del segnale}\}$.
Collega con una freccia un segnale stradale con il suo significato.
\begin{center}
 % (c) 2012 Dimitrios Vrettos - d.vrettos@gmail.com

\pgfdeclareimage[interpolate=true]{segnali}{./img/part08/chapB/segnali.pdf}
\begin{tikzpicture}[x=10mm, y=10mm, scale=.4]
    \pgftext[at=\pgfpoint {0mm}{0mm},left,base]{\pgfuseimage{segnali}};
    \node[right] (a) at (11,11) {Curva pericolosa a destra};
    \node[right] (b) at (11,9) {Divieto di transito};
    \node[right] (c) at (11,7) {Divieto di sosta};
    \node[right] (d) at (11,5) {Fermarsi e dare precedenza};
    \node[right] (e) at (11,3) {Attraversamento ciclabile};

\end{tikzpicture}
\end{center}

\begin{definizione}
Si chiama \emph{corrispondenza} $\Kor$ \emph{fra due insiemi}~$A$ e~$B$, il predicato binario avente come soggetto un elemento di~$A$ e come complemento un elemento di~$B$.
Essa definisce un sottoinsieme~$G_\Kor$ del prodotto cartesiano~$A \times B$, costituito dalle coppie ordinate di elementi corrispondenti:
\[ G_\Kor=\{(a;b)\in A \times B \mid  a \,\Kor\, b \}.\]
\end{definizione}

\osservazione
Nel capitolo precedente abbiamo chiamato relazione un predicato binario che si riferisce a due elementi dello stesso insieme; la differenza di terminologia
sta semplicemente nella sottolineatura del fatto che si considerano appartenenti allo stesso insieme oppure appartenenti a due insiemi diversi il soggetto
e il complemento del predicato binario utilizzato.
A seconda del contesto in cui analizziamo un predicato binario, parleremo di corrispondenza o di relazione. Nelle pagine che seguono tratteremo di corrispondenze,
mettendo in luce le loro caratteristiche.

\begin{definizione}
Si chiama \emph{dominio} $\Dom$ di una corrispondenza l'insieme~$A$ in cui si trova il soggetto della proposizione vera costruita con il predicato~$\Kor$ e \emph{codominio}~$\Cod$
l'insieme degli elementi che costituiscono il complemento della stessa proposizione.
\end{definizione}

Per indicare in linguaggio matematico che si è stabilita una corrispondenza tra due insiemi~$A$ e~$B$ scriviamo:
\[\Kor:A \rightarrow B\text{ ``predicato'',} \quad \text{oppure} \quad \Kor:A \xrightarrow{\Kor:\text{ predicato}} B.\]
Formalizziamo il primo esercizio di questo capitolo:
\[\Kor:A \rightarrow B \textrm{ ``significare'',}\quad \text{oppure} \quad \Kor:A \xrightarrow{\Kor:\text{ significare}} B\text{,}\quad \Dom = A \text{;~~} \Cod = B.\]

\begin{definizione}
Definita una corrispondenza~$\Kor:A \rightarrow B$, nella coppia~$(a;b)$ di elementi corrispondenti, $b$ si chiama \emph{immagine di}~$a$ nella corrispondenza~$\Kor$.
L'insieme delle immagini degli elementi del dominio $\Dom$ è un sottoinsieme del codominio $\Cod$ chiamato \emph{insieme immagine} e Verrà indicato con~$\IM$. Quindi~$\IM \subseteq\Cod$.
\end{definizione}

\section{Rappresentazione di una corrispondenza}
\subsection{Rappresentare una corrispondenza con un grafico cartesiano}
\begin{exrig}
\begin{esempio}\label{ex:C.1}
Consideriamo gli insiemi~$A= \{$Parigi, Roma, Atene$\}$ e~$B= \{$Italia, Francia, Grecia$\}$. Il prodotto cartesiano~$A \times B$ è rappresentato dal grafico cartesiano della
figura~\ref{fig:C.1} (i suoi elementi sono le crocette).
Esso è formato dalle~9 coppie ordinate aventi come primo elemento una città (elemento di~$A$) e come secondo elemento uno stato d'Europa (elemento di~$B$).

Il predicato binario~$\Kor$: ``essere la capitale di'', introdotto nell'insieme~$A \times B$, determina il sottoinsieme~$G_\Kor$ i cui elementi sono le coppie (Parigi; Francia),
(Roma; Italia), (Atene; Grecia). Il dominio della corrispondenza è~$\Dom= \{$Parigi, Roma, Atene$\}$, il codominio è~$\Cod= \{$Italia, Francia, Grecia$\}$ e~$\IM=\Cod$.
\end{esempio}
\end{exrig}

\begin{figure}[b]
\begin{minipage}[t]{.3\textwidth}
 \centering
 % (c) 2012 Dimitrios Vrettos - d.vrettos@gmail.com
\begin{tikzpicture}[ x=7.5mm, y=7.5mm, font=\small]
  \begin{scope}[->]
    \draw (0,0) -- (0,3.5) node[left] {$y$};
    \draw (0,0) -- (3.5,0) node[below] {$x$};
  \end{scope}

  \begin{scope}[Maroon, dotted, step=7.5mm]
    \draw (0,0) grid (3.5,3.5);
  \end{scope}

  \foreach \y/\ytext in {1/Italia,2/Francia,3/Grecia}{
    \node[left] at (0,\y) {\ytext};
    }

  \foreach \x/\xtext in {1/Parigi,2/Roma,3/Atene}{
    \node[below left, rotate=30] at (\x,0) {\xtext};
    }
  \foreach \x in {1,2,3}{
    \draw (\x,1.5pt) -- (\x,-1.5pt);
    \draw (1.5pt,\x) -- (-1.5pt,\x);}

  \begin{scope}[very thick, decoration={crosses, shape size=1.5mm}]
    \begin{scope}[draw=CornflowerBlue]
      \draw decorate {(1,2) -- (1,2.1)};
      \draw decorate {(2,1) -- (2,1.1)};
      \draw decorate {(3,3) -- (3,3.1)};
    \end{scope}
    \draw decorate {(1,1) -- (1,1.1)};
    \draw decorate {(2,2) -- (2,2.1)};
    \draw decorate {(3,1) -- (3,1.1)};
    \draw decorate {(2,3) -- (2,3.1)};
    \draw decorate {(1,3) -- (1,3.1)};
    \draw decorate {(3,2) -- (3,2.1)};
  \end{scope}
  \node[color=CornflowerBlue] at (3.5,3.5) {$\Cod$};
\end{tikzpicture}
 \caption{Esempio C.1.}\label{fig:C.1}
\end{minipage}\hfil
\begin{minipage}[t]{.65\textwidth}
 \centering
 % (c) 2012 Dimitrios Vrettos - d.vrettos@gmail.com
\begin{tikzpicture}[x=10mm, y=10mm, scale=.7]

  \node[ellipse, minimum height=3cm,draw, minimum width=4cm] (A) at (0,0) {};
  \node[above right] () at (A.north east) {$A$};

  \begin{scope}[fill=CornflowerBlue, every node/.style={draw=black, fill}]

    \node[circle, minimum height=1cm]  (a) at (-1.5,.5) {};
    \node[circle,fill=white, minimum height=.5cm]  at (-1.5,.5) {};
    \filldraw (.5,0) ..controls +(90:.25cm) and +(90:.5cm) .. (0,0) .. controls  +(-90:.25cm) and +(90:0cm) .. (.5,-.7) .. controls +(90:0cm) and +(-90:.25cm) ..(1,0)  ..%
    controls +(90:.5cm) and  +(90:.25cm) .. (.5,0) node[draw=CornflowerBlue,below] (h) {};

    \node[regular polygon,regular polygon sides=5, inner sep=2mm] (p) at (-1,-1.2){};

    \node [cylinder, rotate=90, minimum height=1cm, minimum width=.5cm] at (1.5,.5) (c) {};

    \node [single arrow,shape border rotate=90, minimum height=1cm, minimum width=1cm] (f) at (-.4,1.2){};
  \end{scope}

  \begin{scope}[xshift=6cm]
    \node[ellipse, minimum height=3cm,draw, minimum width=4cm] (B) at (0,0) {};

    \node[above right] () at (B.north east) {$B$};

    \node (a1) at (2,0) {anello};
    \node (h1) at (0,1.5) {cuore};
    \node (f1) at (-1.3,.75) {freccia};
    \node (p1) at (0,-1) {pentagono};
    \node (c1) at (0,0) {cilindro};
  \end{scope}
  
  \begin{scope}[->,smooth,thick]
    \draw[Maroon] (a) .. controls +(-80:2cm) and +(-150:1cm) .. (a1);
    \draw[orange] (f) .. controls +(30:2cm) and +(150:1cm) .. (f1);
    \draw[red] (c) .. controls +(-30:2cm) and +(-150:1cm) .. (c1);
    \draw[yellow] (p) .. controls +(-30:2cm) and +(-150:1cm) .. (p1);
    \draw[CornflowerBlue] (h.east) .. controls +(-30:2cm) and +(150:1cm) .. (h1.west);
  \end{scope}
\end{tikzpicture}
 \caption{Esempio C.2.}\label{fig:C.2}
\end{minipage}
\end{figure}
\ovalbox{\risolvii \ref{ese:C.1}, \ref{ese:C.2}}

\subsection{Rappresentare una corrispondenza con un grafico sagittale}

\begin{exrig}
 \begin{esempio}
 Nella figura~\ref{fig:C.2} sono rappresentati gli insiemi~$A$ e~$B$ con diagrammi di Eulero-Venn.
 Collegando con una freccia, ciascun elemento di~$A$ con la sua forma ($B$), possiamo rappresentare
 con un \emph{grafico sagittale} la corrispondenza~$\Kor$: ``essere di forma'', tra gli insiemi
 assegnati. $A$ risulta essere il dominio e~$B$ il codominio della corrispondenza; $\IM=\Cod$.
 La freccia che collega ogni elemento del dominio con la sua immagine rappresenta il predicato~$\Kor$.
\end{esempio}
\end{exrig}

\ovalbox{\risolvi \ref{ese:C.3}}

\begin{exrig}
\begin{esempio}
Consideriamo gli insiemi~$R= \{$regioni d'Italia$\}$ e~$M= \{$Ligure, Ionio, Tirreno, Adriatico$\}$ e la corrispondenza~$\Kor:R \rightarrow M$ ``essere bagnata/o da''; $R$ è il dominio e~$M$
il codominio di questa corrispondenza.

L'insieme~$G_\Kor$ delle coppie ordinate aventi come primo elemento una regione e come secondo elemento un mare
è:~$G_\Kor= \{$(Liguria; Ligure), (Toscana; Tirreno), (Lazio; Tirreno), (Campania; Tirreno), (Basilicata; Tirreno), (Calabria; Tirreno), (Calabria; Ionio), (Puglia; Ionio), 
(Puglia; Adriatico), (Molise; Adriatico), (Abruzzo; Adriatico), (Emilia-Romagna; Adriatico), (Marche; Adriatico), (Veneto; Adriatico), (Friuli Venezia Giulia; Adriatico)$\}$.
Se rappresentiamo questa corrispondenza con un grafico sagittale notiamo che non tutti gli elementi del dominio hanno l'immagine in~$\Kor$. La corrispondenza definita
si può generare solo in un sottoinsieme del dominio.
\end{esempio}
\end{exrig}

\begin{definizione}
Chiamiamo \emph{insieme di definizione} della corrispondenza $\Kor$, indicato con~$\ID$, il sottoinsieme del dominio $\Dom$ i cui elementi hanno effettivamente un corrispondente nel codominio $\Cod$.
\end{definizione}

Nel grafico di figura~\ref{fig:C.3} è rappresentata una generica situazione formatasi dall'aver definito una corrispondenza tra due insiemi; sono in grigio l'insieme di definizione, sottoinsieme del dominio ($\ID \subseteq \Dom$),
e l'insieme immagine, sottoinsieme del codominio ($\IM \subseteq \Cod$).
\begin{figure}[htb]
 \centering% (c) 2012 Dimitrios Vrettos - d.vrettos@gmail.com
\begin{tikzpicture}[x=10mm, y=10mm]

  \node[ellipse, minimum height=3.5cm,draw, minimum width=5cm] (D) at (0,0) {};

  \node[above] (D1) at (D.north) {$\Dom$};

  \begin{scope}[decoration={random steps,segment length=1mm}]
    \draw [decorate,fill=lightgray] (1,0)  arc (00:360:1 and 1.5) -- cycle; 
    \node [above right] at(.8,1) {$\ID$};
  \end{scope}

  \begin{scope}[fill=CornflowerBlue]
    \filldraw (0,1) circle (2pt) node (a) {};
    \filldraw (-.2,.2) circle (2pt) node (b) {};
    \filldraw (.2,-.4) circle (2pt) node (c) {};
    \filldraw (0,-1) circle (2pt) node (d) {};
    \filldraw (-1,-1) circle (2pt) ;
    \filldraw (-1.5,-.5) circle (2pt) ;
    \filldraw (-1,-1) circle (2pt) ;
    \filldraw (-2,0) circle (2pt) ;
    \filldraw (-2.2,.5) circle (2pt) ;
    \filldraw (-1,1) circle (2pt) ;
    \filldraw (-1.3,.5) circle (2pt) ;
    \filldraw (1,-1) circle (2pt) ;
    \filldraw (1.5,-.5) circle (2pt) ;
    \filldraw (1,-1) circle (2pt) ;
    \filldraw (2,0) circle (2pt) ;
    \filldraw (2.2,.5) circle (2pt) ;
    \filldraw (1.3,.5) circle (2pt) ;
  \end{scope}

  \begin{scope}[xshift=6cm]
    \node[ellipse, minimum height=3.5cm,draw, minimum width=5cm] (C) at (0,0) {};
    \node[above] (C1) at (C.north) {$\Cod$};

    \begin{scope}[decoration={random steps,segment length=1mm}]
      \draw [decorate,fill=lightgray] (1,0)  arc (00:360:1 and 1.5) -- cycle; 
      \node [above right] at(.8,1) {$\IM$};
    \end{scope}
    
    \begin{scope}[fill=LimeGreen]
      \filldraw (0,1) circle (2pt) node (a1) {};
      \filldraw (-.2,.2) circle (2pt) node (b1) {};
      \filldraw (.2,-.4) circle (2pt) node (c1) {};
      \filldraw (0,-1) circle (2pt) node (d1) {};

      \filldraw (-1,-1) circle (2pt) ;
      \filldraw (-1.5,-.5) circle (2pt) ;
      \filldraw (-1,-1) circle (2pt) ;
      \filldraw (-2,0) circle (2pt) ;
      \filldraw (-2.2,.5) circle (2pt) ;

      \filldraw (-1.3,.5) circle (2pt) ;
      \filldraw (1.5,-.5) circle (2pt) ;
      \filldraw (1,-1) circle (2pt) ;
      \filldraw (2,0) circle (2pt) ;
      \filldraw (2.2,.5) circle (2pt) ;
    \filldraw (1.3,.5) circle (2pt) ;
    \end{scope}
  \end{scope}
  
  \begin{scope}[->,smooth,thick]
    \draw[blue] (D1.north) .. controls +(80:1cm) and +(80:1cm) .. (C1.north) node [midway, above, black] () {$\Kor$};
    \draw[Maroon] (a) .. controls +(80:1cm) and +(150:1cm) .. (a1);
    \draw[orange] (b) .. controls +(30:2cm) and +(150:1cm) .. (b1);
    \draw[red] (c) .. controls +(-30:2cm) and +(-150:1cm) .. (c1);
    \draw[yellow] (d) .. controls +(-30:2cm) and +(-150:1cm) .. (d1);
  \end{scope}
\end{tikzpicture}
 \caption{Corrispondenza tra due insiemi.}\label{fig:C.3}
\end{figure}

Osserviamo che in alcuni casi si ha la coincidenza dell'insieme di definizione con il dominio e la coincidenza dell'insieme immagine con il codominio:~$\ID=\Dom$ e~$\IM=\Cod$.

\newpage

\section{Caratteristiche di una corrispondenza}
\begin{exrig}
\begin{esempio}
Generalizziamo uno degli esercizi precedenti sulle date di nascita. Prendiamo come dominio~$\Dom= \{$persone italiane viventi$\}$ e come codominio~$\Cod= \{$gli anni dal~1900 al~2012$\}$.
Evidentemente~$\ID=\Dom$: ogni persona ha un determinato anno di nascita, ma più persone sono nate nello stesso anno. Inoltre~$\IM$ potrebbe coincidere con~$\Cod$,
vista la presenza sul territorio nazionale di ultracentenari. Comunque scriveremo~$\IM \subseteq \Cod$. Il grafico sagittale di questa corrispondenza è del tipo rappresentato nella
figura~\ref{fig:C.4}.
\end{esempio}
\begin{figure}[hb]
 \centering% (c) 2012 Dimitrios Vrettos - d.vrettos@gmail.com
\begin{tikzpicture}[x=10mm, y=10mm]

  \node[ellipse, minimum height=3cm,draw, minimum width=4cm] (D) at (0,0) {};

  \node[above] (D1) at (D.north) {$\Dom$};

  \begin{scope}[fill=CornflowerBlue]
    \filldraw (0,1) circle (2pt) node (a) {};
    \node[above left] at (0,1) {$p_1$};
    \filldraw (-.2,.2) circle (2pt) node (b) {};
    \node[above left] at (-.2,.2) {$p_2$};
    \filldraw (.2,-.4) circle (2pt) node (c) {};
    \node[above left] at (.2,-.4) {$p_3$};
    \filldraw (0,-1) circle (2pt) node (d) {};
    \node[above left] at (0,-1) {$p_4$};
    \filldraw (-1,-1) circle (2pt) node (e) {};
    \node[above left] at (-1,-1) {$p_5$};
    \filldraw (-1.5,-.5) circle (2pt) ;
    \filldraw (-1,-1) circle (2pt) ;
    \filldraw (-1,1) circle (2pt) ;
    \filldraw (-1.3,.5) circle (2pt) ;
    \filldraw (1,-1) circle (2pt) ;
    \filldraw (1.5,-.5) circle (2pt) ;
    \filldraw (1,-1) circle (2pt) ;
    \filldraw (1.3,.5) circle (2pt) ;
  \end{scope}

  \begin{scope}[xshift=5cm]
    \node[ellipse, minimum height=3.cm,draw, minimum width=4cm] (C) at (0,0) {};

    \node[above] (C1) at (C.north) {$\Cod$};

    \begin{scope}[fill=LimeGreen]
    \filldraw (0,1) circle (2pt) node (a1) {};
    \filldraw (-.2,.2) circle (2pt);
    \filldraw (.2,-.4) circle (2pt) node (c1) {};
    \filldraw (0,-1) circle (2pt);

    \node[above right]  at (0,1) {1910};
    \node[above right]  at (.2,-.4) {1997};
    \filldraw (-1,-1) circle (2pt) ;
    \filldraw (-1.5,-.5) circle (2pt) ;
    \filldraw (-1,-1) circle (2pt) ;
    \filldraw (-1.3,.5) circle (2pt) ;
    \filldraw (1.5,-.5) circle (2pt) ;
    \filldraw (1,-1) circle (2pt) ;
    \filldraw (1.3,.5) circle (2pt) ;
    \end{scope}
  \end{scope}

  \begin{scope}[->,smooth,thick]
    \draw[blue] (D1.north) .. controls +(80:1cm) and +(80:1cm) .. (C1.north) node [midway, above, black] () {$\Kor$};
    \draw[Maroon] (a) .. controls +(80:1cm) and +(150:1cm) .. (a1);
    \draw[orange] (b) .. controls +(30:2cm) and +(150:1cm) .. (c1);
    \draw[red] (c) .. controls +(-30:2cm) and +(-150:1cm) .. (c1);
    \draw[yellow] (d) .. controls +(-30:2cm) and +(-180:2cm) .. (c1);
    \draw[purple] (e) .. controls +(-30:2cm) and +(-150:1cm) .. (a1);
  \end{scope}
\end{tikzpicture}
 \caption{Corrispondenza \emph{molti a uno}: più persone sono nate nello stesso anno.}\label{fig:C.4}
\end{figure}


\begin{esempio}
Analizziamo la corrispondenza dell'esempio precedente~$\Kor:R \rightarrow M$ ``essere bagnata/o da'' tra l'insieme delle regioni d'Italia $R$ e l'insieme dei mari $M$. 

$\ID\subset\Dom$ poiché alcune regioni non sono bagnate da alcun mare. Moltre regioni sono bagnate dallo stesso mare, ma succede che alcune regioni siano bagnate da
due mari. 

$IM=C$: un mare bagna almeno una regione. Il grafico sagittale di questa corrispondenza è del tipo rappresentato nella
figura~\ref{fig:C.5}.
\end{esempio}

\begin{figure}[hb]
 \centering% (c) 2012 Dimitrios Vrettos - d.vrettos@gmail.com
\begin{tikzpicture}[x=10mm, y=10mm]

\node[ellipse, minimum height=3cm,draw, minimum width=4cm] (D) at (0,0) {};

\node[above] (D1) at (D.north) {Regioni};

\draw[dotted] (-.6,1.4) -- (-.6,-1.4);
\begin{scope}[fill=CornflowerBlue]

\filldraw (.7,1) circle (2pt) node (a) {};
\node[left] at (.7,1) {Liguria};
\filldraw (1,.2) circle (2pt) node (b) {};
\node[left] at (1,.2) {Calabria};
\filldraw (.8,-.5) circle (2pt) node (c) {};
\node[left] at (.8,-.5) {Puglia};
\filldraw (-1.3,0) circle (2pt) node (d) {};
\node[above] at (-1.3,0) {Umbria};

\node at (.4,-1) {$\ID$};
\end{scope}

\begin{scope}[xshift=5cm]
\node[ellipse, minimum height=3cm,draw, minimum width=4cm] (C) at (0,0) {};

\node[above] (C1) at (C.north) {Mari};

\begin{scope}[fill=LimeGreen]
\filldraw (-.2,1) circle (2pt) node (a1) {};
\filldraw (-.2,.2) circle (2pt)node (b1) {};
\filldraw (.2,-.4) circle (2pt) node (c1) {};
\filldraw (0,-1) circle (2pt) node (d1){};

\node[right]  at (-.2,1) {Adriatico};
\node[right]  at (.2,-.4) {Ionio};
\node[right] at (-.2,.2) {Tirreno};
\node[right] at (0,-1) {Ligure};

\end{scope}
\end{scope}
\begin{scope}[->,smooth,thick]
\draw[blue] (D1.north) .. controls +(80:1cm) and +(80:1cm) .. (C1.north) node [midway, above, black] () {$\Kor$};
\draw[Maroon] (a) .. controls +(80:1cm) and +(150:1cm) .. (d1);
 \draw[orange] (b) .. controls +(30:2cm) and +(150:1cm) .. (c1);
\draw[orange] (b) .. controls +(-30:2cm) and +(-150:1cm) .. (b1);
\draw[red] (c) .. controls +(-30:2cm) and +(-180:2cm) .. (c1);
\draw[red] (c) .. controls +(-30:2cm) and +(-180:2cm) .. (a1);
\end{scope}
\end{tikzpicture}
 \caption{Esempio di corrispondenza di tipo \emph{molti a molti}.}\label{fig:C.5}
\end{figure}

\begin{esempio}
Generalizziamo la corrispondenza~$\Kor$: ``essere la capitale di'' tra il dominio~$\Dom= \{$città d'Europa$\}$ e il codominio~$\Cod= \{$stati d'Europa$\}$. È evidente che~$\ID\subset\Dom$:
non tutte le città sono capitali, mentre~$\IM=\Cod$ in quanto ogni stato ha la sua capitale; inoltre due città diverse non possono essere capitali dello stesso stato.
Il grafico sagittale di questa corrispondenza è del tipo rappresentato nella
figura~\ref{fig:C.6}.
\end{esempio}

\begin{figure}[ht]
 \centering% (c) 2012 Dimitrios Vrettos - d.vrettos@gmail.com
\begin{tikzpicture}[x=10mm, y=10mm]

\node[ellipse, minimum height=3cm,draw, minimum width=4cm] (D) at (0,0) {};

\node[above] (D1) at (D.north) {Città};

\draw[dotted] (-.6,1.4) -- (-.6,-1.4);
\begin{scope}[fill=CornflowerBlue]

\filldraw (.7,1) circle (2pt) node (a) {};
\node[left] at (.7,1) {Roma};
\filldraw (1,.2) circle (2pt) node (b) {};
\node[left] at (1,.2) {Parigi};
\filldraw (.8,-.5) circle (2pt) node (c) {};
\node[left] at (.8,-.5) {Londra};
\filldraw (-1.3,0) circle (2pt) node (d) {};
\node[above] at (-1.3,0) {Genova};

\node at (.4,-1) {$\ID$};
\end{scope}

\begin{scope}[xshift=5cm]
\node[ellipse, minimum height=3cm,draw, minimum width=4cm] (C) at (0,0) {};

\node[above] (C1) at (C.north) {Stati};

\begin{scope}[fill=LimeGreen]
\filldraw (-.1,1) circle (2pt) node (a1) {};
\filldraw (-.2,.2) circle (2pt)node (b1) {};
\filldraw (.2,-.8) circle (2pt) node (c1) {};

\node[right]  at (-.1,1) {Francia};
\node[right]  at (.2,-.8) {Italia};
\node[right] at (-.2,.2) {Inghilterra};

\end{scope}
\end{scope}
\begin{scope}[->,smooth,thick]
\draw[blue] (D1.north) .. controls +(80:1cm) and +(80:1cm) .. (C1.north) node [midway, above, black] () {$\Kor$};
\draw[Maroon] (a) .. controls +(30:1cm) and +(150:1cm) .. (c1);
 \draw[orange] (b) .. controls +(30:2cm) and +(150:1cm) .. (a1);
\draw[red] (c) .. controls +(-30:2cm) and +(-180:2cm) .. (b1);
\end{scope}
\end{tikzpicture}
 \caption{Esempio di corrispondenza di tipo \emph{uno a uno}.}\label{fig:C.6}
\end{figure}
\begin{esempio}
Consideriamo, tra l'insieme~$\insN_0$ dei numeri naturali diversi da zero e l'insieme~$\insZ_0$ degli interi relativi diversi da zero, la corrispondenza~$\Kor$: ``essere il valore assoluto di''.
Per la definizione di valore assoluto di un intero, possiamo senz'altro affermare che:~$\insN_0=\Dom=\ID$; $\insZ_0=\Cod=\IM$. Poiché due numeri opposti hanno lo stesso valore assoluto,
ogni elemento di~$\insN_0$ ha due immagini, per cui il grafico sagittale di questa corrispondenza è come nella figura~\ref{fig:C.7}.
\end{esempio}
\begin{figure}[ht]
 \centering% (c) 2012 Dimitrios Vrettos - d.vrettos@gmail.com
\begin{tikzpicture}[x=10mm, y=10mm]

\node[ellipse, minimum height=3cm,draw, minimum width=4cm] (D) at (0,0) {};

\node[above] (D1) at (D.north) {$\Dom=\insN_0$};

\begin{scope}[fill=CornflowerBlue]

\filldraw (0,1) circle (2pt) node (a) {};
\node[left] at (0,1) {1};
\filldraw (1,.2) circle (2pt) node (b) {};
\filldraw (0,-.5) circle (2pt) node (c) {};
\node[left] at (0,-.5) {5};
\filldraw (-1.3,0) circle (2pt) node (d) {};


\end{scope}

\begin{scope}[xshift=5cm]
\node[ellipse, minimum height=3cm,draw, minimum width=4cm] (C) at (0,0) {};

\node[above] (C1) at (C.north) {$\Cod=\insZ_0$};

\begin{scope}[fill=LimeGreen]
\filldraw (-.1,1) circle (2pt) node (a1) {};
\filldraw (-.5,.5) circle (2pt)node (b1) {};
\filldraw (.2,-.8) circle (2pt) node (c1) {};
\filldraw  (-.2,0) circle (2pt) node (d1){};
\filldraw  (.8,1) circle (2pt);
\filldraw  (1.5,0) circle (2pt);
\node[right]  at (-.1,1) {$+1$};
\node[right]  at (-.5,.5) {$-1$};
\node[right]  at (.2,-.8) {$-5$};
\node[right] at (-.2,0) {$+5$};

\end{scope}
\end{scope}
\begin{scope}[->,smooth,thick]
\draw[blue] (D1.north) .. controls +(80:1cm) and +(80:1cm) .. (C1.north) node [midway, above, black] () {$\Kor$};
\draw[Maroon] (a) .. controls +(30:1cm) and +(150:1cm) .. (a1);
 \draw[Maroon] (a) .. controls +(30:1cm) and +(150:1cm) .. (b1);
\draw[red] (c) .. controls +(-30:2cm) and +(-180:2cm) .. (c1);
\draw[red] (c) .. controls +(-30:2cm) and +(-180:2cm) .. (d1);
\end{scope}
\end{tikzpicture}

 \caption{Esempio di corrispondenza di tipo \emph{uno a molti}.}\label{fig:C.7}
\end{figure}
\end{exrig}

\begin{definizione}
Le corrispondenze di tipo \emph{molti a uno} e \emph{uno a uno} sono dette \emph{univoche}; in esse ogni elemento dell'insieme di definizione ha una sola immagine nel codominio.
\end{definizione}
\pagebreak
\begin{exrig}
\begin{esempio}
Consideriamo la corrispondenza~$\Kor$ che associa ad ogni persona il suo codice fiscale: ogni persona ha il proprio codice fiscale, persone diverse hanno codice fiscale diverso.
Dominio e~$\ID$ coincidono e sono l'insieme~$P= \{$persone$\}$. Codominio e~$\IM$ coincidono e sono l'insieme~$F= \{$codici fiscali$\}.$ Il grafico sagittale di questa corrispondenza
è del tipo \emph{uno a uno}. È di questo tipo il grafico sagittale della corrispondenza che associa ad ogni automobile la sua targa, ad ogni moto il suo numero di telaio,
ad ogni cittadino italiano maggiorenne il suo certificato elettorale, \ldots

\end{esempio}
\end{exrig}

\osservazione In tutti i casi in cui la corrispondenza è di tipo \emph{uno a uno}, l'insieme di definizione $\ID$ coincide con il dominio $\Dom$ e l'insieme immagine $\IM$ coincide con il codominio $\Cod$.

\begin{definizione}
Una corrispondenza di tipo \emph{uno a uno} in cui~$\ID=\Dom$ e~$\IM=\Cod$ è detta \emph{corrispondenza biunivoca}.
\end{definizione}

\ovalbox{\risolvii \ref{ese:C.4}, \ref{ese:C.5}, \ref{ese:C.6}, \ref{ese:C.7}, \ref{ese:C.8}, \ref{ese:C.9}, \ref{ese:C.10}}
\newpage
% (c) 2012 Silvia Cibola - silvia.cibola@gmail.com
% (c) 2012-2014 Dimitrios Vrettos - d.vrettos@gmail.com

\section{Esercizi}

\subsection{Esercizi dei singoli paragrafi}

\subsubsection*{C.2 - Rappresentazione di una corrispondenza}

\begin{esercizio}
\label{ese:C.1}
Rappresenta con un grafico cartesiano la corrispondenza~$\Kor$: ``essere nato nell'anno'' di dominio l'insieme~$A=\{$Galileo, Napoleone, Einstein, Fermi, Obama$\}$
e codominio l'insieme~$B=\{$1901, 1564, 1961, 1879, 1769, 1920, 1768$\}$. Rappresenta per elencazione il sottoinsieme~$G_\Kor$ del prodotto cartesiano~$A \times B$.
Stabilisci infine gli elementi dell'immagine~$\IM$.
\end{esercizio}

\begin{esercizio}
\label{ese:C.2}
L'insieme~$S=\{$casa, volume, strada, ufficio, clavicembalo, cantautore, assicurazione$\}$ è il codominio della corrispondenza~$\Kor$: ``essere il numero di sillabe di'' il cui dominio
è~$X=\{x \in\insN \mid  0<x<10\}$. Rappresenta con un grafico cartesiano la corrispondenza assegnata, evidenzia, come nell'esempio~\ref{ex:C.1} a pagina~\pageref{ex:C.1}, l'insieme~$G_\Kor$,
scrivi per elencazione l'insieme~$\IM$.
\end{esercizio}

\begin{esercizio}
\label{ese:C.3}
Completa la rappresentazione con grafico sagittale della corrispondenza $\Kor$ ``essere capitale di''. La freccia che collega gli elementi del dominio $\Dom$ con quelli del codominio $\Cod$ rappresenta
il predicato~$\Kor$.
\begin{center}
 % (c) 2012 Dimitrios Vrettos - d.vrettos@gmail.com
\begin{tikzpicture}[x=10mm, y=10mm]

\node[ellipse, minimum height=3cm,draw, minimum width=4cm] (D) at (0,0) {};

\node[above] (D1) at (D.north) {$\Dom$};

\begin{scope}[fill=CornflowerBlue]

\filldraw (.7,1) circle (2pt) node (a) {};
\node[left] at (.7,1) {Roma};
\filldraw (1,.2) circle (2pt) node (b) {};
\node[left] at (1,.2) {Parigi};

\filldraw (-1.3,-.5) circle (2pt) node (c) {};
\end{scope}

\begin{scope}[xshift=5cm]
\node[ellipse, minimum height=3cm,draw, minimum width=4cm] (C) at (0,0) {};

\node[above] (C1) at (C.north) {$\Cod$};

\begin{scope}[fill=LimeGreen]
\filldraw (-.1,1) circle (2pt) node (a1) {};
\filldraw (-.2,.2) circle (2pt)node (b1) {};
\filldraw (.2,-.8) circle (2pt) node (c1) {};

\node[right]  at (-.1,1) {Francia};
\node[right]  at (.2,-.8) {Italia};
\node[right] at (-.2,.2) {Grecia};

\end{scope}
\end{scope}
\begin{scope}[->,smooth,thick]
\draw[red] (c) .. controls +(-30:2cm) and +(-180:2cm) .. (b1);
\end{scope}
\end{tikzpicture}
\end{center}

\end{esercizio}

\subsubsection*{C.3 - Caratteristiche di una corrispondenza}
\begin{esercizio}
\label{ese:C.4}
\`E univoca la corrispondenza~$\Kor$ definita tra l'insieme~$P= \{$parola del proverbio ``rosso di sera, bel tempo si spera''$\}$ e l'insieme~$A=\{$lettere dell'alfabeto italiano$\}$
che associa ad ogni parola la sua iniziale? Ti sembra corretto affermare che dominio $\Dom$ e insieme di definizione $\ID$ coincidono? Completa con il simbolo corretto
la relazione tra insieme immagine e codominio:~$\IM\ldots\Cod$. Fai il grafico sagittale della corrispondenza.
\end{esercizio}

\begin{esercizio}
\label{ese:C.5}
$\Kor$ è la corrispondenza tra l'insieme ~$\insN$ dei naturali e l'insieme degli interi relativi~$\insZ$ espressa dal predicato ``essere il quadrato di''. Ti sembra corretto affermare che
dominio $\Dom$ e insieme di definizione $\ID$ coincidono? Perché~$\IM=\Cod$? La corrispondenza è univoca?
\end{esercizio}

\pagebreak

\begin{esercizio}
\label{ese:C.6}
Una corrispondenza~$\Kor$ è assegnata con il suo grafico cartesiano.
\begin{center}
 % (c) 2012 Dimitrios Vrettos - d.vrettos@gmail.com
\begin{tikzpicture}[x=10mm, y=10mm]

\begin{scope}[->]
\draw (-.5,0) -- (12,0);
\draw (0,-.5) -- (0,7);
\end{scope}

\foreach \x in {1,2,...,11}
\draw (\x,1.5pt) -- (\x,-1.5pt);

\foreach \y in {1,2,...,6}
\draw (1.5pt,\y) -- (-1.5pt,\y);

\foreach \xi/\xtext in {1/A,2/B,3/C,4/D,5/E,6/F,7/G,8/H,9/I,10/L,11/M}
\node[below] at (\xi,0) {$\xtext$};

\foreach \yi/\ytext in {1/1,2/2,3/3,4/4,5/5,6/6}
\node[left] at (0,\yi){\ytext};

\draw[orange, dotted] (0,0) grid (11,6);

\begin{scope}[fill=CornflowerBlue]
\foreach \x in {1,6}
\filldraw (\x,1) circle (2pt);

\foreach \x in {2,5,8}
\filldraw (\x,2) circle (2pt);

\foreach \x in {4,10}
\filldraw (\x,4) circle (2pt);

\filldraw (3,5) circle (2pt);
\filldraw (11,6) circle (2pt);
\end{scope}
\end{tikzpicture}
\end{center}
Completa e rispondi alle domande:

\begin{enumeratea}
\item $\Dom=$\{\dotfill\};
\item $\Cod=$\{\dotfill\};
\item $\ID=$\{\dotfill\};
\item $\IM=$\{\dotfill\};
\item la corrispondenza è biunivoca?
\item di quali elementi dell'insieme di definizione 2 ne è l'immagine?
\item quale elemento del codominio è l'immagine di~$M$?
\end{enumeratea}
\end{esercizio}

%\newpage
\begin{esercizio}
\label{ese:C.7}
I tre grafici sagittali rappresentano altrettante corrispondenze, $\Kor_1$, $\Kor_2$, $\Kor_3$.
Completa per ciascuna di esse la descrizione schematizzata nel riquadro sottostante:
\begin{center}
 % (c) 2012 Dimitrios Vrettos - d.vrettos@gmail.com
\begin{tikzpicture}[x=10mm, y=10mm]

\node[circle, minimum height=2cm,draw] (A) at (0,0) {};

\node[above] (A1) at (A.north) {$A$};

\begin{scope}[fill=CornflowerBlue]

\filldraw (.5,.5) circle (2pt) node (a) {};
\node[left] at (.5,.5) {1};
\filldraw (.8,.2) circle (2pt) node (b) {};
\node[left] at (.8,.2) {2};
\filldraw (-.4,-.5) circle (2pt) node (c) {};
\node[left] at (-.4,-.5)  {3};
\filldraw (-.5,0) circle (2pt);
\node[left] at (-.5,0)  {4};
\filldraw (-.3,.5) circle (2pt);
\node[left] at (-.3,.5)  {5};
\end{scope}

\begin{scope}[xshift=2.3cm]
\node[circle, minimum height=2cm,draw] (B) at (0,0) {};

\node[above] (B1) at (B.north) {$B$};

\begin{scope}[fill=LimeGreen]
\filldraw (-.1,.6) circle (2pt) node (a1) {};
\filldraw (-.2,.2) circle (2pt)node (b1) {};
\filldraw (.2,-.7) circle (2pt) node (c1) {};
\filldraw(.5,-.2) circle (2pt);

\node[right]  at (-.1,.6) {$a$};
\node[right] at (-.2,.2) {$b$};
\node[right]  at (.2,-.7) {$c$};
\node[right] at (.5,-.2) {$d$};
\end{scope}
\end{scope}

\begin{scope}[->,smooth,thick]
\draw[Maroon] (a) .. controls +(30:1cm) and +(150:.5cm) .. (a1);
\draw[purple] (b) .. controls +(30:.5cm) and +(180:0.5cm) .. (b1);
\draw[orange] (c) .. controls +(30:1cm) and +(-90:1cm) .. (b1);
\draw[orange] (c) .. controls +(30:1cm) and +(-180:2cm) .. (c1);
\end{scope}

\begin{scope}[yshift=-2.5cm]
\matrix (m) [matrix of nodes]
{$\Dom=$&\ldots\\
$\Cod=$&\ldots\\
$\ID=$&\ldots\\
$\IM=$&\ldots\\
Tipo$=$&\ldots\\};
\end{scope}


\begin{scope}[xshift=4.6cm]

\node[circle, minimum height=2cm,draw] (A) at (0,0) {};

\node[above] (A1) at (A.north) {$A$};

\begin{scope}[fill=CornflowerBlue]

\filldraw (0,.7) circle (2pt) node (a) {};
\node[left] at (0,.7) {$a$};
\filldraw (.7,0) circle (2pt) node (b) {};
\node[left] at (.7,0) {$b$};
\filldraw (-.4,-.5) circle (2pt) node (c) {};
\node[left] at (-.4,-.5)  {$c$};
\end{scope}

\begin{scope}[xshift=2.3cm]
\node[circle, minimum height=2cm,draw] (B) at (0,0) {};

\node[above] (B1) at (B.north) {$B$};

\begin{scope}[fill=LimeGreen]
\filldraw (-.1,.6) circle (2pt) node (a1) {};
\filldraw (-.2,.2) circle (2pt)node (b1) {};
\filldraw (.2,-.7) circle (2pt) node (c1) {};

\node[right]  at (-.1,.6) {$m$};
\node[right] at (-.2,.2) {$n$};
\node[right]  at (.2,-.7) {$p$};
\end{scope}
\end{scope}

\begin{scope}[->,smooth,thick]
\draw[Maroon] (a) .. controls +(30:1cm) and +(180:1cm) .. (b1);
\draw[purple] (b) .. controls +(30:1cm) and +(180:1cm) .. (c1);
\draw[orange] (c) .. controls +(30:.5cm) and +(-180:2cm) .. (a1);
\end{scope}

\begin{scope}[yshift=-2.5cm]
\matrix (m) [matrix of nodes]
{$\Dom=$&\ldots\\
$\Cod=$&\ldots\\
$\ID=$&\ldots\\
$\IM=$&\ldots\\
Tipo$=$&\ldots\\};
\end{scope}
\end{scope}

\begin{scope}[xshift=9.2cm]
\node[circle, minimum height=2.cm,draw] (A) at (0,0) {};

\node[above] (A1) at (A.north) {$A$};

\begin{scope}[fill=CornflowerBlue]

\filldraw (.3,.7) circle (2pt) node (a) {};
\node[left] at (.3,.7) {1};
\filldraw (.6,.2) circle (2pt) node (b) {};
\node[left] at (.6,.2) {2};
\filldraw (-.3,-.5) circle (2pt) node (c) {};
\node[left] at (-.3,-.5)  {3};
\filldraw (-.5,0) circle (2pt) node (d){};
\node[left] at (-.5,0)  {4};

\end{scope}

\begin{scope}[xshift=2.3cm]
\node[circle, minimum height=2cm,draw] (B) at (0,0) {};

\node[above] (B1) at (B.north) {$B$};

\begin{scope}[fill=LimeGreen]
\filldraw (-.1,.6) circle (2pt) node (a1) {};
\filldraw (-.2,.2) circle (2pt)node (b1) {};
\filldraw (.1,-.8) circle (2pt) node (c1) {};
\filldraw(.5,-.1) circle (2pt) node (d1) {};
\filldraw(-.7,-.4) circle (2pt) node (e1) {};

\node[right]  at (-.1,.6) {$a$};
\node[right] at (-.2,.2) {$b$};
\node[right]  at (.1,-.8) {$c$};
\node[right] at (.5,-.1) {$d$};
\node[right] at (-.7,-.4) {$e$};
\end{scope}
\end{scope}

\begin{scope}[->,smooth,thick]
\draw[Maroon] (a) .. controls +(30:.5cm) and +(90:.5cm) .. (e1);
\draw[purple] (b) .. controls +(30:.5cm) and +(90:.5cm) .. (e1);
\draw[orange] (c) .. controls +(30:.5cm) and +(-180:2cm) .. (b1);
\draw[red] (d) .. controls +(-30:2cm) and +(-120:1cm) .. (d1);
\end{scope}

\begin{scope}[yshift=-2.5cm]
\matrix (m) [matrix of nodes]
{$\Dom=$&\ldots\\
$\Cod=$&\ldots\\
$\ID=$&\ldots\\
$\IM=$&\ldots\\
Tipo$=$&\ldots\\};
\end{scope}
\end{scope}
\end{tikzpicture}
\end{center}
\end{esercizio}
\pagebreak
\begin{esercizio}
\label{ese:C.8}
Il dominio della corrispondenza~$\Kor$ è l'insieme~$\insZ\times\insZ$ e~$\insZ$ ne è il codominio; l'immagine della coppia~$(a;b)$ è l'intero~$p=a \cdot b$.
\begin{enumeratea}
\item Stabilisci l'insieme di definizione $\ID$ e l'insieme immagine $\IM$;
\item perché questa corrispondenza non è biunivoca?
\item tutte le coppie aventi almeno un elemento uguale a zero hanno come immagine \ldots;
\item 1 è l'immagine di \ldots;
\item se gli elementi della coppia sono numeri concordi, allora l'immagine è \ldots;
\item un numero negativo è immagine di \ldots
\end{enumeratea}
Fai degli esempi che illustrino le tue affermazioni precedenti.
\end{esercizio}

\begin{esercizio}
\label{ese:C.9}
Il dominio $\Dom$ della corrispondenza~$\Kor$ è l'insieme~$\insZ\times\insZ$ e~$\insZ$ ne è il codominio $\Cod$; l'immagine della coppia~$(a;b)$ è il numero razionale~$q=\frac{a}{b}$.
\begin{enumeratea}
\item Stabilisci l'insieme di definizione $\ID$ e l'insieme immagine $\IM$;
\item completa:
\begin{enumeratea}
\item lo zero è immagine delle coppie \ldots;
\item se gli elementi della coppia sono numeri opposti l'immagine è \ldots;
\item se gli elementi della coppia sono numeri concordi allora l'immagine è \ldots;
\item un numero negativo è immagine di \ldots
\end{enumeratea}
\item fai degli esempi che illustrino le tua affermazioni precedenti.
\end{enumeratea}
\end{esercizio}

\begin{esercizio}
\label{ese:C.10}
In un gruppo di~10 persone, due si erano laureate in medicina e tre in legge nell'anno~1961, mentre quattro anni dopo, una si era laureata in fisica, un'altra in scienze e due in legge.
Considerate i seguenti insiemi:~$P=\{x \mid  x$ è una persona del gruppo$\}$; $A=\{$1960, 1961, 1964, 1965$\}$; $F=\{x \mid  x$ è una facoltà universitaria$\}$.
Fatene la rappresentazione con diagramma di Eulero-Venn e studiate le corrispondenze~$\Kor_1$, $\Kor_2$, espresse dai predicati:~$\Kor_1$: ``essersi laureato nell'anno'' e~$K_2$:
``essere laureato in'', mettendo in evidenza per ciascuna dominio, codominio, insieme di definizione, immagine, tipo.

Completate:
\begin{enumeratea}
\item nel gruppo ci sono \ldots persone laureate in legge, di cui \ldots nell'anno~1961 e le altre \ldots nell'anno \ldots;
\item nel~1961 si sono laureate \ldots di cui \ldots in medicina;
\item negli anni \ldots non si è laureata nessuna persona del gruppo considerato;
\item tra le~10 persone \ldots non si è laureata.
\end{enumeratea}
N.B.: ciascuno possiede una sola laurea.

Maria si è laureata in fisica nello stesso anno in cui si è laureato suo marito Luca; Andrea, fratello di Luca, non è medico, ha frequentato una facoltà diversa da quella del fratello
e si è laureato in un anno diverso. Supponendo che Maria, Luca e Andrea siano tra le~10 persone di cui sopra, completate:

Maria si è laureata nell'anno \ldots. Andrea si è laureato nell'anno \ldots in \ldots. Luca si è laureato nell'anno \ldots in \ldots
\end{esercizio}

\cleardoublepage
