% (c) 2012 Claudio Carboncini - claudio.carboncini@gmail.com
% (c) 2012-2014 Dimitrios Vrettos - d.vrettos@gmail.com

\chapter{Riconoscimento di prodotti notevoli}

\section{Quadrato di un binomio}

Uno dei metodi più usati per la scomposizione di polinomi è legato al saper riconoscere i prodotti notevoli.
Se abbiamo un trinomio costituito da due termini che sono quadrati di due monomi ed il terzo termine è uguale al doppio prodotto
degli stessi due monomi, allora il trinomio può essere scritto sotto forma di quadrato di un binomio, secondo la regola che segue (sezione \ref{sect:quadrato_di_un_binomio} a pagina \pageref{sect:quadrato_di_un_binomio})
\begin{equation*}
(A+B)^{2}=A^{2}+2AB+B^{2}\quad \Rightarrow \quad A^{2}+2AB+B^{2}=(A+B)^{2}.
\end{equation*}
Analogamente nel caso in cui il monomio che costituisce il doppio prodotto sia negativo:
\begin{equation*}
(A-B)^{2}=A^{2}-2AB+B^{2}\quad \Rightarrow \quad A^{2}-2AB+B^{2}=(A-B)^{2}.
\end{equation*}
Poiché il quadrato di un numero è sempre positivo, valgono anche le seguenti uguaglianze
\begin{equation*}
(A+B)^{2}=(-A-B)^{2}\quad\Rightarrow\quad A^{2}+2AB+B^{2}=(A+B)^{2}=(-A-B)^{2}\phantom{.}
\end{equation*}
\begin{equation*}
(A-B)^{2}=(-A+B)^{2}\quad \Rightarrow \quad A^{2}-2AB+B^{2}=(A-B)^{2}=(-A+B)^{2}.
\end{equation*}

\begin{exrig}
 \begin{esempio}
Scomporre in fattori~$4a^{2}+12ab^{2}+9b^{4}$.

Notiamo che il primo ed il terzo termine sono quadrati, rispettivamente di~$2a$ e di~$3b^{2}$,
ed il secondo termine è il doppio prodotto degli stessi monomi, pertanto possiamo
scrivere:
\[4a^{2}+12{ab}^{2}+9b^{4}=(2a)^{2}+2\cdot (2a)\cdot (3b^{2})+\left(3b^{2}\right)^{2}=\left(2a+3b^{2}\right)^{2}.\]
 \end{esempio}

 \begin{esempio}
Scomporre in fattori~$x^{2}-6x+9$.

Il primo ed il terzo termine sono quadrati, il secondo termine compare con il segno ``meno''.
Dunque:~$x^{2}-6x+9=x^{2}-2\cdot 3\cdot x+3^{2}=(x-3)^{2}$, ma anche~$x^{2}-6x+9=(-x+3)^{2}$.
 \end{esempio}

 \begin{esempio}
Scomporre in fattori~$x^{4}+4x^{2}+4$.

Può accadere che tutti e tre i termini siano tutti quadrati. $x^{4}+4x^{2}+4$ è formato da tre quadrati, ma il secondo termine, quello di grado intermedio,
è anche il doppio prodotto dei due monomi di cui il primo ed il terzo termine sono i rispettivi quadrati.
Si ha dunque:
\[x^{4}+4x^{2}+4=\left(x^{2}\right)^{2}+2\cdot (2)\cdot (x^{2})+(2)^{2}=\left(x^{2}+2\right)^{2}.\]
 \end{esempio}
\end{exrig}

\begin{procedura}
Individuare il quadrato di un binomio:
\begin{enumeratea}
\item individuare le basi dei due quadrati;
\item verificare se il terzo termine è il doppio prodotto delle due basi;
\item scrivere tra parentesi le basi dei due quadrati e il quadrato fuori dalla parentesi;
\item mettere il segno ``più'' o ``meno'' in accordo al segno del termine che non è un quadrato.
\end{enumeratea}
\end{procedura}

Può capitare che i quadrati compaiano con il coefficiente negativo, ma si può rimediare mettendo in evidenza il segno ``meno''.

\begin{exrig}
 \begin{esempio}
Scomporre in fattori~$-9a^{2}+12{ab}-4b^{2}$.

Mettiamo~$-1$ a fattore comune~$-9a^{2}+12ab-4b^{2}=-(9a^{2}-12{ab}+4b^{2})=-(3a-2b)^{2}$.
 \end{esempio}

 \begin{esempio}
Scomporre in fattori~$-x^{4}-x^{2}-\frac{1}{4}$.
\[-x^{4}-x^{2}-\frac{1}{4}=-\left(x^{4}+x^{2}+\frac{1}{4}\right)=-\left(x^{2}+\frac{1}{2}\right)^{2}.\]
 \end{esempio}

 \begin{esempio}
Scomporre in fattori~$-x^{2}+6xy^{2}-9y^{4}$.
\[x^{2}+6xy^{2}-9y^{4}=-\left(x^{2}-6xy^{2}+9y^{4}\right)=-\left(x-3y^{2}\right)^{2}.\]
 \end{esempio}
\end{exrig}

Possiamo avere un trinomio che ``diventa'' quadrato di binomio dopo aver messo qualche fattore comune in evidenza.

\begin{exrig}
 \begin{esempio}
Scomporre in fattori~$2a^{3}+20a^{2}+50a$.

Mettiamo a fattore comune~$2a$, allora~$2a^{3}+20a^{2}+50a=2a(a^{2}+10a+25)=2a(a+5)^{2}$.
 \end{esempio}

 \begin{esempio}
Scomporre in fattori~$2a^{2}+4a+2$.
\[2a^{2}+4a+2=2\left(a^{2}+2a+1\right)=2(a+1)^{2}.\]
 \end{esempio}

 \begin{esempio}
Scomporre in fattori~$-12a^{3}+12a^{2}-3a$.
\[-12a^{3}+12a^{2}-3a=-3a\left(4a^{2}-4a+1\right)=-3a(2a-1)^{2}.\]
 \end{esempio}

 \begin{esempio}
Scomporre in fattori~$\frac{3}{8}a^{2}+3ab+6b^{2}$.

\[\frac{3}{8}a^{2}+3ab+6b^{2}=\frac{3}{2}\left(\frac{1}{4}a^{2}+2ab+4b^{2}\right)=\frac{3}{2}\left(\frac{1}{2}a+2b\right)^{2}\text{,}\]
o anche
\[\frac{3}{8}a^{2}+3ab+6b^{2}=\frac{3}{8}\left(a^{2}+8ab+16b^{2} \right)=\frac{3}{8}\left(a+4b\right)^{2}.\]
 \end{esempio}
\end{exrig}
\ovalbox{\risolvii \ref{ese:16.1}, \ref{ese:16.2}, \ref{ese:16.3}, \ref{ese:16.4}, \ref{ese:16.5}, \ref{ese:16.6}, \ref{ese:16.7}, \ref{ese:16.8}, \ref{ese:16.9}, \ref{ese:16.10},\ref{ese:16.11}, \ref{ese:16.12}}

\section{Quadrato di un polinomio}

Se siamo in presenza di sei termini, tre dei quali sono quadrati, verifichiamo se il polinomio è il quadrato di un
trinomio secondo le seguenti regole (sezione \ref{sect:quadrato_di_un_polinomio} a pagina \pageref{sect:quadrato_di_un_polinomio})
\begin{equation*}
(A+B+C)^{2}=A^{2}+B^{2}+C^{2}+2AB+2AC+2BC
\end{equation*}
\begin{equation*}
A^{2}+B^{2}+C^{2}+2AB+2AC+2BC=(A+B+C)^{2}=(-A-B-C)^{2}.
\end{equation*}
Notiamo che i doppi prodotti possono essere tutti e tre positivi, oppure uno positivo e due negativi:
indicano se i rispettivi monomi sono concordi o discordi.
%\newpage
\begin{exrig}
 \begin{esempio}
Scomporre in fattori~$16a^{4}+b^{2}+1+8a^{2}b+8a^{2}+2b$.

I primi tre termini sono quadrati rispettivamente di~$4a^{2}$, $b$ e~$1$ e si può verificare poi che gli altri tre termini sono
i doppi prodotti:~$16a^{4}+b^{2}+1+8a^{2}b+8a^{2}+2b=\left(4a^{2}+b+1\right)^{2}$.
 \end{esempio}

 \begin{esempio}
Scomporre in fattori~$x^{4}+y^{2}+z^{2}-2x^{2}y-2x^{2}z+2yz$.
\[x^{4}+y^{2}+z^{2}-2x^{2}y-2x^{2}z+2yz=\left(x^{2}-y-z\right)^{2}=\left(-x^{2}+y+z\right)^{2}.\]
 \end{esempio}

 \begin{esempio}
Scomporre in fattori~$x^{4}-2x^{3}+3x^{2}-2x+1$.

In alcuni casi anche un polinomio di cinque termini può essere il quadrato di un trinomio.
Per far venire fuori il quadrato del trinomio si può scindere il termine~$3x^{2}$ come somma:
\[3x^{2}=x^{2}+2x^{2}.\]
In questo modo si ha:
\[x^{4}-2x^{3}+3x^{2}-2x+1=x^{4}-2x^{3}+x^{2}+2x^{2}-2x+1=(x^{2}-x+1)^{2}.\]
 \end{esempio}
\end{exrig}

Nel caso di un quadrato di un polinomio la regola è sostanzialmente la stessa:
\begin{equation*}
(A+B+C+D)^{2}=A^{2}+B^{2}+C^{2}+D^{2}+2AB+2AC+2AD+2BC+2BD+2CD.
\end{equation*}
\ovalbox{\risolvii \ref{ese:16.13}, \ref{ese:16.14}, \ref{ese:16.15}, \ref{ese:16.16}, \ref{ese:16.17}, \ref{ese:16.18}, \ref{ese:16.19}}

\section{Cubo di un binomio}
I cubi di binomi sono di solito facilmente riconoscibili. Un quadrinomio è lo sviluppo del cubo di un binomio se due suoi termini sono i cubi
di due monomi e gli altri due termini sono i tripli prodotti tra uno dei due monomi ed il quadrato dell’altro, secondo le seguenti formule (sezione \ref{sect:cubo_di_un_binomio} a pagina \pageref{sect:cubo_di_un_binomio})
\begin{equation*}
(A+B)^{3}=A^{3}+3A^{2}B+3AB^{2}+B^{3}\quad \Rightarrow \quad A^{3}+3A^{2}B+3AB^{2}+B^{3}=(A+B)^{3}\phantom{.}
\end{equation*}
\begin{equation*}
(A-B)^{3}=A^{3}-3A^{2}B+3AB^{2}-B^{3}\quad \Rightarrow \quad A^{3}-3A^{2}B+3AB^{2}-B^{3}=(A-B)^{3}.
\end{equation*}
Per il cubo non si pone il problema, come per il quadrato, del segno della base, perché un numero elevato ad esponente dispari,
se è positivo rimane positivo e se è negativo rimane negativo.

\begin{exrig}
 \begin{esempio}
Scomporre in fattori~$8a^{3}+12a^{2}b+6{ab}^{2}+b^{3}$.

Notiamo che il primo ed il quarto termine sono cubi, rispettivamente di~$2a$ e di~$b$, il secondo termine è il triplo
prodotto tra il quadrato di~$2a$ e~$b$, mentre il terzo termine è il triplo prodotto tra~$2a$ e il quadrato di~$b$.
Abbiamo dunque:
\[8a^{3}+12a^{2}b+6ab^{2}+b^{3}=(2a)^{3}+3\cdot (2a)^{2}\cdot (b)+3\cdot (2a)\cdot (b)^{2}=(2a+b)^{3}.\]
 \end{esempio}

 \begin{esempio}
Scomporre in fattori~$-27x^{3}+27x^{2}-9x+1$.

Le basi del cubo sono il primo e il quarto termine, rispettivamente cubi di~$-3x$ e di~$1$.
Dunque:
\[-27x^{3}+27x^{2}-9x+1=(-3x)^{3}+3\cdot (-3x)^{2}\cdot 1+3\cdot (-3x)\cdot 1^{2}+1=(-3x+1)^{3}.\]
 \end{esempio}

 \begin{esempio}
Scomporre in fattori~$x^{6}-x^{4}+\frac{1}{3}x^{2}-\frac{1}{27}$.

Le basi del cubo sono~$x^{2}$ e~$-\frac{1}{3}$ i termini centrali sono i tripli prodotti,
quindi~$\left(x^{2}-\frac{1}{3}\right)^{3}$.
\end{esempio}
\end{exrig}
\ovalbox{\risolvii \ref{ese:16.20}, \ref{ese:16.21}, \ref{ese:16.22}, \ref{ese:16.23}, \ref{ese:16.24},\ref{ese:16.25}, \ref{ese:16.26}, \ref{ese:16.27}, \ref{ese:16.28}}

\section{Differenza di due quadrati}
Un binomio che sia la differenza dei quadrati di due monomi può essere scomposto come prodotto tra la somma dei due monomi
(basi dei quadrati) e la loro differenza (sezione \ref{sect:diffrenza_di_quadrati} a pagina \pageref{sect:diffrenza_di_quadrati})
\begin{equation*}
(A+B)\cdot (A-B)=A^{2}-B^{2}\quad \Rightarrow \quad A^{2}-B^{2}=(A+B)\cdot (A-B).
\end{equation*}

\begin{exrig}
 \begin{esempio}
Scomporre in fattori~$\frac{4}{9}a^{4}-25b^{2}$.
\[\frac{4}{9}a^{4}-25b^{2}=\left(\frac{2}{3}a^{2}\right)^{2}-\left(5b\right)^{2}=\left(\frac{2}{3}a^{2}+5b\right)
\cdot \left(\frac{2}{3}a^{{2}}-5b\right).\]
 \end{esempio}

 \begin{esempio}
Scomporre in fattori~$-x^{6}+16y^{2}$.
\[-x^{6}+16y^{2}=-\left(x^{3}\right)^{2}+\left(4y\right)^{2}=\left(x^{3}+4y\right)\cdot \left(-x^{3}+4y\right).\]
 \end{esempio}

 \begin{esempio}
Scomporre in fattori~$a^{2}-\left(x+1\right)^{2}$.
La formula precedente vale anche se~$A$ e~$B$ sono polinomi. Quindi $a^{2}-\left(x+1\right)^{2}=\left[a+(x+1)\right]\cdot \left[a-(x+1)\right]=(a+x+1)(a-x-1)$.
\end{esempio}

 \begin{esempio}
Scomporre in fattori~$\left(2a-b^{2}\right)^{2}-(4x)^{2}$.
\[\left(2a-b^{2}\right)^{2}-(4x)^{2}=\left(2a-b^{2}+4x\right)\cdot \left(2a-b^{2}-4x\right).\]
 \end{esempio}

 \begin{esempio}
Scomporre in fattori~$(a+3b)^{2}-(2x-5)^{2}$.
\[(a+3b)^{2}-(2x-5)^{2}=(a+3b+2x-5)\cdot (a+3b-2x+5).\]
 \end{esempio}
\end{exrig}

Per questo tipo di scomposizioni, la cosa più difficile è riuscire a riconoscere un quadrinomio o un polinomio
di sei termini come differenza di quadrati. Riportiamo i casi principali:
\begin{itemize*}
 \item $(A+B)^{2}-C^{2}=A^{{2}}+2AB+B^{2}-C^{2}$;
 \item $A^{2}-(B+C)^{2}=A^{2}-B^{2}-2BC-C^{2}$;
 \item $(A+B)^{2}-(C+D)^{2}=A^{2}+2AB+B^{2}-C^{2}-2CD-D^{2}$.
\end{itemize*}

\begin{exrig}
 \begin{esempio}
Scomporre in fattori~$4a^{2}-4b^{2}-c^{2}+4bc$.

Gli ultimi tre termini possono essere raggruppati per formare il quadrati di un binomio.
 \begin{equation*}
   \begin{split}
     4a^{2}-4b^{2}-c^{2}+4bc &=4a^{2}-\left(4b^{2}+c^{2}-4bc\right) \\
                 &= (2a)^{2}-(2b-c)^{2}=(2a+2b-c)\cdot (2a-2b+c).
   \end{split}
  \end{equation*}
 \end{esempio}

 \begin{esempio}
Scomporre in fattori~$4x^{4}-4x^{2}-y^{2}+1$.
\[4x^{4}-4x^{2}-y^{2}+1=\left(2x^{2}-1\right)^{2}-(y)^{2}=(2x^{2}-1+y)\cdot (2x^{2}-1-y).\]
 \end{esempio}

 \begin{esempio}
Scomporre in fattori~$a^{2}+1+2a+6bc-b^{2}-9c^{2}$.
 \begin{equation*}
   \begin{split}
     a^{2}+1+2a+6bc-b^{2}-9c^{2} &=\left(a^{2}+1+2a\right)-\left(b^{2}+9c^{2}-6{bc}\right) \\
                 &= (a+1)^{2}-(b-3c)^{2}=(a+1+b-3c)\cdot (a+1-b+3c).
   \end{split}
  \end{equation*}
 \end{esempio}
\end{exrig}
\ovalbox{\risolvii \ref{ese:16.29}, \ref{ese:16.30}, \ref{ese:16.31}, \ref{ese:16.32}, \ref{ese:16.33}, \ref{ese:16.34}, \ref{ese:16.35}, \ref{ese:16.36},%
\ref{ese:16.37}, \ref{ese:16.38}, \ref{ese:16.39}}

\vspazio\ovalbox{\ref{ese:16.40}, \ref{ese:16.41}}
\newpage
% (c) 2012 Claudio Carboncini - claudio.carboncini@gmail.com
% (c) 2012-2014 Dimitrios Vrettos - d.vrettos@gmail.com
\section{Esercizi}
\subsection{Esercizi dei singoli paragrafi}
\subsubsection*{\thechapter.1 - Quadrato di un binomio}
\begin{multicols}{2}
\begin{esercizio}
\label{ese:16.1}
Quando è possibile, scomponi in fattori, riconoscendo il quadrato di un binomio.
\begin{enumeratea}
 \item $a^{2}-2a+1$;
 \item $x^{2}+4x+4$;
 \item $y^{2}-6y+9$;
 \item $16t^{2}+8t+1$;
 \item $4x^{2}+1+4x$;
 \item $9a^{2}-6a+1$.
\end{enumeratea}
\end{esercizio}

\begin{esercizio}
\label{ese:16.2}
Quando è possibile, scomponi in fattori, riconoscendo il quadrato di un binomio.
\begin{enumeratea}
 \item $4x^{2}-12x+9$;
 \item $\dfrac{1}{4}a^{2}+ab+b^{2}$;
 \item $9x^{2}+4+12x$;
 \item $\dfrac{4}{9}a^{{4}}-4a^{2}+9$;
 \item $\dfrac{1}{4}x^{2}-\dfrac{1}{3}x+\dfrac{1}{9}$;
 \item $16a^{2}+\dfrac{1}{4}b^{2}-4ab$.
\end{enumeratea}
\end{esercizio}

\begin{esercizio}
\label{ese:16.3}
Quando è possibile, scomponi in fattori, riconoscendo il quadrato di un binomio.
\begin{enumeratea}
 \item $-9x^{2}-\dfrac{1}{4}+3x$;
 \item $4x^{2}+4xy+y^{2}$;
 \item $a^{4}+36a^{2}+12a^{3}$;
 \item $144x^{2}-6xa^{2}+\dfrac{1}{16}a^{4}$;
 \item $x^{2}-6xy+9y^{2}$;
 \item $-x^{2}-6xy-9y^{2}$.
\end{enumeratea}
\end{esercizio}

\begin{esercizio}
\label{ese:16.4}
Quando è possibile, scomponi in fattori, riconoscendo il quadrato di un binomio.
\begin{enumeratea}
 \item $25+10x+x^{2}$;
 \item $\dfrac{1}{4}x^{2}+\dfrac{1}{3}xy+\dfrac{1}{9}y^{2}$;
 \item $25-10x+x^{2}$;
 \item $\dfrac{9}{25}a^{4}-6a^{2}+25$;
 \item $4x^{2}+2x^{4}+1$;
 \item $4x^{2}-4x^{4}-1$.
\end{enumeratea}
\end{esercizio}

\begin{esercizio}
\label{ese:16.5}
Quando è possibile, scomponi in fattori, riconoscendo il quadrato di un binomio.
\begin{enumeratea}
 \item $-a^{3}-2a^{2}-a$;
 \item $3a^{7}b-6a^{5}b^{2}+3a^{3}b^{3}$;
 \item $100+a^{2}b^{4}+20ab^{2}$;
 \item $2x^{13}-8x^{8}y+8x^{3}y^{2}$;
 \item $x^{8}+8x^{4}y^{2}+16y^{4}$;
 \item $-x^{2}+6{xy}+9y^{2}$.
\end{enumeratea}
\end{esercizio}

\begin{esercizio}
\label{ese:16.6}
Quando è possibile, scomponi in fattori, riconoscendo il quadrato di un binomio.
\begin{enumeratea}
 \item $4a^{2}b^{4}-12ab^{3}+9b^{6}$;
 \item $a^{2}+a+1$;
 \item $36a^{6}b^{3}+27a^{5}b^{4}+12a^{7}b^{2}$;
 \item $25x^{14}+9y^{6}+30x^{7}y^{3}$;
 \item $-a^{7}-25a^{5}+10a^{6}$;
 \item $25a^{2}+49b^{2}+35ab$.
\end{enumeratea}
\end{esercizio}

\begin{esercizio}
\label{ese:16.7}
Quando è possibile, scomponi in fattori, riconoscendo il quadrato di un binomio.
\begin{enumeratea}
 \item $4y^{6}+4-4y^{2}$;
 \item $\dfrac{1}{4}a^{2}+2ab+b^{2}$;
 \item $25a^{2}-10{ax}-x^{2}$;
 \item $9x^{2}+4y^{2}-6{xy}$.
\end{enumeratea}
\end{esercizio}
\end{multicols}

\begin{esercizio}
\label{ese:16.8}
Individua perché i seguenti polinomi non sono quadrati di un binomio.
\begin{enumeratea}
 \item $4x^{2}+4xy-y^{2}$\, non è un quadrato di binomio perché\,\dotfill;
 \item $x^{2}-6xy+9y$\, non è un quadrato di binomio perché\dotfill;
 \item $25+100x+x^{2}$\, non è un quadrato di binomio perché\dotfill;
 \item $\dfrac{1}{4}x^{2}+\dfrac{2}{3}xy+\dfrac{1}{9}$\, non è un quadrato di binomio perché\dotfill;
 \item $25t^{2}+4-10t$\, non è un quadrato di binomio perché\dotfill%ex103
\end{enumeratea}
\end{esercizio}

\begin{multicols}{2}
\begin{esercizio}[\Ast]
\label{ese:16.9}
Quando è possibile, scomponi in fattori, riconoscendo il quadrato di un binomio.
\begin{enumeratea}
 \item $24a^{3}+6a+24a^{2}$;
 \item $3a^{2}x-12axb+12b^{2}x$;
 \item $5a^{2}+2ax+\dfrac{1}{5}x^{2}$;
 \item $x^{6}y+x^{2}y+2x^{4}y$.
\end{enumeratea}
\end{esercizio}

\begin{esercizio}[\Ast]
\label{ese:16.10}
Quando è possibile, scomponi in fattori, riconoscendo il quadrato di un binomio.
\begin{enumeratea}
 \item $x^{5}+4x^{4}+4x^{3}$;
 \item $2y^{3}-12y^{2}x+18x^{2}y$;
 \item $-50t^{3}-8t+40t^{2}$;
 \item $2^{10}x^{2}+2^{6}\cdot 3^{20}+3^{40}$.
\end{enumeratea}
\end{esercizio}

\begin{esercizio}[\Ast]
\label{ese:16.11}
Quando è possibile, scomponi in fattori, riconoscendo il quadrato di un binomio.
\begin{enumeratea}
 \item $2^{20}x^{40}-2^{26}\cdot x^{50}+2^{30}\cdot x^{60}$;
 \item $10^{100}x^{50}-2\cdot 10^{75}x^{25}+10^{50}$.
\end{enumeratea}
\end{esercizio}

\begin{esercizio}[\Ast]
\label{ese:16.12}
Quando è possibile, scomponi in fattori, riconoscendo il quadrato di un binomio.
\begin{enumeratea}
 \item $10^{11}x^{10}-2\cdot 10^{9}x^{5}+10^{6}$;
 \item $x^{2n}+2x^{n}+1$.
\end{enumeratea}
\end{esercizio}
\end{multicols}

\subsubsection*{\thechapter.2 - Quadrato di un polinomio}
\begin{multicols}{2}
\begin{esercizio}
\label{ese:16.13}
Quando è possibile, scomponi in fattori, riconoscendo il quadrato di un polinomio.
\begin{enumeratea}
 \item $a^{2}+b^{2}+c^{2}+2ab+2ac+2bc$;
 \item $x^{2}+y^{2}+z^{2}+2xy-2xz-2yz$;
 \item $x^{2}+y^{2}+4+4x+2xy+4y$;
 \item $4a^{4}-6{ab}-4a^{2}b+12a^{3}+b^{2}+9a^{2}$.
\end{enumeratea}
\end{esercizio}

\begin{esercizio}
Quando è possibile, scomponi in fattori, riconoscendo il quadrato di un polinomio.
\label{ese:16.14}
\begin{enumeratea}
 \item $9x^{6}+2y^{2}z+y^{4}-6x^{3}z-6x^{3}y^{2}+z^{2}$;
 \item $\dfrac{1}{4}a^{2}+b^{4}+c^{6}+ab^{2}+{ac}^{3}+2b^{2}c^{3}$;
 \item $a^{2}+2ab+b^{2}-2a+1-2b$;
 \item $x^{2}+\dfrac{1}{4}y^{2}+4-xy+4x-2y$.
\end{enumeratea}
\end{esercizio}

\begin{esercizio}
\label{ese:16.15}
Quando è possibile, scomponi in fattori, riconoscendo il quadrato di un polinomio.
\begin{enumeratea}
 \item $a^{2}+b^{2}+c^{2}-2ac-2bc+2ab$;
 \item $-x^{2}-2xy-9-y^{2}+6x+6y$;
 \item $4a^{2}+4ab-8a+b^{2}-4b+4$;
 \item $a^{2}b^{2}+2a^{2}b+a^{2}-2ab^{2}-2ab+b^{2}$.
\end{enumeratea}
\end{esercizio}
\end{multicols}
\begin{esercizio}
Individua perché i seguenti polinomi non sono quadrati.
\label{ese:16.16}
\begin{enumeratea}
 \item $a^{2}+b^{2}+c^{2}$\, non è un quadrato perché\dotfill;
 \item $x^{2}+y^{2}+4+4x+4xy+4y$\, non è un quadrato perché\dotfill;
 \item $a^{2}+b^{2}+c^{2}-2ac-2bc-2ab$\, non è un quadrato perché\dotfill;
 \item $a^{2}+b^{2}-1-2a-2b+2ab$\, non è un quadrato perché\dotfill
\end{enumeratea}
\end{esercizio}

\begin{esercizio}[\Ast]
\label{ese:16.17}
Quando è possibile, scomponi in fattori, riconoscendo il quadrato di un polinomio.
\begin{enumeratea}
 \item $a^{2}+4ab-2a+4b^{2}-4b+1$;
 \item $a^{2}b^{2}+2a^{2}b+a^{2}+4ab^{2}+4ab+4b^{2}$;
 \item $x^{2}-6xy+6x+9y^{2}-18y+9$.
\end{enumeratea}
\end{esercizio}

\begin{esercizio}
\label{ese:16.18}
Quando è possibile, scomponi in fattori, riconoscendo il quadrato di un polinomio.
\begin{enumeratea}
 \item $x^{4}+2x^{3}+3x^{2}+2x+1$\quad  scomponi prima \quad~$3x^{2}=x^{2}+2x^{2}$;
 \item $4a^{4}+8a^{2}+1+8a^{3}+4a$\quad  scomponi prima \quad~$8a^{2}=4a^{2}+4a^{2}$;
 \item $9x^{4}+6x^{3}-11x^{2}-4x+4$ \quad scomponi in maniera opportuna \quad~$-11x^{2}$.
\end{enumeratea}
\end{esercizio}

\begin{esercizio}
\label{ese:16.19}
Quando è possibile, scomponi in fattori, riconoscendo il quadrato di un polinomio.
\begin{enumeratea}
 \item $25x^{2}-20ax-30bx+4a^{2}+12ab+9b^{2}$;
 \item $2a^{10}x+4a^{8}x+2a^{6}x+4a^{5}x+4a^{3}x+2x$;
 \item $a^{2}+b^{2}+c^{2}+d^{2}-2ab+2ac-2ad-2bc+2bd-2cd$;
 \item $x^{6}+x^{4}+x^{2}+1+2x^{5}+2x^{4}+2x^{3}+2x^{3}+2x^{2}+2x$.
\end{enumeratea}
\end{esercizio}

\subsubsection*{\thechapter.3 - Cubo di un binomio}
\begin{multicols}{2}
\begin{esercizio}
\label{ese:16.20}
Quando è possibile, scomponi in fattori, riconoscendo il cubo di un binomio.
\begin{enumeratea}
 \item $8a^{3}+b^{3}+12a^{2}b+6ab^{2}$;
 \item $b^{3}+12a^{2}b-6ab^{2}-8a^{3}$;
 \item $-12a^{2}+8a^{3}-b^{3}+6ab$;
 \item $-12a^{2}b+6ab+8a^{3}-b^{3}$.
\end{enumeratea}
\end{esercizio}

\begin{esercizio}
\label{ese:16.21}
Quando è possibile, scomponi in fattori, riconoscendo il cubo di un binomio.
\begin{enumeratea}
 \item $-x^{3}+6x^{2}-12x+8$;
 \item $-x^{9}-3x^{6}+3x^{3}+8$;
 \item $x^{3}y^{6}+1+3x^{2}y^{2}+3xy^{2}$;
 \item $x^{3}+3x-3x^{2}-1$.
\end{enumeratea}
\end{esercizio}

\begin{esercizio}
\label{ese:16.22}
Quando è possibile, scomponi in fattori, riconoscendo il cubo di un binomio.
\begin{enumeratea}
 \item $-5x^{5}y^{3}-5x^{2}-15x^{4}y^{2}-15x^{3}y$;
 \item $-a^{6}+27a^{3}+9a^{5}-27a^{4}$;
 \item $64a^{3}-48a^{2}+12a-1$;
 \item $a^{6}+9a^{4}+27a^{2}+27$.
\end{enumeratea}
\end{esercizio}

\begin{esercizio}
\label{ese:16.23}
Quando è possibile, scomponi in fattori, riconoscendo il cubo di un binomio.
\begin{enumeratea}
 \item $x^{3}-x^{2}+\dfrac{1}{3}x-\dfrac{1}{27}$;
 \item $0,001x^{6}+0,015x^{4}+0,075x^{2}+0,125$;
 \item $\dfrac{27}{8}a^{3}-\dfrac{27}{2}a^{2}x+18ax^{2}-8x^{3}$;
 \item $x^{3}-x^{2}+\dfrac{1}{3}x-\dfrac{1}{27}$.
\end{enumeratea}
\end{esercizio}

\begin{esercizio}
\label{ese:16.24}
Individua perché i seguenti polinomi non sono cubi.
\begin{enumeratea}
 \item $a^{10}-8a-6a^{7}+12a^{4}$\, non è un cubo perché\dotfill;
 \item $27a^{3}-b^{3}+9a^{2}b-9ab^{2}$\, non è un cubo perché\dotfill;
 \item $8x^{3}+b^{3}+6x^{2}b+6{xb}^{2}$\, non è un cubo perché\dotfill;
 \item $x^{3}+6ax^{2}-6a^{2}x+8a^{3}$\, non è un cubo perché\dotfill
\end{enumeratea}
\end{esercizio}

\begin{esercizio}
\label{ese:16.25}
Quando è possibile, scomponi in fattori, riconoscendo il cubo di un binomio.
\begin{enumeratea}
 \item $x^{3}-6x^{2}+12x-8$;
 \item $a^{3}b^{3}+12ab+48ab+64$;
 \item $216x^{3}-540ax^{2}+450a^{2}x-125a^{3}$;
 \item $8x^{3}+12x^{2}+6x+2$.
\end{enumeratea}
\end{esercizio}

\begin{esercizio}[\Ast]
\label{ese:16.26}
Quando è possibile, scomponi in fattori, riconoscendo il cubo di un binomio.
\begin{enumeratea}
 \item $a^{6}+3a^{4}b^{2}+3a^{2}b^{4}+b^{6}$;
 \item $8a^{3}-36a^{2}b+54ab^{2}-27b^{3}$;
 \item $a^{6}+3a^{5}+3a^{4}+a^{3}$;
 \item $a^{10}-8a-6a^{7}+12a^{4}$.%ex154b trovato risultato: a\left(a^3-2\right)^3
\end{enumeratea}
\end{esercizio}

\begin{esercizio}
\label{ese:16.27}
Quando è possibile, scomponi in fattori, riconoscendo il cubo di un binomio.
\begin{enumeratea}
 \item $8x^{3}-36x^{2}+54x-27$;
 \item $x^{6}+12ax^{4}+12a^{2}x^{2}+8a^{3}$;
 \item $x^{300}-10^{15}-3\cdot 10^{5}x^{200}+3\cdot 10^{10}x^{100}$;
 \item $a^{6n}+3a^{4n}x^{n}+3a^{2n}x^{2n}+x^{3n}$.
\end{enumeratea}
\end{esercizio}
\end{multicols}
\begin{esercizio}
\label{ese:16.28}
Quando è possibile, scomponi in fattori, riconoscendo il cubo di un binomio.
\begin{enumeratea}
 \item $10^{15}a^{60}+3\cdot 10^{30}a^{45}+3\cdot 10^{45}a^{30}+10^{60}a^{15}$;
 \item $10^{-33}x^{3}-3\cdot 10^{-22}x^{2}+3\cdot 10^{-11}x-1$.
\end{enumeratea}
\end{esercizio}

\subsubsection*{\thechapter.4 - Differenza di due quadrati}
\begin{multicols}{2}
\begin{esercizio}
\label{ese:16.29}
Scomponi i seguenti polinomi come differenza di quadrati.
\begin{enumeratea}
 \item $a^{2}-25b^{2}$;
 \item $16-x^{2}y^{2}$;
 \item $25-9x^{2}$;
 \item $4a^{4}-9b^{2}$;
 \item $x^{2}-16y^{2}$;
 \item $144x^{2}-9y^{2}$.
\end{enumeratea}
\end{esercizio}

\begin{esercizio}
\label{ese:16.30}
Scomponi i seguenti polinomi come differenza di quadrati.
\begin{enumeratea}
 \item $16x^{4}-81z^{2}$;
 \item $a^{2}b^{4}-c^{2}$;
 \item $4x^{6}-9y^{4}$;
 \item $-36x^{8}+25b^{2}$;
 \item $-1+a^{2}$;
 \item $\dfrac{1}{4}x^{4}-\dfrac{1}{9}y^{4}$.
\end{enumeratea}
\end{esercizio}

\begin{esercizio}
\label{ese:16.31}
Scomponi i seguenti polinomi come differenza di quadrati.
\begin{enumeratea}
 \item $\dfrac{a^{2}}{4}-\dfrac{y^{2}}{9}$;
 \item $2a^{2}-50$;
 \item $a^{3}-16{ab}^{6}$;
 \item $-4x^{2}y^{2}+y^{2}$;
 \item $-4a^{2}+b^{2}$;
 \item $25x^{2}y^{2}-\dfrac{1}{4}z^{6}$.
\end{enumeratea}
\end{esercizio}

\begin{esercizio}
\label{ese:16.32}
Scomponi i seguenti polinomi come differenza di quadrati.
\begin{enumeratea}
 \item $-a^{2}b^{4}+49$;
 \item $16y^{4}-z^{4}$;
 \item $a^{8}-b^{8}$;
 \item $a^{4}-16$;
 \item $16a^{2}-9b^{2}$;
 \item $9-4x^{2}$.
\end{enumeratea}
\end{esercizio}

\begin{esercizio}
\label{ese:16.33}
Scomponi i seguenti polinomi come differenza di quadrati.
\begin{enumeratea}
 \item $\dfrac{1}{4}x^{2}-1$;
 \item $a^{2}-9b^{2}$;
 \item $\dfrac{25}{16}a^{2}-1$;
 \item $-16+25x^{2}$;
 \item $25a^{2}b^{2}-\dfrac{9}{16}y^{6}$;
 \item $-4x^{8}+y^{12}$.
\end{enumeratea}
\end{esercizio}

\begin{esercizio}
\label{ese:16.34}
Scomponi i seguenti polinomi come differenza di quadrati.
\begin{enumeratea}
 \item $\dfrac{1}{4}x^{2}-0,01y^{4}$;
 \item $x^{6}-y^{8}$;
 \item $x^{4}-y^{8}$.
\end{enumeratea}
\end{esercizio}

\begin{esercizio}[\Ast]
\label{ese:16.35}
Quando è possibile, scomponi in fattori, riconoscendo la differenza di due quadrati.
\begin{enumeratea}
 \item $(b+3)^{2}-x^{2}$;
 \item $a^{8}-(b-1)^{2}$;
 \item $(x-1)^{2}-a^{2}$.
\end{enumeratea}
\end{esercizio}

\begin{esercizio}
\label{ese:16.36}
Quando è possibile, scomponi in fattori, riconoscendo la differenza di due quadrati.
\begin{enumeratea}
 \item $(x-y)^{2}-(y+z)^{2}$;
 \item $-(2a-1)^{2}+(3b+3)^{2}$;
 \item $x^{2}-b^{2}-9-6b$.
\end{enumeratea}
\end{esercizio}

\begin{esercizio}[\Ast]
\label{ese:16.37}
Quando è possibile, scomponi in fattori, riconoscendo la differenza di due quadrati.
\begin{enumeratea}
 \item $(2x-3)^{2}-9y^{2}$;
 \item $(x+1)^{2}-(y-1)^{2}$;
 \item $x^{2}+2x+1-y^{2}$.
\end{enumeratea}
\end{esercizio}

\begin{esercizio}
\label{ese:16.38}
Quando è possibile, scomponi in fattori, riconoscendo la differenza di due quadrati.
\begin{enumeratea}
 \item $b^{2}-x^{4}+1+2b$;
 \item $a^{4}+4a^{2}+4-y^{2}$;
 \item $x^{2}-y^{2}-1+2y$.
\end{enumeratea}
\end{esercizio}

\begin{esercizio}[\Ast]
\label{ese:16.39}
Quando è possibile, scomponi in fattori, riconoscendo la differenza di due quadrati.
\begin{enumeratea}
 \item $(2x+3)^{2}-(2y+1)^{2}$;
 \item $a^{2}-2{ab}+b^{2}-4$;
 \item $(2x-3a)^{2}-(x-a)^{2}$.
\end{enumeratea}
\end{esercizio}

\begin{esercizio}
\label{ese:16.40}
Quando è possibile, scomponi in fattori, riconoscendo la differenza di due quadrati.
\begin{enumeratea}
 \item $-(a+1)^{2}+9$;
 \item $16x^{2}y^{6}-(xy^{3}+1)^{2}$;
 \item $a^{2}+1+2a-9$;
 \item $x^{2}y^{4}-z^{2}+9+6xy^{2}$.
\end{enumeratea}
\end{esercizio}

\begin{esercizio}[\Ast]
\label{ese:16.41}
Quando è possibile, scomponi in fattori, riconoscendo la differenza di due quadrati.
\begin{enumeratea}
 \item $a^{2}-6a+9-x^{2}-16-8x$;
 \item $x^{2}+25+10x-y^{2}+10y-25$.
\end{enumeratea}
\end{esercizio}

\begin{esercizio}
\label{ese:16.42}
Quando è possibile, scomponi in fattori, riconoscendo la differenza di due quadrati.
\begin{enumeratea}
 \item $(a-1)^{2}-(a+1)^{2}$;
 \item $a^{2n}-4$;
 \item $a^{2m}-b^{2n}$;
 \item $x^{2}n-y^{4}$.
\end{enumeratea}
\end{esercizio}
\end{multicols}
\subsection{Risposte}

\paragraph{\thechapter.9}
a)~$6a(2a+1)^{2}$,\quad b)~$3x(a-2b)^{2}$, \quad c)~$\dfrac{1}{5}(x+5a)^{2}$, \quad d)~$x^{2}y\left(x^{2}+1\right)^{2}$.

\paragraph{\thechapter.10}
a)~$x^{3}(x+2)^{2}$,\quad b)~$2y(3x-y)^{2}$, \quad c)~$-2t(5t-2)^{2}$, \quad d)~$\left(2^{5}x+3^{20}\right)^{2}$.

\paragraph{\thechapter.11}
a)~$2^{20} x^{40}\left(1-2^{5}x^{10} \right)^2$, \quad b)~$10^{50}\left(10^{25} x^{25}-1 \right)^2$.

\paragraph{\thechapter.12}
a)~$10^{6} \left(10^{5} x^{10}-2 \cdot 10^{3}x^{5}+1\right)$,\quad b)~$\left(x^{n}+1\right)^2$.

\paragraph{\thechapter.17}
a)~$(a+2b-1)^{2}$,\quad b)~$(ab+a+2b)^{2}$, \quad c)~$(x-3y+3)^{2}$.

\paragraph{\thechapter.26}
a)~$\left(a^{2}+b^{2}\right)^{3}$,\quad b)~$(2a-3b)^{3}$,\quad c)~$a^{3}(a+1)^{3}$,\quad d)~$a\left(a^3-2\right)^3$.

\paragraph{\thechapter.35}
a)~$(b+3-x)(b+3+x)$,\quad b)~$(a^{4}-b+1)(a^{4}+b-1)$, \quad c)~$(x+a-1)(x-a-1)$.

\paragraph{\thechapter.37}
a)~$(2x+3y-3)(2x-3y-3)$,\quad b)~$(x+y)(x-y+2)$, \quad c)~$(x+y+1)(x-y+1)$.

\paragraph{\thechapter.39}
a)~$4(x+y+2)(x-y+1)$,\quad b)~$(a-b-2)(a-b+2)$, \quad c)~$(3x-4a)(x-2a)$.

\paragraph{\thechapter.41}
a)~$-(x+a+1)(x-a+7)$,\quad b)~$(x+y)(x-y+10)$.

\cleardoublepage
