% (c) 2012 Silvia Cibola - silvia.cibola@gmail.com
% (c) 2012-2014 Dimitrios Vrettos - d.vrettos@gmail.com

\section{Esercizi}

\subsection{Esercizi dei singoli paragrafi}

\subsubsection*{\thechapter.2 - Rappresentazione di una corrispondenza}

\begin{esercizio}
\label{ese:C.1}
Rappresenta con un grafico cartesiano la corrispondenza~$\Kor$: ``essere nato nell'anno'' di dominio l'insieme~$A=\{$Galileo, Napoleone, Einstein, Fermi, Obama$\}$
e codominio l'insieme~$B=\{$1901, 1564, 1961, 1879, 1769, 1920, 1768$\}$. Rappresenta per elencazione il sottoinsieme~$G_\Kor$ del prodotto cartesiano~$A \times B$.
Stabilisci infine gli elementi dell'immagine~$\IM$.
\end{esercizio}

\begin{esercizio}
\label{ese:C.2}
L'insieme~$S=\{$casa, volume, strada, ufficio, clavicembalo, cantautore, assicurazione$\}$ è il codominio della corrispondenza~$\Kor$: ``essere il numero di sillabe di'' il cui dominio
è~$X=\{x \in\insN \mid  0<x<10\}$. Rappresenta con un grafico cartesiano la corrispondenza assegnata, evidenzia, come nell'esempio~\ref{ex:C.1} a pagina~\pageref{ex:C.1}, l'insieme~$G_\Kor$,
scrivi per elencazione l'insieme~$\IM$.
\end{esercizio}

\begin{esercizio}
\label{ese:C.3}
Completa la rappresentazione con grafico sagittale della corrispondenza $\Kor$ ``essere capitale di''. La freccia che collega gli elementi del dominio $\Dom$ con quelli del codominio $\Cod$ rappresenta
il predicato~$\Kor$.
\begin{center}
 % (c) 2012 Dimitrios Vrettos - d.vrettos@gmail.com
\begin{tikzpicture}[x=10mm, y=10mm]

\node[ellipse, minimum height=3cm,draw, minimum width=4cm] (D) at (0,0) {};

\node[above] (D1) at (D.north) {$\Dom$};

\begin{scope}[fill=CornflowerBlue]

\filldraw (.7,1) circle (2pt) node (a) {};
\node[left] at (.7,1) {Roma};
\filldraw (1,.2) circle (2pt) node (b) {};
\node[left] at (1,.2) {Parigi};

\filldraw (-1.3,-.5) circle (2pt) node (c) {};
\end{scope}

\begin{scope}[xshift=5cm]
\node[ellipse, minimum height=3cm,draw, minimum width=4cm] (C) at (0,0) {};

\node[above] (C1) at (C.north) {$\Cod$};

\begin{scope}[fill=LimeGreen]
\filldraw (-.1,1) circle (2pt) node (a1) {};
\filldraw (-.2,.2) circle (2pt)node (b1) {};
\filldraw (.2,-.8) circle (2pt) node (c1) {};

\node[right]  at (-.1,1) {Francia};
\node[right]  at (.2,-.8) {Italia};
\node[right] at (-.2,.2) {Grecia};

\end{scope}
\end{scope}
\begin{scope}[->,smooth,thick]
\draw[red] (c) .. controls +(-30:2cm) and +(-180:2cm) .. (b1);
\end{scope}
\end{tikzpicture}
\end{center}

\end{esercizio}

\subsubsection*{\thechapter.3 - Caratteristiche di una corrispondenza}
\begin{esercizio}
\label{ese:C.4}
\`E univoca la corrispondenza~$\Kor$ definita tra l'insieme~$P= \{$parola del proverbio ``rosso di sera, bel tempo si spera''$\}$ e l'insieme~$A=\{$lettere dell'alfabeto italiano$\}$
che associa ad ogni parola la sua iniziale? Ti sembra corretto affermare che dominio $\Dom$ e insieme di definizione $\ID$ coincidono? Completa con il simbolo corretto
la relazione tra insieme immagine e codominio:~$\IM\ldots\Cod$. Fai il grafico sagittale della corrispondenza.
\end{esercizio}

\begin{esercizio}
\label{ese:C.5}
$\Kor$ è la corrispondenza tra l'insieme ~$\insN$ dei naturali e l'insieme degli interi relativi~$\insZ$ espressa dal predicato ``essere il quadrato di''. Ti sembra corretto affermare che
dominio $\Dom$ e insieme di definizione $\ID$ coincidono? Perché~$\IM=\Cod$? La corrispondenza è univoca?
\end{esercizio}

\pagebreak

\begin{esercizio}
\label{ese:C.6}
Una corrispondenza~$\Kor$ è assegnata con il suo grafico cartesiano.
\begin{center}
 % (c) 2012 Dimitrios Vrettos - d.vrettos@gmail.com
\begin{tikzpicture}[x=10mm, y=10mm]

\begin{scope}[->]
\draw (-.5,0) -- (12,0);
\draw (0,-.5) -- (0,7);
\end{scope}

\foreach \x in {1,2,...,11}
\draw (\x,1.5pt) -- (\x,-1.5pt);

\foreach \y in {1,2,...,6}
\draw (1.5pt,\y) -- (-1.5pt,\y);

\foreach \xi/\xtext in {1/A,2/B,3/C,4/D,5/E,6/F,7/G,8/H,9/I,10/L,11/M}
\node[below] at (\xi,0) {$\xtext$};

\foreach \yi/\ytext in {1/1,2/2,3/3,4/4,5/5,6/6}
\node[left] at (0,\yi){\ytext};

\draw[orange, dotted] (0,0) grid (11,6);

\begin{scope}[fill=CornflowerBlue]
\foreach \x in {1,6}
\filldraw (\x,1) circle (2pt);

\foreach \x in {2,5,8}
\filldraw (\x,2) circle (2pt);

\foreach \x in {4,10}
\filldraw (\x,4) circle (2pt);

\filldraw (3,5) circle (2pt);
\filldraw (11,6) circle (2pt);
\end{scope}
\end{tikzpicture}
\end{center}
Completa e rispondi alle domande:

\begin{enumeratea}
\item $\Dom=$\{\dotfill\};
\item $\Cod=$\{\dotfill\};
\item $\ID=$\{\dotfill\};
\item $\IM=$\{\dotfill\};
\item la corrispondenza è biunivoca?
\item di quali elementi dell'insieme di definizione 2 ne è l'immagine?
\item quale elemento del codominio è l'immagine di~$M$?
\end{enumeratea}
\end{esercizio}

%\newpage
\begin{esercizio}
\label{ese:C.7}
I tre grafici sagittali rappresentano altrettante corrispondenze, $\Kor_1$, $\Kor_2$, $\Kor_3$.
Completa per ciascuna di esse la descrizione schematizzata nel riquadro sottostante:
\begin{center}
 % (c) 2012 Dimitrios Vrettos - d.vrettos@gmail.com
\begin{tikzpicture}[x=10mm, y=10mm]

\node[circle, minimum height=2cm,draw] (A) at (0,0) {};

\node[above] (A1) at (A.north) {$A$};

\begin{scope}[fill=CornflowerBlue]

\filldraw (.5,.5) circle (2pt) node (a) {};
\node[left] at (.5,.5) {1};
\filldraw (.8,.2) circle (2pt) node (b) {};
\node[left] at (.8,.2) {2};
\filldraw (-.4,-.5) circle (2pt) node (c) {};
\node[left] at (-.4,-.5)  {3};
\filldraw (-.5,0) circle (2pt);
\node[left] at (-.5,0)  {4};
\filldraw (-.3,.5) circle (2pt);
\node[left] at (-.3,.5)  {5};
\end{scope}

\begin{scope}[xshift=2.3cm]
\node[circle, minimum height=2cm,draw] (B) at (0,0) {};

\node[above] (B1) at (B.north) {$B$};

\begin{scope}[fill=LimeGreen]
\filldraw (-.1,.6) circle (2pt) node (a1) {};
\filldraw (-.2,.2) circle (2pt)node (b1) {};
\filldraw (.2,-.7) circle (2pt) node (c1) {};
\filldraw(.5,-.2) circle (2pt);

\node[right]  at (-.1,.6) {$a$};
\node[right] at (-.2,.2) {$b$};
\node[right]  at (.2,-.7) {$c$};
\node[right] at (.5,-.2) {$d$};
\end{scope}
\end{scope}

\begin{scope}[->,smooth,thick]
\draw[Maroon] (a) .. controls +(30:1cm) and +(150:.5cm) .. (a1);
\draw[purple] (b) .. controls +(30:.5cm) and +(180:0.5cm) .. (b1);
\draw[orange] (c) .. controls +(30:1cm) and +(-90:1cm) .. (b1);
\draw[orange] (c) .. controls +(30:1cm) and +(-180:2cm) .. (c1);
\end{scope}

\begin{scope}[yshift=-2.5cm]
\matrix (m) [matrix of nodes]
{$\Dom=$&\ldots\\
$\Cod=$&\ldots\\
$\ID=$&\ldots\\
$\IM=$&\ldots\\
Tipo$=$&\ldots\\};
\end{scope}


\begin{scope}[xshift=4.6cm]

\node[circle, minimum height=2cm,draw] (A) at (0,0) {};

\node[above] (A1) at (A.north) {$A$};

\begin{scope}[fill=CornflowerBlue]

\filldraw (0,.7) circle (2pt) node (a) {};
\node[left] at (0,.7) {$a$};
\filldraw (.7,0) circle (2pt) node (b) {};
\node[left] at (.7,0) {$b$};
\filldraw (-.4,-.5) circle (2pt) node (c) {};
\node[left] at (-.4,-.5)  {$c$};
\end{scope}

\begin{scope}[xshift=2.3cm]
\node[circle, minimum height=2cm,draw] (B) at (0,0) {};

\node[above] (B1) at (B.north) {$B$};

\begin{scope}[fill=LimeGreen]
\filldraw (-.1,.6) circle (2pt) node (a1) {};
\filldraw (-.2,.2) circle (2pt)node (b1) {};
\filldraw (.2,-.7) circle (2pt) node (c1) {};

\node[right]  at (-.1,.6) {$m$};
\node[right] at (-.2,.2) {$n$};
\node[right]  at (.2,-.7) {$p$};
\end{scope}
\end{scope}

\begin{scope}[->,smooth,thick]
\draw[Maroon] (a) .. controls +(30:1cm) and +(180:1cm) .. (b1);
\draw[purple] (b) .. controls +(30:1cm) and +(180:1cm) .. (c1);
\draw[orange] (c) .. controls +(30:.5cm) and +(-180:2cm) .. (a1);
\end{scope}

\begin{scope}[yshift=-2.5cm]
\matrix (m) [matrix of nodes]
{$\Dom=$&\ldots\\
$\Cod=$&\ldots\\
$\ID=$&\ldots\\
$\IM=$&\ldots\\
Tipo$=$&\ldots\\};
\end{scope}
\end{scope}

\begin{scope}[xshift=9.2cm]
\node[circle, minimum height=2.cm,draw] (A) at (0,0) {};

\node[above] (A1) at (A.north) {$A$};

\begin{scope}[fill=CornflowerBlue]

\filldraw (.3,.7) circle (2pt) node (a) {};
\node[left] at (.3,.7) {1};
\filldraw (.6,.2) circle (2pt) node (b) {};
\node[left] at (.6,.2) {2};
\filldraw (-.3,-.5) circle (2pt) node (c) {};
\node[left] at (-.3,-.5)  {3};
\filldraw (-.5,0) circle (2pt) node (d){};
\node[left] at (-.5,0)  {4};

\end{scope}

\begin{scope}[xshift=2.3cm]
\node[circle, minimum height=2cm,draw] (B) at (0,0) {};

\node[above] (B1) at (B.north) {$B$};

\begin{scope}[fill=LimeGreen]
\filldraw (-.1,.6) circle (2pt) node (a1) {};
\filldraw (-.2,.2) circle (2pt)node (b1) {};
\filldraw (.1,-.8) circle (2pt) node (c1) {};
\filldraw(.5,-.1) circle (2pt) node (d1) {};
\filldraw(-.7,-.4) circle (2pt) node (e1) {};

\node[right]  at (-.1,.6) {$a$};
\node[right] at (-.2,.2) {$b$};
\node[right]  at (.1,-.8) {$c$};
\node[right] at (.5,-.1) {$d$};
\node[right] at (-.7,-.4) {$e$};
\end{scope}
\end{scope}

\begin{scope}[->,smooth,thick]
\draw[Maroon] (a) .. controls +(30:.5cm) and +(90:.5cm) .. (e1);
\draw[purple] (b) .. controls +(30:.5cm) and +(90:.5cm) .. (e1);
\draw[orange] (c) .. controls +(30:.5cm) and +(-180:2cm) .. (b1);
\draw[red] (d) .. controls +(-30:2cm) and +(-120:1cm) .. (d1);
\end{scope}

\begin{scope}[yshift=-2.5cm]
\matrix (m) [matrix of nodes]
{$\Dom=$&\ldots\\
$\Cod=$&\ldots\\
$\ID=$&\ldots\\
$\IM=$&\ldots\\
Tipo$=$&\ldots\\};
\end{scope}
\end{scope}
\end{tikzpicture}
\end{center}
\end{esercizio}
\pagebreak
\begin{esercizio}
\label{ese:C.8}
Il dominio della corrispondenza~$\Kor$ è l'insieme~$\insZ\times\insZ$ e~$\insZ$ ne è il codominio; l'immagine della coppia~$(a;b)$ è l'intero~$p=a \cdot b$.
\begin{enumeratea}
\item Stabilisci l'insieme di definizione $\ID$ e l'insieme immagine $\IM$;
\item perché questa corrispondenza non è biunivoca?
\item tutte le coppie aventi almeno un elemento uguale a zero hanno come immagine \ldots;
\item 1 è l'immagine di \ldots;
\item se gli elementi della coppia sono numeri concordi, allora l'immagine è \ldots;
\item un numero negativo è immagine di \ldots
\end{enumeratea}
Fai degli esempi che illustrino le tue affermazioni precedenti.
\end{esercizio}

\begin{esercizio}
\label{ese:C.9}
Il dominio $\Dom$ della corrispondenza~$\Kor$ è l'insieme~$\insZ\times\insZ$ e~$\insZ$ ne è il codominio $\Cod$; l'immagine della coppia~$(a;b)$ è il numero razionale~$q=\frac{a}{b}$.
\begin{enumeratea}
\item Stabilisci l'insieme di definizione $\ID$ e l'insieme immagine $\IM$;
\item completa:
\begin{enumeratea}
\item lo zero è immagine delle coppie \ldots;
\item se gli elementi della coppia sono numeri opposti l'immagine è \ldots;
\item se gli elementi della coppia sono numeri concordi allora l'immagine è \ldots;
\item un numero negativo è immagine di \ldots
\end{enumeratea}
\item fai degli esempi che illustrino le tua affermazioni precedenti.
\end{enumeratea}
\end{esercizio}

\begin{esercizio}
\label{ese:C.10}
In un gruppo di~10 persone, due si erano laureate in medicina e tre in legge nell'anno~1961, mentre quattro anni dopo, una si era laureata in fisica, un'altra in scienze e due in legge.
Considerate i seguenti insiemi:~$P=\{x \mid  x$ è una persona del gruppo$\}$; $A=\{$1960, 1961, 1964, 1965$\}$; $F=\{x \mid  x$ è una facoltà universitaria$\}$.
Fatene la rappresentazione con diagramma di Eulero-Venn e studiate le corrispondenze~$\Kor_1$, $\Kor_2$, espresse dai predicati:~$\Kor_1$: ``essersi laureato nell'anno'' e~$K_2$:
``essere laureato in'', mettendo in evidenza per ciascuna dominio, codominio, insieme di definizione, immagine, tipo.

Completate:
\begin{enumeratea}
\item nel gruppo ci sono \ldots persone laureate in legge, di cui \ldots nell'anno~1961 e le altre \ldots nell'anno \ldots;
\item nel~1961 si sono laureate \ldots di cui \ldots in medicina;
\item negli anni \ldots non si è laureata nessuna persona del gruppo considerato;
\item tra le~10 persone \ldots non si è laureata.
\end{enumeratea}
N.B.: ciascuno possiede una sola laurea.

Maria si è laureata in fisica nello stesso anno in cui si è laureato suo marito Luca; Andrea, fratello di Luca, non è medico, ha frequentato una facoltà diversa da quella del fratello
e si è laureato in un anno diverso. Supponendo che Maria, Luca e Andrea siano tra le~10 persone di cui sopra, completate:

Maria si è laureata nell'anno \ldots. Andrea si è laureato nell'anno \ldots in \ldots. Luca si è laureato nell'anno \ldots in \ldots
\end{esercizio}
