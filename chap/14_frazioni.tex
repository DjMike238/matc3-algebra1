% (c) 2012-2013 Claudio Carboncini - claudio.carboncini@gmail.com
% (c) 2012-2013 Dimitrios Vrettos - d.vrettos@gmail.com

\chapter{Frazioni algebriche}

\section{Definizione di frazione algebrica}

Diamo la seguente definizione:
\begin{definizione}
Si definisce \emph{frazione algebrica} un'espressione del tipo~$\dfrac{A}{B}$ dove~$A$ e~$B$ sono polinomi.
\end{definizione}
Osserviamo che un'espressione di questo tipo si ottiene talvolta quando ci si propone di ottenere il quoziente di due monomi.

\begin{exrig}
 \begin{esempio}
Determinare il quoziente tra~$m_{1}=5a^{3}b^{2}c^{5}$ e~$m_{2}=-3a^{2}bc^{5}$.

Questa operazione si esegue applicando, sulla parte letterale, le proprietà delle potenze e sul coefficiente la divisione tra numeri
razionali:~$q=5a^{3}b^{2}c^{5}:\left(-3a^{2}bc^{5}\right)=-{\dfrac{5}{3}}ab$.
Il quoziente è quindi un monomio.
 \end{esempio}

 \begin{esempio}
Determinare il quoziente tra~$m_{1}=5a^{3}b^{2}c^{5}$ e~$m_{2}=-3a^{7}bc^{5}$.

In questo caso l'esponente della~$a$ nel dividendo è minore dell'esponente della stessa variabile nel divisore quindi
si ottiene~$q_{1}=5a^{3}b^{2}c^{5}:\left(-3a^{7}bc^{5}\right)=-{\dfrac{5}{3}}a^{-4}b$. 

Questo non è un monomio per la presenza dell'esponente
negativo alla variabile~$a$ (sezione \ref{sect:potenza_esponente_negativo} a pagina \pageref{sect:potenza_esponente_negativo}).
Quindi:~$q_{1}=5a^{3}b^{2}c^{5}:\left(-3a^{7}bc^{5}\right)={\dfrac{5b}{3a^{4}}}$. Il quoziente è una frazione algebrica.
 \end{esempio}
\end{exrig}

Quando vogliamo determinare il quoziente di una divisione tra un monomio e un polinomio o tra due polinomi, si presentano diversi casi.

\paragraph{Caso I }Monomio diviso un polinomio.
\begin{itemize*}
 \item Determinare il quoziente tra:~$D=2a^{3}b$ e~$d=a^{2}+b$.
\end{itemize*}
Il dividendo è un monomio e il divisore un polinomio.
Questa operazione non ha come risultato un polinomio ma una
frazione. $q=2a^{3}b:\left(a^{2}+b\right)=\dfrac{2a^{3}b}{a^{2}+b}$.
\pagebreak
\paragraph{Caso II}Un polinomio diviso un monomio.
\begin{itemize*}
 \item Determinare il quoziente tra:~$D=2a^{3}b+a^{5}b^{3}-3ab^{2}$ e~$d=\dfrac{1}{2}ab$.
\end{itemize*}
$q=\left(2a^{3}b+a^{5}b^{3}-3ab^{2}\right):\left(\dfrac{1}{2}ab\right)=4a^{2}+2a^{4}b^{2}-6b$.
Il quoziente è un polinomio.
\begin{itemize*}
 \item Determinare il quoziente tra:~$D=2a^{3}b+a^{5}b^{3}-3ab^{2}$ e~$d=\dfrac{1}{2}a^{5}b$.
\end{itemize*}
Dividiamo ciascun termine del polinomio per il monomio assegnato: il quoziente sarà
$q=\left(2a^{3}b+a^{5}b^{3}-3ab^{2}\right):\left(\dfrac{1}{2}a^{5}b\right)=\dfrac{4}{a^{2}}+2b^{2}-\dfrac{6b}{a^{4}}$.
Il quoziente è una somma di frazioni algebriche.

\paragraph{Caso III}Un polinomio diviso un altro polinomio.
\begin{itemize*}
 \item Determinare il quoziente tra:~$D=x-3$ e~$d=x^{2}+1$.
\end{itemize*}
La divisione tra polinomi in una sola variabile è possibile, quando il grado del dividendo è maggiore o uguale al grado del divisore;
questa condizione non si verifica nel caso proposto.
Il quoziente è la frazione algebrica~$q=\dfrac{x-3}{x^{2}+1}$.

\paragraph{Conclusione}Una frazione algebrica può essere considerata come il quoziente indicato tra due polinomi.
Ogni frazione algebrica è dunque un'espressione letterale fratta o frazionaria.

\section{Condizioni di esistenza per una frazione algebrica}
Per \emph{discussione} di una frazione algebrica intendiamo la ricerca dei valori che attribuiti alle variabili non la rendano
priva di significato. Poiché non è possibile dividere per~$0$, una frazione algebrica perde di significato per quei valori
che attribuiti alle variabili rendono il denominatore uguale a zero. Quando abbiamo una frazione algebrica
tipo~$\frac{A}{B}$ poniamo sempre la condizione di esistenza (abbreviato con~$\CE$):~$B\neq~0$.

\begin{exrig}
 \begin{esempio}
Determinare le condizioni di esistenza di~$\dfrac{1+x}{x}$.

Questa frazione perde di significato quando il denominatore si annulla:~$\CE x\neq~0$.
 \end{esempio}

 \begin{esempio}
Determinare le condizioni di esistenza di~$\dfrac{x}{x+3}$.

Questa frazione perde di significato quando il denominatore si annulla:~$\CE x-3\neq 0\:\Rightarrow\: x\neq -3$.
 \end{esempio}

 \begin{esempio}
Determinare le condizioni di esistenza di~$\dfrac{3a+5b-7}{ab}$.

$\CE ab\neq~0$. Sappiamo che un prodotto è nullo quando almeno uno dei suoi fattori è nullo,
dunque affinché il denominatore non si annulli non si deve annullare né $a$ né $b$,
quindi~$a\neq~0$ e~$b\neq~0$. Concludendo,~$\CE a\neq~0\wedge b\neq~0$.
 \end{esempio}

 \begin{esempio}
Determinare le condizioni di esistenza di~$\dfrac{-6}{2x+5}$.

$\CE~2x+5\neq~0$, per risolvere questa disuguaglianza si procede come per le usuali
equazioni:~$2x+5\neq~0 \Rightarrow~2x\neq -5\Rightarrow x\neq -{\frac{5}{2}}$ si può concludere~$\CE x\neq -{\frac{5}{2}}$.
 \end{esempio}

 \begin{esempio}
Determinare le condizioni di esistenza di~$\dfrac{-x^{3}-8x}{x^{2}+2}$.

$\CE x^{2}+2\neq~0$, il binomio è sempre maggiore di~$0$ perché somma di due grandezze positive.
Pertanto la condizione~$x^{2}+2\neq~0$ è sempre verificata e la frazione esiste sempre. Scriveremo~$\CE \forall x\in \insR$ (si legge ``per ogni $x$ appartenente a $\insR$'' o ``qualunque $x$ appartenente a $\insR$'').
 \end{esempio}

 \begin{esempio}
Determinare le condizioni di esistenza di~$\dfrac{2x}{x^{2}-4}$.

$\CE x^{2}-4\neq~0$; per rendere nullo il denominatore si dovrebbe avere~$x^2 = 4$ e questo si verifica se~$x = +2$
oppure se~$x = -2$; possiamo anche osservare che il denominatore è una differenza di quadrati e che quindi la
condizione di esistenza si può scrivere come~$\CE (x-2)(x+2)\neq~0$, essendo un prodotto possiamo
scrivere~$\CE x-2\neq~0\wedge x+2\neq~0$ e concludere:~$\CE x\neq~2\wedge x\neq -2$.
 \end{esempio}
\end{exrig}

\begin{procedura}
Determinare la condizione di esistenza di una frazione algebrica:
\begin{enumeratea}
\item porre il denominatore della frazione diverso da zero;
\item scomporre in fattori il denominatore;
\item porre ciascun fattore del denominatore diverso da zero;
\item escludere i valori che annullano il denominatore.
\end{enumeratea}
\end{procedura}
\ovalbox{\risolvii \ref{ese:14.1}, \ref{ese:14.2}}

\section{Semplificazione di una frazione algebrica}

\emph{Semplificare} una frazione algebrica significa dividere numeratore e denominatore per uno stesso fattore diverso da zero.
In questo modo, infatti, la proprietà invariantiva della divisione garantisce che la frazione ottenuta è equivalente a quella data.
Quando semplifichiamo una frazione numerica dividiamo il numeratore e il denominatore per il loro~$\mcd$
che è sempre un numero diverso da zero, ottenendo così una frazione ridotta ai minimi termini equivalente a quella assegnata.
Quando ci poniamo lo stesso problema su una frazione algebrica, dobbiamo porre attenzione a escludere quei valori
che, attribuiti alle variabili, rendono nullo il~$\mcd$.

\begin{exrig}
 \begin{esempio}
Semplificare~$\dfrac{16x^{3}y^{2}z}{10xy^{2}}$.

$\CE xy^{2}\neq~0 \rightarrow x\neq~0 \wedge y\neq~0$.
Puoi semplificare la parte numerica.
Per semplificare la parte letterale applica la proprietà delle potenze relativa al quoziente di potenze con la stessa
base:~$x^{3}:x=x^{3-1}=x^{2}$ e~$y^{2}:y^{2}=1$. Quindi:
\[\frac{16x^{3}y^{2}z}{10xy^{2}}=\frac{8x^{2}z}{5}=\frac{8}{5}x^{2}z.\]
 \end{esempio}

 \begin{esempio}
Ridurre ai minimi termini la frazione:~$\dfrac{a^{2}-6a+9}{a^{4}-81}$.
\begin{itemize*}
 \item Scomponiamo in fattori
  \begin{itemize*}
  \item il numeratore:~$a^2 - 6a +9 = (a - 3 )^2$;
  \item il denominatore:~$a^4 - 81 = \left(a^2 - 9\right)\left(a^2 + 9\right) = (a - 3)(a + 3)\left(a^2 + 9\right)$;
  \end{itemize*}
 \item riscriviamo la frazione~$\dfrac{\left(a-3\right)^{2}}{(a-3)\cdot(a+3)\cdot \left(a^{2}+9\right)}$;
 \item $\CE (a-3)\cdot(a+3)\cdot\left(a^{2}+9\right)\neq~0$ da cui~$\CE a \neq +3$ e~$a \neq -3$,
    il terzo fattore non si annulla mai perché somma di un numero positivo e un quadrato;
 \item semplifichiamo:~$\dfrac{(a-3)^{\cancel{2}}}{\cancel{(a-3)}\cdot (a+3)\cdot \left(a^{2}+9\right)}=\dfrac{a-3}{\left(a+3)(a^{2}+9\right)}$.
\end{itemize*}
 \end{esempio}

 \begin{esempio}
Ridurre ai minimi termini la frazione in due variabili:~$\dfrac{x^{4}+x^{2}y^{2}-x^{3}y-xy^{3}}{x^{4}-x^{2}y^{2}+x^{3}y-xy^{3}}$.
\begin{itemize*}
 \item Scomponiamo in fattori
  \begin{itemize*}
  \item $x^{4}+x^{2}y^{2}-x^{3}y-xy^{3}=x^{2} \left(x^{2}+y^{2}\right)-xy \left(x^{2}+y^{2}\right)=x \left(x^{2}+y^{2}\right) (x-y)$;
  \item $x^{4}-x^{2}y^{2}+x^{3}y-xy^{3}=x^{2} \left(x^{2}-y^{2}\right)+xy \left(x^{2}-y^{2}\right)=x (x+y)^{2} (x-y)$;
  \end{itemize*}
 \item la frazione diventa:~$\dfrac{x^{4}+x^{2}y^{2}-x^{3}y-xy^{3}}{x^{4}-x^{2}y^{2}+x^{3}y-xy^{3}}=\dfrac{x \left(x^{2}+y^{2}\right) (x-y)}{x(x+y)^{2} (x-y)}$;
 \item $\CE x\cdot (x+y)^{2}\cdot (x^{2}+y^{2})\neq~0$ cioè~$\CE x\neq~0\wedge x\neq -y$;
 \item semplifichiamo i fattori uguali:~$\dfrac{\cancel{x} \left(x^{2}+y^{2}\right) \cancel{(x-y)}}{\cancel{x}(x+y)^{2} \cancel{(x-y)}}=\dfrac{x^2+y^2}{(x+y)^2}$.
\end{itemize*}
 \end{esempio}
\end{exrig}

Le seguenti semplificazioni sono errate.
\begin{itemize*}
 \item $\dfrac{\cancel{a}+b}{\cancel{a}}$ questa semplificazione è errata perché $a$ e~$b$ sono addendi, non sono fattori;
 \item $\dfrac{\cancel{x^2}+x+4}{\cancel{x^2}+2}$ questa semplificazione è errata perché $x^2$ è un addendo, non un fattore;
 \item $\dfrac{x^{\cancel{2}}+y^{\cancel{2}}}{(x+y)^{\cancel{2}}}=1$,\quad$\dfrac{\cancel{3a}(a-2)}{\cancel{3a}x-7}=\dfrac{a-2}{x-7}$,
    \quad~$\dfrac{\cancel{\left(x-y^2\right)}\cancel{(a-b)}}{\cancel{\left(y^2-x\right)}\cancel{(a-b)}}=1$;
 \item $\dfrac{\cancel{(2x-3y)}}{(3y-2x)^{\cancel{2}}}=\dfrac{1}{3y-2x}$,\qquad~$\dfrac{a^2+ab}{a^3}=\dfrac{\cancel a(a+b)}{a^{\cancel{3}2}}=
    \dfrac{\cancel{a}+b}{a^{\cancel{2}}}=\dfrac{1+b}{a}$.
\end{itemize*}
\ovalbox{\risolvii \ref{ese:14.3}, \ref{ese:14.4}, \ref{ese:14.5}, \ref{ese:14.6}, \ref{ese:14.7}, \ref{ese:14.8}, \ref{ese:14.9}, \ref{ese:14.10}, \ref{ese:14.11}, \ref{ese:14.12}}


\section{Moltiplicazione di frazioni algebriche}

Il \emph{prodotto} di due frazioni è una frazione avente per numeratore il prodotto dei numeratori e per denominatore il prodotto dei denominatori.

Si vuole determinare il prodotto~$p=\frac{7}{15}\cdot \frac{20}{21}$; possiamo scrivere prima il risultato dei prodotti
dei numeratori e dei denominatori e poi ridurre ai minimi termini la frazione
ottenuta:~$p=\frac{7}{15}\cdot \frac{20}{21}=\frac{\cancel{140}^4}{\cancel{315}^9}=\frac{4}{9}$,
oppure prima semplificare i termini delle frazioni e poi
moltiplicare:~$p=\frac{7}{15}\cdot \frac{20}{21}=\frac{\cancel{7}^1}{\cancel{15}^3}\cdot \frac{\cancel{20}^4}{\cancel{21}^3}=\frac{4}{9}$.
\pagebreak

\begin{exrig}
 \begin{esempio}
Prodotto delle frazioni algebriche~$f_{1}=-{\dfrac{3a^{2}}{10b^{3}c^{4}}}$ e~$f_{2}=\dfrac{25ab^{2}c^{7}}{ab}$.

Poniamo le~$\CE$ per ciascuna frazione assegnata ricordando che tutti i fattori letterali dei denominatori devono essere diversi da zero,
quindi~$\CE a\neq~0\wedge b\neq~0\wedge c\neq~0$.
Il prodotto è la frazione~$f=-\dfrac{3a^{2}}{10b^{3}c^{4}}\cdot\dfrac{25ab^{2}c^{7}}{ab}=-\dfrac{15a^{2}c^{3}}{2b^{2}}$.
 \end{esempio}

 \begin{esempio}
Prodotto delle frazioni algebriche~$f_{1}=-{d\dfrac{3a}{2b+1}}$ e~$f_{2}=\dfrac{10b}{a-3}$.

L’espressione è in due variabili, i denominatori sono polinomi di primo grado irriducibili; poniamo le condizioni di
esistenza:~$\CE~2b+1\neq~0\wedge a-3\neq~0$ dunque~$\CE b\neq -{\frac{1}{2}}\wedge a\neq~3$.
Il prodotto è la frazione~$f=-{\dfrac{3a}{2b+1}}\cdot\dfrac{10b}{a-3}=-{\dfrac{30ab}{(2b+1)(a-3)}}$ in cui non è possibile alcuna semplificazione.

\osservazione
$f=-{\dfrac{3\cancel{a}}{2\cancel{b}+1}}\cdot\dfrac{10\cancel{b}}{\cancel{a}-3}$. Questa semplificazione
contiene errori in quanto la variabile~$a$ è un fattore del numeratore ma è un addendo nel denominatore; analogamente la
variabile~$b$.
 \end{esempio}

 \begin{esempio}
Prodotto delle frazioni algebriche in cui numeratori e denominatori sono polinomi~$f_{{1}}=\dfrac{2x^{2}-x}{x^{2}-3x+2}$
e~$f_{2}=\dfrac{5x-5}{x-4x^{2}+4x^{3}}$.
\begin{itemize*}
 \item Scomponiamo in fattori tutti i denominatori (servirà per la determinazione delle~$\CE$) e
    tutti i numeratori (servirà per le eventuali semplificazioni)
  \begin{itemize*}
  \item $f_{1}=\dfrac{2x^{2}-x}{x^{2}-3x+2}=\dfrac{x\cdot(2x-1)}{(x-1)\cdot(x-2)}$,
  \item $f_{2}=\dfrac{5x-5}{x-4x^{2}+4x^{3}}=\dfrac{5\cdot(x-1)}{x\cdot(2x-1)^{2}}$;
  \end{itemize*}
 \item poniamo le~$\CE$ ricordando che tutti i fattori dei denominatori devono essere diversi da
	zero:~$\CE x-1\neq~0\wedge x-2\neq~0\wedge x\neq~0\wedge~2x-1\neq~0$ da cui~$\CE x\neq~1\wedge x\neq~2\wedge x\neq~0\wedge x\neq \frac{1}{2}$;
 \item determiniamo la frazione prodotto, effettuando le eventuali semplificazioni:
  \begin{itemize*}
  \item $f=\dfrac{\cancel{x}\cdot\cancel{(2x-1)}}{\cancel{(x-1)}\cdot{(x-2)}}\cdot\dfrac{5\cdot\cancel{(x-1)}}{\cancel{x}\cdot(2x-1)^{\cancel{2}}}=
     \dfrac{5}{(x-2)(2x-1)}$.
  \end{itemize*}
\end{itemize*}
\end{esempio}
\end{exrig}

\ovalbox{\risolvii \ref{ese:14.13}, \ref{ese:14.14}, \ref{ese:14.15}, \ref{ese:14.16}, \ref{ese:14.17}}
\pagebreak
\section{Potenza di una frazione algebrica}

La \emph{potenza} di esponente~$n$, naturale diverso da zero, della frazione algebrica~$\dfrac{A}{B}$ con~$B{\neq}0$ ($\CE$) è la frazione
avente per numeratore la potenza di esponente~$n$ del numeratore e per denominatore la potenza di esponente~$n$
del denominatore:~$\left(\dfrac{A}{B}\right)^{n}=\dfrac{A^{n}}{B^{n}}$.

\begin{exrig}
 \begin{esempio}
Calcoliamo  $\left(\dfrac{x-2}{x^{2}-1}\right)^{3}$.

Innanzi tutto determiniamo le~$\CE$ per la frazione
assegnata
\[\frac{x-2}{x^{2}-1}=\frac{x-2}{(x-1)\cdot(x+1)}\quad\Rightarrow\quad(x-1)\cdot(x+1)\neq~0\text{,}\] da cui~$\CE x\neq~1\wedge x\neq -1$.
Dunque si ha
\[\left(\frac{x-2}{x^{2}-1}\right)^{3}=\frac{(x-2)^{3}}{(x-1)^{3}\cdot (x+1)^{3}}.\]
 \end{esempio}
\end{exrig}

\subsection{Casi particolari dell'esponente}
Se~$n = 0$ sappiamo che qualsiasi numero diverso da zero elevato a zero è uguale a~$1$; lo stesso si può dire se
la base è una frazione algebrica, purché essa non sia nulla.
$\left(\dfrac{A}{B}\right)^{0}=1$ con~$A\neq~0$ e~$B\neq~0$.
\begin{exrig}
 \begin{esempio}
Quali condizioni deve rispettare la variabile~$a$ per avere~$\left(\dfrac{3a-2}{5a^{2}+10a}\right)^{0}=1$?
\begin{itemize*}
 \item Scomponiamo in fattori numeratore e denominatore della frazione:~$\left(\dfrac{3a-2}{5a\cdot (a+2)}\right)^{0}$;
 \item determiniamo le~$\CE$ del denominatore:~$a\neq~0\wedge a+2\neq~0$ da cui, $\CE a\neq~0\wedge a\neq -2$.
    Poniamo poi la condizione, affinché la frazione non sia nulla, che anche il numeratore sia diverso da zero.
    Indichiamo con~$C_0$ questa condizione, dunque~$C_0$:~$3a-2\neq~0$, da cui~$a\neq \frac{2}{3}$;
 \item le condizioni di esistenza sono allora~$a\neq -2\wedge a\neq~0\wedge a\neq \frac{2}{3}$.
\end{itemize*}
 \end{esempio}
\end{exrig}

Se~$n$ è intero negativo la potenza con base diversa da zero è uguale alla potenza che ha
per base l’inverso della base e per esponente l’opposto
dell’esponente. $\left(\dfrac{A}{B}\right)^{-n}=\left(\dfrac{B}{A}\right)^{+n}$ con~$A\neq~0$ e~$B\neq~0$.
\begin{exrig}
 \begin{esempio}
Determinare~$\left(\dfrac{x^{2}+5x+6}{x^{3}+x}\right)^{-2}$.
\begin{itemize*}
 \item Scomponiamo in fattori numeratore e denominatore:~$\left(\dfrac{\left(x+2\right)\cdot \left(x+3\right)}{x\cdot \left(x^{2}+1\right)}\right)^{-2}$;
 \item $\CE$ del denominatore~$x\neq~0$ e~$x^2+1\neq~0$ da cui~$\CE x\neq~0$ essendo l’altro fattore sempre diverso da~0.
      Per poter determinare la frazione inversa dobbiamo porre le condizioni perché la frazione non sia nulla e
      cioè che anche il numeratore sia diverso da zero, quindi si deve avere~$C_0: (x+2)\cdot(x+3)\neq~0$ da cui~$C_0: x\neq -2$ e
      $x \neq -3$;
 \item quindi se~$x\neq~0$, $x\neq-2$ e~$x\neq-3$ si ha~$\left(\dfrac{(x+2)\cdot (x+3)}
    {x\cdot \left(x^{2}+1\right)}\right)^{-2}=\dfrac{x^{2}\cdot \left(x^{2}+1\right)^{2}}{(x+2)^{2}\cdot (x+3)^{2}}$.
\end{itemize*}
 \end{esempio}
\end{exrig}

\ovalbox{\risolvi \ref{ese:14.18}}

\section{Divisione di frazioni algebriche}

Il \emph{quoziente} di due frazioni è la frazione che si ottiene moltiplicando la prima con l’inverso della seconda.
Lo schema di calcolo può essere illustrato nel modo seguente, come del resto abbiamo visto nell’insieme dei numeri
razionali:
\[\frac{m}{n}:\frac{p}{q}=\frac{m}{n}\cdot {\frac{q}{p}}=\frac{m\cdot q}{n\cdot p}.\]

Si vuole determinare il quoziente~$q=\dfrac{5}{12}:\dfrac{7}{4}$. L'inverso di~$\dfrac{7}{4}$ è la frazione~$\dfrac{4}{7}$,
dunque
\[q=\frac{5}{12}:\frac{7}{4}=\frac{5}{\cancel{12}^3}\cdot\frac{\cancel{4}^1}{7}=\frac{5}{21}.\]

\begin{exrig}
 \begin{esempio}
Determinare il quoziente delle frazioni algebriche~$f_{1}=\dfrac{3a-3b}{2a^{2}b}$ e~$f_{2}=\dfrac{a^{2}-ab}{b^{2}}$.
\begin{itemize*}
 \item Scomponiamo in fattori le due frazioni algebriche:
\[f_{1}=\frac{3a-3b}{2a^{2}b}=\frac{3\cdot(a-b)}{2a^{2}b}\quad\text{e}\quad f_{2}=\frac{a^{2}-ab}{b^{2}}=\frac{a\cdot (a-b)}{b^{2}}\text{;}\]
 \item poniamo le condizioni di esistenza dei denominatori:~$2a^{2}b\neq~0\wedge b^{2}\neq~0$
    da cui~$\CE$ $a\neq~0\wedge b\neq~0$;
 \item determiniamo la frazione inversa di~$f_2$. Per poter determinare l’inverso dobbiamo porre le condizioni perché la frazione non sia nulla.
    Poniamo il numeratore diverso da zero, $C_0: a\neq~0\wedge a-b\neq~0$ da cui~$C_0: a\neq~0\wedge a\neq b$;
 \item aggiorniamo le condizioni~$\CE a\neq~0\wedge b\neq~0\wedge a\neq b$;
 \item cambiamo la divisione in moltiplicazione e semplifichiamo:
\begin{equation*}
\frac{3\cdot(a-b)}{2a^{2}b}:\frac{a\cdot(a-b)}{b^2}=\frac{3\cdot\cancel{(a-b)}}{2a^{2}\cancel{b}}\cdot\frac{b^{\cancel{2}}}{a\cdot\cancel{(a-b)}}=\frac{3b}{2a^3}.
\end{equation*}
\end{itemize*}
 \end{esempio}
\end{exrig}

\ovalbox{\risolvii \ref{ese:14.19}, \ref{ese:14.20}, \ref{ese:14.21}}

\section{Addizione di frazioni algebriche}

\subsection{Proprietà della addizione tra frazioni algebriche}

Nell’insieme delle frazioni algebriche la somma:
\begin{itemize*}
\item è commutativa:~$f_1+ f_2 = f_2 + f_1$;
\item è associativa:~$(f_1+ f_2) + f_3 = f_1 + (f_2 + f_3) = f_1 + f_2 + f_3$;
\item possiede l’elemento neutro, cioè esiste una frazione~$F^0$ tale che per qualunque frazione~$f$ si abbia~$F^{0} + f = f + F^0= f$ cioè~$F^0 = 0$;
\item per ogni frazione algebrica~$f$ esiste la \emph{frazione opposta}~$(-f)$ tale che
\[(-f) + f = f + (- f) = F^0 = 0.\]
\end{itemize*}
Quest’ultima proprietà ci permette di trattare contemporaneamente l’operazione di addizione e di sottrazione con la somma algebrica, come abbiamo fatto
tra numeri relativi; $(+1) + (-2)$ omettendo il segno di addizione~``$+$'' e togliendo le parentesi diventa~$1 - 2$;
$(+1) - (-2)$ omettendo il segno di sottrazione~``$-$'' e togliendo le parentesi diventa~$1 + 2$.
Come per i numeri relativi, quando si parlerà di somma di frazioni si intenderà ``somma algebrica''.
\begin{exrig}
 \begin{esempio}
Le frazioni~$\dfrac{2x-3y}{x+y}+\dfrac{x+2y}{x+y}$ hanno lo stesso denominatore.

Poniamo le~$\CE$ $x + y{\neq}0$ da cui~$\CE x{\neq}-y$, quindi
\begin{equation*}
\frac{2x-3y}{x+y}+\frac{x+2y}{x+y}=\frac{(2x-3y)+(x+2y)}{x+y}=\frac{3x-y}{x+y}.
\end{equation*}
 \end{esempio}
\osservazione A questo caso ci si può sempre ricondurre trasformando le frazioni in maniera che abbiamo lo stesso denominatore. Si potrebbe scegliere un qualunque denominatore comune,
ad esempio il prodotto di tutti i denominatori ma per semplificare i calcoli scegliamo il~$\mcm$ dei denominatori delle frazioni addendi.

 \begin{esempio}
$\dfrac{x+y}{3x^{2}y}-\dfrac{2y-x}{2xy^{3}}$.

Dobbiamo trasformare le frazioni in modo che abbiano lo stesso denominatore:
\begin{itemize*}
 \item calcoliamo il~$\mcm(3x^{2}y\text{,~}2xy^{3}) = 6x^{2}y^{3}$;
 \item poniamo le~$\CE$ $6x^{2}y^{3} \neq 0$ da cui~$\CE x \neq 0$ e~$y \neq 0$;
 \item dividiamo il~$\mcm$ per ciascun denominatore e moltiplichiamo il quoziente ottenuto per il relativo numeratore:
    \[\frac{2y^{2}\cdot(x+y)}{6x^{2}y^{3}}-\frac{3x\cdot(2y-x)}{6x^{2}y^{3}};\]
 \item la frazione somma ha come denominatore lo stesso denominatore e come numeratore la somma dei numeratori:
    \begin{equation*}
    \frac{2y^{2}\cdot(x+y)}{6x^{2}y^{3}}-\frac{3x\cdot(2y-x)}{6x^{2}y^{3}}=
    \frac{2xy^{2}+2y^{3}+2x^{2}y-6xy+3x^{2}}{6x^{2}y^{3}}.
    \end{equation*}
\end{itemize*}
 \end{esempio}
\pagebreak
 \begin{esempio}
$\dfrac{x+2}{x^{2}-2x}-\dfrac{x-2}{2x+x^{2}}+\dfrac{-4x}{x^{2}-4}$.
\begin{itemize*}
 \item Scomponiamo in fattori i denominatori:
 \[\frac{x+2}{x(x-2)}-\frac{x-2}{x(2+x)}+\frac{-4x}{(x+2)(x-2)}\text{,}\]
    il~$\mcm$ è~$x\cdot (x+2)\cdot (x-2)$;
 \item poniamo le~$\CE$ $x(x+2)(x-2) \neq 0$ da cui~$\CE$ $x \neq 0$, $x \neq 2$ e~$ x \neq -2$;
 \item dividiamo il~$\mcm$ per ciascun denominatore e moltiplichiamo il quoziente ottenuto per il relativo numeratore:
    \[\frac{(x+2)^{2}-(x-2)^{2}-4x^{2}}{x\cdot(x+2)\cdot(x-2)};\]
 \item eseguiamo le operazioni al numeratore:
    \begin{equation*}
     \frac{x^{2}+4x+4-x^{2}+4x-4-4x^{2}}{x\cdot(x+2)\cdot(x-2)}=\frac{8x-4x^{2}}{x\cdot(x+2)\cdot(x-2)};
     \end{equation*}
 \item semplifichiamo la frazione ottenuta, dopo aver scomposto il numeratore:
    \begin{equation*}
    \frac{-4\cancel{x}\cdot\cancel{(x-2)}}{\cancel{x}\cdot(x+2)\cdot\cancel{(x-2)}}=\frac{-4}{x+2}.
    \end{equation*}
\end{itemize*}
 \end{esempio}

 \begin{esempio}
$\dfrac{x}{x-2}-\dfrac{2x}{x+1}+\dfrac{x}{x-1}-\dfrac{5x^{2}-7}{x^{3}-2x^{2}+2-x}$.
\begin{itemize*}
 \item Scomponiamo in fattori~$x^{3}-2x^{2}+2-x$, essendo gli altri denominatori
 irriducibili:  $x^{3}-2x^{2}+2-x=x^2(x-2)-1(x-2)=(x-2)\left(x^2-1\right)=(x-2)(x+1)(x-1)$ che è anche il~$\mcm$ dei denominatori;
 \item poniamo le~$\CE$ $(x-2)(x+1)(x-1) \neq 0$ da cui~$\CE$ $x \neq 2$, $ x \neq -1$ e~$x \neq 1$;
 \item dividiamo il~$\mcm$ per ciascun denominatore e moltiplichiamo il quoziente ottenuto per il relativo numeratore:
 \[\frac{x(x+1)(x-1)-2x(x-2)(x-1)+x(x-2)(x+1)-(5x^{2}-7)}{(x-2)(x+1)(x-1)};\]
 \item eseguiamo le operazioni al numeratore:
 \[\frac{\dotfill}{(x-2)(x+1)(x-1)};\]
 \item semplifichiamo la frazione ottenuta, dopo aver scomposto il numeratore. La frazione somma è:
 \[-{\frac{7}{(x-2)(x+1)}}.\]
 \end{itemize*}
 \end{esempio}

\end{exrig}

\ovalbox{\risolvii \ref{ese:14.22}, \ref{ese:14.23}, \ref{ese:14.24}, \ref{ese:14.25}, \ref{ese:14.26}, \ref{ese:14.27}, \ref{ese:14.28}, \ref{ese:14.29}, \ref{ese:14.30}, \ref{ese:14.31}}
\newpage
% (c) 2012 Silvia Cibola - silvia.cibola@gmail.com
% (c) 2012-2014 Dimitrios Vrettos - d.vrettos@gmail.com

Gli esercizi indicati con (\croce) sono tratti da \emph{Matematica~1}, Dipartimento di Matematica, ITIS V.~Volterra, San Donà di Piave, Versione [11-12][S-A11], pg.~90;
licenza CC,BY-NC-BD, per gentile concessione dei professori che hanno redatto il libro.
%Il libro è scaricabile da \url{http://www.istitutovolterra.it/dipartimenti/matematica/dipmath/docs/M1_1112.pdf}

\section{Esercizi}
\subsection{Problemi con i numeri}
\begin{multicols}{2}
\begin{esercizio}[\Ast]
Determina due numeri, sapendo che la loro somma vale~$70$ e il secondo supera di~$16$ il doppio del primo.
\end{esercizio}

\begin{esercizio}[\Ast]
Determina due numeri, sapendo che il secondo supera di~$17$ il triplo del primo e che la loro somma è~$101$.
\end{esercizio}

\begin{esercizio}[\Ast]
Determinare due numeri dispari consecutivi sapendo che il minore supera di~$10$ i~$\frac{3}{7}$ del maggiore.
\end{esercizio}

\begin{esercizio}[\Ast]
Sommando~$15$ al doppio di un numero si ottengono i~$\frac{7}{2}$ del numero stesso. Qual è il numero?
\end{esercizio}

\begin{esercizio}
Determinare due numeri consecutivi sapendo che i~$\frac{4}{9}$ del maggiore superano di~$8$ i~$\frac{2}{13}$ del minore.
\end{esercizio}

\begin{esercizio}[\Ast]
Se ad un numero sommiamo il suo doppio, il suo triplo, il suo quintuplo e sottraiamo~$21$, otteniamo~$100$. Qual è il numero?
\end{esercizio}

\begin{esercizio}[\Ast]
Trova il prodotto tra due numeri, sapendo che: se al primo numero sottraiamo~$50$ otteniamo~$50$ meno il primo numero; se al doppio del secondo aggiungiamo il suo consecutivo, otteniamo~$151$.
\end{esercizio}

\begin{esercizio}[\Ast]
Se a~$\frac{1}{25}$ sottraiamo un numero, otteniamo la quinta parte del numero stesso. Qual è questo numero?
\end{esercizio}

\begin{esercizio}[\Ast]
Carlo ha~$152$ caramelle e vuole dividerle con le sue due sorelline. Quante caramelle resteranno a Carlo se le ha distribuite in modo che ogni sorellina ne abbia la metà delle sue?
\end{esercizio}

\begin{esercizio}[\Ast]
Se a~$\frac{5}{2}$ sottraiamo un numero, otteniamo il numero stesso aumentato di~$\frac{2}{3}$. Di quale numero si tratta?
\end{esercizio}

\begin{esercizio}[\Ast]
Se ad un numero sottraiamo~$34$ e sommiamo~$75$, otteniamo~$200$. Qual è il numero?
\end{esercizio}

\begin{esercizio}[\Ast]
Se alla terza parte di un numero sommiamo~$45$ e poi sottraiamo~$15$, otteniamo~$45$. Qual è il numero?
\end{esercizio}

\begin{esercizio}[\Ast]
Se ad un numero sommiamo il doppio del suo consecutivo otteniamo~$77$. Qual è il numero?
\end{esercizio}

\begin{esercizio}[\Ast]
Se alla terza parte di un numero sommiamo la sua metà, otteniamo il numero aumentato di~$2$. Qual è il numero?
\end{esercizio}

\begin{esercizio}[\Ast]
Il doppio di un numero equivale alla metà del suo consecutivo più $1$. Qual è il numero?
\end{esercizio}

\begin{esercizio}[\Ast]
Un numero è uguale al suo consecutivo meno~$1$. Trova il numero.
\end{esercizio}

\begin{esercizio}[\Ast]
La somma tra un numero e il suo consecutivo è uguale al numero aumentato di~$2$. Trova il numero.
\end{esercizio}

\begin{esercizio}[\Ast]
La somma tra un numero ed il suo consecutivo aumentato di~$1$ è uguale a~$18$. Qual è il numero?
\end{esercizio}

\begin{esercizio}
La somma tra un numero e lo stesso numero aumentato di~$3$ è uguale a~$17$. Qual è il numero?
\end{esercizio}

\begin{esercizio}[\Ast]
La terza parte di un numero aumentata di~$3$ è uguale a~$27$. Trova il numero.
\end{esercizio}

\begin{esercizio}[\Ast]
La somma tra due numeri~$x$ e~$y$ vale~$80$. Del numero~$x$ sappiamo che questo stesso numero aumentato della sua metà è uguale a~$108$.
\end{esercizio}

\begin{esercizio}[\Ast]
Sappiamo che la somma fra tre numeri~$(x$, $y$, $z)$ è uguale a~$180$. Il numero~$x$ è uguale a se stesso diminuito di~$50$ e poi moltiplicato per~$6$. 
Il numero~$y$ aumentato di~$60$ è uguale a se stesso diminuito di~$40$ e poi moltiplicato per~$6$, trova~$x$, $y$, $z$.
\end{esercizio}

\begin{esercizio}[\Ast]
La somma tra la terza parte di un numero e la sua quarta parte è uguale alla metà del numero aumentata di~$1$. Trova il numero.
\end{esercizio}

\begin{esercizio}
Determina due numeri interi consecutivi tali che la differenza dei loro quadrati è uguale a~$49$.
\end{esercizio}

\begin{esercizio}
Trova tre numeri dispari consecutivi tali che la loro somma sia uguale a~$87$.
\end{esercizio}

\begin{esercizio}
Trova cinque numeri pari consecutivi tali che la loro somma sia uguale a~$1000$.
\end{esercizio}

\begin{esercizio}[\Ast]
Determinare il numero naturale la cui metà, aumentata di~$20$, è uguale al triplo del numero stesso diminuito di~$95$.
\end{esercizio}

\begin{esercizio}[\Ast]
Trova due numeri dispari consecutivi tali che la differenza dei loro cubi sua uguale a~$218$.
\end{esercizio}

\begin{esercizio}[\Ast]
Trova un numero tale che se calcoliamo la differenza tra il quadrato del numero stesso e il quadrato del precedente otteniamo~$111$.
\end{esercizio}

\begin{esercizio}
Qual è il numero che sommato alla sua metà è uguale a~$27$?
\end{esercizio}

\begin{esercizio}[\Ast]
Moltiplicando un numero per~9 e sommando il risultato per la quarta parte del numero si ottiene~$74$. Qual è il numero?
\end{esercizio}

\begin{esercizio}
La somma di due numeri pari e consecutivi è~$46$. Trova i due numeri.
\end{esercizio}

\begin{esercizio}[\Ast]
La somma della metà di un numero con la sua quarta parte è uguale al numero stesso diminuito della sua quarta parte. Qual è il numero?
\end{esercizio}

\begin{esercizio}[\Ast]
Di~$y$ sappiamo che il suo triplo è uguale al suo quadruplo diminuito di~$2$; trova~$y$.
\end{esercizio}

\begin{esercizio}
Il numero~$z$ aumentato di~$60$ è uguale a se stesso diminuito di~$30$ e moltiplicato per~$4$.
\end{esercizio}

\begin{esercizio}[\Ast]
Determinare un numero di tre cifre sapendo che la cifra delle centinaia è~$\frac{2}{3}$ di quella delle unità, la cifra delle decine è~$\frac{1}{3}$ delle unità e la somma delle tre cifre è~$12$.
\end{esercizio}

\begin{esercizio}[\Ast]
Dividere il numero~$576$ in due parti tali che~$\frac{5}{6}$ della prima parte meno~$\frac{3}{4}$ della seconda parte sia uguale a~$138$.
\end{esercizio}

\begin{esercizio}[\Ast]
Determina due numeri naturali consecutivi tali che la differenza dei loro quadrati è uguale a~$49$.
\end{esercizio}
\end{multicols}

\subsection{Problemi dalla realtà}
\begin{multicols}{2}
\begin{esercizio}[\Ast]
Luca e Andrea posseggono rispettivamente \officialeuro~$200$ e \officialeuro~$180$; Luca spende \officialeuro~$10$ al giorno e Andrea \officialeuro~$8$ al giorno. Dopo quanti giorni avranno la stessa somma?
\end{esercizio}

\begin{esercizio}[\Ast]
Ad un certo punto del campionato la Fiorentina ha il doppio dei punti della Juventus e l'Inter ha due terzi dei punti della Fiorentina. Sapendo che in totale i punti delle tre squadre sono~$78$, determinare i punti delle singole squadre.
\end{esercizio}

\begin{esercizio}[\Ast]
Per organizzare una gita collettiva, vengono affittati due pulmini dello stesso modello, per i quali ciascun partecipante deve pagare \officialeuro~$12$. Sui pulmini restano, in tutto, quattro posti liberi. Se fossero stati occupati anche questi posti, ogni partecipante avrebbe risparmiato \officialeuro~$1,50$. Quanti posti vi sono su ogni pulmino? (``La settimana enigmistica'')
\end{esercizio}

\begin{esercizio}
Un rubinetto, se aperto, riempie una vasca in~$5$ ore; un altro rubinetto riempie la stessa vasca in~$7$ ore. Se vengono aperti contemporaneamente, quanto tempo ci vorrà per riempire~$\frac{1}{6}$ della vasca?
\end{esercizio}

\begin{esercizio}[\Ast]
L'età di Antonio è i~$\frac{3}{8}$ di quella della sua professoressa. Sapendo che tra~$16$ anni l'età della professoressa sarà doppia di quella di Antonio, quanti anni ha la professoressa?
\end{esercizio}

\begin{esercizio}[\Ast]
Policrate, tiranno di Samos, domanda a Pitagora il numero dei suoi allievi. Pitagora risponde che: ``la metà studia le belle scienze matematiche; l'eterna Natura è oggetto dei lavori di un quarto; un settimo si esercita al silenzio e alla meditazione; vi sono inoltre tre donne''. Quanti allievi aveva Pitagora? (``Matematica dilettevole e curiosa'')
\end{esercizio}

\begin{esercizio}
Trovare un numero di due cifre sapendo che la cifra delle decine è inferiore di~$3$ rispetto alla cifra delle unità e sapendo che invertendo l'ordine delle cifre e sottraendo il numero stesso, si ottiene~$27$. (``Algebra ricreativa'')
\end{esercizio}

\begin{esercizio}
Al cinema ``Matematico'' hanno deciso di aumentare il biglietto del~$10\%$. Il numero degli spettatori è calato, però, del~$10\%$. È stato un affare?
\end{esercizio}

\begin{esercizio}
A mezzogiorno le lancette dei minuti e delle ore sono sovrapposte. Quando saranno di nuovo sovrapposte?
\end{esercizio}

\begin{esercizio}
Con due qualità di caffè da~$3$~\officialeuro/$\unit{kg}$ e~$5$~\officialeuro/$\unit{kg}$ si vuole ottenere un quintale di miscela da~$\np{3,25}$~\officialeuro/$\unit{kg}$. Quanti kg della prima e quanti della seconda qualità occorre prendere?
\end{esercizio}

\begin{esercizio}[\Ast]
In un supermercato si vendono le uova in due diverse confezioni, che ne contengono rispettivamente~$10$ e~$12$. In un giorno è stato venduto un numero di contenitori da~$12$ uova doppio di quelli da~$10$, per un totale di~$544$ uova. Quanti contenitori da~$10$ uova sono stati venduti?
\end{esercizio}

\begin{esercizio}[\Ast]
Ubaldo, per recarsi in palestra, passa sui mezzi di trasporto~$20$ minuti, tuttavia il tempo totale per completare il tragitto è maggiore a causa dei tempi di attesa. Sappiamo che Ubaldo utilizza~$3$ mezzi, impiega i~$\frac{3}{10}$ del tempo totale per l'autobus, i~$\frac{3}{5}$ del tempo totale per la metropolitana e~$10$ minuti per il treno. Quanti minuti è costretto ad aspettare i mezzi di trasporto? (\emph{poni x il tempo di attesa})
\end{esercizio}

\begin{esercizio}[\Ast]
Anna pesa un terzo di Gina e Gina pesa la metà di Alfredo. Se la somma dei tre pesi è~$200\unit{kg}$, quanto pesa Anna?
\end{esercizio}

\begin{esercizio}
In una partita a dama dopo i primi~$10$ minuti sulla scacchiera restano ancora~$18$ pedine. Dopo altri~$10$ minuti un giocatore perde~$4$ pedine nere e l'altro~$6$ pedine bianche ed entrambi rimangono con lo stesso numero di pedine. Calcolate quante pedine aveva ogni giocatore dopo i primi~$10$ minuti di gioco.
\end{esercizio}

\begin{esercizio}[\Ast]
Due numeri naturali sono tali che la loro somma è~$16$ e il primo, aumentato di~$1$, è il doppio del secondo diminuito di~$3$. Trovare i due numeri.
\end{esercizio}

\begin{esercizio}
Un dvd recoder ha due modalità di registrazione: SP e LP. Con la seconda modalità è possibile registrare il doppio rispetto alla modalità SP. Con un dvd dato per~$2$ ore in SP, come è possibile registrare un film della durata di~$3$ ore e un quarto? Se voglio registrare il più possibile in SP (di qualità migliore rispetto all'altra) quando devo necessariamente passare alla modalità LP?
\end{esercizio}

\begin{esercizio}[\Ast]
Tizio si reca al casinò e gioca tutti i soldi che ha; dopo la prima giocata, perde la metà dei suoi soldi. Gli vengono prestati \officialeuro~$2$ e gioca ancora una volta tutti i suoi soldi; questa volta vince e i suoi averi vengono quadruplicati. Torna a casa con \officialeuro~$100$. Con quanti soldi era arrivato al casinò?
\end{esercizio}

\begin{esercizio}[\Ast]
I sette nani mangiano in tutto~$127$ bignè; sapendo che il secondo ne ha mangiati il doppio del primo, il terzo il doppio del secondo e così via, quanti bignè ha mangiato ciascuno di loro?
\end{esercizio}

\begin{esercizio}[\Ast]
Babbo Natale vuole mettere in fila le sue renne in modo tale che ogni fila abbia lo stesso numero di renne. Se le mette in fila per quattro le file sono due di meno rispetto al caso in cui le mette in fila per tre. Quante sono le renne?
\end{esercizio}

\begin{esercizio}[\Ast]
Cinque fratelli si devono spartire un'eredità di \officialeuro~$\np{180000}$ in modo tale che ciascuno ottenga \officialeuro~$\np{8000}$ in più del fratello immediatamente minore. Quanto otterrà il fratello più piccolo?
\end{esercizio}

\begin{esercizio}[\Ast]
Giovanni ha tre anni in più di Maria. Sette anni fa la somma delle loro età era~$19$. Quale età hanno attualmente?
\end{esercizio}

\begin{esercizio}[\Ast]
Lucio ha acquistato un paio di jeans e una maglietta spendendo complessivamente \officialeuro~$518$. Calcolare il costo dei jeans e quello della maglietta, sapendo che i jeans costano \officialeuro~$88$ più della maglietta.
\end{esercizio}

\begin{esercizio}[\Ast]
Francesca ha il triplo dell'età di Anna. Fra sette anni Francesca avrà il doppio dell'età di Anna. Quali sono le loro età attualmente?
\end{esercizio}

\begin{esercizio}[\Ast]
In una fattoria ci sono tra polli e conigli~$40$ animali con~$126$ zampe. Quanti sono i conigli?
\end{esercizio}

\begin{esercizio}[\Ast]
Due anni fa ho comprato un appartamento. Ho pagato alla consegna~$\frac{1}{3}$ del suo prezzo, dopo un anno~$\frac{3}{4}$ della rimanenza; oggi ho saldato il debito sborsando \officialeuro~$\np{40500}$. Qual è stato il prezzo dell'appartamento?
\end{esercizio}

\begin{esercizio}[\Ast]
Un ciclista pedala in una direzione a~$30\unit{km/h}$, un marciatore parte a piedi dallo stesso punto e alla stessa ora e va nella direzione contraria a~$6\unit{km/h}$. Dopo quanto tempo saranno lontani~$150\unit{km}$?
\end{esercizio}

\begin{esercizio}[\Ast]
Un banca mi offre il~$2\%$ di interesse su quanto depositato all'inizio dell'anno. Alla fine dell'anno vado a ritirare i soldi depositati più l'interesse: se ritiro \officialeuro~$\np{20400}$, quanto avevo depositato all'inizio? Quanto dovrebbe essere la percentuale di interesse per ricevere \officialeuro~$\np{21000}$ depositando i soldi calcolati al punto precedente?
\end{esercizio}

\begin{esercizio}[\Ast]
Si devono distribuire \officialeuro~$\np{140800}$ fra~$11$ persone che hanno vinto un concorso. Alcune di esse rinunciano alla vincita e quindi la somma viene distribuita tra le persone rimanenti. Sapendo che ad ognuna di esse sono stati dati \officialeuro~$\np{4800}$ in più, quante sono le persone che hanno rinunciato al premio?
\end{esercizio}

\begin{esercizio}[\Ast]
Un treno parte da una stazione e viaggia alla velocità costante di~$120\unit{km/h}$. Dopo~$80$ minuti parte un secondo treno dalla stessa stazione e nella stessa direzione alla velocità di~$150\unit{km/h}$. Dopo quanti~$\unit{km}$ il secondo raggiungerà il primo?
\end{esercizio}

\begin{esercizio}[\Ast]
Un padre ha~$32$ anni, il figlio~$5$. Dopo quanti anni l'età del padre sarà~$10$ volte maggiore di quella del figlio? Si interpreti il risultato ottenuto.
\end{esercizio}

\begin{esercizio}[\Ast]
Uno studente compra~$4$ penne, $12$ quaderni e~$7$ libri per un totale di \officialeuro~$180$. Sapendo che un libro costa quanto~$8$ penne e che~$16$ quaderni costano quanto~$5$ libri, determinare il costo dei singoli oggetti.
\end{esercizio}

\begin{esercizio}[\Ast]
Un mercante va ad una fiera, riesce a raddoppiare il proprio capitale e vi spende \officialeuro~$500$; ad una seconda fiera triplica il suo avere e spende \officialeuro~$900$; ad una terza poi quadruplica il suo denaro e spende \officialeuro~$\np{1200}$. Dopo ciò gli sono rimasti \officialeuro~$800$. Quanto era all'inizio il suo capitale?
\end{esercizio}

\begin{esercizio}[\Ast]
L'epitaffio di Diofanto. ``Viandante! Qui furono sepolti i resti di Diofanto. E i numeri possono mostrare, oh, miracolo! Quanto lunga fu la sua vita, la cui sesta parte costituì la sua felice infanzia. Aveva trascorso ormai la dodicesima parte della sua vita, quando di peli si coprì la guancia. E la settima parte della sua esistenza trascorse in un matrimonio senza figli. Passò ancora un quinquennio e gli fu fonte di gioia la nascita del suo primogenito, che donò il suo corpo, la sua bella esistenza alla terra, la quale durò solo la metà di quella del padre. Il quale, con profondo dolore discese nella sepoltura, essendo sopravvenuto solo quattro anni al proprio figlio. Dimmi quanti anni visse Diofanto.''
\end{esercizio}

\begin{esercizio}[\Ast, \croce]
Un cane cresce ogni mese di~$\frac{1}{3}$ della sua altezza. Se dopo~$3$ mesi dalla nascita è alto~$64\unit{cm}$, quanto era alto appena nato?
\end{esercizio}

\begin{esercizio}[\Ast, \croce]
La massa di una botte colma di vino è di~$192\unit{kg}$ mentre se la botte è riempita di vino per un terzo la sua massa è di~$74\unit{kg}$. Trovare la massa della botte vuota.
\end{esercizio}

\begin{esercizio}[\Ast, \croce]
Carlo e Luigi percorrono in auto, a velocità costante, un percorso di~$400\unit{km}$, ma in senso opposto. Sapendo che partono alla stessa ora dagli estremi del percorso e che Carlo corre a~$120\unit{km/h}$ mentre Luigi viaggia a~$80\unit{km/h}$, calcolare dopo quanto tempo si incontrano.
\end{esercizio}

\begin{esercizio}[\Ast, \croce]
Un fiorista ordina dei vasi di stelle di Natale che pensa di rivendere a \officialeuro~$12$ al vaso con un guadagno complessivo di \officialeuro~$320$. Le piantine però sono più piccole del previsto, per questo è costretto a rivendere ogni vaso a \officialeuro~$7$ rimettendoci complessivamente \officialeuro~$80$. Quanti sono i vasi comprati dal fiorista?
\end{esercizio}

\begin{esercizio}[\Ast, \croce]
Un contadino possiede~$25$ tra galline e conigli; determinare il loro numero sapendo che in tutto hanno~$70$ zampe.
\end{esercizio}

\begin{esercizio}[\Ast, \croce]
Un commerciante di mele e pere carica nel suo autocarro~$139$ casse di frutta per un peso totale di~$\np{23,5}$ quintali. Sapendo che ogni cassa di pere e mele pesa rispettivamente~$20\unit{kg}$ e~$15\unit{kg}$, determinare il numero di casse per ogni tipo caricate.
\end{esercizio}

\begin{esercizio}[\Ast, \croce]
Determina due numeri uno triplo dell'altro sapendo che dividendo il maggiore aumentato di~$60$ per l'altro diminuito di~$20$ si ottiene~$5$.
\end{esercizio}

\begin{esercizio}[\Ast, \croce]
Un quinto di uno sciame di api si posa su una rosa, un terzo su una margherita. Tre volte la differenza dei due numeri vola sui fiori di pesco, e rimane una sola ape che si libra qua e là nell'aria. Quante sono le api dello sciame?
\end{esercizio}

\begin{esercizio}[\Ast, \croce]
Per organizzare un viaggio di~$540$ persone un'agenzia si serve di~$12$ autobus, alcuni con~$40$ posti a sedere e altri con~$52$; quanti sono gli autobus di ciascun tipo?
\end{esercizio}

\begin{esercizio}[\croce]
Il papà di Paola ha venti volte l'età che lei avrà tra due anni e la mamma, cinque anni più giovane del marito, ha la metà dell'età che avrà quest'ultimo fra venticinque anni; dove si trova Paola oggi?
\end{esercizio}
\end{multicols}
\subsection{Problemi di geometria}
\begin{multicols}{2}
\begin{esercizio}[\Ast]
In un triangolo rettangolo uno degli angoli acuti è~$\frac{3}{7}$ dell'altro angolo acuto. Quanto misurano gli angoli del triangolo?
\end{esercizio}

\begin{esercizio}[\Ast]
In un triangolo un angolo è il~$\frac{3}{4}$ del secondo angolo, il terzo angolo supera di~$10^{\circ}$ la somma degli altri due. Quanto misurano gli angoli?
\end{esercizio}

\begin{esercizio}
In un triangolo~$ABC$, l'angolo in~$A$ è doppio dell'angolo in~$B$ e l'angolo in~$C$ è doppio dell'angolo in~$B$. Determina i tre angoli.
\end{esercizio}

\begin{esercizio}
Un triangolo isoscele ha il perimetro di~$39$. Determina le lunghezze dei lati del triangolo sapendo che la base è~$\frac{3}{5}$ del lato.
\end{esercizio}

\begin{esercizio}[\Ast]
Un triangolo isoscele ha il perimetro di~$122\unit{m}$, la base di~$24\unit{m}$. Quanto misura ciascuno dei due lati obliqui congruenti?
\end{esercizio}

\begin{esercizio}[\Ast]
Un triangolo isoscele ha il perimetro di~$188\unit{cm}$, la somma dei due lati obliqui supera di~$25\unit{cm}$ i~$\frac{2}{3}$ della base. Calcola la lunghezza dei lati.
\end{esercizio}

\begin{esercizio}[\Ast]
In un trinagolo~$ABC$ di perimetro~$186\unit{cm}$ il lato~$AB$ è~$\frac{5}{7}$ di~$AC$ e~$BC$ è~$\frac{3}{7}$ di~$AC$. Quanto misurano i lati del triangolo?
\end{esercizio}

\begin{esercizio}[\Ast]
Un trapezio rettangolo ha la base minore che è~$\frac{2}{5}$ della base minore e l'altezza è~$\frac{5}{4}$ della base minore. Sapendo che il perimetro è~$\np[m]{294,91}$, calcola l'area del trapezio.
\end{esercizio}

\begin{esercizio}[\Ast]
Determina l'area di un rettangolo che ha la base che è~$\frac{2}{3}$ dell'altezza, mentre il perimetro è~$144\unit{cm}$.
\end{esercizio}

\begin{esercizio}[\Ast]
Un trapezio isoscele ha la base minore pari a~$\frac{7}{13}$ della base maggiore, il lato obliquo è pari ai~$\frac{5}{6}$ della differenza tra le due basi. Sapendo che il perimetro misura~$124\unit{cm}$, calcola l'area del trapezio.
\end{esercizio}

\begin{esercizio}[\Ast]
Il rettangolo~$ABCD$ ha il perimetro di~$78\unit{cm}$, inoltre sussiste la seguente relazione tra i lati:~$\overline{AD}=\frac{8}{5}\overline{AB}+12\unit{cm}$. Calcola l'area del rettangolo.
\end{esercizio}

\begin{esercizio}[\Ast]
Un rettangolo ha il perimetro che misura~$240\unit{cm}$, la base è tripla dell'altezza. Calcola l'area del rettangolo.
\end{esercizio}

\begin{esercizio}[\Ast]
In un rettangolo l'altezza supera di~$3\unit{cm}$ i~$\frac{3}{4}$ della base, inoltre i~$\frac{3}{2}$ della base hanno la stessa misura dei~$\frac{2}{3}$ dell'altezza. Calcola le misure della base e dell'altezza.
\end{esercizio}

\begin{esercizio}[\Ast]
In un triangolo isoscele la base è gli~$\frac{8}{5}$ del lato ed il perimetro misura~$108\unit{cm}$. Trovare l'area del triangolo e la misura dell'altezza relativa ad uno dei due lati obliqui.
\end{esercizio}

\begin{esercizio}[\Ast]
In un rombo la differenza tra le due diagonali è di~$3\unit{cm}$. Sapendo che la diagonale maggiore è~$\frac{4}{3}$ della minore, calcolare il perimetro del rombo.
\end{esercizio}

\begin{esercizio}[\Ast]
Determinare le misure delle dimensioni di un rettangolo, sapendo che la minore è uguale a~$\frac{1}{3}$ della maggiore e che la differenza tra il doppio della minore e la metà della maggiore è di~$10\unit{cm}$. Calcolare inoltre il lato del quadrato avente la stessa area del rettangolo dato.
\end{esercizio}

\begin{esercizio}[\Ast]
Antonello e Gianluigi hanno avuto dal padre l'incarico di arare due campi, l'uno di forma quadrata e l'altro rettangolare. ``Io scelgo il campo quadrato -- dice Antonello, -- dato che il suo perimetro è di~$4$ metri inferiore a quello dell'altro''. ``Come vuoi! -- commenta il fratello -- Tanto, la superficie è la stessa, dato che la lunghezza di quello rettangolare è di~$18$ metri superiore alla larghezza''. Qual è l'estensione di ciascun campo?
\end{esercizio}

\begin{esercizio}[\Ast]
In un trapezio rettangolo il lato obliquo e la base minore hanno la stessa lunghezza. La base maggiore supera di~$7\unit{cm}$ i~$\frac{4}{3}$ della base minore. Calcolare l'area del trapezio sapendo che la somma delle basi è~$42\unit{cm}$.
\end{esercizio}

\begin{esercizio}[\Ast]
L'area di un trapezio isoscele è~$168\unit{cm^2}$, l'altezza è~$8\unit{cm}$, la base minore è~$\frac{5}{9}$ della maggiore. Calcolare le misure delle basi, del perimetro del trapezio e delle sue diagonali.
\end{esercizio}

\begin{esercizio}[\Ast]
Le due dimensioni di un rettangolo differiscono di~$4\unit{cm}$. Trovare la loro misura sapendo che aumentandole entrambe di~$3\unit{cm}$ l'area del rettangolo aumenta di~$69\unit{cm^2}$.
\end{esercizio}

\begin{esercizio}[\Ast]
In un quadrato~$ABCD$ il lato misura~$12\unit{cm}$. Detto~$M$ il punto medio del lato~$AB$, determinare sul lato opposto~$CD$ un punto~$N$ tale che l'area del trapezio~$AMND$ sia metà di quella del trapezio~$MBCN$.
\end{esercizio}

\begin{esercizio}[\Ast]
Nel rombo~$ABCD$ la somma delle diagonali è~$20\unit{cm}$ ed il loro rapporto è~$\frac{2}{3}$. Determinare sulla diagonale maggiore~$AC$ un punto~$P$ tale che l'area del triangolo~$APD$ sia metà di quella del triangolo~$ABD$.
\end{esercizio}

\begin{esercizio}
In un rettangolo~$ABCD$ si sa che~$\overline{AB}=91\unit{m}$ e~$\overline{BC}=27\unit{m}$; dal punto~$E$ del lato~$AB$, traccia la perpendicolare a~$DC$ e indica con~$F$ il punto d'intersezione con lo stesso lato. Determina la misura di~$AE$, sapendo che~$\Area(AEFD)=\frac{3}{4}\Area(EFCB)$.
\end{esercizio}
\end{multicols}
\subsection{Risposte}
\begin{multicols}{2}
\paragraph{\thechapter.1.}
$18$; $52$.

\paragraph{\thechapter.2.}
$21$; $80$.

\paragraph{\thechapter.3.}
$19$; $21$.

\paragraph{\thechapter.4.}
$10$.

\paragraph{\thechapter.6.}
$11$.

\paragraph{\thechapter.7.}
$\np{2500}$.

\paragraph{\thechapter.8.}
$\frac{1}{30}$.

\paragraph{\thechapter.9.}
$76$.

\paragraph{\thechapter.10.}
$\frac{11}{12}$.

\paragraph{\thechapter.11.}
$159$.

\paragraph{\thechapter.12.}
$45$.

\paragraph{\thechapter.13.}
$25$.

\paragraph{\thechapter.14.}
$-12$.

\paragraph{\thechapter.15.}
$1$.

\paragraph{\thechapter.16.}
Indeterminato.

\paragraph{\thechapter.17.}
$1$.

\paragraph{\thechapter.18.}
$8$.

\paragraph{\thechapter.20.}
$72$.

\paragraph{\thechapter.21.}
$72$; $8$.

\paragraph{\thechapter.22.}
$60$; $60$; $60$.

\paragraph{\thechapter.23.}
$12$.

\paragraph{\thechapter.27.}
$46$.

\paragraph{\thechapter.28.}
$5$; $7$.

\paragraph{\thechapter.29.}
$56$.

\paragraph{\thechapter.31.}
$8$.

\paragraph{\thechapter.33.}
Indeterminato.

\paragraph{\thechapter.34.}
$2$.

\paragraph{\thechapter.36.}
$426$.

\paragraph{\thechapter.37.}
$216$; $360$.

\paragraph{\thechapter.38.}
$24$; $25$.

\paragraph{\thechapter.39.}
$10$.

\paragraph{\thechapter.40.}
$36$; $24$; $18$.

\paragraph{\thechapter.41.}
$16$.

\paragraph{\thechapter.43.}
$64$.

\paragraph{\thechapter.44.}
$28$.

\paragraph{\thechapter.49.}
$16$.

\paragraph{\thechapter.50.}
$80'$.

\paragraph{\thechapter.51.}
$20\;\unit{kg}$.

\paragraph{\thechapter.53.}
Impossibile.

\paragraph{\thechapter.55.}
\officialeuro~$46$.

\paragraph{\thechapter.56.}
$1\text{,~}2\text{,~}4\text{,~}6\text{,~}16\text{,~}\ldots$

\paragraph{\thechapter.57.}
$24$.

\paragraph{\thechapter.58.}
\officialeuro~$\np{20000}$.

\paragraph{\thechapter.59.}
$15$; $18$.

\paragraph{\thechapter.60.}
\officialeuro~$303$; \officialeuro~$215$.

\paragraph{\thechapter.61.}
$7$; $21$.

\paragraph{\thechapter.62.}
$23$.

\paragraph{\thechapter.63.}
\officialeuro~$\np{243000}$.

\paragraph{\thechapter.64.}
$250'$.

\paragraph{\thechapter.65.}
\officialeuro~$\np{20000}$; $5\%$.

\paragraph{\thechapter.66.}
\officialeuro~$3$.

\paragraph{\thechapter.67.}
$800\;\unit{km}$.

\paragraph{\thechapter.68.}
$2$ anni fa.

\paragraph{\thechapter.69.}
\officialeuro~$2$ penna, \officialeuro~$16$ libro, \officialeuro~$5$ quaderno.

\paragraph{\thechapter.70.}
\officialeuro~$\np{483,33}$.

\paragraph{\thechapter.71.}
$84$.

\paragraph{\thechapter.72.}
$27\;\unit{cm}$.

\paragraph{\thechapter.73.}
$15\;\unit{kg}$.

\paragraph{\thechapter.74.}
$2$ ore.

\paragraph{\thechapter.75.}
$80$.

\paragraph{\thechapter.76.}
$15$ galline e~$10$ conigli.

\paragraph{\thechapter.77.}
$80$; $50$.

\paragraph{\thechapter.78.}
$240$; $80$.

\paragraph{\thechapter.79.}
$15$.

\paragraph{\thechapter.80.}
$7$ da~$40$ posti e~$5$ da~$52$.

\paragraph{\thechapter.82.}
$63^{\circ}$; $27^{\circ}$; $90^{\circ}$.

\paragraph{\thechapter.83.}
$\np{36,43}^{\circ}$; $\np{48,57}^{\circ}$; $95^{\circ}$.

\paragraph{\thechapter.86.}
$49\unit{m}$.

\paragraph{\thechapter.87.}
$\np[cm]{97,8}$; $\np[cm]{45,1}$; $\np[cm]{45,1}$.

\paragraph{\thechapter.88.}
$\np[cm]{32,82}$; $\np[cm]{45,95}$; $\np[cm]{107,22}$.

\paragraph{\thechapter.89.}
$\np[cm^2]{4235}$.

\paragraph{\thechapter.91.}
$\np[cm^2]{683,38}$.

\paragraph{\thechapter.92.}
$\np[cm^2]{297,16}$.

\paragraph{\thechapter.93.}
$\np[cm^2]{2700}$.

\paragraph{\thechapter.94.}
$2$; $\frac{9}{2}$.

\paragraph{\thechapter.95.}
$432\;\unit{cm^2}$; $\np[cm^2]{28,8}$.

\paragraph{\thechapter.96.}
$30\;\unit{cm}$.

\paragraph{\thechapter.97.}
$60\;\unit{cm}$; $20\;\unit{cm}$; $20\sqrt{3}\;\unit{cm}$.

\paragraph{\thechapter.98.}
$\np[m^2]{1600}$.

\paragraph{\thechapter.99.}
$189\;\unit{cm^2}$.

\paragraph{\thechapter.100.}
$27\;\unit{cm}$; $15\;\unit{cm}$; $62\;\unit{cm}$; $\np[cm]{22,47}$.

\paragraph{\thechapter.101.}
$12\;\unit{cm}$; $8\;\unit{cm}$.

\paragraph{\thechapter.102.}
$DN=2\;\unit{cm}$.

\paragraph{\thechapter.103.}
$AP=6\;\unit{cm}$.

\end{multicols}


\cleardoublepage
