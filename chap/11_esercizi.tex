% (c) 2012-2014 Dimitrios Vrettos - d.vrettos@gmail.com
% (c) 2012, 2014 Claudio Carboncini - claudio.carboncini@gmail.com
% (c) 2012 Silvia Cibola - silvia.cibola@gmail.com
\section{Esercizi}
\subsection{Esercizi dei singoli paragrafi}
\subsubsection*{11.1 - Definizioni fondamentali}
\begin{multicols}{2}
\begin{esercizio}
\label{ese:11.1}
Riduci in forma normale il seguente polinomio:
\[5a^3-4ab-1+2a^3+2ab-a-3a^3.\]
\emph{Svolgimento}: Evidenziamo i termini simili e sommiamoli tra di loro:
\[\underline{5a^3}-\overline{4ab}+1+\underline{2a^3}+\overline{2ab}-a-\underline{3a^3}.\]
%\[\mmevid{ev_rosso}{5a^{3}}-\mmevid{ev_verde}{4ab}+1+\mmevid{ev_rosso}{2a^{3}}+\mmevid{ev_verde}{2ab}-a-\mmevid{ev_rosso}{3a^{3}}\]
%così otteniamo \dotfill Il termine noto è \dotfill
\end{esercizio}

\begin{esercizio}
\label{ese:11.2}
Il grado di:
\begin{enumeratea}
\item $x^2y^2−3y^3+5yx−6y^2x^3$ rispetto alla lettera~$y$ è \dotfill, il grado complessivo è \dotfill
\item $5a^2−b+4ab$ rispetto alla~$b$ è \dotfill,\\ il grado complessivo è \dotfill
\end{enumeratea}
\end{esercizio}


\begin{esercizio}
\label{ese:11.3}
Quali polinomi sono omogenei:
\begin{enumeratea}
\item $x^3y+2y^2x^2−4x^4$;
\item $2x+3−xy$;
\item $2x^3y^3−y^4x^2+5x^6$.
\end{enumeratea}
\end{esercizio}

\begin{esercizio}
\label{ese:11.4}
Quali dei seguenti polinomi sono ordinati rispetto alla lettera~$x$ con potenze crescenti:

\begin{enumeratea}
\item $2-\dfrac{1}{2}x^2+x$;
\item $\dfrac{2}{3}-x+3x^2+5x^3$;
\item $3x^4-\dfrac{1}{2}x^3+2x^2-x+\dfrac{7}{8}$.
\end{enumeratea}
\end{esercizio}

\begin{esercizio}
\label{ese:11.5}
Relativamente al polinomio~$b^2+a^4+a^3+a^2$ Il grado massimo è \ldots il grado rispetto alla lettera~$a$ è \ldots 

Rispetto alla lettera~$b$ è \ldots\, Il polinomio è ordinato rispetto alla $a$? È completo? È omogeneo?
%\begin{itemize*}
%\item Il grado massimo è \ldots. Il grado rispetto alla lettera~$a$ è \ldots. Rispetto alla lettera~$b$ è \ldots
%\item il polinomio è ordinato rispetto alla $a$? %\tab\qquad\boxV\qquad\boxF
%\item è completo? %\tab\qquad\boxV\qquad\boxF
%\item è omogeneo? %\tab\qquad\boxV\qquad\boxF
%\end{itemize*}
\end{esercizio}

\begin{esercizio}
\label{ese:11.6}
Scrivi un polinomio di terzo grado nelle variabili~$a$ e~$b$ che sia omogeneo.
\end{esercizio}

\begin{esercizio}
\label{ese:11.7}
Scrivi un polinomio di quarto grado nelle variabili~$x$ e~$y$ che sia omogeneo e ordinato secondo le
potenze decrescenti della seconda indeterminata.
\end{esercizio}

\begin{esercizio}
\label{ese:11.8}
Scrivi un polinomio di quinto grado nelle variabili~$r$ e~$s$ che sia omogeneo e ordinato secondo le
potenze crescenti della prima indeterminata.
\end{esercizio}

\begin{esercizio}
\label{ese:11.9}
Scrivi un polinomio di quarto grado nelle variabili~$z$ e~$w$ che sia omogeneo e ordinato secondo le
potenze crescenti della prima indeterminata e decrescenti della seconda.
\end{esercizio}

\begin{esercizio}
\label{ese:11.10}
Scrivi un polinomio di sesto grado nelle variabili~$x$, $y$ e~$z$ che sia completo e ordinato secondo le
potenze decrescenti della seconda variabile.
\end{esercizio}


\begin{esercizio}
\label{ese:11.11}
Calcola il valore numerico dei polinomi per i valori a fianco indicati.

\begin{enumeratea}
\item $x^2+x$ per $x=-1$;
\item $2x^2-3x+1$ per $x=0$;
\item $3x^2-2x-1$ per $x=2$;
\item $3x^3-2x+x$ per $x=-2$;
\item $\dfrac{3}{4}a+\dfrac{1}{2}b-\dfrac{1}{6}ab$ per $a=-\dfrac{1}{2}$, $b=3$;
\item $4x-6y+\dfrac{1}{5}x^2$ per $x=-5$, $y=\dfrac{1}{2}$.
\end{enumeratea}
\end{esercizio}
\end{multicols}


\subsubsection*{11.2 - Somma algebrica di polinomi}
\begin{esercizio}
\label{ese:11.12}
Calcolare la somma dei due polinomi:~$2x^2+5−3y^2x$, $x^2−xy+2−y^2x+y^3$.

\emph{Svolgimento}: Indichiamo la somma~$(2x^2+5−3y^2x)+(x^2−xy+2−y^2x+y^3)$, eliminando le parentesi otteniamo
il polinomio~$2x^2+5−3y^2x+x^2−xy+2−y^2x+y^3$, sommando i monomi simili otteniamo~$3x^2−4x^{\ldots}y^{\ldots}-\ldots xy+y^3+\ldots$
\end{esercizio}
%\newpage
\begin{esercizio}
\label{ese:11.13}
 Esegui le seguenti somme di polinomi.
\begin{multicols}{3}
 \begin{enumeratea}
 \item $a+b-b$;
 \item $a+b-2b$;
 \item $a+b-(-2b)$;
 \item $a-(b-2b)$;
 \item $2a+b+(3a+b)$;
 \item $2a+2b+(2a+b)+2a$;
 \item $2a+b-(-3a-b)$;
 \item $2a-3b-(-3b-2a)$;
 \item $(a+1)-(a-3)$.
\end{enumeratea}
\end{multicols}
\end{esercizio}

\begin{esercizio}[\Ast]
\label{ese:11.14}
 Esegui le seguenti somme di polinomi.

 \begin{enumeratea}
 \item $\left(2a^{2}-3b\right)+\left(4b+3a^{2}\right)+\left(a^{2}-2b\right)$;
 \item $\left(3a^{3}-3b^{2}\right)+\left(6a^{3}+b^{2}\right)+\left(a^{3}-b^{2}\right)$;
 \item $\left(\dfrac{1}{5}x^{3}-5x^{2}+\dfrac{1}{5}x-1\right)-\left(3x^{3}-\dfrac{7}{3}x^{2}+\dfrac{1}{4}x-1\right)$;
 \item $\left(\dfrac{1}{2}+2a^{2}+x\right)-\left(\dfrac{2}{5}a^{2}+\dfrac{1}{2}{ax}\right)+\left[-\left(-{\dfrac{3}{2}}-2{ax}+x^{2}\right)+\dfrac{1}{3}a^{2}\right]-\left(\dfrac{3}{2}{ax}+2\right)$;
 \item $\left(\dfrac{3}{4}a+\dfrac{1}{2}b-\dfrac{1}{6}{ab}\right)-\left(\dfrac{9}{8}{ab}+\dfrac{1}{2}a^{2}-2b\right)+{ab}-\dfrac{3}{4}a$.
\end{enumeratea}
\end{esercizio}

\begin{esercizio}[\Ast]
\label{ese:11.15} %nuovo
 Esegui le seguenti somme di polinomi.

 \begin{enumeratea}
 \item $\left(a+b^{2}+c^{3}\right)+\left(-4a-5c^{3}\right)+\left(8a-7b^{2}+10c^{3}\right)+\left(6b^{2}-7c^{3}\right)$;
 \item $\left(\dfrac{3}{2}x^{2}-\dfrac{5}{3}xy+2y^{2}\right)+\left(\dfrac{3}{4}x^{2}+\dfrac{1}{5}xy-\dfrac{4}{3}y^{2}\right)$;
 \item $\left(\dfrac{1}{2}x^{2}-2x+3\right)+\left(\dfrac{3}{2}x^{2}-x+\dfrac{1}{3}\right)+\left(\dfrac{2}{3}x^{2}-\dfrac{1}{2}x+\dfrac{2}{3}\right)+\left(\dfrac{7}{5}x^{2}-2+\dfrac{3}{4}x\right)$;
 \item $\left(2a^{3}-\dfrac{1}{4}\right)+\left(-3a^{3}-\dfrac{2}{5}a^{2}+\dfrac{3}{4}\right)+\left(\dfrac{2}{5}a^{2}-\dfrac{1}{2}a+\dfrac{1}{4}a^{3}\right)$;
 \item $\left(x^{4}-\dfrac{1}{2}x^{2}+2x^{3}-\dfrac{1}{3}x\right)+\left(-\dfrac{2}{5}x^{4}-\dfrac{2}{3}x^{3}+\dfrac{5}{3}x^{2}+x-1\right)+\left(2x^{2}-1-\dfrac{4}{3}x^{3}-\dfrac{2}{3}x\right)-\dfrac{1}{6}x^2$.
\end{enumeratea}
\end{esercizio}

\begin{esercizio}[\Ast]
\label{ese:11.16} %nuovo
 Esegui le seguenti somme di polinomi.

 \begin{enumeratea}
 \item $(2ab-3)+\left(a^{2}b-2ab\right)-\left(4+a^{2}b\right)$;
 \item $\dfrac{2ab+3}{2}-\dfrac{4a+b-5}{3}+3ab-\dfrac{19+8a-2b}{6}$;
 \item $(3a-2+b)-\left(\dfrac{4}{3}+\dfrac{a}{2}-\dfrac{b}{3}\right)-\dfrac{9a+2b-20}{6}$;
 \item $\dfrac{4-3ab}{2}-\dfrac{3+4ab}{4}-\dfrac{10ab-5}{4}$;
 \item $\left(2a^{2}b-7ab+{3}\right)-\left(a^{2}b-6ab-3\right)+\left(3ab+3a^{2}b\right)$;
 \item $\dfrac{7ab-3a^{2}+b^{2}}{3}-\left(2b^{2}-a^{2}+2ab\right)+\dfrac{b^{2}}{3}$.
\end{enumeratea}
\end{esercizio}

\begin{esercizio}[\Ast]
\label{ese:11.17} %nuovo
 Esegui le seguenti somme di polinomi.

 \begin{enumeratea}
 \item $5y+3x-[7x-3y-(5x-7y)]+(x-y)-(x-y)$;
 \item $\left(3-\dfrac{1}{2}x-2x^{2}\right)-\left(4x^{2}+\dfrac{1}{2}x+3x^{4}-2\right)+\left(1+3x^{2}-3x\right)-\left(-5x-5x^{2}+6+x^{4}\right)$;
 \item $\left(a^{3}-4a^{2}b+6ab^{3}-b^{3}\right)-\left(a^{3}-b^{3}-4a^{2}b+3ab^{3}\right)$;
 \item $\left[7a-\left(a^{2}-2\right)\right]+\left\lbrace 3a^{2}-4a+\left[6a^{2}-(2a-10)\right]-2 \right\rbrace$;
 \item $\left(b-\dfrac{a}{18}\right)-\left(\dfrac{7b}{8}-\dfrac{a}{6}\right)-\left(\dfrac{3}{4}b-\dfrac{5}{9}a\right)$.
\end{enumeratea}
\end{esercizio}

\subsubsection*{11.3 - Prodotto di un polinomio per un monomio}

\begin{esercizio}
\label{ese:11.18} %{ese:11.15}
 Esegui i seguenti prodotti di un monomio per un polinomio.
 \begin{multicols}{3}
\begin{enumeratea}
 \item $(a + b)b$;
 \item $(a - b)b$;
 \item $(a +b)(-b)$;
 \item $(a - b + 51)b$;
 \item $(-a - b -51)(-b)$;
 \item $(a^{2} - a)a$;
 \item $(a^{2} - a)(-a)$;
 \item $(a^{2}- a - 1)a^{2}$;
 \item $(a^{2}b-ab - 1)(ab)$;
 \item $(ab- ab - 1)(ab)$;
 \item $(a^{2}b- ab -1)(a^{2}b^{2})$;
 \item $(a^{2}b-ab - 1)(ab)^{2}$;
 \item $ab(a^{2}b- ab -1)ab$;
 \item $-2a(a^{2} - a - 1)(-a^{2})$;
 \item $(x^{2}a- ax+2)(2x^{2}a^{3})$.
\end{enumeratea}
\end{multicols}
\end{esercizio}

\begin{esercizio}
\label{ese:11.19} %{ese:11.16}
 Esegui i seguenti prodotti di un monomio per un polinomio.
 \begin{multicols}{2}
\begin{enumeratea}
 \item $\dfrac{3}{4}x^{2}y\cdot\left(2{xy}+\dfrac{1}{3}x^{3}y^{2}\right)$;
 \item $\left(\dfrac{a^{4}}{4}+\dfrac{a^{3}}{8}+\dfrac{a^{2}}{2}\right)\left(2a^{2}\right)$;
 \item $\left(\dfrac{1}{2}a-3+a^{2}\right)\left(-{\dfrac{1}{2}}a\right)$;
 \item $\left(5x+3{xy}+\dfrac{1}{2}y^{2}\right)\left(3x^{2}y\right)$;
 \item $\left(\dfrac{2}{3}xy^{2}+\dfrac{1}{2}x^{3}-\dfrac{3}{4}{xy}\right)(6{xy})$;
 \item $-\dfrac{1}{3}y\left(6x^{2}y-3{xy}\right)$;
 \item $-3xy^2\left(\dfrac{1}{3}x+1\right)$;
 \item $\left(\dfrac{7}{3}b-b\right)\left(a-\dfrac{1}{2}b+1\right)(3a-2a)$.
\end{enumeratea}
\end{multicols}
\end{esercizio}
%\newpage
\subsubsection*{11.4 - Quoziente tra un polinomio e un monomio}
\begin{esercizio}
\label{ese:11.20} %{ese:11.17}
 Svolgi le seguenti divisioni tra polinomi e monomi.
 \begin{multicols}{2}
\begin{enumeratea}
 \item $\left(2x^{2}y+8{xy}^{2}\right):\left(2{xy}\right)$;
 \item $\left(a^{2}+a\right):a$;
 \item $\left(a^{2}-a\right):(-a)$;
 \item $\left(\dfrac{1}{2}a-\dfrac{1}{4}\right):\dfrac{1}{2}$;
 \item $\left(\dfrac{1}{2}a-\dfrac{1}{4}\right):2$;
 \item $(2a-2):\dfrac{1}{2}$;
 \item $\left(\dfrac{1}{2}a-\dfrac{a^{2}}{4}\right):\dfrac{a}{2}$.
\end{enumeratea}
\end{multicols}
\end{esercizio}

\begin{esercizio}
\label{ese:11.21} %{ese:11.18}
 Svolgi le seguenti divisioni tra polinomi e monomi.
 \begin{multicols}{2}
\begin{enumeratea}
 \item $\left(a^{2}-a\right):a$;
 \item $\left(a^{3}+a^{2}-a\right):a$;
 \item $\left(8a^{3}+4a^{2}-2a\right):2a$;
 \item $\left(a^{3}b^{2}+a^{2}b-ab\right):b$;
 \item $\left(a^{3}b^{2}-a^{2}b^{3}-ab^{4}\right):(-{ab}^{2})$;
 \item $\left(a^{3}b^{2}+a^{2}b-ab\right):ab$;
 \item $\left(16x^{4}-12x^{3}+24x^{2}\right):\left(4x^{2}\right)$.
 \item $\left(-x^{3}+3x^{2}-10x+5\right):(-5)$;
\end{enumeratea}
\end{multicols}
\end{esercizio}

\begin{esercizio}
\label{ese:11.22} %{ese:11.19}
 Svolgi le seguenti divisioni tra polinomi e monomi.

\begin{enumeratea}
 \item $\left[\left(-3a^{2}b^{3}-2a^{2}b^{2}+6a^{3}b^{2}\right):(-3{ab})\right]\cdot\left(\dfrac{1}{2}b^{2}\right)$;
 \item $\left(\dfrac{4}{3}a^{2}b^{3}-\dfrac{3}{4}a^{3}b^{2}\right):\left(-{\dfrac{3}{2}a^{2}b^{2}}\right)$;
 \item $\left(2a+\dfrac{a^{2}}{2}-\dfrac{a^{3}}{4}\right):\left(\dfrac{a}{2}\right)$;
 \item $\left(\dfrac{1}{2}a-\dfrac{a^{2}}{4}-\dfrac{a^{3}}{8}\right):\left(\dfrac{1}{2}a\right)$;
 \item $\left(-4x^{3}+\dfrac{1}{2}x^{2}\right):\left(2x^{2}-3x^{2}+\dfrac{1}{2}x^{2}\right)$;
 \item $\left(a^{3}b^{2}-a^{4}b+a^{2}b^{3}\right):\left(a^{2}b\right)$;
 \item $\left(a^{2}-a^{4}+a^{3}\right):\left(a^{2}\right)$.
\end{enumeratea}
\end{esercizio}

\subsubsection*{11.5 - Prodotto di polinomi}
\begin{esercizio}
\label{ese:11.23} %{ese:11.20}
Esegui i seguenti prodotti di polinomi.
\begin{multicols}{2}
\begin{enumeratea}
 \item $\left(\dfrac{1}{2}a^{2}b-2{ab}^{2}+\dfrac{3}{4}a^{3}b\right)\cdot\left(\dfrac{1}{2}{ab}+b\right)$;
 \item $\left(x^{3}-x^{2}+x-1\right)({x}-1)$;
 \item $\left(a^{2}+2{ab}+b^{2}\right)(a+b)$;
 \item $(a-1)(a-2)(a-3)$;
 \item $(a+1)(2a-1)(3a-1)$;
 \item $(a+1)\left(a^{2}+a\right)\left(a^{3}-a^{2}\right)$.
\end{enumeratea}
\end{multicols}
\end{esercizio}

\subsubsection*{11.6 - Divisioni tra due polinomi}

\begin{esercizio}
\label{ese:11.24}
Completa la divisione
\begin{center}
 % (c) 2012 Dimitrios Vrettos - d.vrettos@gmail.com
\begin{tikzpicture}[font=\small]

\matrix  (a) [matrix of  nodes, anchor=south, minimum width=9mm, ,nodes={text depth=2.5mm}]{
$7x^4$&$+0x^3$&$-5x^2$&$+x$&$-1$ &$2x^2$&$+0x$&$-1$\\
{}&{}&$\ldots$&{}&{}&$\displaystyle\frac{7}{2}x$&\ldots\\
{}&{}&$-\displaystyle\frac{3}{2}x^2$&$+x$&$-1$\\
{}&{}&{}&\ldots&{}\\
&&&$x$&$-\displaystyle\frac{7}{4}$\\
};

\draw(a-1-6.north west)--(a-2-6.south west);
\draw(a-1-6.south west)--(a-1-8.south east);
 \draw (a-2-1.south west) -- (a-2-5.south east);
 \draw (a-4-2.south west) -- (a-4-5.south east);
\end{tikzpicture}
\end{center}
\end{esercizio}

\begin{esercizio}[\Ast]
\label{ese:11.25}
Esegui le divisioni tra polinomi.
\begin{multicols}{2}
 \begin{enumeratea}
 \item $\left(3x^{2}-5x+4\right):\left(2x-2\right)$;
 \item $\left(4x^{3}-2x^{2}+2x-4\right):\left(3x-1\right)$;
 \item $\left(5a^{3}-a^{2}-4\right)\text{:}\left(a-2\right)$;
 \item $\left(6y^{5}-5y^{4}+y^{2}-1\right):\left(2y^{2}-3\right)$.
 \end{enumeratea}
\end{multicols}
\end{esercizio}

\begin{esercizio}[\Ast]
\label{ese:11.26}
Esegui le divisioni tra polinomi.
 \begin{enumeratea}
 \item $\left(-7a^{4}+3a^{2}-4+a\right):\left(a^{3}-2\right)$;
 \item $\left(x^{7}-4\right):\left(x^{3}-2x^{2}+3x-7\right)$;
 \item $\left(x^{3}-\dfrac{1}{2}x^{2}-4x+\dfrac{3}{2}\right):\left(x^{2}+3x\right)$;
 \item $\left(2x^{4}+2x^{3}-\dfrac{15}{2}x^{2}-15x-7\right):(2x+3)$.
 \end{enumeratea}
\end{esercizio}

\begin{esercizio}[\Ast]
\label{ese:11.27}
Esegui le divisioni tra polinomi.
 \begin{enumeratea}
 \item $\left(6-7a+3a^{2}-4a^{3}+a^{5}\right):\left(1-2a^{3}\right)$;
 \item $(a^{6}-1):(1+a^{3}+2a^{2}+2a)$;
 \item $\left(a^{4}-\dfrac{5}{4}a^{3}+\dfrac{11}{8}a^{2}-\dfrac{a}{2}\right):\left(a^{2}-\dfrac{a}{2}\right)$;
 \item $\left(2x^{3}-6x^{2}+6x-2\right):\left(2x-2\right)$.
 \end{enumeratea}
\end{esercizio}

\begin{esercizio}
\label{ese:11.28}
Esegui le divisioni tra polinomi.
 \begin{enumeratea}
 \item $\left(2x^{5}-11x^{3}+2x+2\right):\left(x^{3}-2x^{2}+1\right)$;
 \item $\left(15x^{4}-2x+5\right):\left(2x^{2}+3\right)$;
 \item $\left(-{\dfrac{9}{2}}x^{2}-2x^{4}+\dfrac{1}{2}x^{3}-\dfrac{69}{8}x-\dfrac{9}{4}-\dfrac{4}{3}x^{5}\right):\left(-2x^{2}-3x-\dfrac{3}{4}\right)$.
 \end{enumeratea}
\end{esercizio}

%\subsubsection*{11.2 - Polinomi in più variabili}

\begin{esercizio}
\label{ese:11.29}
Dividi il polinomio~$A(x\text{,~}y)=x^{3}+3x^{2}y+2xy^{2}$ per il polinomio~$B(x\text{,~}y)=x+y$ rispetto alla variabile~$x$.
Il quoziente è~$Q(x\text{,~}y)=\ldots \ldots \ldots$, il resto è~$R(x\text{,~}y)=0$.

Ordina il polinomio~$A(x\text{,~}y)$ in modo decrescente rispetto alla variabile~$y$ ed esegui
nuovamente la divisione. Il quoziente è sempre lo stesso? Il resto è sempre zero?
\end{esercizio}

\begin{esercizio}
\label{ese:11.30}
Esegui le divisioni tra polinomi rispetto alla variabile~$x$.
 \begin{enumeratea}
 \item $\left(3x^{4}+5ax^{3}-a^{2}x^{2}-6a^{3}x+2a^{4}\right):\left(3x^{2}-ax-2a^{2}\right)$;
 \item $\left(-4x^{5}+13x^{3}y^{2}-12y^{3}x^{2}+17x^{4}y-12y^{5}\right):\left(2x^{3}-3yx^{2}+2y^{2}x-3y^{3}\right)$;
 \item $\left(x^{5}-x^{4}-2ax^{3}+3ax^{2}-2a\right):\left(x^{2}-2a\right)$.
 \end{enumeratea}
\end{esercizio}

\subsubsection*{11.7 - Regola di Ruffini}

\begin{esercizio}
\label{ese:11.31}
Completa la seguente divisione utilizzando la regola di Ruffini:\:$\left(x^{2}-3x+1\right):(x-3)$.
\begin{itemize*}
\item Calcolo del resto:~$(+3)^{2}-3(+3)+1=\ldots$;
\item calcolo del quoziente:~$Q(x)=1x+0=x$ \quad~$R=\ldots$;
\item verifica:~$(x-3)\cdot x+\ldots =x^{2}-3x+1$.
\end{itemize*}
\end{esercizio}

\begin{esercizio}[\Ast]
\label{ese:11.32}
Risolvi le seguenti divisioni utilizzando la regola di Ruffini.
\begin{multicols}{2}
 \begin{enumeratea}
 \item $\left(3x^{3}-4x^{2}+5x-1\right):(x-2)$;
 \item $\left(x^{5}-x^{3}+x^{2}-1\right):(x-1)$;
 \item $\left(x^{4}-10x^{2}+9\right):(x-3)$;
 \item $\left(2x^{4}+6x^{3}-x-9\right):(x+3)$.
 \end{enumeratea}
\end{multicols}
\end{esercizio}

\begin{esercizio}[\Ast]
\label{ese:11.33}
Risolvi le seguenti divisioni utilizzando la regola di Ruffini.
\begin{multicols}{2}
 \begin{enumeratea}
 \item $\left(x^{4}+5x^{2}+5x^{3}-5x-6 \right):(x+2)$;%ex~444
 \item $\left(4x^{3}-2x^{2}+2x-4 \right):(x+1)$;%ex~445
 \item $\left(\dfrac{4}{3}y^{4}-2y^{2}+\dfrac{3}{2}y-2\right):\left(y+\dfrac{1}{2}\right)$.%ex~446
 \end{enumeratea}
\end{multicols}
\end{esercizio}

\begin{esercizio}[\Ast]
\label{ese:11.34}
Risolvi le seguenti divisioni utilizzando la regola di Ruffini.
 \begin{enumeratea}
 \item $\left(\dfrac{1}{3}x^{5}-\dfrac{3}{2}x-2\right):(x+2)$;
 \item $\left(2a-\dfrac{4}{3}a^{4}-2a^{2}-\dfrac{1}{3}\right):\left(a-\dfrac{1}{2}\right)$;
 \item $\left(\dfrac{4}{3}y^{4}-\dfrac{3}{2}y^{3}+\dfrac{3}{2}y-2\right):\left(y+3\right)$.
 \end{enumeratea}
\end{esercizio}

\begin{esercizio}
\label{ese:11.35}
Risolvi le seguenti divisioni utilizzando la regola di Ruffini.
\begin{multicols}{2}
 \begin{enumeratea}
 \item $\left(27x^{3}-3x^{2}+2x+1\right):(x+3)$;
 \item $\left(2x^{4}-5x^{3}-3x+2\right):(x-1)$;
 \item $\left(\dfrac{3}{4}x^{2}-\dfrac{x^{3}}{3}+2x^{4}\right):\left(2x-\dfrac{3}{2}\right)$.
 \end{enumeratea}
\end{multicols}
\end{esercizio}
\pagebreak
\begin{esercizio}[\Ast]
\label{ese:11.36}
Risolvi le seguenti divisioni utilizzando la regola di Ruffini.
 \begin{enumeratea}
 \item $\left(6a^{3}-9a^{2}+9a-6\right):(3a-2)$;
 \item $(2x^{4}-3x^{2}-5x+1):(2x-3)$;
 \item $\left(x^{5}+\dfrac{1}{3}x^{4}-2x^{2}-\dfrac{2}{3}x\right):\left(x+\dfrac{1}{3}\right)$.
 \end{enumeratea}
\end{esercizio}

\begin{esercizio}[\Ast]
\label{ese:11.37}
Risolvi le seguenti divisioni utilizzando la regola di Ruffini.
 \begin{enumeratea}
 \item $\left(x^{3}-2x^{2}+2x-4\right):(2x-2)$;
 \item $\left(3x^{4}-2x^{3}+x-1\right):(2x-3)$;
 \item $\left(\dfrac{3}{2}a^{4}-2a^{2}+a-\dfrac{1}{2}\right):(3a-1)$.
 \end{enumeratea}
\end{esercizio}

\begin{esercizio}[\Ast]
\label{ese:11.38}
Risolvi le seguenti divisioni nella variabile~$a$.
 \begin{enumeratea}
 \item $\left(3a^{4}b^{4}+a^{2}b^{2}+2ab+2\right):(ab-1)$;
 \item $\left(3a^{4}b^{2}-2a^{2}b\right):(a^{2}b-3)$.
 \end{enumeratea}
\end{esercizio}

\begin{esercizio}[\Ast]
\label{ese:11.39}
Risolvi le seguenti divisioni nella variabile~$x$ utilizzando la regola di Ruffini.
 \begin{enumeratea}
 \item $\left(x^{4}-ax^{3}-4a^{2}x^{2}+7a^{3}x-6a^{4}\right):(x-2a)$;
 \item $\left(x^{4}-2ax^{3}+2a^{3}x-a^{4}\right):(x+a)$.
 \end{enumeratea}
\end{esercizio}

\begin{esercizio}[\Ast]
\label{ese:11.40}
Risolvi utilizzando, quando puoi, il teorema di Ruffini.
 \begin{enumeratea}
 \item Per quale valore di~$k$ il polinomio~$x^{3}-2x^{2}+kx+2$ è divisibile per~$x^{2}-1$?
 \item Per quale valore di~$k$ il polinomio~$x^{3}-2x^{2}+kx$ è divisibile per~$x^{2}-1$?
 \item Per quale valore di~$k$ il polinomio~$x^{3}-3x^{2}+x-k$ è divisibile per~$x+2$?
 \item Scrivi, se possibile, un polinomio nella variabile~$a$ che, diviso per~$a^{2}-1$ dà come quoziente~$a^{2}+1$ e come resto~$-1$.
 \end{enumeratea}
\end{esercizio}

\begin{esercizio}[\Ast]
\label{ese:11.41}
Risolvi utilizzando il teorema di Ruffini.
 \begin{enumeratea}
 \item Trovare un polinomio di secondo grado nella variabile~$x$ che risulti divisibile per~$(x-1)$ e per~$(x-2)$ e tale che
     il resto della divisione per~$(x-3)$ sia uguale a~$-4$;
 \item Per quale valore di~$a$ la divisione~$\left(2x^{2}-ax+3\right):(x+1)$ dà resto~$5$?
 \item Per quale valore di~$k$ il polinomio~$2x^{3}-x^{2}+kx-3k$ è divisibile per~$x+2$?
 \item I polinomi~$A(x)=x^3+2x^2-x+3k-2$ e~$B(x)=kx^2-(3k-1)x-4k+7$ divisi entrambi per~$x+1$ per quale valore di~$k$ hanno lo stesso resto?
 \end{enumeratea}
\end{esercizio}

\subsection{Esercizi riepilogativi}

\begin{esercizio}[\Ast]
\label{ese:11.42}
Risolvi le seguenti espressioni con i polinomi.
 \begin{enumeratea}
 \item $(-a-1-2)-(-3-a+a)$;
 \item $\left(2a^{2}-3b\right)-\left[\left(4b+3a^{2}\right)-\left(a^{2}-2b\right)\right]$;
 \item $\left(2a^{2}-5b\right)-\left[\left(2b+4a^{2}\right)-\left(2a^{2}-2b\right)\right]-9b$;
 \item $3a\left[2(a-2{ab})+3a\left(\dfrac{1}{2}-3b\right)-\dfrac{1}{2}a(3-5b)\right]$;
 \item $2(x-1)(3x+1)-\left(6x^{2}+3x+1\right)+2x(x-1)$.
 \end{enumeratea}
\end{esercizio}
\pagebreak
\begin{esercizio}
\label{ese:11.43}
Risolvi le seguenti espressioni con i polinomi.
 \begin{enumeratea}
 \item $\left(\dfrac{1}{3}x-1\right)(3x+1)-2x\left(\dfrac{5}{4}x-\dfrac{1}{2}\right)(x+1)-\dfrac{1}{2}x\left(x-\dfrac{2}{3}\right)$;
 \item $\left(b^{3}-b\right)(x-b)+(x+b)\left(ab^{2}-a\right)+(b+a)\left(ab-ab^{3}\right)+2ab\left(b-b^{3}\right)$;
 \item $ab\left(a^{2}-b^{2}\right)+2b\left(x^{2}-a^{2}\right)(a-b)-2bx^{2}(a-b)$;
 \item $\left(\dfrac{3}{2}x^{2}y-\dfrac{1}{2}{xy}\right)\left(2x-\dfrac{1}{3}y\right)4x$;
 \item $\left(\dfrac{1}{2}a-\dfrac{1}{2}a^{2}\right)(1-a)\left[a^{2}+2a-\left(a^{2}+a+1\right)\right]$.
 \end{enumeratea}
\end{esercizio}

\begin{esercizio}[\Ast]
\label{ese:11.44}
Risolvi le seguenti espressioni con i polinomi.
 \begin{enumeratea}
 \item $(1-3x)(1-3x)-(-3x)^{2}+5(x+1)-3(x+1)-7$;
 \item $3\left(x-\dfrac{1}{3}y\right)\left[2x+\dfrac{1}{3}y-(x-2y)\right]-2\left(x-\dfrac{1}{3}y+2\right)(2x+3y)$;
 \item $\dfrac{1}{24}(29x+7)-\dfrac{1}{2}x^{2}+\dfrac{1}{2}(x-3)(x-3)-2-\left[\dfrac{1}{3}-\dfrac{3}{2}\left(\dfrac{3}{4}x+\dfrac{2}{3}\right)\right]$;
 \item $-{\dfrac{1}{4}}\left(2 abx+2a^{2}b^{2}+3 ax\right)+a^{2}(b^{2}+x^{2})-\left[\left(\dfrac{1}{3} ax\right)^{2}-\left(\dfrac{2}{3}bx\right)^{2}\right]$;
 \item $\left(\dfrac{1}{3}x+\dfrac{1}{2}y-\dfrac{3}{5}\right)\left(\dfrac{1}{3}x-\dfrac{1}{2}y+\dfrac{3}{5}\right)-\left[\left(\dfrac{1}{3}x\right)^{2}-\left(\dfrac{1}{2}y\right)^{2}\right]$.
 \end{enumeratea}
\end{esercizio}

\begin{esercizio}[\Ast]
\label{ese:11.45} %nuovo
Risolvi le seguenti espressioni con i polinomi.
 \begin{enumeratea}
 \item $\left(x+x^{2}-1\right)(x-1)-(x+1)\left(1+x^{2}-x\right)+4-2x^{2}$;
 \item $(a-3b)(5b-a)+15b^{2}-(b-3a)(2b-5a)+37a^{2}+(b+7a)(2b-3a)$;
 \item $\left(1-\dfrac{x}{2}+\dfrac{y}{3}\right)\left(\dfrac{y}{3}+1+\dfrac{x}{2}\right)$;
 \item $\left(36x^{5}y^{7}-24x^{6}y^{6}+4x^{7}y^{5}\right):4xy$;
 \item $\left(-5ab^{3}+\dfrac{2}{3}ab-\dfrac{3}{4}a^{2}b\right):\left(-\dfrac{3}{5}ab\right)$.
 \end{enumeratea}
\end{esercizio}

\begin{esercizio}[\Ast]
\label{ese:11.46} %nuovo
Risolvi le seguenti espressioni con i polinomi.
 \begin{enumeratea}
 \item $\left(\dfrac{1}{2}x-1\right)\left(\dfrac{1}{4}x^{2}+\dfrac{1}{2}x+1\right)+\left(-{\dfrac{1}{2}}x\right)^{3}+2\left(\dfrac{1}{2}x+1\right)$;
 \item $(3a-2)(3a+2)-(a-1)(2a-2)+a(a-1)\left(a^{2}+a+1\right)$;
 \item $-4x(5-2x)+\left(1-4x+x^{2}\right)\left(1-4x-x^{2}\right)$;
 \item $-(2x-1)(2x-1)+\left[x^{2}-\left(1+x^{2}\right)\right]^{2}-\left(x^{2}-1\right)\left(x^{2}+1\right)$.
 \end{enumeratea}
\end{esercizio}

\begin{esercizio}[\Ast]
\label{ese:11.47}
Risolvi le seguenti espressioni con i polinomi.
 \begin{enumeratea}
 \item $\left(5y+\dfrac{4}{3}x\right)+\left(\dfrac{1}{6}x-4\right)-\left\lbrace\left[(-3x)^3:(-2x)^2\right]-(9+5y)\right\rbrace$;
 \item $3x^{2}y^{2}-\left(-2x^{2}y^{2}\right)^{3}-\left[\dfrac{1}{2}xy(-2xy)^{5}+3x^{2}y^{2}\right]-\left[-(-xy)^{2}\right]^{3}$;
 \item $\left(7a^{2}b+10a^{3}-\dfrac{5}{4}ab^{2}\right)\left(-\dfrac{3}{5}ab^{3}\right)$;
 \item $2a^{3}-\left\lbrace-\dfrac{a}{2}\left[-2\left(a^{2}-b^{2}\right)+2a^{2}\right]+2a^{3}\right\rbrace$.
% \item $\left(\dfrac{12}{25}x^{2}y^{4}-\dfrac{6}{5}x^{2}y^{3}\right)\left(\dfrac{5}{6}x^{2}y^{3}\right)$.
 \end{enumeratea}
\end{esercizio}

\begin{esercizio}
\label{ese:11.48}
Risolvi le seguenti espressioni con i polinomi.
 \begin{enumeratea}
 \item $4(x+1)-3x(1-x)-(x+1)(x-1)-\left(4+2x^{2}\right)$;
 \item $\dfrac{1}{2}(x+1)+\dfrac{1}{4}(x+1)(x-1)-\left(x^{2}-1\right)$;
 \item $(3x+1)\left(\dfrac{5}{2}+x\right)-(2x-1)(2x+1)(x-2)+2x^{3}$.
 \end{enumeratea}
\end{esercizio}

\begin{esercizio}[\Ast]
\label{ese:11.49}
Risolvi le seguenti espressioni con i polinomi.
 \begin{enumeratea}
 \item $\left(a-\dfrac{1}{2}b\right)a^{3}-\left(\dfrac{1}{3}{ab}-1\right)\left[2a^{2}(a-b)-a\left(a^{2}-2{ab}\right)\right]$;
 \item $\left(3x^2+6xy-4y^2\right)\left(\dfrac{1}{2}xy-\dfrac{2}{3}y^2\right)$;
 \item $(2a-3b)\left(\dfrac{5}{4}a^{2}+\dfrac{1}{2}{ab}-\dfrac{1}{6}b^{2}\right)-\dfrac{1}{6}a\left(12a^{2}-\dfrac{18}{5}b^{2}\right)+\dfrac{37}{30}ab^{2}-\dfrac{1}{2}a\left(a^{2}-\dfrac{11}{2}{ab}\right)$;
 \item $\dfrac{1}{3}xy\left[\left(x-y^{2}\right)\left(x^{2}-\dfrac{1}{2}y\right)-3x\left(-{\dfrac{1}{9}xy}\right)\left(3y\right)\right]-\dfrac{1}{3}x\left(x^{3}y+\dfrac{1}{4}xy^{2}\right)$.
 \end{enumeratea}
\end{esercizio}

\begin{esercizio}[\Ast]
\label{ese:11.50}
Risolvi le seguenti espressioni con i polinomi.
 \begin{enumeratea}
 \item $(a-1)\left(a^{2}-a+\dfrac{1}{2}\right)-(a+1)\left(2a^{2}+a-\dfrac{1}{2}\right)$;
 \item $\left[\dfrac{2}{3}+x\left(\dfrac{4}{3}x-\dfrac{4}{3}\right)\right]\left[\dfrac{2}{3}x(x-2)+\dfrac{4}{3}x\right]-(1-2x)\left(\dfrac{4}{9}x^{2}\right)$;
 \item $\left(a^2-\dfrac{3}{2}ab+3b^2\right)\left(a^2+\dfrac{2}{3}ab\right)-ab\left(\dfrac{1}{2}a^{2}-6b^{2}\right)$;
 \item $\dfrac{10}{3}ab^{3}\left[\dfrac{2}{3}a^{2}b-\dfrac{1}{5}ab^{2}\left(\dfrac{3}{4}a^{3}+\dfrac{1}{6}a^{2}b-b^{3}\right)+\dfrac{1}{2}a^{3}b^{3}+ab^{5}\right]-\dfrac{2}{9}a^{3}b^{2}\left(7ab^{4}+10b^{2}\right)$;
 \item $\dfrac{5}{3}xy^{2}\left\lbrace 6x^{3}+\dfrac{2}{3}x\left[3y\left(3x-\dfrac{3}{4}y\right)-4x\left(\dfrac{3}{4}y-\dfrac{9}{4}x\right)\right]\right\rbrace+5x^{2}y^{2} \left(\dfrac{1}{2}y^{2}-4x^{2}\right)$.
 \end{enumeratea}
\end{esercizio}

\begin{esercizio}[\Ast] %esercizi accorpati
	\label{ese:11.51}
	Risolvi la seguente espressione con i polinomi.
	
	 	\begin{multline*}
			\text{a)} \quad \dfrac{1}{2}x\left[\left(x-y^{2}\right)\left(x^{2}+\dfrac{1}{2}y\right)-5x\left(-{\dfrac{1}{10}}{xy}\right)(4y)\right]-\dfrac{1}{2}x\left(x^{3}y+\dfrac{1}{2}xy^{2}\right)+\\
			-\dfrac{1}{2}x^{2}\left(x^{2}+\dfrac{1}{2}y+{xy}^{2}\right)+\dfrac{1}{4}{xy}\left(y^{2}+2x^{3}+{xy}\right);
		\end{multline*}
		
		\begin{multline*}
			\text{b)}\quad \left(\dfrac{2}{3}a-2b\right)\left(\dfrac{3}{2}a+2b\right)\left(\dfrac{9}{4}a^{2}+4b^{2}\right)-\dfrac{3}{4}\left(\dfrac{9}{4}a^{2}\right)-a^{2}\left(\dfrac{9}{4}a^{2}-5b^{2}\right)+\\
			+5{ab}\left(\dfrac{3}{4}a^{2}+\dfrac{4}{3}b^{2}\right);
		\end{multline*}
		
		\begin{multline*}
			\text{c)}\quad \left(\dfrac{1}{2}x+2y\right)\left(\dfrac{1}{2}x-2y\right)\left(\dfrac{1}{4}x^{2}-4y^{2}\right)-\dfrac{1}{4}x\left(\dfrac{27}{4}x^{3}-\dfrac{61}{3}xy^{2}\right)+\\
			-16\left(y^{4}+x^{4}\right)-\dfrac{37}{12}x^{2}y^{2}+\dfrac{141}{8}x^{4};
		\end{multline*}
		
		\begin{multline*}
			\text{d)}\quad x\left(\dfrac{2}{3}y^{2}-\dfrac{27}{8}x^{2}\right)-\left[-\left(\dfrac{3}{2}x-\dfrac{2}{3}y\right)\left(\dfrac{9}{4}x^{2}+xy+\dfrac{4}{3}y^{2}\right)+\dfrac{2}{3}x^{2}\left(\dfrac{9}{4}y^{2}+\dfrac{1}{3}y\right)\right]+\\
			+\dfrac{2}{9}y\left(x^{2}+4y^{2}-9xy\right);
		\end{multline*}
		
		\begin{multline*}
			\text{e)}\quad \left(\dfrac{1}{2}ab+\dfrac{2}{3}xy\right)\left(\dfrac{1}{2}ab-\dfrac{2}{3}xy\right)-\left[\left(\dfrac{1}{2}ab\right)^{2}-\left(\dfrac{2}{3}xy\right)^{2}\right]\left(\dfrac{1}{2}ax\right)+\dfrac{3}{2}ax\left(\dfrac{2}{3}a-\dfrac{2}{3}y\right)+\\
			-x\left(\dfrac{1}{2}ax+\dfrac{3}{4}xy\right)-\dfrac{2}{9}x^{2}y^{2}(ax-2)+\dfrac{1}{4}a^{2}b^{2}\left(\dfrac{1}{2}ax-1\right)+\dfrac{3}{4}x^{2}\left(y+\dfrac{2}{3}a\right).
		\end{multline*}
	\end{esercizio}


\begin{esercizio}[\Ast]
	\label{ese:11.52} %accorpati+nuovi
	Risolvi la seguente espressione con i polinomi.

		\begin{multline*}
			\text{a)} \quad \dfrac{1}{6}ab-\dfrac{1}{3}a^{2}-\left\{\dfrac{3}{4}ab+\dfrac{1}{2}a\left[\dfrac{3}{2}b-\left(\dfrac{1}{6}a-\dfrac{4}{5}a\cdot {\dfrac{25}{3}a}\right)\left(-{\dfrac{2}{3}ab}\right)-\left(3ab^{2}\right)\right]\right\}+\\
			+\dfrac{1}{3}a\left(a-5b-9a^{3}b+\dfrac{1}{6}a^{2}b\right);
		\end{multline*}

		\begin{multline*}
			\text{b)} \quad \dfrac{1}{5}x^{2}+\left\{\left[2x-\left(\dfrac{3}{2}x^{2}y-\dfrac{7}{4}xy+\dfrac{1}{8}y^{3}\right):\left(-{\dfrac{1}{2}y}\right)\right] 2x-\dfrac{7}{10}xy\right\}\left(-{\dfrac{1}{6}x^{2}}\right)+\\
			+x^{2}y-\dfrac{1}{3}x\left(\dfrac{3}{5}x\right)-x^{2}\left(y-x^{3}-\dfrac{1}{12}xy^{2}\right);
		\end{multline*}
		
		c) \quad $\dfrac{1}{2}ax\left(\dfrac{4}{3}a+\dfrac{5}{2}x\right)-\left[\dfrac{1}{9}a^{2}b^{2}-\left(\dfrac{2}{5}xy\right)^2\right]+\left(\dfrac{1}{3}ab+\dfrac{2}{5}xy\right)\left(\dfrac{1}{3}ab-\dfrac{2}{5}xy\right)$;
		
		d) \quad $\dfrac{2}{3}b^{2}\left(\dfrac{4}{3}a-\dfrac{5}{2}b\right)+\left(\dfrac{3}{2}a^{2}+\dfrac{5}{2}b\right)\left(\dfrac{2}{3}b^{2}-\dfrac{4}{3}a\right)-\left(\dfrac{3}{2}a^{2}+\dfrac{4}{3}a\right)\left(\dfrac{2}{3}b^{2}-\dfrac{5}{2}b\right)$;
		
		e) \quad $\left(\dfrac{3}{2}x^{2}-\dfrac{1}{3}x\right)\left(\dfrac{9}{4}x^{4}+\dfrac{1}{9}x^{2}\right)-\left(\dfrac{3}{2}x^{2}+\dfrac{1}{3}x\right)\left(\dfrac{9}{4}x^{4}-\dfrac{1}{9}x^{2}\right)-x^{4}\left(\dfrac{1}{3}-\dfrac{3}{2}x\right)$.
\end{esercizio}

\begin{esercizio}[\Ast]
\label{ese:11.53}
Se $A=x-1$, $B=2x+2$, $C=x^2-1$ determina
\begin{multicols}{3}
\begin{enumeratea}
\item $A+B+C$;
\item $A\cdot B-C$;
\item $A+B\cdot C$;
\item $A\cdot B\cdot C$;
\item $2AC-2BC$;
\item $(A+B)\cdot C$.
\end{enumeratea}
\end{multicols}
\end{esercizio}

\begin{esercizio}[\Ast]
\label{ese:11.54}
 Operazioni tra polinomi con esponenti letterali.

\begin{enumeratea}
\item $\left(a^{n+1}-a^{n+2}+a^{n+3}\right):\left(a^{1+n}\right)$;
\item $\left(1+a^{n+1}\right)\left(1-a^{n-1}\right)$;
\item $\left(16a^{n+1}b^{n+2}-2a^{2n}b^{n+3}+5a^{n+2}b^{n+1}\right):\left(2a^{n}b^{n}\right)$;
\item $\left(a^{n+1}-a^{n+2}+a^{n+3}\right)\left(a^{n+1}-a^{n}\right)$;
\item $\left(a^{n}-a^{n+1}+a^{n+2}\right)\left(a^{n+1}-a^{n-1}\right)$;
\item $\left(a^{n}+a^{n+1}+a^{n+2}\right)\left(a^{n+1}-a^{n}\right)$;
\item $\left(a^{n+2}+a^{n+1}\right)\left(a^{n+1}+a^{n+2}\right)$;
\item $\left(1+a^{n+1}\right)\left(a^{n+1}-2\right)$;
\item $\left(a^{n+1}-a^{n}\right)\left(a^{n+1}+a^{n}\right)\left(a^{2n+2}+a^{2n}\right)$;
\item $\left(\dfrac{1}{2}x^{n}-\dfrac{3}{2}x^{2n}\right)\left(\dfrac{1}{3}x^{n}-\dfrac{1}{2}\right)-\left(\dfrac{1}{3}x^{n}-1\right)\left(x^{n}+x\right)$.
\end{enumeratea}
\end{esercizio}
\begin{multicols}{2}
\begin{esercizio}
\label{ese:11.55}
 Se si raddoppiano i lati di un rettangolo, come varia il suo
perimetro?
\end{esercizio}

\begin{esercizio}
\label{ese:11.56}
 Se si raddoppiano i lati di un triangolo rettangolo, come varia la sua
area?
\end{esercizio}

\begin{esercizio}
\label{ese:11.57}
 Se si raddoppiano gli spigoli~$a$, $b$ e~$c$ di un parallelepipedo, come
varia il suo volume?
\end{esercizio}

\begin{esercizio}
\label{ese:11.58}
 Come varia l'area di un cerchio se si triplica il suo
raggio?
\end{esercizio}

\begin{esercizio}
\label{ese:11.59}
 Determinare l'area di un rettangolo avente come
dimensioni~$\frac{1}{2}a$ e~$\frac{3}{4}a^{2}b$.
\end{esercizio}

\begin{esercizio}
\label{ese:11.60}
 Determinare la superficie laterale di un cilindro avente raggio di
base~$x^{2}y$ e altezza~$\frac{1}{5}{xy}^{2}$.
\end{esercizio}
\end{multicols}

\begin{esercizio}[\Ast]
\label{ese:11.61}
Esegui le seguenti divisioni tra polinomi.
 \begin{enumeratea}
 \item $\left(4x^{2}-11x+4x^{3}+4\right):\left(3x+2x^{2}-4\right)$;
 \item $\left(x^{3}-27\right):\left(x^{2}+3x+9\right)$;
 \item $\left(\dfrac{1}{6}x^{2}+x^{4}-\dfrac{19}{6}x^{3}+4x-2\right):\left(2+3x^{2}-5x\right)$.
 \end{enumeratea}
\end{esercizio}

\begin{esercizio}[\Ast]
\label{ese:11.62}
Esegui le seguenti divisioni tra polinomi.
 \begin{enumeratea}
 \item $\left(5x^{2}-12x+6x^{3}+3x^{4}+6\right):(x-2)$;
 \item $\left(12x^{3}-16x-10+10x^{2}\right):(4x+6)$;
 \item $\left(2a^{3}+4a^{2}+a\right):\left(a^{2}+1\right)$;
 \item $\left(-3x^{3}-3x^{2}+2x^{4}-3x+1\right):\left(-3x+2x^{2}+1\right)$.
 \end{enumeratea}
\end{esercizio}

\begin{esercizio}[\Ast]
\label{ese:11.63}
Esegui le seguenti divisioni utilizzando il metodo di Ruffini.
 \begin{enumeratea}
 \item $\left(3a-a^{2}+10+2a^{3}+a^{4}\right):(a+2)$;
 \item $\left(2a^{3}-56a+3a^{5}\right):(a+2)$;
 \item $\left(8a^{2}-5a+1-5a^{3}+2a^{4}\right):\left(a-\dfrac{1}{2}\right)$;
 \item $\left(21a-17a^{2}+6a^{3}-12\right):(3a-4)$.
 \end{enumeratea}
\end{esercizio}

\begin{esercizio}[\Ast]
\label{ese:11.64}
Esegui la divisione prima rispetto ad $a$ e poi rispetto a $y$. In entrambi in casi si deve ottenere lo stesso risultato.

$\left(9a^{3}-5a^{2}y-8ay^{2}+4y^{3}\right):\left(3a^{2}+ay-2y^{2}\right)$
\end{esercizio}

\begin{esercizio}[\Ast]
\label{ese:11.65}
Esegui la divisione prima rispetto a $x$ e poi rispetto a $y$. In entrambi in casi si deve ottenere lo stesso risultato.

$\left(x^{3}+y^{3}-2xy^{2}-2x^{2}y\right):(x+y)$
\end{esercizio}
\newpage
\subsection{Risposte}
%\begin{multicols}{2}
\paragraph{11.14.} d)~$-x^{2}+x+\frac{29}{15}a^{2}$,\quad e)~$-{\frac{a^{2}}{2}}-\frac{7}{24}ab+\frac{5}{2}b$.
\paragraph{11.15.} a)~$5a-c^{3}$,\quad b)~$\frac{9}{4}x^{2}-\frac{22}{15}xy+\frac{2}{3}y^{2}$,\quad c)~$\frac{61}{15}x^{2}-\frac{11}{4}x+2$,\quad d)~$-\frac{3}{4}a^{3}-\frac{1}{2}a+\frac{1}{2}$,\protect\\ e)~$\frac{3}{5}x^{4}+3x^{2}-2$.
\paragraph{11.16.} a)~$-7$,\quad b)~$4ab-\frac{8}{3}a$,\quad c)~$a+b$,\quad d)~$-5ab+\frac{5}{2}$,\quad e)~$4a^{2}b+2ab+6$,\quad f)~$\frac{ab+b}{3}$.
\paragraph{11.17.} a)~$x+y$,\quad b)~$-4x^{4}+2x^{2}+x$,\quad c)~$3ab^{3}$,\quad d)~$8a^{2}+a+10$,\quad e)~$\frac{2}{3}a-\frac{5}{8}b$.
\paragraph{11.25.}
a)~$Q(x)=\frac{3}{2}x-1; R(x)=2$,\quad b)~$Q(x)=\frac{4}{3}x^{2}-\frac{2}{9}x+\frac{16}{27}; R(x)=-{\frac{92}{27}}$,\quad
c)~{$Q(a)=5a^{2}+9a+18$}; $R(a)=32$,\quad d)~$Q(y)=3y^{3}-\frac{5}{2}y^{2}+\frac{9}{2}y-\frac{13}{4}$; $R(y)=\frac{27}{2}y-\frac{43}{4}$.
\paragraph{11.26.}
a)~$Q(a)=-7a; R(a)=3a^{2}-13a-4$,\quad b)~$Q(x)=x^{4}+2x^{3}+x^{2}+3x+17$;\protect\\ ${R(x)=32x^{2}-30x+115}$,\quad
c)~$Q(x)=x-\frac{7}{2}$; $R(x)=\frac{13}{2}x+\frac{3}{2}$, d)~$Q(x)=x^{3}-\frac{1}{2}x^{2}-3x-~3$; $R(x)=2$.
\paragraph{11.27.}
a)~$Q(a)=2-\frac{1}{2}a^{2}; R(a)=\frac{7}{2}a^{2}-7a+4$,\quad b)~$Q(a)=a^{3}-2a^{2}+2a-1; R(a)=0$,\quad
c)~$Q(a)=a^{2}-\frac{3}{4}a+1; R(a)=0$,\quad d)~$Q(x)=x^{2}-2x+1; R(x)=0$.
\paragraph{11.32.}
a)~$Q(x)=3x^{2}+2x+9; R(x)=17$,\quad b)~$Q(x)=x^{4}+x^{3}+x+1; R(x)=0$, \protect\\ c)~$Q(x)=~x^{3}+3x^{2}-x-3; R(x)=0$.
\paragraph{11.33.}
a)~$Q(x)=x^{3}+3x^{2}-x-3; R(x)=0$,\quad b)~$Q(x)=4x^{2}-6x+8; R(x)=-12$,\protect\\ c)~$Q(y)=\frac{4}{3}y^{3}-\frac{2}{3}y^{2}-\frac{5}{3}y+\frac{7}{3}; R(y)=-\frac{19}{6}$.
\paragraph{11.34.}
a)~$Q(x)=\frac{1}{3}x^{4}-\frac{2}{3}x^{3}+\frac{4}{3}x^{2}-\frac{8}{3}x+\frac{23}{6}; R(x)=-\frac{29}{3}$,
\,b)~$Q(a)=~{-\frac{4}{3}a^{3}-\frac{2}{3}a^{2}-\frac{7}{3}a+\frac{5}{6}}$; $R(a)=\frac{1}{12}$,\quad c)~$Q(y)=\frac{4}{3}y^{3}-\frac{11}{2}y^{2}+\frac{33}{2}y-48; R(y)=142$.
\paragraph{11.36.}
c)~$Q(x)=x^{4}-\frac{8}{3}x^{3}+8x^{2}-26x+\frac{232}{3}; R(x)=-232$.
\paragraph{11.37.}
a)~$Q(x)=\frac{1}{2}x^{2}-\frac{1}{2}x+\frac{1}{2}$; $R(x)=-3$,\quad b)~$Q(x)=\frac{3}{2}x^{3}+\frac{5}{4}x^{2}+\frac{15}{8}x+\frac{53}{16}$; $R(x)=\frac{143}{16}$,
\quad c)~$Q(a)=\frac{1}{2}a^{3}+\frac{1}{6}a^{2}-\frac{11}{18}a+\frac{7}{54}$; $R(a)=-{\frac{10}{27}}$.
\paragraph{11.38.}
a)~$Q(a)=3a^{3}b^{3}+3a^{2}b^{2}+4ab+6$; $R(a)=8$,\quad b)~$Q(a)=3a^{2}b+7$; $R(a)=21$.
\paragraph{11.39.}
a)~$Q(x)=x^{3}+ax^{2}-2a^{2}x+3a^{3}$; $R(x)=0$\quad b)~$Q(x)=x^3-3ax^2+3a^2 x-a^3$; $R(x)=0$.
\paragraph{11.40.}
a)~$k=-1$,\quad b)~nessuno,\quad c)~$k=-22$,\quad d)~$a^{4}-2$.
\paragraph{11.41.}
a)~$-2x^2+6x-4$,\quad b)~$a=0$,\quad c)~$k=-4$,\quad d)~$k=2$.
\paragraph{11.42.} a)~$-a$,\quad b)~$-9b$,\quad c)~$-18b$,\quad d)~$6a^{2}-\frac{63}{2}a^{2}b$,\quad e)~$2x^2-9x-3$.
\paragraph{11.44.} e)~$\frac{3}{5}y-\frac{9}{25}$.
\paragraph{11.45.} a)~$-2x^{2}-2x+4$,\quad b)~$30ab$,\quad c)~$\frac{y^{2}}{9}-\frac{x^{2}}{4}+\frac{2y}{3}+1$,\quad d)~$9x^{4}y^{6}-6x^{5}y^{5}+x^{6}y^{4}$,\protect\\ e)~$\frac{25}{3}b^{2}-\frac{10}{9}+\frac{5}{4}a$.
\paragraph{11.46.} a)~$x+1$,\quad b)~$a^{4}+7a^{2}-5a-2$.
\paragraph{11.47.} a)~$\frac{33}{4}x+10y+5$,\quad b)~$25x^{6}y^{6}$,\quad c)~$-6a^{4}b^{2}-\frac{21}{5} a^{3}b^{4}+\frac{3}{4}a^{2}b^{5}$,\quad d)~$ab^{2}$. %,\protect\\ e)~$\frac{2}{5}x^{4}y^{7}-x^{4}y^{6}$.
\paragraph{11.49.} a)~$a^{4}-\frac{1}{2}a^{3}b-\frac{1}{3}a^{4}b+a^{3}$,\quad b)~$\frac{3}{2}x^{3}y+x^{2}y^{2}-6{xy}^{3}+\frac{8}{3}y^{4}$,\quad c)~$\frac{1}{2}b^{3}$,\quad d)~$\frac{1}{6}xy^{4}-\frac{1}{4}x^{2}y^{2}$.
\paragraph{11.50.} a)~$-a^{3}-5a^{2}+a$,\quad b)~$\frac{8}{9}x^{4}$,\; c)~$a^{4}-\frac{4}{3}a^{3}b+2a^{2}b^{2}+8ab^{3}$,\; d)~$4a^{2}b^{8}-\frac{1}{2}a^{5}b^{5}$,\; e)~$\frac{20}{3}x^{3}y^{3}$.
\paragraph{11.51.} a)~$0$,\quad b)~$-16b^{4}-\frac{27}{16}a^{2}$,\quad c)~$0$,\quad d)~$-\frac{3}{2}x^{2}y^{2}$,\quad e)~$a^{2}x-axy$.
\paragraph{11.52.} a)~$-\frac{7}{9}a^{4}b+\frac{3}{2}a^2b^2-3ab$,\quad b)~$\frac{1}{2}x^{4}+\frac{7}{60}x^{3}y$,\quad c)~$\frac{2}{3}a^{2}x-\frac{5}{4}ax^{2}$,\quad d)~$\frac{15}{4}a^{2}b-2a^{3}$,\quad e)~$0$.
\paragraph{11.53.} a)~$x^{2}+3x$,\quad b)~$x^{2}-1$.
\paragraph{11.54.} a)~$1-a+a^{2}$,\quad b)~$1-a^{n-1}+a^{n+1}-a^{2}n$,\quad c)~$8ab^2-a^nb^3+\frac{5}{2}a^2b$, \protect\\
d)~$a^{2n+4}-2a^{2n+3}+2a^{2n+2}-a^{2n+1}$,\quad e)~$a^{2n+3}-a^{2n+2}-a^{2n-1}+a^{2n}$,\protect\\ f)~$-a^{2}n+a^{2n+3}$,\quad
g)~$a^{2n+4}+2a^{2n+3}+a^{2n+2}$,\quad h)~$a^{2n+2}-a^{n+1}-2$,\protect\\ i)~$a^{4n+4}-a^{4n}$,\quad
j)~$\frac{7}{12}x^{2n}+\frac{3}{4}x^{n}-\frac{1}{2}x^{3n}-\frac{1}{3}x^{n+1}+x$.
\paragraph{11.61.}
a)~$Q(x)=2x-1$; $R(x)=0$,\quad b)~$Q(x)=x-3$; $R(x)=0$,\quad c)~$Q(x)=\frac{1}{3}x^{2}-\frac{1}{2}x-1$; $R(x)=0$.
\paragraph{11.62.}
a)~$Q(x)=3x^{3}+12x^{2}+29x+46$; $R(x)=98$,\quad b)~$Q(x)=3x^{2}-2x-1$; $R(x)=-4$,\protect\\ c)~$Q(a)=2a+4$; $R(a)=-a-4$,\quad d)~$Q(x)=x^{2}-2$; $R(x)=-9x+3$.
\paragraph{11.63.}
a)~$Q(a)=a^{3}-a+5$; $R(x)=0$,\quad b)~$Q(a)=3a^{4}-6a^{3}+14a^{2}-28a$; $R(a)=0$,\quad c)~$Q(a)=2a^{3}-4a^{2}+6a-2$; $R(a)=0$,\quad d)~$Q(a)=2a^{2}-3a+3$; $R(a)=0$.
\paragraph{11.64.}
$Q(a)=3a-2y$; $R(a)=0$.
\paragraph{11.65.}
$Q(x)=x^{2}-3xy+y^{2}$; $R(x)=0$.
