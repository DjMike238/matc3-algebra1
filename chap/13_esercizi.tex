% (c) 2012 Claudio Carboncini - claudio.carboncini@gmail.com
% (c) 2012 -2014 Dimitrios Vrettos - d.vrettos@gmail.com

\section{Esercizi}
\subsection{Esercizi dei singoli paragrafi}
\subsubsection*{\thechapter.2 - Identità ed equazioni}

\begin{esercizio}
\label{ese:13.1}
Risolvi in~$\insZ$ la seguente equazione:~$-x+3=-1$.

\emph{Suggerimento}. Lo schema operativo è: entra~$x$, cambia il segno in~$-x$, aggiunge~$3$, si ottiene~$-1$.
Ora ricostruisci il cammino inverso: da~$-1$ togli~$3$ ottieni \ldots cambia segno ottieni come soluzione~$x = \ldots$.
\end{esercizio}

\subsubsection*{\thechapter.3 - Risoluzione di equazioni numeriche intere di primo grado}

\begin{esercizio}
\label{ese:13.2}
Risolvi le seguenti equazioni applicando il~1° principio di equivalenza.
\begin{multicols}{3}
\begin{enumeratea}
\spazielenx
 \item $x+2=7$;
 \item $2+x=3$;
 \item $16+x=26$;
 \item $x-1=1$;
 \item $3+x=-5$;
 \item $12+x=-22$.
\end{enumeratea}
\end{multicols}
\end{esercizio}

\begin{esercizio}
\label{ese:13.3}%es-3
Risolvi le seguenti equazioni applicando il~1° principio di equivalenza.
\begin{multicols}{3}
\begin{enumeratea}
\spazielenx
 \item $3x=2x-1$;
 \item $8x=7x+4$;
 \item $2x=x-1$;
 \item $5x=4x+2$;
 \item $3x=2x-3$;
 \item $3x=2x-2$.
\end{enumeratea}
\end{multicols}
\end{esercizio}

\begin{esercizio}
\label{ese:13.4}
Risolvi le seguenti equazioni applicando il~1° principio di equivalenza.
\begin{multicols}{3}
\begin{enumeratea}
\spazielenx
 \item $7+x=0$;
 \item $7=-x$;
 \item $-7=x$;
 \item $1+x=0$;
 \item $1-x=0$;
 \item $0=2-x$.
\end{enumeratea}
\end{multicols}
\end{esercizio}

\begin{esercizio}
\label{ese:13.5}
Risolvi le seguenti equazioni applicando il~1° principio di equivalenza.
\begin{multicols}{3}
\begin{enumeratea}
\spazielenx
 \item $3x-1=2x-3$;
 \item $7x-2x-2=4x-1$;
 \item $-5x+2=-6x+6$;
 \item $-2+5x=8+4x$;
 \item $7x+1=6x+2$;
 \item $-1-5x=3-6x$.
\end{enumeratea}
\end{multicols}
\end{esercizio}

%%%%%%%%%%%%%%%%%%%%%%%%%%%%%%%%%%%%%%%%%%%%%%%%%%%%%%%

\begin{esercizio}
\label{ese:13.6}
Risolvi le seguenti equazioni applicando il~2° principio di equivalenza.
\begin{multicols}{3}
\begin{enumeratea}
\spazielenx
 \item $2x=8$;
 \item $2x=3$;
 \item $6x=24$;
 \item $0x=1$;
 \item $\dfrac{1}{3}x=-1$;
 \item $\dfrac{1}{2}x=\dfrac{1}{4}$.
\end{enumeratea}
\end{multicols}
\end{esercizio}

\begin{esercizio}
\label{ese:13.7}
Risolvi le seguenti equazioni applicando il~2° principio di equivalenza.
\begin{multicols}{3}
\begin{enumeratea}
\spazielenx
 \item $\dfrac{3}{2}x=12$;
 \item $2x=-2$;
 \item $3x=\dfrac{1}{6}$;
 \item $\dfrac{1}{2}x=4$;
 \item $\dfrac{3}{4}x=\dfrac{12}{15}$;
 \item $2x=\dfrac{1}{2}$.
\end{enumeratea}
\end{multicols}
\end{esercizio}

%\newpage
\begin{esercizio}
\label{ese:13.8}
Risolvi le seguenti equazioni applicando il~2° principio di equivalenza.
\begin{multicols}{4}
\begin{enumeratea}
\spazielenx
 \item $3x=6$;
 \item $\dfrac{1}{3}x=\dfrac{1}{3}$;
 \item $\dfrac{2}{5}x=\dfrac{10}{25}$;
 \item $-{\dfrac{1}{2}}x=-{\dfrac{1}{2}}$;
 \item $\np{0,1}x=1$;
 \item $\np{0,1}x=10$;
 \item $\np{0,1}x=\np{0,5}$;
 \item $\np{-0,2}x=5$.
\end{enumeratea}
\end{multicols}
\end{esercizio}

%%%%%%%%%%%%%%%%%%%%%%%%%%%%%%%%%%%%%%%%%%%%%%%%%%%%%%%%%%%%5

\begin{esercizio}
\label{ese:13.9}
Risolvi le seguenti equazioni applicando entrambi i principi.
\begin{multicols}{3}
\begin{enumeratea}
\spazielenx
 \item $2x+1=7$;
 \item $3-2x=3$;
 \item $6x-12=24$;
 \item $3x+3=4$;
 \item $5-x=1$;
 \item $7x-2=5$.
\end{enumeratea}
\end{multicols}
\end{esercizio}

\begin{esercizio}
\label{ese:13.10}
Risolvi le seguenti equazioni applicando entrambi i principi.
\begin{multicols}{3}
\begin{enumeratea}
\spazielenx
 \item $2x+8=8-x$;
 \item $2x-3=3-2x$;
 \item $6x+24=3x+12$;
 \item $2+8x=6-2x$;
 \item $6x-6=5-x$;
 \item $-3x+12=3x+18$.
\end{enumeratea}
\end{multicols}
\end{esercizio}

\begin{esercizio}
\label{ese:13.11}
Risolvi le seguenti equazioni applicando entrambi i principi.
\begin{multicols}{3}
\begin{enumeratea}
\spazielenx
 \item $3-2x=8+2x$;
 \item $\dfrac{2}{3}x-3=\dfrac{1}{3}x+1$;
 \item $\dfrac{6}{5}x=\dfrac{24}{5}-x$;
 \item $3x-2x+1=2+3x-1$;
 \item $\dfrac{2}{5}x-\dfrac{3}{2}=\dfrac{3}{2}x+\dfrac{1}{10}$;
 \item $\dfrac{5}{6}x+\dfrac{3}{2}=\dfrac{25}{3}-\dfrac{10}{2}x$.
\end{enumeratea}
\end{multicols}
\end{esercizio}

%%%%%%%%%%%%%%%%%%%%%%%%%%%%%%%%%%%%%%%%%%%%%%%%%%%%%%%%%%%%%%%%%%%%%%%%

\begin{esercizio}
\label{ese:13.12}
Risolvi l'equazione~$10x+4=-2\cdot (x+5)-x$ seguendo la traccia:
\begin{enumerate}
\spazielenx
 \item svolgi i calcoli al primo e al secondo membro: \dotfill;
 \item somma i monomi simili in ciascun membro dell'equazione: \dotfill;
 \item applica il primo principio d'equivalenza per lasciare in un membro solo monomi con l'incognita e nell'altro membro solo numeri: \dotfill;
 \item somma i termini del primo membro e somma i termini del secondo membro: \dotfill;
 \item applica il secondo principio d'equivalenza dividendo ambo i membri per il coefficiente dell'incognita: \dotfill in forma canonica: \dotfill;
 \item scrivi l'Insieme Soluzione:~$\IS = \ldots \ldots \ldots$.
\end{enumerate}
\end{esercizio}

\begin{esercizio}
\label{ese:13.13}
Risolvi, seguendo la traccia, l'equazione~$x-(3x+5)=(4x+8)-4\cdot (x+1)$:
\begin{enumerate}
\spazielenx
 \item svolgi i calcoli: \dotfill;
 \item somma i monomi simili: \dotfill;
 \item porta al primo membro i monomi con la~$x$ e al secondo quelli senza:~$\dotfill$;
 \item somma i monomi simili al primo membro e al secondo membro:~$\dotfill$;
 \item dividi ambo i membri per il coefficiente dell'incognita:~$\dotfill$;
 \item l'insieme soluzione è:~$\dotfill$
\end{enumerate}
\end{esercizio}

%%%%%%%%%%%%%%%%%%%%%%%%%%%%%%%%%%%%%%%%%%%%%%%%%%%%%%%%%%%%%%
%\newpage
\begin{esercizio}[\Ast]
\label{ese:13.14}
Risolvi le seguenti equazioni con le regole pratiche indicate.
 \begin{enumeratea}
 \item $3(x-1)+2(x-2)+1=2x$;
 \item $x-(2x+2)=3x-(x+2)-1$;
 \item $-2(x+1)-3(x-2)=6x+2$;
 \item $x+2-3(x+2)=x-2$.
 \end{enumeratea}
\end{esercizio}

\begin{esercizio}[\Ast]
\label{ese:13.15}
Risolvi le seguenti equazioni con le regole pratiche indicate.
 \begin{enumeratea}
 \item $2(1-x)-(x+2)=4x-3(2-x)$;
 \item $(x+2)^{2}=x^{2}-4x+4$;
 \item $5(3x-1)-7(2x-4)=28$;
 \item $(x+1)(x-1)+2x=5+x(2+x)$.
 \end{enumeratea}
\end{esercizio}

\begin{esercizio}[\Ast]
\label{ese:13.16}
Risolvi le seguenti equazioni con le regole pratiche indicate.
 \begin{enumeratea}
 \item $2x+(x+2)(x-2)+5=(x+1)^{2}$;
 \item $4(x-2)+3(x+2)=2(x-1)-(x+1)$;
 \item $(x+2)(x+3)-(x+3)^{2}=(x+1)(x-1)-x(x+1)$;
 \item $x^{3}+6x^{2}+(x+2)^{3}+11x+(x+2)^{2}=(x+3)\left(2x^{2}+7x\right)$.
 \end{enumeratea}
\end{esercizio}

\begin{esercizio}[\Ast]
\label{ese:13.17}
Risolvi le seguenti equazioni con le regole pratiche indicate.
 \begin{enumeratea}
 \item $(x+2)^{3}-(x-1)^{3}=9(x+1)^{2}-9x$;
 \item $(x+1)^{2}+2x+2(x-1)=(x+2)^{2}$;
 \item $2(x-2)(x+3)-3(x+1)(x-4)=-9(x-2)^{2}+\left(8x^{2}-25x+36\right)$.
 \end{enumeratea}
\end{esercizio}

\begin{esercizio}
\label{ese:13.18}
Risolvi le seguenti equazioni con le regole pratiche indicate.
 \begin{enumeratea}
 \item $(2x-3)^{2}-4x(2-5x)-4=-8x(x+4)$;
 \item $(x-1)\left(x^{2}+x+1\right)-3x^{2}=(x-1)^{3}+1$;
 \item $(2x-1)\left(4x^{2}+2x+1\right)=(2x-1)^{3}-12x^{2}$.
 \end{enumeratea}
\end{esercizio}

%%%%%%%%%%%%%%%%%%%%%%%%%%%%%%%%%%%%%%%%%%%%%%%%%%%%%%%%%%%%%%%%%%%%%%%

\subsubsection*{\thechapter.4 - Equazioni a coefficienti frazionari}

\begin{esercizio}
\label{ese:13.19}
Risolvi l'equazione~$\dfrac{3\cdot (x-11)}{4}=\dfrac{3\cdot (x+1)}{5}-\dfrac{1}{10}$.
\begin{enumerate}
\spazielenx
 \item calcola~$\mcm(4\text{,~}5\text{,~}10) = \ldots \ldots$;
 \item moltiplica ambo i membri per \dotfill e ottieni: \dotfill;
 \item \dotfill
\end{enumerate}
\end{esercizio}

%%%%%%%%%%%%%%%%%%%%%%%%%%%%%%%%%%%%%%%%%%%%%%%%%%%%%%%%%%%%%%%%%%%%%%%
\begin{esercizio}
\label{ese:13.20}
Risolvi le seguenti equazioni nell'insieme a fianco indicato.
\begin{multicols}{3}
\begin{enumeratea}
\spazielenx
 \item $x+7=8\quad$ in $\insN$;
 \item $4+x=2\quad$ in $\insZ$;
 \item $x-3=4\quad$ in $\insN$;
 \item $x=0\quad$ in $\insN$;
 \item $x+1=0\quad$ in $\insZ$;
 \item $5x=0\quad$ in $\insZ$.
\end{enumeratea}
\end{multicols}
\end{esercizio}

%\newpage
\begin{esercizio}
\label{ese:13.21}
Risolvi le seguenti equazioni nell'insieme a fianco indicato.
\begin{multicols}{3}
\begin{enumeratea}
\spazielenx
 \item $\dfrac{x}{4}=0$ in $\insQ$;
 \item $-x=0\quad$ in $\insZ$;
 \item $7+x=0\quad$ in $\insZ$;
 \item $-2x=0\quad$ in $\insZ$;
 \item $-x-1=0\quad$ in $\insZ$;
 \item $\dfrac{-x}{4}=0\quad$ in $\insQ$.
\end{enumeratea}
\end{multicols}
\end{esercizio}

\begin{esercizio}
\label{ese:13.22}
Risolvi le seguenti equazioni nell'insieme a fianco indicato.
\begin{multicols}{3}
\begin{enumeratea}
\spazielenx
 \item $x-\dfrac{2}{3}=0\quad$ in $\insQ$;
 \item $\dfrac{x}{-3}=0\quad$ in $\insZ$;
 \item $2(x-1)=0\quad$ in $\insZ$;
 \item $-3x=1\quad$ in $\insQ$;
 \item $3x=-1\quad$ in $\insQ$;
 \item $\dfrac{x}{3}=1\quad$ in $\insQ$.
\end{enumeratea}
\end{multicols}
\end{esercizio}

\begin{esercizio}
\label{ese:13.23}
Risolvi le seguenti equazioni nell'insieme a fianco indicato.
\begin{multicols}{3}
\begin{enumeratea}
\spazielenx
 \item $\dfrac{x}{3}=2\quad$ in $\insQ$;
 \item $\dfrac{x}{3}=-2\quad$ in $\insQ$;
 \item $0x=0\quad$ in $\insQ$;
 \item $0x=5\quad$ in $\insQ$;
 \item $0x=-5\quad$ in $\insQ$;
 \item $\dfrac{x}{1}=0\quad$ in $\insQ$.
\end{enumeratea}
\end{multicols}
\end{esercizio}

\begin{esercizio}
\label{ese:13.24}
Risolvi le seguenti equazioni nell'insieme a fianco indicato.
\begin{multicols}{3}
\begin{enumeratea}
\spazielenx
 \item $\dfrac{x}{1}=1\quad$ in $\insQ$;
 \item $-x=10\quad$ in $\insZ$;
 \item $\dfrac{x}{-1}=-1\quad$ in $\insZ$;
 \item $3x=3\quad$ in $\insN$;
 \item $-5x=2\quad$ in $\insZ$;
 \item $3x+2=0\quad$ in $\insQ$.
\end{enumeratea}
\end{multicols}
\end{esercizio}

\begin{esercizio}
\label{ese:13.25}
Risolvi le seguenti equazioni nell'insieme~$\insQ$.
\begin{multicols}{3}
\begin{enumeratea}
\spazielenx
 \item $3x=\dfrac{1}{3}$;
 \item $-3x=-{\dfrac{1}{3}}$;
 \item $x+2=0$;
 \item $4x-4=0$;
 \item $4x-0=1$;
 \item $2x+3=x+3$.
\end{enumeratea}
\end{multicols}
\end{esercizio}

\begin{esercizio}
\label{ese:13.26}
Risolvi le seguenti equazioni nell'insieme~$\insQ$.
\begin{multicols}{3}
\begin{enumeratea}
\spazielenx
 \item $4x-4=1$;
 \item $4x-1=1$;
 \item $4x-1=0$;
 \item $3x=12-x$;
 \item $4x-8=3x$;
 \item $-x-2=-2x-3$.
\end{enumeratea}
\end{multicols}
\end{esercizio}

\begin{esercizio}
\label{ese:13.27}
Risolvi le seguenti equazioni nell'insieme~$\insQ$.
\begin{multicols}{3}
\begin{enumeratea}
\spazielenx
 \item $-3(x-2)=3$;
 \item $x+2=2x+3$;
 \item $-x+2=2x+3$;
 \item $3(x-2)=0$;
 \item $3(x-2)=1$;
 \item $3(x-2)=3$.
\end{enumeratea}
\end{multicols}
\end{esercizio}

\begin{esercizio}
\label{ese:13.28}
Risolvi le seguenti equazioni nell'insieme~$\insQ$.
\begin{multicols}{3}
\begin{enumeratea}
\spazielenx
 \item $0(x-2)=1$;
 \item $0(x-2)=0$;
 \item $12+x=-9x$;
 \item $40x+3=30x-100$;
 \item $4x+8x=12x-8$;
 \item $-2-3x=-2x-4$.
\end{enumeratea}
\end{multicols}
\end{esercizio}

%\newpage
\begin{esercizio}
\label{ese:13.29}
Risolvi le seguenti equazioni nell'insieme~$\insQ$.
\begin{multicols}{3}
\begin{enumeratea}
\spazielenx
 \item $2x+2=2x+3$;
 \item $\dfrac{x+2}{2}=\dfrac{x+1}{2}$;
 \item $\dfrac{2x+1}{2}=x+1$;
 \item $\dfrac{x}{2}+\dfrac{1}{4}=3x-\dfrac{1}{2}$;
 \item $\pi x=0$;
 \item $2\pi x=\pi$.
\end{enumeratea}
\end{multicols}
\end{esercizio}

\begin{esercizio}
\label{ese:13.30}
Risolvi le seguenti equazioni nell'insieme~$\insQ$.
\begin{multicols}{2}
\begin{enumeratea}
\spazielenx
 \item $\np{0,12}x=\np{0,1}$;
 \item $-{\dfrac{1}{2}}x-\np{0,3}=-{\dfrac{2}{5}}x-\np{0,15}$;
 \item $892x-892=892x-892$;
 \item $892x-892=893x-892$;
 \item $348x-347=340x-347$;
 \item $340x+740=\np{8942}+340x$.
\end{enumeratea}
\end{multicols}
\end{esercizio}

\begin{esercizio}
\label{ese:13.31}
Risolvi le seguenti equazioni nell'insieme~$\insQ$.
\begin{multicols}{3}
\begin{enumeratea}
\spazielenx
 \item $2x+3=2x+4$;
 \item $2x+3=2x+3$;
 \item $2(x+3)=2x+5$;
 \item $2(x+4)=2x+8$;
 \item $3x+6=6x+6$;
 \item $-2x+3=-2x+4$.
\end{enumeratea}
\end{multicols}
\end{esercizio}

\begin{esercizio}
\label{ese:13.32}
Risolvi le seguenti equazioni nell'insieme~$\insQ$.
\begin{multicols}{2}
\begin{enumeratea}
\spazielenx
 \item $\dfrac{x}{2}+\dfrac{1}{4}=\dfrac{x}{4}-\dfrac{1}{2}$;
 \item $\dfrac{x}{2}+\dfrac{1}{4}=\dfrac{x}{2}-\dfrac{1}{2}$;
 \item $\dfrac{x}{2}+\dfrac{1}{4}=3\dfrac{x}{2}-\dfrac{1}{2}$;
 \item $\dfrac{x}{200}+\dfrac{1}{100}=\dfrac{1}{200}$;
 \item $\np{1000}x-100=\np{2000}x-200$;
 \item $100x-\np{1000}=\np{-1000}x+100$.
\end{enumeratea}
\end{multicols}
\end{esercizio}

\begin{esercizio}[\Ast]
\label{ese:13.33}
Risolvi le seguenti equazioni nell'insieme~$\insQ$.
\begin{multicols}{2}
\begin{enumeratea}
\spazielenx
 \item $x-5(1-x)=5+5x$;
 \item $2(x-5)-(1-x)=3x$;
 \item $3(2+x)=5(1+x)-3(2-x)$;
 \item $4(x-2)-3(x+2)=2(x-1)$;
 \item $\dfrac{x+\np{1000}}{3}+\dfrac{x+\np{1000}}{4}=1$;
 \item $\dfrac{x-4}{5}=\dfrac{2x+1}{3}$.
\end{enumeratea}
\end{multicols}
\end{esercizio}

\begin{esercizio}[\Ast]
\label{ese:13.34}
Risolvi le seguenti equazioni nell'insieme~$\insQ$.
\begin{multicols}{2}
\begin{enumeratea}
\spazielenx
 \item $\dfrac{x+1}{2}+\dfrac{x-1}{5}=\dfrac{1}{10}$;
 \item $\dfrac{x}{3}-\dfrac{1}{2}\;=\;\dfrac{x}{4}-\dfrac{x}{6}$;
 \item $8x-\dfrac{x}{6}=2x+11$;
 \item $3(x-1)-\dfrac{1}{7}=4(x-2)+1$;
 \item $537x+537\dfrac{x}{4}-\dfrac{537x}{7}=0$;
 \item $\dfrac{2x+3}{5}=x-1$.
\end{enumeratea}
\end{multicols}
\end{esercizio}

%\newpage
\begin{esercizio}[\Ast]
\label{ese:13.35}
Risolvi le seguenti equazioni nell'insieme~$\insQ$.
\begin{multicols}{2}
\begin{enumeratea}
\spazielenx
 \item $\dfrac{x}{2}-\dfrac{x}{6}-1=\dfrac{x}{3}$;
 \item $\dfrac{4-x}{5}+\dfrac{3-4x}{2}=3$;
 \item $\dfrac{x+3}{2}=3x-2$;
 \item $\dfrac{x+\np{0,25}}{5}=\np{1,75}-\np{0,}\overline{{3}}x$;
 \item $3(x-2)-4(5-x)=3x\left(1-\dfrac{1}{3}\right)$;
 \item $4(2x-1)+5=1-2(-3x-6)$.
\end{enumeratea}
\end{multicols}
\end{esercizio}

\begin{esercizio}[\Ast]
\label{ese:13.36}
Risolvi le seguenti equazioni nell'insieme~$\insQ$.
\begin{multicols}{2}
\begin{enumeratea}
\spazielenx
 \item $\dfrac{3}{2}(x+1)-\dfrac{1}{3}(1-x)=x+2$;
 \item $\dfrac{1}{2}(x+5)-x=\dfrac{1}{2}(3-x)$;
 \item $(x+3)^{2}\;=\;(x-2)(x+2)+\dfrac{1}{3}x$;
 \item $\dfrac{(x+1)^{2}}{4}-\dfrac{2+3x}{2}\;=\;\dfrac{(x-1)^{2}}{4}$;
 \item $2\left(x-\dfrac{1}{3}\right)+x\;=\;3x-2$;
 \item $\dfrac{3}{2}x+\dfrac{x}{4}\;=\;5\left(\dfrac{2}{3}x-\dfrac{1}{2}\right)-x$.
\end{enumeratea}
\end{multicols}
\end{esercizio}

\begin{esercizio}[\Ast]
\label{ese:13.37}
Risolvi le seguenti equazioni nell'insieme~$\insQ$.
\begin{multicols}{2}
\begin{enumeratea}
\spazielenx
 \item $(2x-3)(5+x)+\dfrac{1}{4}=2(x-1)^{2}-\dfrac{1}{2}$;
 \item $(x-2)(x+5)+\dfrac{1}{4}=x^{2}-\dfrac{1}{2}$;
 \item $\left(x-\dfrac{1}{2}\right)\left(x-\dfrac{1}{2}\right)=x^{2}+\dfrac{1}{2}$;
 \item $(x+1)^{2}=(x-1)^{2}$;
 \item $\dfrac{(1-x)^{2}}{2}-\dfrac{x^{2}-1}{2}=1$;
 \item $\dfrac{(x+1)^{2}}{3}=\dfrac{1}{3}(x^{2}-1)$.
\end{enumeratea}
\end{multicols}
\end{esercizio}

\begin{esercizio}[\Ast]
\label{ese:13.38}
Risolvi le seguenti equazioni nell'insieme~$\insQ$.
\begin{enumeratea}
\spazielenx
 \item $4(x+1)-3x(1-x)=(x+1)(x-1)+4+2x^{2}$;
 \item $\dfrac{1-x}{3}\cdot (x+1)=1-x^{2}+\dfrac{2}{3}\left(x^{2}-1\right)$;
 \item $(x+1)^{2}=x^{2}-1$;
 \item $(x+1)^{3}=(x+2)^{3}-3x(x+3)$;
 \item $\dfrac{1}{3}x\left(\dfrac{1}{3}x-1\right)+\dfrac{5}{3}x\left(1+\dfrac{1}{3}x\right)=\dfrac{2}{3}x(x+3)$;
 \item $\dfrac{1}{2}\left(3x+\dfrac{1}{3}\right)-(1-x)+2\left(\dfrac{1}{3}x-1\right)=-{\dfrac{3}{2}}x+1$.
\end{enumeratea}
\end{esercizio}

\begin{esercizio}[\Ast]
\label{ese:13.39}
Risolvi le seguenti equazioni nell'insieme~$\insQ$.
\begin{enumeratea}
\spazielenx
 \item $3+2x-\dfrac{1}{2}\left(\dfrac{x}{2}+1\right)-\dfrac{3}{4}x=\dfrac{3}{4}x+\dfrac{x+3}{2}$;
 \item $\dfrac{1}{2}\left[\dfrac{x+2}{2}-\left(x+\dfrac{1}{2}\right)+\dfrac{x+1}{2}\right]+\dfrac{1}{4}x=\dfrac{x-2}{4}-\left(x+\dfrac{2-x}{3}\right)$;
 \item $2\left(x-\dfrac{1}{2}\right)^{2}+\left(x+\dfrac{1}{2}\right)^{2}=(x+1)(3x-1)-5x-\dfrac{1}{2}$;
 \item $\dfrac{2\left(x-1\right)}{3}+\dfrac{x+1}{5}-\dfrac{3}{5}=\dfrac{x-1}{5}+\dfrac{7}{15}x$;
 \item $\dfrac{1}{2}(x-2)-\left(\dfrac{x+1}{2}-\dfrac{1+x}{2}\right)=\dfrac{1}{2}-\dfrac{2-x}{6}+\dfrac{1+x}{3}$;
 \item $-\left(\dfrac{1}{2}x+3\right)-\dfrac{1}{2}\left(x+\dfrac{5}{2}\right)+\dfrac{3}{4}(4x+1)=\dfrac{1}{2}(x-1)$.
\end{enumeratea}
\end{esercizio}

\begin{esercizio}[\Ast]
\label{ese:13.40}
Risolvi le seguenti equazioni nell'insieme~$\insQ$.
\begin{enumeratea}
\spazielenx
 \item $\dfrac{(x+1)(x-1)}{9}-\dfrac{3x-3}{6}=\dfrac{(x-1)^{2}}{9}-\dfrac{2-2x}{6}$;
 \item $\left(x-\dfrac{1}{2}\right)^{3}-\left(x+\dfrac{1}{2}\right)^{2}-x(x+1)(x-1)=\dfrac{-5}{2}x(x+1)$;
 \item $\dfrac{1}{2}\left(3x-\dfrac{1}{3}\right)-\dfrac{1}{3}(1+x)(-1+x)+3\left(\dfrac{1}{3}x-1\right)^{2}=\dfrac{2}{3}x$;
 \item $(x-2)(x-3)-6=(x+2)^2 +5$;
 \item $(x-3)(x-4)-\dfrac{1}{3}(1-3x)(2-x)=\dfrac{1}{3}x-5\left(\dfrac{2x-9}{6}\right)$;
 \item $\dfrac{2w-1}{3}+\dfrac{w-5}{4}=\dfrac{w+1}{3}-4$.
\end{enumeratea}
\end{esercizio}

\begin{esercizio}[\Ast]
\label{ese:13.41}
Risolvi le seguenti equazioni nell'insieme~$\insQ$.
\begin{enumeratea}
\spazielenx
 \item $(2x-5)^2 +2(x-3)=(4x-2)(x+3)-28x+25$;
 \item $\dfrac{(x-3)(x+3)+(x-2)(2-x)-3(x-2)}{\dfrac{1}{3}-3}=\dfrac{\dfrac{2}{3}x+\dfrac{1}{2}x}{2}$;
 \item $2\left(\dfrac{1}{2}x-1\right)^{2}-\dfrac{(x+2)(x-2)}{2}+2x=x+\dfrac{1}{2}$;
 \item $\left(\np{0,}\overline{{1}}x-10\right)^{2}+\np{0,1}(x-\np{0,2})+\left(\dfrac{1}{3}x+\np{0,3}\right)^{2}=\dfrac{10}{81}x^{2}+\np{0,07}$;
 \item $5x+\dfrac{1}{6}-\left(\dfrac{2x+1}{2}\right)^{2}+\left(\dfrac{3x-1}{3}\right)^{2}+\dfrac{1}{3}x+(2x-1)(2x+1)=(2x+1)^{2}+\dfrac{1}{36}$;
 \item \begin{multline*}\left(1+\dfrac{1}{2}x\right)^{3}-2\left(\dfrac{1}{2}x-2\right)^{2}+\left(\dfrac{3x-1}{3}\right)^{2}-\left(1-\dfrac{1}{3}x\right)x+\dfrac{1}{3}x={\dfrac{1}{3}}(2x+1)^{2} \\
     +\dfrac{1}{4}x^{2}-\dfrac{5}{9}+\dfrac{1}{2}x\left(\dfrac{1}{2}x+1\right)\left(\dfrac{1}{2}x-1\right).\end{multline*}
\end{enumeratea}
\end{esercizio}

\begin{esercizio}[\Ast]
\label{ese:13.42}%es-42 una colonna
Risolvi le seguenti equazioni nell'insieme~$\insQ$.
\begin{enumeratea}
\spazielenx
 \item $\left(\dfrac{1}{2}x+\dfrac{1}{3}\right)\left(\dfrac{1}{2}x-\dfrac{1}{3}\right)+\left(\dfrac{1}{2}+\dfrac{1}{3}\right)x=\left(\dfrac{1}{2}x+1\right)^{2}$;
 \item $\dfrac{3}{20}+\dfrac{6x+8}{10}-\dfrac{2x-1}{12}+\dfrac{2x-3}{6}=\dfrac{x-2}{4}$;
 \item $\dfrac{x^{3}-1}{18}+\dfrac{(x+2)^{3}}{9}=\dfrac{(x+1)^{3}}{4}-\dfrac{x^{3}+x^{2}-4}{12}$;
 \item $\dfrac{2}{3}x+\dfrac{5x-1}{3}+\dfrac{(x-3)^{2}}{6}+\dfrac{1}{3}(x+2)(x-2)=\dfrac{1}{2}(x-1)^{2}$;
 \item $\dfrac{5}{12}x-12+\dfrac{x-6}{2}-\dfrac{x-24}{3}=\dfrac{x+4}{4}-\left(\dfrac{5}{6}x-6\right)$;
 \item $\left(1-\dfrac{x+\dfrac{1}{2}}{1-\dfrac{1}{2}}\right)\left(1+\dfrac{\dfrac{1}{2}x+1}{\dfrac{1}{2}-1}\right)+\left(\dfrac{\dfrac{1}{2}x+1}{\dfrac{1}{2}+1}-1\right)\cdot {\dfrac{\dfrac{1}{2}+x}{\dfrac{1}{2}-1}}-\dfrac{x\left(\dfrac{1}{2}x+1\right)}{\dfrac{1}{2}+1}=x^{2}$.
\end{enumeratea}
\end{esercizio}

\begin{esercizio}
\label{ese:13.43}%es-43 una colonna
Risolvi le seguenti equazioni nell'insieme~$\insQ$.
\begin{enumeratea}
\spazielenx
 \item $x+\dfrac{1}{2}=\dfrac{x+3}{3}-1$;
 \item $\dfrac{2}{3}x+\dfrac{1}{2}=\dfrac{1}{6}x+\dfrac{1}{2}x$;
 \item $\dfrac{3}{2}=2x-\left[\dfrac{x-1}{3}-\left(\dfrac{\text{2x}+1}{2}-\text{5x}\right)-\dfrac{2-x}{3}\right]$;
 \item $\dfrac{x+5}{3}+3+\dfrac{2\cdot \left(x-1\right)}{3}=x+4$;
 \item $\dfrac{1}{5}x-1+\dfrac{2}{3}x-2=\dfrac{10}{15}+\dfrac{3}{5}x$;
 \item $\dfrac{1}{2}(x-2)^{2}-\dfrac{8x^{2}-25x+36}{18}+\dfrac{1}{9}(x-2)(x+3)=\dfrac{1}{6}(x+1)(x-4)$.
\end{enumeratea}
\end{esercizio}

\begin{esercizio}
\label{ese:13.44}
Per una sola delle seguenti equazioni, definite in~$\insZ$, l'insieme soluzione è vuoto. Per quale?
\[\boxA\quad x=x+1\qquad\boxB\quad x+1=0\qquad\boxC\quad x-1=+1\qquad\boxD\quad x+1=1\]
\end{esercizio}

\begin{esercizio}
\label{ese:13.45}
Una sola delle seguenti equazioni è di primo grado nella sola incognita~$x$. Quale?
\[\boxA\quad x+y=5\qquad\boxB\quad x^{2}+1=45\qquad\boxC\quad x-\dfrac{7}{89}=+1\qquad\boxD\quad x+x^{2}=1\]
\end{esercizio}

\begin{esercizio}
\label{ese:13.46}
Tra le seguenti una sola equazione non è equivalente alle altre. Quale?
\[\boxA\quad \dfrac{1}{2}x-1=3x\qquad\boxB\quad~6x=x-2\qquad\boxC\quad x-2x=3x\qquad\boxD\quad~3x=\dfrac{1}{2}(x-2)\]
\end{esercizio}

\begin{esercizio}
\label{ese:13.47}
Da~$8x=2$ si ottiene:
\[\boxA\quad x=-6\qquad\boxB\quad x=4\qquad\boxC\quad x=\dfrac{1}{4}\qquad\boxD\quad x=-{\dfrac{1}{4}}\]
\end{esercizio}

\begin{esercizio}
\label{ese:13.48}
Da~$-9x=0$ si ottiene:
\[\boxA\quad x=9\qquad\boxB\quad x=-{\dfrac{1}{9}}\qquad\boxC\quad x=0\qquad\boxD\quad x=\dfrac{1}{9}\]
\end{esercizio}

\begin{esercizio}
\label{ese:13.49}
L'insieme soluzione dell'equazione~$2\cdot \left(x+1\right)=5\cdot \left(x-1\right)-11$ è:
\[\boxA\quad \IS=\Bigl\{-6\Bigr\}\qquad\boxB\quad \IS=\Bigl\{6\Bigr\}\qquad\boxC\quad \IS=\left\{\dfrac{11}{3}\right\}\qquad\boxD\quad \IS=\left\{\dfrac{1}{6}\right\}\]
\end{esercizio}

\begin{esercizio}
\label{ese:13.50}
Per ogni equazione, individua quali tra gli elementi dell'insieme $Q$ indicato a fianco sono soluzioni:
\begin{enumeratea}
\spazielenx
 \item $\dfrac{x+5}{2}+\dfrac{1}{5}=0$, $\qquad Q=\left\{1\text{,~}-5\text{,~}7\text{,~}-\dfrac{27}{5}\right\}$;
 \item $x-\dfrac{3}{4}x=4$, $\qquad Q=\Bigl\{1\text{,~}-1\text{,~}0\text{,~}16\Bigr\}$;
 \item $x(x+1)+4=5-2x+x^{2}$,$\qquad Q=\left\{-9\text{,~}3\text{,~}\dfrac{1}{3}\text{,~}-\dfrac{1}{3}\right\}$.
\end{enumeratea}
\end{esercizio}

\subsection{Risposte}
\begin{multicols}{2}
\paragraph{\thechapter.14.}
a)~$x=2$,\quad b)~$x=\frac{1}{3}$, \quad c)~$x=\frac{2}{11}$, \quad d)~$x=-\frac{2}{3}$.

\paragraph{\thechapter.15.}
a)~$x=\frac{3}{5}$,\quad b)~$x=0$, \quad c)~$x=5$, \quad d)~Impossibile.

\paragraph{\thechapter.16.}
a)~Indeterminata,\quad b)~$x=-\frac{1}{6}$, \protect\\ c)~Impossibile, \quad d)~$x=-2$.

\paragraph{\thechapter.17.}
a)~Indeterminata,\quad b)~$x=\frac{5}{2}$, \quad c)~Indeterminata.

\paragraph{\thechapter.33.}
a)~$x=10$,\quad b)~Impossibile, \protect\\ c)~$x=\frac{7}{5}$, \quad d)~$x=-12$, \quad e)~$x=-\frac{\np{6988}}{7}$, \quad f)~$x=-\frac{17}{7}$.

\paragraph{\thechapter.34.}
a)~$x=-{\frac{2}{7}}$,\quad b)~$x=2$, \quad c)~$x=\frac{66}{35}$, \quad d)~$x=\frac{27}{7}$, \quad e)~$x=0$, \quad f)~$x=\frac{8}{3}$.

\paragraph{\thechapter.35.}
a)~Impossibile,\quad b)~$x=-{\frac{7}{22}}$, \protect\\ c)~$x=\frac{7}{5}$, \quad d)~$x=\frac{51}{16}$, \quad e)~$x=\frac{26}{5}$, \quad f)~$x=6$.

\paragraph{\thechapter.36.}
a)~$x=1$,\quad b)~Impossibile,\protect\\ c)~$x=-{\frac{39}{17}}$, \quad d)~$x=-2$, \quad e)~Impossibile, \quad f)~$x=\frac{30}{7}$.

\paragraph{\thechapter.37.}
a)~$x=\frac{65}{44}$,\quad b)~$x=\frac{37}{12}$, \quad c)~$x=-{\frac{1}{4}}$, \quad d)~$x=0$, \quad e)~$x=0$, \quad f)~$x=-1$.

\paragraph{\thechapter.38.}
a)~$x=-1$,\quad b)~Indeterminata, \protect\\c)~$x=-1$, \quad d)~Impossibile, \quad e)~$x=0$, \quad f)~$x=\frac{23}{28}$.

\paragraph{\thechapter.39.}
a)~$x=4$,\quad b)~$x=-{\frac{5}{2}}$, \quad c)~$x=-{\frac{9}{8}}$, \quad d)~$x=\frac{13}{3}$, \quad e)~Impossibile, \quad f)~$x=2$.

\paragraph{\thechapter.40.}
a)~$x=1$,\quad b)~$x=\frac{3}{26}$, \quad c)~$x=\frac{19}{7}$, \quad d)~$x=-1$, \quad e)~$x=\frac{23}{20}$, \quad f)~$x=-{\frac{25}{7}}$.

\paragraph{\thechapter.41.}
a)~Indeterminata,\quad b)~$x=\frac{63}{23}$, \protect\\ c)~$x=\frac{7}{2}$, \quad d)~$x=\frac{\np{9000}}{173}$, \quad e)~$x=-6$, \protect\\ f)~$x=2$.

\paragraph{\thechapter.42.}
a)~$x=-{\frac{20}{3}}$,\quad b)~$x=-2$, \quad c)~$x=-{\frac{3}{7}}$, \quad d)~$x=\frac{2}{7}$, \quad e)~$x=12$, \quad f)~$x=-{\frac{1}{5}}$.
\end{multicols}
