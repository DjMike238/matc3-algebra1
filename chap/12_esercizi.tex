% (c) 2012 Claudio Carboncini - claudio.carboncini@gmail.com
% (c) 2012 -2014 Dimitrios Vrettos - d.vrettos@gmail.com

\section{Esercizi}
\subsection{Esercizi dei singoli paragrafi}
\subsubsection*{\thechapter.1 - Divisioni in una variabile}

\begin{esercizio}
\label{ese:12.1}
Completa la divisione
\begin{center}
 % (c) 2012 Dimitrios Vrettos - d.vrettos@gmail.com
\begin{tikzpicture}[font=\small]

\matrix  (a) [matrix of  nodes, anchor=south, minimum width=9mm, ,nodes={text depth=2.5mm}]{
$7x^4$&$+0x^3$&$-5x^2$&$+x$&$-1$ &$2x^2$&$+0x$&$-1$\\
{}&{}&$\ldots$&{}&{}&$\displaystyle\frac{7}{2}x$&\ldots\\
{}&{}&$-\displaystyle\frac{3}{2}x^2$&$+x$&$-1$\\
{}&{}&{}&\ldots&{}\\
&&&$x$&$-\displaystyle\frac{7}{4}$\\
};

\draw(a-1-6.north west)--(a-2-6.south west);
\draw(a-1-6.south west)--(a-1-8.south east);
 \draw (a-2-1.south west) -- (a-2-5.south east);
 \draw (a-4-2.south west) -- (a-4-5.south east);
\end{tikzpicture}
\end{center}
\end{esercizio}

\begin{esercizio}[\Ast]
\label{ese:12.2}
Esegui le divisioni tra polinomi.
 \begin{enumeratea}
 \item $\left(3x^{2}-5x+4\right):\left(2x-2\right)$;
 \item $\left(4x^{3}-2x^{2}+2x-4\right):\left(3x-1\right)$;
 \item $\left(5a^{3}-a^{2}-4\right)\text{:}\left(a-2\right)$;
 \item $\left(6y^{5}-5y^{4}+y^{2}-1\right):\left(2y^{2}-3\right)$.
 \end{enumeratea}
\end{esercizio}

\begin{esercizio}[\Ast]
\label{ese:12.3}
Esegui le divisioni tra polinomi.
 \begin{enumeratea}
 \item $\left(-7a^{4}+3a^{2}-4+a\right):\left(a^{3}-2\right)$;
 \item $\left(x^{7}-4\right):\left(x^{3}-2x^{2}+3x-7\right)$;
 \item $\left(x^{3}-\dfrac{1}{2}x^{2}-4x+\dfrac{3}{2}\right):\left(x^{2}+3x\right)$;
 \item $\left(2x^{4}+2x^{3}-\dfrac{15}{2}x^{2}-15x-7\right):(2x+3)$.
 \end{enumeratea}
\end{esercizio}

\begin{esercizio}[\Ast]
\label{ese:12.4}
Esegui le divisioni tra polinomi.
 \begin{enumeratea}
 \item $\left(6-7a+3a^{2}-4a^{3}+a^{5}\right):\left(1-2a^{3}\right)$;
 \item $(a^{6}-1):(1+a^{3}+2a^{2}+2a)$;
 \item $\left(a^{4}-\dfrac{5}{4}a^{3}+\dfrac{11}{8}a^{2}-\dfrac{a}{2}\right):\left(a^{2}-\dfrac{a}{2}\right)$;
 \item $\left(2x^{3}-6x^{2}+6x-2\right):\left(2x-2\right)$.
 \end{enumeratea}
\end{esercizio}

\begin{esercizio}
\label{ese:12.5}
Esegui le divisioni tra polinomi.
 \begin{enumeratea}
 \item $\left(2x^{5}-11x^{3}+2x+2\right):\left(x^{3}-2x^{2}+1\right)$;
 \item $\left(15x^{4}-2x+5\right):\left(2x^{2}+3\right)$;
 \item $\left(-{\dfrac{9}{2}}x^{2}-2x^{4}+\dfrac{1}{2}x^{3}-\dfrac{69}{8}x-\dfrac{9}{4}-\dfrac{4}{3}x^{5}\right):\left(-2x^{2}-3x-\dfrac{3}{4}\right)$.
 \end{enumeratea}
\end{esercizio}

\subsubsection*{\thechapter.2 - Polinomi in più variabili}

\begin{esercizio}
\label{ese:12.6}
Dividi il polinomio~$A(x\text{,~}y)=x^{3}+3x^{2}y+2xy^{2}$ per il polinomio~$B(x\text{,~}y)=x+y$ rispetto alla variabile~$x$.
Il quoziente è~$Q(x\text{,~}y)=\ldots \ldots \ldots$, il resto è~$R(x\text{,~}y)=0$.

Ordina il polinomio~$A(x\text{,~}y)$ in modo decrescente rispetto alla variabile~$y$ ed esegui
nuovamente la divisione. Il quoziente è sempre lo stesso? Il resto è sempre zero?
\end{esercizio}

\begin{esercizio}
\label{ese:12.7}
Esegui le divisioni tra polinomi rispetto alla variabile~$x$.
 \begin{enumeratea}
 \item $\left(3x^{4}+5ax^{3}-a^{2}x^{2}-6a^{3}x+2a^{4}\right):\left(3x^{2}-ax-2a^{2}\right)$;
 \item $\left(-4x^{5}+13x^{3}y^{2}-12y^{3}x^{2}+17x^{4}y-12y^{5}\right):\left(2x^{3}-3yx^{2}+2y^{2}x-3y^{3}\right)$;
 \item $\left(x^{5}-x^{4}-2ax^{3}+3ax^{2}-2a\right):\left(x^{2}-2a\right)$.
 \end{enumeratea}
\end{esercizio}

\subsubsection*{\thechapter.3 - Regola di Ruffini}

\begin{esercizio}
\label{ese:12.8}
Completa la seguente divisione utilizzando la regola di Ruffini:\:$\left(x^{2}-3x+1\right):(x-3)$.
\begin{itemize*}
\item Calcolo del resto:~$(+3)^{2}-3(+3)+1=\ldots$;
\item calcolo del quoziente:~$Q(x)=1x+0=x$ \quad~$R=\ldots$;
\item verifica:~$(x-3)\cdot x+\ldots =x^{2}-3x+1$.
\end{itemize*}
\end{esercizio}

\begin{esercizio}[\Ast]
\label{ese:12.9}
Risolvi le seguenti divisioni utilizzando la regola di Ruffini.
 \begin{enumeratea}
 \item $\left(3x^{3}-4x^{2}+5x-1\right):(x-2)$;%ex~441
 \item $\left(x^{5}-x^{3}+x^{2}-1\right):(x-1)$;%ex~442
 \item $\left(x^{4}-10x^{2}+9\right):(x-3)$.%ex~443
 \end{enumeratea}
\end{esercizio}

\begin{esercizio}[\Ast]
\label{ese:12.10}
Risolvi le seguenti divisioni utilizzando la regola di Ruffini.
 \begin{enumeratea}
 \item $\left(x^{4}+5x^{2}+5x^{3}-5x-6 \right):(x+2)$;%ex~444
 \item $\left(4x^{3}-2x^{2}+2x-4 \right):(x+1)$;%ex~445
 \item $\left(\dfrac{4}{3}y^{4}-2y^{2}+\dfrac{3}{2}y-2\right):\left(y+\dfrac{1}{2}\right)$.%ex~446
 \end{enumeratea}
\end{esercizio}

\begin{esercizio}[\Ast]
\label{ese:12.11}
Risolvi le seguenti divisioni utilizzando la regola di Ruffini.
 \begin{enumeratea}
 \item $\left(\dfrac{1}{3}x^{5}-\dfrac{3}{2}x-2\right):(x+2)$;
 \item $\left(2a-\dfrac{4}{3}a^{4}-2a^{2}-\dfrac{1}{3}\right):\left(a-\dfrac{1}{2}\right)$;
 \item $\left(\dfrac{4}{3}y^{4}-\dfrac{3}{2}y^{3}+\dfrac{3}{2}y-2\right):\left(y+3\right)$.
 \end{enumeratea}
\end{esercizio}

\begin{esercizio}
\label{ese:12.12}
Risolvi le seguenti divisioni utilizzando la regola di Ruffini.
 \begin{enumeratea}
 \item $\left(27x^{3}-3x^{2}+2x+1\right):(x+3)$;
 \item $\left(2x^{4}-5x^{3}-3x+2\right):(x-1)$;
 \item $\left(\dfrac{3}{4}x^{2}-\dfrac{x^{3}}{3}+2x^{4}\right):\left(2x-\dfrac{3}{2}\right)$.
 \end{enumeratea}
\end{esercizio}

%\newpage

\begin{esercizio}
\label{ese:12.13}
Risolvi le seguenti divisioni utilizzando la regola di Ruffini.
 \begin{enumeratea}
 \item $\left(6a^{3}-9a^{2}+9a-6\right):(3a-2)$;
 \item $(2x^{4}-3x^{2}-5x+1):(2x-3)$;
 \item $\left(x^{5}+\dfrac{1}{3}x^{4}-2x^{2}-\dfrac{2}{3}x\right):\left(x+\dfrac{1}{3}\right)$.
 \end{enumeratea}
\end{esercizio}

\begin{esercizio}[\Ast]
\label{ese:12.14}
Risolvi le seguenti divisioni utilizzando la regola di Ruffini.
 \begin{enumeratea}
 \item $\left(x^{3}-2x^{2}+2x-4\right):(2x-2)$;
 \item $\left(3x^{4}-2x^{3}+x-1\right):(2x-3)$;
 \item $\left(\dfrac{3}{2}a^{4}-2a^{2}+a-\dfrac{1}{2}\right):(3a-1)$.
 \end{enumeratea}
\end{esercizio}

\begin{esercizio}[\Ast]
\label{ese:12.15}
Risolvi le seguenti divisioni nella variabile~$a$.
 \begin{enumeratea}
 \item $\left(3a^{4}b^{4}+a^{2}b^{2}+2ab+2\right):(ab-1)$;
 \item $\left(3a^{4}b^{2}-2a^{2}b\right):(a^{2}b-3)$.
 \end{enumeratea}
\end{esercizio}

\begin{esercizio}[\Ast]
\label{ese:12.16}
Risolvi le seguenti divisioni nella variabile~$x$ utilizzando la regola di Ruffini.
 \begin{enumeratea}
 \item $\left(x^{4}-ax^{3}-4a^{2}x^{2}+7a^{3}x-6a^{4}\right):(x-2a)$;
 \item $\left(x^{4}-2ax^{3}+2a^{3}x-a^{4}\right):(x+a)$.
 \end{enumeratea}
\end{esercizio}

\begin{esercizio}[\Ast]
\label{ese:12.17}
Risolvi utilizzando, quando puoi, il teorema di Ruffini.
 \begin{enumeratea}
 \item Per quale valore di~$k$ il polinomio~$x^{3}-2x^{2}+kx+2$ è divisibile per~$x^{2}-1$?
 \item Per quale valore di~$k$ il polinomio~$x^{3}-2x^{2}+kx$ è divisibile per~$x^{2}-1$?
 \item Per quale valore di~$k$ il polinomio~$x^{3}-3x^{2}+x-k$ è divisibile per~$x+2$?
 \item Scrivi, se possibile, un polinomio nella variabile~$a$ che, diviso per~$a^{2}-1$ dà come quoziente~$a^{2}+1$ e come resto~$-1$.
 \end{enumeratea}
\end{esercizio}

\begin{esercizio}[\Ast]
\label{ese:12.18}
Risolvi utilizzando il teorema di Ruffini.
 \begin{enumeratea}
 \item Trovare un polinomio di secondo grado nella variabile~$x$ che risulti divisibile per~$(x-1)$ e per~$(x-2)$ e tale che
     il resto della divisione per~$(x-3)$ sia uguale a~$-4$;
 \item Per quale valore di~$a$ la divisione~$\left(2x^{2}-ax+3\right):(x+1)$ dà resto~$5$?
 \item Per quale valore di~$k$ il polinomio~$2x^{3}-x^{2}+kx-3k$ è divisibile per~$x+2$?
 \item I polinomi~$A(x)=x^3+2x^2-x+3k-2$ e~$B(x)=kx^2-(3k-1)x-4k+7$ divisi entrambi per~$x+1$ per quale valore di~$k$ hanno lo stesso resto?
 \end{enumeratea}
\end{esercizio}

\subsection{Risposte}
\paragraph{\thechapter.2.}
a)~$Q(x)=\frac{3}{2}x-1; R(x)=2$,\quad b)~$Q(x)=\frac{4}{3}x^{2}-\frac{2}{9}x+\frac{16}{27}; R(x)=-{\frac{92}{27}}$,\quad
c)~{$Q(a)=5a^{2}+9a+18$}; $R(a)=32$,\quad d)~$Q(y)=3y^{3}-\frac{5}{2}y^{2}+\frac{9}{2}y-\frac{13}{4}$; $R(y)=\frac{27}{2}y-\frac{43}{4}$.
\paragraph{\thechapter.3.}
a)~$Q(a)=-7a; R(a)=3a^{2}-13a-4$,\quad b)~$Q(x)=x^{4}+2x^{3}+x^{2}+3x+17$;\protect\\ ${R(x)=32x^{2}-30x+115}$,\quad
c)~$Q(x)=x-\frac{7}{2}$; $R(x)=\frac{13}{2}x+\frac{3}{2}$, d)~$Q(x)=x^{3}-\frac{1}{2}x^{2}-3x-~3$; $R(x)=2$.
\paragraph{\thechapter.4.}
a)~$Q(a)=2-\frac{1}{2}a^{2}; R(a)=\frac{7}{2}a^{2}-7a+4$,\quad b)~$Q(a)=a^{3}-2a^{2}+2a-1; R(a)=0$,\quad
c)~$Q(a)=a^{2}-\frac{3}{4}a+1; R(a)=0$,\quad d)~$Q(x)=x^{2}-2x+1; R(x)=0$.
\paragraph{\thechapter.9.}
a)~$Q(x)=3x^{2}+2x+9; R(x)=17$,\quad b)~$Q(x)=x^{4}+x^{3}+x+1; R(x)=0$, \protect\\ c)~$Q(x)=~x^{3}+3x^{2}-x-3; R(x)=0$.
\paragraph{\thechapter.10.}
a)~$Q(x)=x^{3}+3x^{2}-x-3; R(x)=0$,\quad b)~$Q(x)=4x^{2}-6x+8; R(x)=-12$,\protect\\ c)~$Q(y)=\frac{4}{3}y^{3}-\frac{2}{3}y^{2}-\frac{5}{3}y+\frac{7}{3}; R(y)=-\frac{19}{6}$.
\paragraph{\thechapter.11.}
a)~$Q(x)=\frac{1}{3}x^{4}-\frac{2}{3}x^{3}+\frac{4}{3}x^{2}-\frac{8}{3}x+\frac{23}{6}; R(x)=-\frac{29}{3}$,
\,b)~$Q(a)=~{-\frac{4}{3}a^{3}-\frac{2}{3}a^{2}-\frac{7}{3}a+\frac{5}{6}}$; $R(a)=\frac{1}{12}$,\quad c)~$Q(y)=\frac{4}{3}y^{3}-\frac{11}{2}y^{2}+\frac{33}{2}y-48; R(y)=142$.
\paragraph{\thechapter.14.}
a)~$Q(x)=\frac{1}{2}x^{2}-\frac{1}{2}x+\frac{1}{2}$; $R(x)=-3$,\quad b)~$Q(x)=\frac{3}{2}x^{3}+\frac{5}{4}x^{2}+\frac{15}{8}x+\frac{53}{16}$; $R(x)=\frac{143}{16}$,
\quad c)~$Q(a)=\frac{1}{2}a^{3}+\frac{1}{6}a^{2}-\frac{11}{18}a+\frac{7}{54}$; $R(a)=-{\frac{10}{27}}$.
\paragraph{\thechapter.15.}
a)~$Q(a)=3a^{3}b^{3}+3a^{2}b^{2}+4ab+6$; $R(a)=8$,\quad b)~$Q(a)=3a^{2}b+7$; $R(a)=21$.
\paragraph{\thechapter.16.}
a)~$Q(x)=x^{3}+ax^{2}-2a^{2}x+3a^{3}$; $R(x)=0$\quad b)~$Q(x)=x^3-3ax^2+3a^2 x-a^3$; $R(x)=0$.
\paragraph{\thechapter.17.}
a)~$k=-1$,\quad b)~nessuno,\quad c)~$k=-22$,\quad d)~$a^{4}-2$.
\paragraph{\thechapter.18.}
a)~$-2x^2+6x-4$,\quad b)~$a=0$,\quad c)~$k=-4$,\quad d)~$k=2$.
