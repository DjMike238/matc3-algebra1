% (c) 2014 Daniele Masini - d.masini.it@gmail.com
\section{Esercizi}
%\subsection{Esercizi dei singoli paragrafi}
%\subsubsection*{\thechapter.4 - Operazioni con i numeri naturali}

\begin{esercizio}
\label{ese:0.1}
Quali delle seguenti frasi sono proposizioni logiche?
\begin{enumeratea}
\item I matematici sono intelligenti				\tab\qquad\boxV\qquad\boxF
\item 12 è un numero dispari						\tab\qquad\boxV\qquad\boxF
\item Pascoli è stato un grande poeta			\tab\qquad\boxV\qquad\boxF
\item Pascoli ha scritto La Divina Commedia		\tab\qquad\boxV\qquad\boxF
\item Pascoli ha scritto poesie					\tab\qquad\boxV\qquad\boxF
\item Lucia è una bella ragazza					\tab\qquad\boxV\qquad\boxF
\item Lucia ha preso 8 al compito di matematica	\tab\qquad\boxV\qquad\boxF
\item Che bella serata!							\tab\qquad\boxV\qquad\boxF
\item Il rombo è una figura storta				\tab\qquad\boxV\qquad\boxF
\item Per favore fate silenzio!					\tab\qquad\boxV\qquad\boxF
\item 2+2=5										\tab\qquad\boxV\qquad\boxF
\item I miei insegnanti sono tutti laureati		\tab\qquad\boxV\qquad\boxF
\end{enumeratea}
\end{esercizio}

\begin{esercizio}
\label{ese:0.2}
A partire dalle due proposizioni $p =\;$<<16 è divisibile per 2>> e $q =\;$<<16 è divisibile per 4>>,
costruisci le proposizioni $p\vee q$ e $p\wedge q$.
\end{esercizio}

\begin{esercizio}
\label{ese:0.3}
A partire dalle proposizioni $p=\;$<<18 è divisibile per 3>> e $q=\;$<<18 è numero dispari>>
costruisci le proposizioni di seguito indicate e stabilisci il loro valore di verità
\begin{multicols}{2}
\begin{enumeratea}
\item $p\vee q$				\tab\qquad\boxV\qquad\boxF
\item $p\wedge q$			\tab\qquad\boxV\qquad\boxF
\item $\neg p$				\tab\qquad\boxV\qquad\boxF
\item $\neg q$				\tab\qquad\boxV\qquad\boxF
\item $p\vee \neg q$			\tab\qquad\boxV\qquad\boxF
\item $\neg p\wedge q$		\tab\qquad\boxV\qquad\boxF
\item $p\wedge \neg q$		\tab\qquad\boxV\qquad\boxF
\item $\neg p\vee\neg q$		\tab\qquad\boxV\qquad\boxF
\item $\neg p\wedge\neg q$	\tab\qquad\boxV\qquad\boxF
\item $\neg (p\wedge q)$		\tab\qquad\boxV\qquad\boxF
\end{enumeratea}
\end{multicols}
\end{esercizio}

\begin{esercizio}
\label{ese:0.4}
A partire dalle proposizioni $a=\;$<<20 è minore di 10>>, $b=\;$<<20 è maggiore di 1>>, $c=\;$<<20 è multiplo di 5>>, $d=\;$<<20 è dispari>>, stabilisci il valore di verità delle seguenti proposizioni:
\begin{multicols}{2}
\begin{enumeratea}
\item $a\vee b$				\tab\boxV\quad\boxF
\item $a\wedge c$			\tab\boxV\quad\boxF
\item $d\wedge a$			\tab\boxV\quad\boxF
\item $\neg a \wedge b$		\tab\boxV\quad\boxF
\item $a\vee \neg b$			\tab\boxV\quad\boxF
\item $\neg b\wedge \neg a$	\tab\boxV\quad\boxF
\item $(\neg a\wedge \neg b)\vee(c\wedge d)$	\tab\boxV\quad\boxF
\item $(a\vee \neg b)\wedge(c \vee \neg d)$	\tab\boxV\quad\boxF
\end{enumeratea}
\end{multicols}
\end{esercizio}

\begin{esercizio}
\label{ese:0.5}
Date le proposizioni $p=\;$<<Oggi è lunedì>> e $q=\;$<<Oggi studio matematica>>, scrivi in simboli le seguenti proposizioni:
\begin{enumeratea}
\item <<Oggi è lunedì e studio matematica>>
\item <<Oggi non è lunedì e studio matematica>>
\item <<Oggi è lunedì e non studio matematica>>
\item <<Oggi non è lunedì e non studio matematica>>
\end{enumeratea}
\end{esercizio}

\begin{esercizio}
\label{ese:0.6}
In quale delle seguenti proposizioni si deve usare la ``o'' inclusiva e in quali la ``o'' esclusiva?
\begin{enumeratea}
\item <<Nelle fermate a richiesta l'autobus si ferma se qualche persona deve scendere o salire>>
\item <<Luca sposerà Maria o Claudia>>
\item <<Fammi chiamare da Laura o da Elisa>>
\item <<Si raggiunge l'unanimità quando sono tutti favorevoli o tutti contrari>>
\item <<Vado al cinema con Carla o con Luisa>>
\item <<Per le vacanze andrò al mare o a Firenze>>
\end{enumeratea}
\end{esercizio}

\begin{esercizio}
\label{ese:0.7}
A partire dalle proposizioni $p=\;$<<Oggi pioverà>> e $\neg p=\;$<<Oggi non pioverà>>, scrivere le proposizioni $p\veebar \neg p$, $p\vee \neg p$, $p\wedge \neg p$. Scrivere quindi la loro tabella della verità.
\end{esercizio}

\begin{esercizio}
\label{ese:0.8}
Scrivere le tabelle di verità delle formule 
\begin{multicols}{3}
\begin{enumeratea}
\item $p\wedge(p\vee q)$
\item $p\vee(p\wedge q)$
\item $p\veebar(p\wedge q)$
\item $p\wedge(p\veebar q)$
\item $(p\vee \neg q)\wedge (\neg p\vee q)$
\item $(p\vee q)\wedge q$
\item $(\neg p\vee q)\wedge (p\wedge q)$
\item $\neg(p\vee q)\wedge (p\vee \neg q)$
\item $(p\vee \neg q)\wedge \neg p$
\item $(\neg p\vee q)\wedge \neg q$
\item $(p\wedge q)\wedge \neg p$
\item $(p\vee r)\vee \neg q$
\end{enumeratea}
\end{multicols}
\end{esercizio}

\begin{esercizio}
\label{ese:0.9}
Costruisci la tavola di verità delle seguenti proposizioni
\begin{multicols}{3}
\begin{enumeratea}
\item $(\neg p\vee q)\wedge(p\wedge r)$
\item $\neg(p\vee q)\wedge(r\vee \neg q)$
\item $(p\vee\neg q)\wedge \neg r$
\item $p\vee\neg(r\veebar q)$
\item $(\neg p\veebar q)\wedge(r\vee q)$
\item $\neg(p\vee r)\wedge(r\veebar q)$
\item $(p\vee\neg q)\wedge\neg r$
\item $(\neg p\vee\neg q)\wedge(r\vee\neg q)$
\item $\neg((p\veebar q)\wedge \neg r)$
\end{enumeratea}
\end{multicols}
\end{esercizio}

\begin{esercizio}
\label{ese:0.10}
Verificare che, date due proposizioni $p$ e $q$, la proposizione composta $(\neg p\wedge q)\vee(p\wedge \neg q)$ è equivalente alla proposizione $p\veebar q$. Dimostrare l'equivalenza verificando che le tavole della verità sono uguali.
\end{esercizio}

\begin{esercizio}
\label{ese:0.11}
Se $p\wedge q$ è falso, quale dei seguenti enunciati è vero?
\begin{multicols}{4}
\begin{enumeratea}
\item $p\wedge \neg q$
\item $\neg p\wedge q$
\item $\neg p\wedge\neg q$
\item $\neg p\vee\neg q$
\end{enumeratea}
\end{multicols}
%Risposta [D]
\end{esercizio}

\begin{esercizio}
\label{ese:0.12}
Qual è la negazione della frase <<Ogni volta che ho preso l'ombrello non è piovuto>>?
\begin{enumeratea}
\item <<Almeno una volta sono uscito con l'ombrello ed è piovuto>>
\item <<Quando esco senza ombrello piove sempre>>
\item <<Tutti i giorni in cui non piove esco con l'ombrello>>
\item <<Tutti i giorni che è piovuto ho preso l'ombrello>>
\end{enumeratea}
\end{esercizio}

\begin{esercizio}
\label{ese:0.13}
Scrivi le negazioni delle seguenti frasi che contengono dei quantificatori:
\begin{enumeratea}
\item <<Al compito di matematica eravamo tutti presenti>>
\item <<Ogni giorno il professore ci dà sempre compiti per casa>>
\item <<Ogni giorno Luca vede il telegiornale>>
\item <<Tutti i miei familiari portano gli occhiali>>
\item <<Tutti hanno portato i soldi per la gita>>
\end{enumeratea}
\end{esercizio}

\begin{esercizio}
\label{ese:0.14}
Sono date le frasi $p=\;$<<Mario è cittadino romano>> e $q=\;$<<Mario è cittadino italiano>>, scrivi per esteso le seguenti implicazioni e indica quale di esse è vera.
\begin{multicols}{3}
\begin{enumeratea}
\item $p\Rightarrow q$		\tab\boxV\quad\boxF
\item $q\Rightarrow p$		\tab\boxV\quad\boxF
\item $q\Leftrightarrow p$	\tab\boxV\quad\boxF
\end{enumeratea}
\end{multicols}
\end{esercizio}

\begin{esercizio}
\label{ese:0.15}
Trasforma nella forma <<Se \ldots{} allora \ldots{}>> le seguenti frasi:
\begin{enumeratea}
\item <<Un oggetto lanciato verso l'alto ricade a terra>>
\item <<Quando piove prendo l'ombrello>>
\item <<I numeri la cui ultima cifra è 0 sono divisibili per 5>>
\item <<Per essere promosso occorre aver raggiunto la sufficienza>>
\end{enumeratea}
\end{esercizio}

\begin{esercizio}
\label{ese:0.16}
Date le proposizioni $p$, $q$, e $r$ costruire la tavola di verità delle seguenti proposizioni
\begin{multicols}{3}
\begin{enumeratea}
\item $p\Rightarrow\neg q$
\item $\neg p\Rightarrow q$
\item $(p\vee q)\Rightarrow \neg q$
\item $p\Rightarrow (q\vee \neg q)$
\item $(p\wedge q)\Rightarrow(p\vee q)$
\item $p\vee(p\Rightarrow q)$
\item $(p\wedge q)\Rightarrow(\neg q\vee r)$
\item $(p\wedge q)\Leftrightarrow(\neg p\vee \neg q)$
\item $(p\Rightarrow q)\wedge \neg q$
\item $(p\Rightarrow q)\vee(q\Rightarrow p)$
\item $(\neg p\vee\neg q)\Leftrightarrow(p\wedge q)$
\item $\neg(\neg p\wedge r)\Leftrightarrow(q\vee\neg r)$
\end{enumeratea}
\end{multicols}
\end{esercizio}

\begin{esercizio}
\label{ese:0.17}
Completa i seguenti ragionamenti:
\begin{enumeratea}
\item <<Se un numero è multiplo di 10 allora è pari>>; <<il numero n non è pari quindi \ldots\ldots\ldots\ldots>>
\item <<Se il sole tramonta fa buio>>; <<il sole è tramontato quindi \ldots\ldots\ldots\ldots>>
\end{enumeratea}
\end{esercizio}

\begin{esercizio}
\label{ese:0.18}
Dimostra con un controesempio che l'affermazione <<Tutti i multipli di 3 sono dispari>> non è vera.
\end{esercizio}

\begin{esercizio}[Giochi d'autunno, 2010]
\label{ese:0.19}
Ecco le dichiarazioni rilasciate da quattro amiche:\\
Anna: <<Io sono la più anziana>>;\\
Carla: <<Io non sono né la più giovane né la più anziana>>;\\
Liliana: <<Io non sono la più giovane>>;\\
Milena: <<Io sono la più giovane>>.\\
Il fatto è che una di loro (e solo una) ha mentito. Chi è, delle quattro amiche, effettivamente la più giovane?
\end{esercizio}

\begin{esercizio}[I Giochi di Archimede, 2011]
\label{ese:0.20}
Dopo una rissa in campo l'arbitro vuole espellere il capitano di una squadra di calcio. \`E uno tra Paolo, Andrea e Gabriele ma, siccome nessuno ha la fascia al braccio, non sa qual è dei tre. Paolo dice di non essere il capitano; Andrea dice che il capitano è Gabriele; Gabriele dice che il capitano è uno degli altri due. Sapendo che uno solo dei tre dice la verità, quale delle affermazioni seguenti è sicuramente vera?
\begin{enumeratea}
\item Gabriele non è il capitano;
\item Andrea dice la verità;
\item Paolo dice la verità;
\item Andrea è il capitano;
\item Gabriele mente.
\end{enumeratea}
\end{esercizio}

\begin{esercizio}[I Giochi di Archimede, 2010]
\label{ese:0.21}
 21   Un celebre investigatore sta cercando il colpevole di un omicidio tra cinque sospettati: Anna, Bruno, Cecilia, Dario ed Enrico. Egli sa che il colpevole mente sempre e gli altri dicono sempre la verità. Anna afferma: <<Il colpevole è un maschio>>, Cecilia dice: <<\`E stata Anna oppure è stato Enrico>>. Infine Enrico dice: <<Se Bruno è colpevole allora Anna è innocente>>. Chi ha commesso l'omicidio?
\end{esercizio}

\begin{esercizio}[I Giochi di Archimede, 2009]
\label{ese:0.22}
Quattro amici, Anna, Bea, Caio e Dino, giocano a poker con 20 carte di uno stesso mazzo: i quattro re, le quattro regine, i quattro fanti, i quattro assi e i quattro dieci. Vengono distribuite cinque carte a testa. Anna dice: <<Io ho un poker!>> (quattro carte dello stesso valore). Bea dice: <<Io ho tutte e cinque le carte di cuori>>. Caio dice: <<Io ho cinque carte rosse>>. Infine Dino dice: <<Io ho tre carte di uno stesso valore e anche le altre due hanno lo stesso valore>>. Sappiamo che una e una sola delle affermazioni è falsa; chi sta mentendo?
\end{esercizio}

\begin{esercizio}[I Giochi di Archimede, 2008]
\label{ese:0.23}
Un satellite munito di telecamera inviato sul pianeta Papilla ha permesso di stabilire che è falsa la convinzione di qualcuno che: <<su Papilla sono tutti grassi e sporchi>>. Quindi adesso sappiamo che:
\begin{enumeratea}
\item <<su Papilla almeno un abitante è magro e pulito>>;
\item <<su Papilla tutti gli abitanti sono magri e puliti>>;
\item <<almeno un abitante di Papilla è magro>>;
\item <<almeno un abitante di Papilla è pulito>>;
\item <<se su Papilla tutti gli abitanti sono sporchi, almeno uno di loro è magro>>.
\end{enumeratea}
\end{esercizio}

\begin{esercizio}[I Giochi di Archimede, 2000]
\label{ese:0.24}
Anna, Barbara, Chiara e Donatella si sono sfidate in una gara di nuoto fino alla boa. All'arrivo non ci sono stati ex-equo. Al ritorno, Anna dice: <<Chiara è arrivata prima di Barbara>>; Barbara dice: <<Chiara è arrivata prima di Anna>>; Chiara dice: <<Io sono arrivata seconda>>. Sapendo che una sola di esse ha detto la verità
\begin{enumeratea}
\item si può dire solo chi ha vinto, 
\item si può dire solo chi è arrivata seconda, 
\item si può dire solo chi è arrivata terza, 
\item si può dire solo chi è arrivata ultima, 
\item non si può stabile la posizione in classifica di nessuna. 
\end{enumeratea}
\end{esercizio}

\begin{esercizio}[I Giochi di Archimede, 1999]
\label{ese:0.25}
<<In ogni scuola c'è almeno una classe in cui sono tutti promossi>>. Volendo negare questa affermazione, quale dei seguenti enunciati sceglieresti?
\begin{enumeratea}
\item <<In ogni scuola c'è almeno una classe in cui sono tutti bocciati>>;
\item <<In ogni scuola c'è almeno un bocciato in tutte le classi;
\item <<C'è almeno una scuola che ha almeno un bocciato in ogni classe>>;
\item <<C'è almeno una scuola in cui c'è una classe che ha almeno un bocciato>>.
\end{enumeratea}
\end{esercizio}

\begin{esercizio}[I Giochi di Archimede, 1997]
\label{ese:0.26}
<<Se il pomeriggio ho giocato a tennis, la sera ho fame e se la sera ho fame, allora mangio troppo>>. Quale delle seguenti conclusioni non posso trarre da queste premesse?
\begin{enumeratea}
\item <<Se gioco a tennis il pomeriggio, allora la sera ho fame e mangio troppo>>;
\item <<Se la sera ho fame, allora mangio troppo, oppure ho giocato a tennis il pomeriggio>>;
\item <<Se la sera non ho fame, allora non ho giocato a tennis il pomeriggio>>;
\item <<Se la sera non ho fame, allora non mangio troppo>>;
\item <<Se la sera non mangio troppo, allora non ho giocato a tennis il pomeriggio>>.
\end{enumeratea}
\end{esercizio}

\begin{esercizio}[I Giochi di Archimede, 1998]
\label{ese:0.27}
Su un isola vivono tre categorie di persone: i cavalieri, che dicono sempre la verità, i furfanti, che mentono sempre, ed i paggi che dopo una verità dicono sempre una menzogna e viceversa. Sull'isola incontro un vecchio, un ragazzo e una ragazza. Il vecchio afferma: <<Io sono paggio>> e <<Il ragazzo è cavaliere>>. Il ragazzo dice: <<Io sono cavaliere>> e <<La ragazza è paggio>>. La ragazza afferma infine: <<Io sono furfante>> e <<Il vecchio è paggio>>. Si può allora affermare che:
\begin{enumeratea}
\item c'è esattamente un paggio;
\item ci sono esattamente due paggi;
\item ci sono esattamente tre paggi;
\item non c'è alcun paggio;
\item il numero dei paggi non è sicuro.
\end{enumeratea}
\end{esercizio}

\begin{esercizio}
\label{ese:0.28}
Dimostra che in ogni festa c'è sempre una coppia di persone che balla con lo stesso numero di invitati.
\end{esercizio}

\begin{esercizio}
\label{ese:0.29}
Mr.~Smith, Mr.~Taylor e Mr.~Elder insegnano 6 diverse materie (Biologia, Geografia, Matematica, Storia, Inglese e Fisica), ciascuno di essi due materie. Abbiamo le seguenti informazioni: Gli insegnanti di Fisica ed Inglese sono vicini di casa; Mr.~Smith è il più giovane dei tre. Mr.~Elder gioca a poker con l'insegnante d'Inglese e con quello di Biologia ogni domenica. L'insegnante di Biologia è più vecchio di quello di Matematica. L'insegnante di Geografia, quello di Matematica e Mr.~Smith andranno a fare un giro in bici il prossimo weekend. Associare ogni insegnante alle materie che insegna.
% [Sm: S e F; T: B e I; E: G e M].
\end{esercizio}

\begin{esercizio}[Test di ammissione a Medicina 1997]
\label{ese:0.30}
Un alano, un boxer, un collie e un dobermann vincono i primi 4 premi ad una mostra canina. I loro padroni sono il Sig.~Estro, il Sig.~Forti, il Sig.~Grassi ed il Sig.~Rossi, non necessariamente in quest'ordine. I nomi dei cani sono Jack, Kelly, Lad, Max, non necessariamente in quest'ordine. Disponiamo inoltre delle seguenti informazioni:
\begin{enumerate}
\item cane del Sig.~Grassi non ha vinto né il primo, né il secondo premio;
\item il collie ha vinto il primo premio;
\item Max ha vinto il secondo premio;
\item l'alano si chiama Jack;
\item il cane del Sig.~Forti, il dobermann, ha vinto il quarto premio;
\item il cane del Sig.~Rossi si chiama Kelly.
\end{enumerate}
Da quale cane e' stato vinto il primo premio?
\begin{multicols}{2}
\begin{enumeratea}
\item Il cane del Sig.~Estro
\item Il cane del Sig.~Rossi
\item Max
\item Jack
\item Lad
\end{enumeratea}
\end{multicols}
\end{esercizio}

\begin{esercizio}[Test di ammissione a Ingegneria 1999]
\label{ese:0.31}
In una squadra di calcio giocano Amilcare, Bertoldo e Carletto nei ruoli di portiere, centravanti, libero (non necessariamente in quest'ordine). Si sa che:
\begin{enumerate}
\item Il centravanti è il più basso di statura ed è scapolo;
\item Amilcare è il suocero di Carletto ed è più alto del portiere.
\end{enumerate}
Quale delle seguenti affermazioni è necessariamente vera?
\begin{multicols}{2}
\begin{enumeratea}
\item Bertoldo è il genero di Carletto;
\item Bertoldo ha sposato la sorella di Carletto;
\item Carletto è il portiere;
\item Carletto è scapolo;
\item Amilcare è il centravanti.
\end{enumeratea}
\end{multicols}
\end{esercizio}
