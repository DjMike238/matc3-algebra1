% (c) 2012 Claudio Carboncini - claudio.carboncini@gmail.com
% (c) 2012-2014 Dimitrios Vrettos - d.vrettos@gmail.com
\section{Esercizi}
\subsection{Esercizi dei singoli paragrafi}
\subsubsection*{16.1 - Quadrato di un binomio}
\begin{multicols}{2}
\begin{esercizio}
\label{ese:16.1}
Quando è possibile, scomponi in fattori, riconoscendo il quadrato di un binomio.
\begin{enumeratea}
 \item $a^{2}-2a+1$;
 \item $x^{2}+4x+4$;
 \item $y^{2}-6y+9$;
 \item $16t^{2}+8t+1$;
 \item $4x^{2}+1+4x$;
 \item $9a^{2}-6a+1$.
\end{enumeratea}
\end{esercizio}

\begin{esercizio}
\label{ese:16.2}
Quando è possibile, scomponi in fattori, riconoscendo il quadrato di un binomio.
\begin{enumeratea}
 \item $4x^{2}-12x+9$;
 \item $\dfrac{1}{4}a^{2}+ab+b^{2}$;
 \item $9x^{2}+4+12x$;
 \item $\dfrac{4}{9}a^{{4}}-4a^{2}+9$;
 \item $\dfrac{1}{4}x^{2}-\dfrac{1}{3}x+\dfrac{1}{9}$;
 \item $16a^{2}+\dfrac{1}{4}b^{2}-4ab$.
\end{enumeratea}
\end{esercizio}

\begin{esercizio}
\label{ese:16.3}
Quando è possibile, scomponi in fattori, riconoscendo il quadrato di un binomio.
\begin{enumeratea}
 \item $-9x^{2}-\dfrac{1}{4}+3x$;
 \item $4x^{2}+4xy+y^{2}$;
 \item $a^{4}+36a^{2}+12a^{3}$;
 \item $144x^{2}-6xa^{2}+\dfrac{1}{16}a^{4}$;
 \item $x^{2}-6xy+9y^{2}$;
 \item $-x^{2}-6xy-9y^{2}$.
\end{enumeratea}
\end{esercizio}

\begin{esercizio}
\label{ese:16.4}
Quando è possibile, scomponi in fattori, riconoscendo il quadrato di un binomio.
\begin{enumeratea}
 \item $25+10x+x^{2}$;
 \item $\dfrac{1}{4}x^{2}+\dfrac{1}{3}xy+\dfrac{1}{9}y^{2}$;
 \item $25-10x+x^{2}$;
 \item $\dfrac{9}{25}a^{4}-6a^{2}+25$;
 \item $4x^{2}+2x^{4}+1$;
 \item $4x^{2}-4x^{4}-1$.
\end{enumeratea}
\end{esercizio}

\begin{esercizio}
\label{ese:16.5}
Quando è possibile, scomponi in fattori, riconoscendo il quadrato di un binomio.
\begin{enumeratea}
 \item $-a^{3}-2a^{2}-a$;
 \item $3a^{7}b-6a^{5}b^{2}+3a^{3}b^{3}$;
 \item $100+a^{2}b^{4}+20ab^{2}$;
 \item $2x^{13}-8x^{8}y+8x^{3}y^{2}$;
 \item $x^{8}+8x^{4}y^{2}+16y^{4}$;
 \item $-x^{2}+6{xy}+9y^{2}$.
\end{enumeratea}
\end{esercizio}

\begin{esercizio}
\label{ese:16.6}
Quando è possibile, scomponi in fattori, riconoscendo il quadrato di un binomio.
\begin{enumeratea}
 \item $4a^{2}b^{4}-12ab^{3}+9b^{6}$;
 \item $a^{2}+a+1$;
 \item $36a^{6}b^{3}+27a^{5}b^{4}+12a^{7}b^{2}$;
 \item $25x^{14}+9y^{6}+30x^{7}y^{3}$;
 \item $-a^{7}-25a^{5}+10a^{6}$;
 \item $25a^{2}+49b^{2}+35ab$.
\end{enumeratea}
\end{esercizio}

\begin{esercizio}
\label{ese:16.7}
Quando è possibile, scomponi in fattori, riconoscendo il quadrato di un binomio.
\begin{enumeratea}
 \item $4y^{6}+4-4y^{2}$;
 \item $\dfrac{1}{4}a^{2}+2ab+b^{2}$;
 \item $25a^{2}-10{ax}-x^{2}$;
 \item $9x^{2}+4y^{2}-6{xy}$.
\end{enumeratea}
\end{esercizio}
\end{multicols}

\begin{esercizio}
\label{ese:16.8}
Individua perché i seguenti polinomi non sono quadrati di un binomio.
\begin{enumeratea}
 \item $4x^{2}+4xy-y^{2}$\, non è un quadrato di binomio perché\,\dotfill;
 \item $x^{2}-6xy+9y$\, non è un quadrato di binomio perché\dotfill;
 \item $25+100x+x^{2}$\, non è un quadrato di binomio perché\dotfill;
 \item $\dfrac{1}{4}x^{2}+\dfrac{2}{3}xy+\dfrac{1}{9}$\, non è un quadrato di binomio perché\dotfill;
 \item $25t^{2}+4-10t$\, non è un quadrato di binomio perché\dotfill%ex103
\end{enumeratea}
\end{esercizio}

\begin{multicols}{2}
\begin{esercizio}[\Ast]
\label{ese:16.9}
Quando è possibile, scomponi in fattori, riconoscendo il quadrato di un binomio.
\begin{enumeratea}
 \item $24a^{3}+6a+24a^{2}$;
 \item $3a^{2}x-12axb+12b^{2}x$;
 \item $5a^{2}+2ax+\dfrac{1}{5}x^{2}$;
 \item $x^{6}y+x^{2}y+2x^{4}y$.
\end{enumeratea}
\end{esercizio}

\begin{esercizio}[\Ast]
\label{ese:16.10}
Quando è possibile, scomponi in fattori, riconoscendo il quadrato di un binomio.
\begin{enumeratea}
 \item $x^{5}+4x^{4}+4x^{3}$;
 \item $2y^{3}-12y^{2}x+18x^{2}y$;
 \item $-50t^{3}-8t+40t^{2}$;
 \item $2^{10}x^{2}+2^{6}\cdot 3^{20}+3^{40}$.
\end{enumeratea}
\end{esercizio}

\begin{esercizio}[\Ast]
\label{ese:16.11}
Quando è possibile, scomponi in fattori, riconoscendo il quadrato di un binomio.
\begin{enumeratea}
 \item $2^{20}x^{40}-2^{26}\cdot x^{50}+2^{30}\cdot x^{60}$;
 \item $10^{100}x^{50}-2\cdot 10^{75}x^{25}+10^{50}$.
\end{enumeratea}
\end{esercizio}

\begin{esercizio}[\Ast]
\label{ese:16.12}
Quando è possibile, scomponi in fattori, riconoscendo il quadrato di un binomio.
\begin{enumeratea}
 \item $10^{11}x^{10}-2\cdot 10^{9}x^{5}+10^{6}$;
 \item $x^{2n}+2x^{n}+1$.
\end{enumeratea}
\end{esercizio}
\end{multicols}

\subsubsection*{16.2 - Quadrato di un polinomio}
\begin{multicols}{2}
\begin{esercizio}
\label{ese:16.13}
Quando è possibile, scomponi in fattori, riconoscendo il quadrato di un polinomio.
\begin{enumeratea}
 \item $a^{2}+b^{2}+c^{2}+2ab+2ac+2bc$;
 \item $x^{2}+y^{2}+z^{2}+2xy-2xz-2yz$;
 \item $x^{2}+y^{2}+4+4x+2xy+4y$;
 \item $4a^{4}-6{ab}-4a^{2}b+12a^{3}+b^{2}+9a^{2}$.
\end{enumeratea}
\end{esercizio}

\begin{esercizio}
Quando è possibile, scomponi in fattori, riconoscendo il quadrato di un polinomio.
\label{ese:16.14}
\begin{enumeratea}
 \item $9x^{6}+2y^{2}z+y^{4}-6x^{3}z-6x^{3}y^{2}+z^{2}$;
 \item $\dfrac{1}{4}a^{2}+b^{4}+c^{6}+ab^{2}+{ac}^{3}+2b^{2}c^{3}$;
 \item $a^{2}+2ab+b^{2}-2a+1-2b$;
 \item $x^{2}+\dfrac{1}{4}y^{2}+4-xy+4x-2y$.
\end{enumeratea}
\end{esercizio}

\begin{esercizio}
\label{ese:16.15}
Quando è possibile, scomponi in fattori, riconoscendo il quadrato di un polinomio.
\begin{enumeratea}
 \item $a^{2}+b^{2}+c^{2}-2ac-2bc+2ab$;
 \item $-x^{2}-2xy-9-y^{2}+6x+6y$;
 \item $4a^{2}+4ab-8a+b^{2}-4b+4$;
 \item $a^{2}b^{2}+2a^{2}b+a^{2}-2ab^{2}-2ab+b^{2}$.
\end{enumeratea}
\end{esercizio}
\end{multicols}
\begin{esercizio}
Individua perché i seguenti polinomi non sono quadrati.
\label{ese:16.16}
\begin{enumeratea}
 \item $a^{2}+b^{2}+c^{2}$\, non è un quadrato perché\dotfill;
 \item $x^{2}+y^{2}+4+4x+4xy+4y$\, non è un quadrato perché\dotfill;
 \item $a^{2}+b^{2}+c^{2}-2ac-2bc-2ab$\, non è un quadrato perché\dotfill;
 \item $a^{2}+b^{2}-1-2a-2b+2ab$\, non è un quadrato perché\dotfill
\end{enumeratea}
\end{esercizio}

\begin{esercizio}[\Ast]
\label{ese:16.17}
Quando è possibile, scomponi in fattori, riconoscendo il quadrato di un polinomio.
\begin{enumeratea}
 \item $a^{2}+4ab-2a+4b^{2}-4b+1$;
 \item $a^{2}b^{2}+2a^{2}b+a^{2}+4ab^{2}+4ab+4b^{2}$;
 \item $x^{2}-6xy+6x+9y^{2}-18y+9$.
\end{enumeratea}
\end{esercizio}

\begin{esercizio}
\label{ese:16.18}
Quando è possibile, scomponi in fattori, riconoscendo il quadrato di un polinomio.
\begin{enumeratea}
 \item $x^{4}+2x^{3}+3x^{2}+2x+1$\quad  scomponi prima \quad~$3x^{2}=x^{2}+2x^{2}$;
 \item $4a^{4}+8a^{2}+1+8a^{3}+4a$\quad  scomponi prima \quad~$8a^{2}=4a^{2}+4a^{2}$;
 \item $9x^{4}+6x^{3}-11x^{2}-4x+4$ \quad scomponi in maniera opportuna \quad~$-11x^{2}$.
\end{enumeratea}
\end{esercizio}

\begin{esercizio}
\label{ese:16.19}
Quando è possibile, scomponi in fattori, riconoscendo il quadrato di un polinomio.
\begin{enumeratea}
 \item $25x^{2}-20ax-30bx+4a^{2}+12ab+9b^{2}$;
 \item $2a^{10}x+4a^{8}x+2a^{6}x+4a^{5}x+4a^{3}x+2x$;
 \item $a^{2}+b^{2}+c^{2}+d^{2}-2ab+2ac-2ad-2bc+2bd-2cd$;
 \item $x^{6}+x^{4}+x^{2}+1+2x^{5}+2x^{4}+2x^{3}+2x^{3}+2x^{2}+2x$.
\end{enumeratea}
\end{esercizio}

\subsubsection*{16.3 - Cubo di un binomio}
\begin{multicols}{2}
\begin{esercizio}
\label{ese:16.20}
Quando è possibile, scomponi in fattori, riconoscendo il cubo di un binomio.
\begin{enumeratea}
 \item $8a^{3}+b^{3}+12a^{2}b+6ab^{2}$;
 \item $b^{3}+12a^{2}b-6ab^{2}-8a^{3}$;
 \item $-12a^{2}+8a^{3}-b^{3}+6ab$;
 \item $-12a^{2}b+6ab+8a^{3}-b^{3}$.
\end{enumeratea}
\end{esercizio}

\begin{esercizio}
\label{ese:16.21}
Quando è possibile, scomponi in fattori, riconoscendo il cubo di un binomio.
\begin{enumeratea}
 \item $-x^{3}+6x^{2}-12x+8$;
 \item $-x^{9}-3x^{6}+3x^{3}+8$;
 \item $x^{3}y^{6}+1+3x^{2}y^{2}+3xy^{2}$;
 \item $x^{3}+3x-3x^{2}-1$.
\end{enumeratea}
\end{esercizio}

\begin{esercizio}
\label{ese:16.22}
Quando è possibile, scomponi in fattori, riconoscendo il cubo di un binomio.
\begin{enumeratea}
 \item $-5x^{5}y^{3}-5x^{2}-15x^{4}y^{2}-15x^{3}y$;
 \item $-a^{6}+27a^{3}+9a^{5}-27a^{4}$;
 \item $64a^{3}-48a^{2}+12a-1$;
 \item $a^{6}+9a^{4}+27a^{2}+27$.
\end{enumeratea}
\end{esercizio}

\begin{esercizio}
\label{ese:16.23}
Quando è possibile, scomponi in fattori, riconoscendo il cubo di un binomio.
\begin{enumeratea}
 \item $x^{3}-x^{2}+\dfrac{1}{3}x-\dfrac{1}{27}$;
 \item $0,001x^{6}+0,015x^{4}+0,075x^{2}+0,125$;
 \item $\dfrac{27}{8}a^{3}-\dfrac{27}{2}a^{2}x+18ax^{2}-8x^{3}$;
 \item $x^{3}-x^{2}+\dfrac{1}{3}x-\dfrac{1}{27}$.
\end{enumeratea}
\end{esercizio}

\begin{esercizio}
\label{ese:16.24}
Individua perché i seguenti polinomi non sono cubi.
\begin{enumeratea}
 \item $a^{10}-8a-6a^{7}+12a^{4}$\, non è un cubo perché\dotfill;
 \item $27a^{3}-b^{3}+9a^{2}b-9ab^{2}$\, non è un cubo perché\dotfill;
 \item $8x^{3}+b^{3}+6x^{2}b+6{xb}^{2}$\, non è un cubo perché\dotfill;
 \item $x^{3}+6ax^{2}-6a^{2}x+8a^{3}$\, non è un cubo perché\dotfill
\end{enumeratea}
\end{esercizio}

\begin{esercizio}
\label{ese:16.25}
Quando è possibile, scomponi in fattori, riconoscendo il cubo di un binomio.
\begin{enumeratea}
 \item $x^{3}-6x^{2}+12x-8$;
 \item $a^{3}b^{3}+12ab+48ab+64$;
 \item $216x^{3}-540ax^{2}+450a^{2}x-125a^{3}$;
 \item $8x^{3}+12x^{2}+6x+2$.
\end{enumeratea}
\end{esercizio}

\begin{esercizio}[\Ast]
\label{ese:16.26}
Quando è possibile, scomponi in fattori, riconoscendo il cubo di un binomio.
\begin{enumeratea}
 \item $a^{6}+3a^{4}b^{2}+3a^{2}b^{4}+b^{6}$;
 \item $8a^{3}-36a^{2}b+54ab^{2}-27b^{3}$;
 \item $a^{6}+3a^{5}+3a^{4}+a^{3}$;
 \item $a^{10}-8a-6a^{7}+12a^{4}$.%ex154b trovato risultato: a\left(a^3-2\right)^3
\end{enumeratea}
\end{esercizio}

\begin{esercizio}
\label{ese:16.27}
Quando è possibile, scomponi in fattori, riconoscendo il cubo di un binomio.
\begin{enumeratea}
 \item $8x^{3}-36x^{2}+54x-27$;
 \item $x^{6}+12ax^{4}+12a^{2}x^{2}+8a^{3}$;
 \item $x^{300}-10^{15}-3\cdot 10^{5}x^{200}+3\cdot 10^{10}x^{100}$;
 \item $a^{6n}+3a^{4n}x^{n}+3a^{2n}x^{2n}+x^{3n}$.
\end{enumeratea}
\end{esercizio}
\end{multicols}
\begin{esercizio}
\label{ese:16.28}
Quando è possibile, scomponi in fattori, riconoscendo il cubo di un binomio.
\begin{enumeratea}
 \item $10^{15}a^{60}+3\cdot 10^{30}a^{45}+3\cdot 10^{45}a^{30}+10^{60}a^{15}$;
 \item $10^{-33}x^{3}-3\cdot 10^{-22}x^{2}+3\cdot 10^{-11}x-1$.
\end{enumeratea}
\end{esercizio}

\subsubsection*{16.4 - Differenza di due quadrati}
\begin{multicols}{2}
\begin{esercizio}
\label{ese:16.29}
Scomponi i seguenti polinomi come differenza di quadrati.
\begin{enumeratea}
 \item $a^{2}-25b^{2}$;
 \item $16-x^{2}y^{2}$;
 \item $25-9x^{2}$;
 \item $4a^{4}-9b^{2}$;
 \item $x^{2}-16y^{2}$;
 \item $144x^{2}-9y^{2}$.
\end{enumeratea}
\end{esercizio}

\begin{esercizio}
\label{ese:16.30}
Scomponi i seguenti polinomi come differenza di quadrati.
\begin{enumeratea}
 \item $16x^{4}-81z^{2}$;
 \item $a^{2}b^{4}-c^{2}$;
 \item $4x^{6}-9y^{4}$;
 \item $-36x^{8}+25b^{2}$;
 \item $-1+a^{2}$;
 \item $\dfrac{1}{4}x^{4}-\dfrac{1}{9}y^{4}$.
\end{enumeratea}
\end{esercizio}

\begin{esercizio}
\label{ese:16.31}
Scomponi i seguenti polinomi come differenza di quadrati.
\begin{enumeratea}
 \item $\dfrac{a^{2}}{4}-\dfrac{y^{2}}{9}$;
 \item $2a^{2}-50$;
 \item $a^{3}-16{ab}^{6}$;
 \item $-4x^{2}y^{2}+y^{2}$;
 \item $-4a^{2}+b^{2}$;
 \item $25x^{2}y^{2}-\dfrac{1}{4}z^{6}$.
\end{enumeratea}
\end{esercizio}

\begin{esercizio}
\label{ese:16.32}
Scomponi i seguenti polinomi come differenza di quadrati.
\begin{enumeratea}
 \item $-a^{2}b^{4}+49$;
 \item $16y^{4}-z^{4}$;
 \item $a^{8}-b^{8}$;
 \item $a^{4}-16$;
 \item $16a^{2}-9b^{2}$;
 \item $9-4x^{2}$.
\end{enumeratea}
\end{esercizio}

\begin{esercizio}
\label{ese:16.33}
Scomponi i seguenti polinomi come differenza di quadrati.
\begin{enumeratea}
 \item $\dfrac{1}{4}x^{2}-1$;
 \item $a^{2}-9b^{2}$;
 \item $\dfrac{25}{16}a^{2}-1$;
 \item $-16+25x^{2}$;
 \item $25a^{2}b^{2}-\dfrac{9}{16}y^{6}$;
 \item $-4x^{8}+y^{12}$.
\end{enumeratea}
\end{esercizio}

\begin{esercizio}
\label{ese:16.34}
Scomponi i seguenti polinomi come differenza di quadrati.
\begin{enumeratea}
 \item $\dfrac{1}{4}x^{2}-0,01y^{4}$;
 \item $x^{6}-y^{8}$;
 \item $x^{4}-y^{8}$.
\end{enumeratea}
\end{esercizio}

\begin{esercizio}[\Ast]
\label{ese:16.35}
Quando è possibile, scomponi in fattori, riconoscendo la differenza di due quadrati.
\begin{enumeratea}
 \item $(b+3)^{2}-x^{2}$;
 \item $a^{8}-(b-1)^{2}$;
 \item $(x-1)^{2}-a^{2}$.
\end{enumeratea}
\end{esercizio}

\begin{esercizio}
\label{ese:16.36}
Quando è possibile, scomponi in fattori, riconoscendo la differenza di due quadrati.
\begin{enumeratea}
 \item $(x-y)^{2}-(y+z)^{2}$;
 \item $-(2a-1)^{2}+(3b+3)^{2}$;
 \item $x^{2}-b^{2}-9-6b$.
\end{enumeratea}
\end{esercizio}

\begin{esercizio}[\Ast]
\label{ese:16.37}
Quando è possibile, scomponi in fattori, riconoscendo la differenza di due quadrati.
\begin{enumeratea}
 \item $(2x-3)^{2}-9y^{2}$;
 \item $(x+1)^{2}-(y-1)^{2}$;
 \item $x^{2}+2x+1-y^{2}$.
\end{enumeratea}
\end{esercizio}

\begin{esercizio}
\label{ese:16.38}
Quando è possibile, scomponi in fattori, riconoscendo la differenza di due quadrati.
\begin{enumeratea}
 \item $b^{2}-x^{4}+1+2b$;
 \item $a^{4}+4a^{2}+4-y^{2}$;
 \item $x^{2}-y^{2}-1+2y$.
\end{enumeratea}
\end{esercizio}

\begin{esercizio}[\Ast]
\label{ese:16.39}
Quando è possibile, scomponi in fattori, riconoscendo la differenza di due quadrati.
\begin{enumeratea}
 \item $(2x+3)^{2}-(2y+1)^{2}$;
 \item $a^{2}-2{ab}+b^{2}-4$;
 \item $(2x-3a)^{2}-(x-a)^{2}$.
\end{enumeratea}
\end{esercizio}

\begin{esercizio}
\label{ese:16.40}
Quando è possibile, scomponi in fattori, riconoscendo la differenza di due quadrati.
\begin{enumeratea}
 \item $-(a+1)^{2}+9$;
 \item $16x^{2}y^{6}-(xy^{3}+1)^{2}$;
 \item $a^{2}+1+2a-9$;
 \item $x^{2}y^{4}-z^{2}+9+6xy^{2}$.
\end{enumeratea}
\end{esercizio}

\begin{esercizio}[\Ast]
\label{ese:16.41}
Quando è possibile, scomponi in fattori, riconoscendo la differenza di due quadrati.
\begin{enumeratea}
 \item $a^{2}-6a+9-x^{2}-16-8x$;
 \item $x^{2}+25+10x-y^{2}+10y-25$.
\end{enumeratea}
\end{esercizio}

\begin{esercizio}
\label{ese:16.42}
Quando è possibile, scomponi in fattori, riconoscendo la differenza di due quadrati.
\begin{enumeratea}
 \item $(a-1)^{2}-(a+1)^{2}$;
 \item $a^{2n}-4$;
 \item $a^{2m}-b^{2n}$;
 \item $x^{2}n-y^{4}$.
\end{enumeratea}
\end{esercizio}
\end{multicols}
\subsection{Risposte}

\paragraph{16.9}
a)~$6a(2a+1)^{2}$,\quad b)~$3x(a-2b)^{2}$, \quad c)~$\dfrac{1}{5}(x+5a)^{2}$, \quad d)~$x^{2}y\left(x^{2}+1\right)^{2}$.

\paragraph{16.10}
a)~$x^{3}(x+2)^{2}$,\quad b)~$2y(3x-y)^{2}$, \quad c)~$-2t(5t-2)^{2}$, \quad d)~$\left(2^{5}x+3^{20}\right)^{2}$.

\paragraph{16.11}
a)~$2^{20} x^{40}\left(1-2^{5}x^{10} \right)^2$, \quad b)~$10^{50}\left(10^{25} x^{25}-1 \right)^2$.

\paragraph{16.12}
a)~$10^{6} \left(10^{5} x^{10}-2 \cdot 10^{3}x^{5}+1\right)$,\quad b)~$\left(x^{n}+1\right)^2$.

\paragraph{16.17}
a)~$(a+2b-1)^{2}$,\quad b)~$(ab+a+2b)^{2}$, \quad c)~$(x-3y+3)^{2}$.

\paragraph{16.26}
a)~$\left(a^{2}+b^{2}\right)^{3}$,\quad b)~$(2a-3b)^{3}$,\quad c)~$a^{3}(a+1)^{3}$,\quad d)~$a\left(a^3-2\right)^3$.

\paragraph{16.35}
a)~$(b+3-x)(b+3+x)$,\quad b)~$(a^{4}-b+1)(a^{4}+b-1)$, \quad c)~$(x+a-1)(x-a-1)$.

\paragraph{16.37}
a)~$(2x+3y-3)(2x-3y-3)$,\quad b)~$(x+y)(x-y+2)$, \quad c)~$(x+y+1)(x-y+1)$.

\paragraph{16.39}
a)~$4(x+y+2)(x-y+1)$,\quad b)~$(a-b-2)(a-b+2)$, \quad c)~$(3x-4a)(x-2a)$.

\paragraph{16.41}
a)~$-(x+a+1)(x-a+7)$,\quad b)~$(x+y)(x-y+10)$.
