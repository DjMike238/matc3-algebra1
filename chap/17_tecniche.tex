% (c) 2012-2014 Dimitrios Vrettos - d.vrettos@gmail.com

\chapter{Altre tecniche di scomposizione}

\section{Trinomi particolari}

Consideriamo il seguente prodotto:

\[(x+3)(x+2)=x^{2}+3x+2x+6=x^{2}+5x+6.\]

Poniamoci ora l'obiettivo opposto: se abbiamo il
polinomio~$x^{2}+5x+6$ come facciamo a trovare ritrovare il prodotto
che lo ha originato? Possiamo notare che il~5 deriva dalla somma tra
il~3 e il~2, mentre il~6 deriva dal prodotto tra~3 e~2. Generalizzando:

\[\left(x+a\right)\cdot \left(x+b\right)=x^{{2}}+ax+bx+ab=x^{2}+\left(a+b\right)x+a\cdot b.\]
Leggendo la formula precedente da destra verso sinistra:

\[x^{{2}}+\left(a+b\right)x+a\cdot b=\left(x+a\right)\cdot\left(x+b\right).\]

Possiamo allora concludere che se abbiamo un trinomio di secondo grado
in una sola lettera, a coefficienti interi, avente il termine di
secondo grado con coefficiente~1, se riusciamo a trovare due numeri~$a$ e~$b$
tali che la loro somma è uguale al
coefficiente del termine di primo grado ed il loro prodotto è uguale
al termine noto, allora il polinomio è scomponibile nel prodotto~$(x+a)(x+b)$.

Osserva che il termine noto, poiché è dato dal prodotto dei numeri
che cerchiamo, ci dice se i due numeri sono concordi o discordi.
Inoltre, se il numero non è particolarmente grande è sempre
possibile scrivere facilmente tutte le coppie di numeri che danno come
prodotto il numero cercato, tra tutte queste coppie dobbiamo poi
individuare quella che ha per somma il coefficiente del termine di
primo grado.

\begin{exrig}
 \begin{esempio}
 $x^{2}+7x+12.$

 I coefficienti sono positivi e quindi i due numeri da trovare sono
entrambi positivi.
Il termine noto~12 può essere scritto sotto forma di prodotto di due
numeri naturali solo come:

\[12\cdot 1\text{,}\quad~6\cdot 2\text{,}\quad~3\cdot 4.\]

Le loro somme sono rispettivamente~13, 8, 7. La coppia di numeri che
dà per somma $+7$ e prodotto $+12$ è pertanto~$+3$ e~$+4$. Dunque il
trinomio si scompone come:

\[x^{2}+7x+12=\left(x+4\right)\cdot \left(x+3\right).\]
 \end{esempio}

 \begin{esempio}
 $x^{2}-\overset{S}{8}x+\overset{P}{15}.$

I segni dei coefficienti ci dicono che i due numeri, dovendo avere somma (S)
negativa e prodotto (P) positivo, sono entrambi negativi. Dobbiamo cercare
due numeri negativi la cui somma (S) sia~$-8$ e il cui prodotto (P) sia~15. Le
coppie di numeri negativi che danno~15 come prodotto (P) sono $(-15\text{,~}-1)$ e~$(-5\text{,~}-3)$.
Allora i due numeri cercati sono~$-5$ e~$-3$. Il trinomio si scompone
come:
\[x^{2}-8x+15=\left(x-5\right)\cdot \left(x-3\right).\]
 \end{esempio}

\begin{esempio}
 $x^{2}+\overset{S}{4}x-\overset{P}{5}.$

I due numeri sono discordi, il maggiore in valore assoluto è quello
positivo. C'è una sola coppia di numeri che dà~$-5$
come prodotto, precisamente~$+5$ e~$-1$. Il polinomio si scompone:
\[x^{2}+4x-5=\left(x+5\right)\cdot \left(x-1\right).\]
\end{esempio}

\begin{esempio}
 $x^{2}-\overset{S}{3}x-\overset{P}{10}.$

I due numeri sono discordi, in valore assoluto il più grande è quello
negativo. Le coppie di numeri che danno~$-10$ come prodotto sono~$(-10\text{,~}+1)$ e
$(-5\text{,~}+2)$. Quella che dà~$-3$ come somma è~$(-5\text{,~}+2)$. Quindi
\[x^{2}-3x-10=\left(x-5\right)\cdot \left(x+2\right).\]
\end{esempio}

\begin{esempio}
 In alcuni casi si può applicare questa regola anche quando il trinomio
non è di secondo grado, è necessario però che il termine di grado
intermedio sia esattamente di grado pari alla metà di quello di grado
maggiore.

\begin{itemize*}
\item $x^{4}+5x^{2}+6=\left(x^{2}+3\right)\cdot \left(x^{2}+2\right)$;
\item $x^{6}+x^{3}-12=\left(x^{3}+4\right)\cdot \left(x^{3}-3\right)$;
\item $a^{4}-10a^{2}+9=\underbrace{\left(a^{2}-9\right)\cdot\left(a^{2}-1\right)}_{\text{differenze di quadrati}}=\left(a+3\right)\cdot \left(a-3\right)\cdot \left(a+1\right)\cdot\left(a-1\right)$;
\item $-x^{4}-x^{2}+20=-\left(x^{4}+x^{2}-20\right)=-\left(x^{2}+5\right)\cdot\left(x^{2}-4\right)=-\left(x^{2}+5\right)\cdot%
\left(x+2\right)\cdot \left(x-2\right)$;
\item $2x^{5}-12x^{3}-14x=2x\cdot \left(x^{4}-6x^{2}-7\right)=2x\cdot%
\left(x^{2}-7\right)\cdot \left(x^{2}+1\right)$;
\item $-2a^{7}+34a^{5}-32a^{3}=-2a^{3}\left(a^{4}-17a^{2}+16\right)
    =-2a^{3}\left(a^{2}-1\right)\left(a^{2}-16\right)\\
    =-2a^{3}\left(a-1\right)\left(a+1\right)\left(a-4\right)\left(a+4\right).$
\end{itemize*}
\end{esempio}

\end{exrig}

È possibile applicare questo metodo anche quando il
polinomio è in due variabili.

\begin{exrig}
 \begin{esempio}
 $x^{2}+5xy+6y^{2}$.

 Per capire come applicare la regola precedente, possiamo scrivere il
trinomio in questo modo:~$x^{2}+\overset{S}{5}xy+\overset{P}{6}y{^{2}}$.

Bisogna cercare due monomi~$A$ e~$B$ tali che~$A+B=5y$
e~$A\cdot B=6y^{2}$. Partendo dal fatto che i due numeri che danno~5
come somma e~6 come prodotto sono~$+3$ e~$+2$, i monomi cercati sono~$+3y$ e~$+2y$,
infatti~$+3y+3y=+5y$ e~$+3y\cdot (+2y)=+6y^{2}$. Pertanto si
può scomporre come segue:~$x^{2}+5xy+6y^{2}=(x+3y)(x+2y)$.
 \end{esempio}
\end{exrig}

La regola, opportunamente modificata, vale anche se il primo
coefficiente non è~1. Vediamo un esempio.
%\newpage
\begin{exrig}
 \begin{esempio}
 $2x^{2}-x-1$.

Non possiamo applicare la regola del trinomio caratteristico, con somma
e prodotto, ma con un accorgimento, possiamo riscrivere il polinomio in un
altro modo. Cerchiamo due numeri la cui somma sia~$-1$ e il prodotto sia
pari al prodotto tra il primo e l'ultimo coefficiente,
o meglio tra il coefficiente del termine di secondo grado e il termine
noto, in questo caso~$2\cdot (-1)=-2$. I numeri sono~$-2$ e~$+1$.
Spezziamo il monomio centrale in somma di due monomi in questo modo
\[2x^{2}-x-1=2x^{2}-2x+x-1.\]

Ora possiamo applicare il raccoglimento a fattore comune parziale
\[2x^{2}-x-1=\mmevid{ev_rosso}{2x^{2}}\underbrace{\mmevid{ev_rosso}{-2x}+x}_{-x}-1=2x\cdot
\mmevid{ev_verde}{(x-1)}+1\cdot
\mmevid{ev_verde}{(x-1)}=\left(x-1\right)\cdot \left(2x+1\right).\]
 \end{esempio}
\end{exrig}

\begin{procedura}
Sia da scomporre un trinomio di secondo grado a coefficienti interi~$ax^{2}+bx+c$
con~$a\neq 1$, cerchiamo due numeri~$m$ ed~$n$ tali che $m+n=b$ e~$m\cdot n=a\cdot c$;
se riusciamo a trovarli, li useremo per dissociare
il coefficiente~$b$ e riscrivere il polinomio nella forma~$ax^{2}+\left(m+n\right)\cdot x+c$
su cui poi eseguire un raccoglimento parziale.
\end{procedura}

\ovalbox{\risolvii \ref{ese:17.1}, \ref{ese:17.2}, \ref{ese:17.3}, \ref{ese:17.4}, \ref{ese:17.5}, \ref{ese:17.6}, \ref{ese:17.7}, \ref{ese:17.8}, \ref{ese:17.9}, \ref{ese:17.10}}

\section{Scomposizione con la regola Ruffini}
Anche il teorema di Ruffini permette di scomporre in fattori i polinomi.
Dato il polinomio~$P(x)$, se riusciamo a trovare un numero~$k$
per il quale~$P(k)=0$, allora~$P(x)$ è
divisibile per il binomio~$x-k$, allora possiamo scomporre~$P(x)=(x-k)\cdot Q(x)$, dove~$Q(x)$
è il quoziente della divisione
tra~$P(x)$ e~$(x-k)$.

Il problema di scomporre un polinomio~$P(x)$ si riconduce quindi a quello
della ricerca del numero~$k$ che sostituito alla~$x$ renda nullo il
polinomio. Un numero di questo tipo si dice anche \emph{radice del polinomio}.

Il numero~$k$ non va cercato del tutto a caso, abbiamo degli elementi per
restringere il campo di ricerca di questo numero quando il polinomio
è a coefficienti interi.

\osservazione Le radici intere del polinomio vanno cercate tra i divisori del termine
noto.

\begin{exrig}
 \begin{esempio}
 $P(x)=x^{3}+x^{2}-10x+8$.
 \end{esempio}
Le radici intere del polinomio sono da ricercare
nell'insieme dei divisori di~8, precisamente in~$\{\pm 1$, $\pm 2$, $\pm 4$, $\pm 8\}$.
Sostituiamo questi numeri nel polinomio,
finché non troviamo quello che lo annulla.

Per~$x=1$ si ha~$p(1)=(1)^{3}+(1)^{2}-10\cdot (1)+8=1+1-10+8=0$,
pertanto il polinomio è divisibile per~$x-1$.

Utilizziamo la regola di Ruffini per dividere~$P(x)$ per~$x-1$.

\begin{wrapfloat}{figure}{l}{0pt}
 % (c) 2012 Dimitrios Vrettos - d.vrettos@gmail.com

\begin{tikzpicture}[x=5mm,y=5mm]
\matrix (a)[matrix of nodes, nodes in empty cells,nodes={ text width=8mm, text depth=1mm, text centered}]{
&1&1&$-10$&8\\
1&&1&2&$-8$\\
&1&2&$-8$&\\
};  
\begin{scope}[blue]
\draw(a-1-2.north west)--(a-3-1.south east);
\draw(a-2-1.south west)--(a-2-5.south east);
\draw(a-1-4.north east)--(a-3-4.south east);
      \end{scope}
\end{tikzpicture}
\end{wrapfloat}

Predisponiamo una griglia come quella a fianco, nella prima riga mettiamo i
coefficienti di~$P(x)$, nella seconda riga mettiamo come primo numero la
radice che abbiamo trovato, cioè~1. Poi procediamo come abbiamo già
indicato per la regola di Ruffini (sezione \ref{sect:regola_di_Ruffini} a pagina \pageref{sect:regola_di_Ruffini}).

I numeri che abbiamo ottenuto nell'ultima riga sono i
coefficienti del polinomio quoziente:~$q(x)=x^{2}+2x-8$.

Possiamo allora scrivere:
\[x^{3}+x^{2}-10x+8=(x-1)\cdot (x^{2}+2x-8).\]

Per fattorizzare il polinomio di secondo grado~$x^{2}+2x-8$ possiamo
ricorrere al metodo del trinomio notevole. Cerchiamo due numeri la sui
somma sia~$+2$ e il cui prodotto sia~$-8$. Questi numeri vanno cercati tra
le coppie che danno per prodotto~$-8$ e precisamente tra le seguenti
coppie~$(+8;-1)$, $(-8;+1)$, $(+4;-2)$, $(-4;+2)$. La coppia che dà per
somma~$+2$ è~$(+4;-2)$. In definitiva si ha:
\[x^{3}+x^{2}-10x+8=(x-1)\cdot (x^{2}+2x-8)=(x-1)(x-2)(x+4).\]


 \begin{esempio}
$x^{4}-5x^{3}-7x^{2}+29x+30$.
\end{esempio}

Le radici intere vanno cercate tra i divisori di~30, precisamente in~$\{\pm 1$, $\pm 2$, $\pm 3$, $\pm 5$, $\pm 6$, $\pm 10$, $\pm 15$, $\pm 30\}$.
Sostituiamo questi numeri al posto della~$x$, finché non troviamo la
radice.

Per~$x=1$ si ha~$P(1)=1-5-7+29+30$ senza effettuare il calcolo si
nota che i numeri positivi superano quelli negativi, quindi~1 non è
una radice.

Per~$x=-1$ si ha
\begin{align*}
P(-1)&=(-1)^{4}-5\cdot (-1)^{3}-7\cdot(-1)^{2}+29\cdot (-1)+30\\
&=+1+5-7-29+30\\
&=0.
\end{align*}

Una radice del polinomio è quindi~$-1$; utilizzando la regola di Ruffini
abbiamo:
\begin{center}
 % (c) 2012 Dimitrios Vrettos - d.vrettos@gmail.com

\begin{tikzpicture}[x=5mm,y=5mm]
\matrix (a)[matrix of nodes, nodes in empty cells,nodes={ text width=8mm, text depth=1mm, text centered}]{
&1&$-5$&$-7$&29&30\\
$-1$&&$-1$&6&1&$-30$\\
&1&$-6$&$-1$&30&0\\
};  
\begin{scope}[blue]
\draw(a-1-2.north west)--(a-3-1.south east);
\draw(a-2-1.south west)--(a-2-6.south east);
\draw(a-1-5.north east)--(a-3-5.south east);
      \end{scope}
\end{tikzpicture}
\end{center}
Con i numeri che abbiamo ottenuto nell'ultima riga
costruiamo il polinomio quoziente:~$x^{3}-6x^{2}-1x+30$. Possiamo allora
scrivere:
\[x^{4}-5x^{3}-7x^{2}+29x+30=(x+1)(x^{3}-6x^{2}-x+30).\]

Con lo stesso metodo scomponiamo il polinomio~$x^{3}-6x^{2}-1x+30$.
Cerchiamone le radici tra i divisori di~30, precisamente
nell'insieme~$\{\pm 1$, $\pm 2$, $\pm 3$, $\pm 5$, $\pm 6$, $\pm 10$, $\pm 15$, $\pm 30\}$. Bisogna ripartire dall'ultima
radice trovata, cioè da~$-1$.

Per~$x=-1$ si ha~$P(-1)=(-1)^{3}-6\cdot (-1)^{2}-1\cdot (-1)+30=-1-6+1+30\neq~0$.

Per~$x=+2$ si ha~$P(+2)=(+2)^{3}-6\cdot (+2)^{2}-1\cdot (+2)+30=+8-24-2+30\neq~0$.

Per~$x=-2$ si ha~$P(-2)=(-2)^{3}-6\cdot (-2)^{2}-1\cdot (-2)+30=-8-24+2+30=0$.

Quindi~$-2$ è una radice del polinomio. Applichiamo la regola di
Ruffini, ricordando che nella prima riga dobbiamo mettere i coefficienti
del polinomio da scomporre, cioè~$x^{3}-6x^{2}-1x+30$.
\begin{center}
 % (c) 2012 Dimitrios Vrettos - d.vrettos@gmail.com

\begin{tikzpicture}[x=5mm,y=5mm]
\matrix (a)[matrix of nodes, nodes in empty cells,nodes={ text width=8mm, text depth=1mm, text centered}]{
&1&$-6$&$-1$&30\\
$-2$&&$-2$&16&$-30$\\
&1&$-8$&$15$&0\\
};  
\begin{scope}[blue]
\draw(a-1-2.north west)--(a-3-1.south east);
\draw(a-2-1.south west)--(a-2-5.south east);
\draw(a-1-4.north east)--(a-3-4.south east);
      \end{scope}
\end{tikzpicture}
\end{center}
Il polinomio~$q(x)$ si scompone nel
prodotto~$x^{3}-6x^{2}-x+30=(x+2)\cdot (x^{2}-8x+15)$.

Infine possiamo scomporre~$x^{2}-8x+15$ come trinomio notevole: i due
numeri che hanno per somma~$-8$ e prodotto~$+15$ sono~$-3$ e~$-5$. In
conclusione possiamo scrivere la scomposizione:
\[x^{4}-5x^{3}-7x^{2}+29x+30=(x+1)\cdot (x+2)\cdot (x-3)\cdot (x-5).\]

Non sempre è possibile scomporre un polinomio utilizzando solo numeri
interi. In alcuni casi possiamo provare con le frazioni, in particolare
quando il coefficiente del termine di grado maggiore non è~1. In
questi casi possiamo cercare la radice del polinomio tra le frazioni
del tipo~$\frac{p}{q}$, dove~$p$ è un divisore del termine noto e~$q$ è
un divisore del coefficiente del termine di grado maggiore.


\begin{esempio}
$6x^{2}-x-2.$
\end{esempio}

Determiniamo prima di tutto l'insieme nel quale
possiamo cercare le radici del polinomio. Costruiamo tutte le frazioni
del tipo~$\frac{p}{q}$, con~$p$ divisore di~$-2$ e~$q$ divisore di~$6$. I
divisori di~2 sono~$\{\pm 1$, $\pm 2\}$ mentre i divisori di~6 sono~$\{\pm 1$, $\pm 2$, $\pm 3$, $\pm 6\}$.
Le frazioni tra cui cercare sono
\[\left\{\pm {\frac{1}{1}}\text{, }\pm \frac{1}{2}\text{, }\pm \frac{2}{1}\text{, }\pm
\frac{2}{3}\text{, }\pm \frac{2}{6}\right\}\]
cioè \[\left\{\pm 1\text{, }\pm\frac{1}{2}\text{, }\pm 2\text{, }\pm \frac{2}{3}\text{, }\pm \frac{1}{3}\right\}.\]

Si ha~$\quad A(1)=-3;\quad A(-1)=5;\quad A\left(\frac{1}{2}\right)=-1;\quad A\left(-{\frac{1}{2}}\right)=0$.

\begin{wrapfloat}{figure}{l}{0pt}
 % (c) 2012 Dimitrios Vrettos - d.vrettos@gmail.com

\begin{tikzpicture}[x=5mm,y=5mm]
\matrix (a)[matrix of nodes, nodes in empty cells,nodes={ text width=8mm, text depth=1mm, text centered}]{
&6&$-1$&$-2$\\
$-\frac{1}{2}$&&$-3$&2\\
&6&$-4$&0\\
};  
\begin{scope}[blue]
\draw(a-1-2.north west)--(a-3-1.south east);
\draw(a-2-1.south west)--(a-2-4.south east);
\draw(a-1-3.north east)--(a-3-3.south east);
      \end{scope}
\end{tikzpicture}
\end{wrapfloat}
Sappiamo dal teorema di Ruffini che il polinomio~$A(x)=6x^{2}-x-2$ è
divisibile per~$\left(x+\frac{1}{2}\right)$ dobbiamo quindi trovare il
polinomio~$Q(x)$ per scomporre~$6x^{2}-x-2$ come~$Q(x)\cdot \left(x+\frac{1}{2}\right)$.

Applichiamo la regola di Ruffini per trovare il quoziente. Il quoziente è~$Q(x)=6x-4$.
Il polinomio sarà scomposto in~$(6x-4)\cdot\left(x+\frac{1}{2}\right)$.
Mettendo a fattore comune~2 nel primo binomio si ha:
\[6x^{2}-x-2\ =\ (6x-4)%
\left(x+\frac{1}{2}\right)\ =\ 2(3x-2)\left(x+\frac{1}{2}\right)\ =(3x-2)(2x+1).\]


\end{exrig}

\ovalbox{\risolvii \ref{ese:17.11}, \ref{ese:17.12}, \ref{ese:17.13}, \ref{ese:17.14}, \ref{ese:17.15}}

\section{Somma e differenza di due cubi}

Per scomporre i polinomi del tipo~$A^{3}+B^{3}$ e~$A^{3}-B^{3}$
possiamo utilizzare il metodo di Ruffini.

\begin{exrig}
 \begin{esempio}
 $x^{3}-8$.
\end{esempio}
Il polinomio si annulla per~$x=2$, che è la radice cubica di~8.
Calcoliamo il quoziente.
\begin{wrapfloat}{figure}{l}{0pt}
 % (c) 2012 Dimitrios Vrettos - d.vrettos@gmail.com

\begin{tikzpicture}[x=5mm,y=5mm]
\matrix (a)[matrix of nodes, nodes in empty cells,nodes={ text width=8mm, text depth=1mm, text centered}]{
&1&0&0&$-8$\\
2&&2&4&8\\
&1&2&4&/\\
};  
\begin{scope}[blue]
\draw(a-1-2.north west)--(a-3-1.south east);
\draw(a-2-1.south west)--(a-2-5.south east);
\draw(a-1-4.north east)--(a-3-4.south east);
      \end{scope}
\end{tikzpicture}
\end{wrapfloat}
Il polinomio quoziente è~$Q(x)=x^{2}+2x+4$ e la scomposizione risulta
\[x^{3}-8\ =\ (x-2)(x^{2}+2x+4).\]

Notiamo che il quoziente somiglia al quadrato di un binomio, ma non lo
è in quanto il termine intermedio è il prodotto e non il doppio
prodotto dei due termini, si usa anche dire che è un ``falso quadrato''.
Un trinomio di questo tipo non è ulteriormente scomponibile.


 \begin{esempio}
 $x^{3}+27$.
 \end{esempio}
 \begin{wrapfloat}{figure}{l}{0pt}
 % (c) 2012 Dimitrios Vrettos - d.vrettos@gmail.com

\begin{tikzpicture}[x=5mm,y=5mm]
\matrix (a)[matrix of nodes, nodes in empty cells,nodes={ text width=8mm, text depth=1mm, text centered}]{
&1&0&0&27\\
$-3$&&$-3$&9&$-27$\\
&1&$-3$&9&/\\
};  
\begin{scope}[blue]
\draw(a-1-2.north west)--(a-3-1.south east);
\draw(a-2-1.south west)--(a-2-5.south east);
\draw(a-1-4.north east)--(a-3-4.south east);
      \end{scope}
\end{tikzpicture}
\end{wrapfloat}
Il polinomio si annulla per~$x=-3$, cioè~$P(-3)=(-3)^{3}+27=-27+27=0$.
Il polinomio quindi è divisibile per~$x+3$. Calcoliamo il quoziente
attraverso la regola di Ruffini.

Il polinomio quoziente è~$Q(x)=x^{2}-3x+9$ e la scomposizione risulta
\[x^{3}+27=(x+3)(x^{2}-3x+9).\]

\end{exrig}

In generale possiamo applicare le seguenti regole per la scomposizione
di somma e differenza di due cubi:

\[A^{3}+B^{3}=(A+B)(A^{2}-AB+B^{2})\text{,}\]
\[A^{3}-B^{3}=(A-B)(A^{2}+AB+B^{2}).\]

\ovalbox{\risolvii \ref{ese:17.16}, \ref{ese:17.17}, \ref{ese:17.18}}

\section{Scomposizione mediante metodi combinati}

Nei paragrafi precedenti abbiamo analizzato alcuni metodi per ottenere
la scomposizione in fattori di un polinomio e talvolta abbiamo mostrato
che la scomposizione si ottiene combinando metodi diversi.
Sostanzialmente non esiste una regola generale per la scomposizione di
polinomi, cioè non esistono criteri di divisibilità semplici come
quelli per scomporre un numero nei suoi fattori primi. In questo
paragrafo vediamo alcuni casi in cui si applicano vari metodi combinati
tra di loro.

Un buon metodo per ottenere la scomposizione è procedere tenendo conto
di questi suggerimenti:


\begin{enumerate}
\item analizzare se si può effettuare \emph{un raccoglimento totale};
\item \emph{contare il numero di termini} di cui si compone il polinomio:
 \begin{enumerate}
  \item \emph{due} termini. Analizzare se il binomio è
   \begin{enumerate}
	\item una \emph{differenza di quadrati} $A^{2}-B^{2}=(A-B)(A+B)$;
	\item una \emph{differenza di cubi} $A^{3}-B^{3}=(A-B)\left(A^{2}+AB+B^{2}\right)$;
	\item una \emph{somma di cubi} $A^{3}+B^{3}=(A+B)\left(A^{2}-AB+B^{2}\right)$;
	\item una \emph{somma di quadrati} $A^{2}+B^{2}$, nel qual caso è \emph{irriducibile}.
   \end{enumerate}
  \item \emph{tre} termini. Analizzare se è
   \begin{enumerate}
	\item un \emph{quadrato di un binomio} $A^{2}\pm~2AB+B^{2}=\left(A\pm B\right)^{2}$;
	\item un \emph{trinomio particolare} del tipo~$x^{2}+Sx+P=(x+a)(x+b)$ con~$a+b=S$ e~$a\cdot b=P$;
	\item un \emph{falso quadrato}~$A^{2}\pm AB+B^{2}$, che è irriducibile.
   \end{enumerate}
  \item \emph{quattro} termini. Analizzare se è
   \begin{enumerate}
	\item un \emph{cubo di un binomio} $A^{3}\pm~3A^{2}B+3AB^{2}\pm B^{3}=\left(A\pm B\right)^{3}$;
	\item una \emph{particolare differenza di quadrati}
	 \subitem $A^{2}\pm~2AB+B^{2}-C^{2}=(A\pm B+C)(A\pm B-C)$;
	\item un \emph{raccoglimento parziale}, tipo $ax+bx+ay+by=(a+b)(x+y)$.
   \end{enumerate}
  \item \emph{sei} termini. Analizzare se è
   \begin{enumerate}
	\item un \emph{quadrato di un trinomio} $A^{2}+B^{2}+C^{2}+2AB+2{AC}+2{BC}=\left(A+B+C\right)^{2}$;
	\item un \emph{raccoglimento parziale}, tipo
	 \subitem $ax+{bx}+{cx}+{ay}+{by}+{cy}=(a+b+c)(x+y)$.
   \end{enumerate}
  \end{enumerate}
 \item se non riuscite ad individuare nessuno dei casi precedenti, provate ad applicare la \emph{regola di Ruffini}.
\end{enumerate}


Ricordiamo infine alcune formule per somma e differenza di potenze
dispari.

\[A^5+B^5=(A+B)\left(A^4-A^3B+A^2B^2-AB^3+B^4\right)\text{,}\]
\[A^5-B^5=(A-B)\left(A^4+A^3B+A^2B^2+AB^3+B^4\right)\text{,}\]
\[A^{7}\pm B^{7}=(A\pm B)\left(A^{6}\mp A^{5}B+A^{4}B^{2}\mp A^{3}B^{3}+A^{2}B^{4}\mp AB^{5}+B^{6}\right)\text{,}\]
\begin{equation*}
\begin{split}
 (A^{11}-B^{11})=(A-B)(A^{10}+A^{9}B+A^{8}B^{2}&+A^{7}B^{3}+A^{6}B^{4}+\\
 &+A^{5}B^{5}+A^{4}B^{6}+A^{3}B^{7}+A^{2}B^{8}+AB^{9}+B^{10}).\\
\end{split}
\end{equation*}

La differenza di due potenze ad esponente pari (uguale o differente tra le basi dei due addendi)
rientra nel caso della differenza di quadrati:

\[A^{8}-B^{10}=\left(A^{4}-B^{5}\right)\left(A^{4}+B^{5}\right).\]

In alcuni casi si può scomporre anche la somma di potenze pari:

\[A^{6}+B^{6}=\left(A^{2}\right)^{3}+\left(B^{2}\right)^{3}=\left(A^{2}+B^{2}\right)\left(A^{4}-A^{2}B^{2}+B^{4}\right)\text{,}\]
\[A^{10}+B^{10}=\left(A^{2}\right)^{5}+\left(B^{2}\right)^{5}=\left(A^{2}+B^{2}\right)\left(A^{8}-A^{6}B^{2}+A^{4}B^{4}-A^2B^6+B^8\right).\]

Proponiamo di seguito alcuni esercizi svolti o da completare in modo che
possiate acquisire una certa abilità nella scomposizione di polinomi.
%\newpage
\begin{exrig}
 \begin{esempio}
 $a^{2}x+5abx-36b^{2}x$.

Il polinomio ha~3 termini, è di terzo grado in~2 variabili, è
omogeneo;
tra i suoi monomi si ha~$\mcd= x$; effettuiamo il raccoglimento
totale:~$x\cdot\left(a^{2}+5ab-36b^{2}\right)$.
Il trinomio ottenuto come secondo fattore è di grado~2 in~2 variabili,
omogeneo e può essere riscritto
\[a^{2}+\left(5b\right)\cdot a-36b^{2}.\]
Proviamo a scomporlo come trinomio particolare:
cerchiamo due monomi~$m$ ed~$n$ tali che~$m+n=5b$
e~$m\cdot n=-36b^{2}$; i due monomi sono~$m=9b$
ed~$n=-4b$;

\[a^{2}x+5abx-36b^{2}x=x\cdot\left(a+9b\right)\cdot \left(a-4b\right).\]
 \end{esempio}

 \begin{esempio}
 $x^{2}+y^{2}+2xy-2x-2y$.

Facendo un raccoglimento parziale del coefficiente~2 tra gli ultimi tre
monomi otterremmo~$x^{2}+y^{2}+2\cdot(xy-x-y)$ su cui non possiamo
fare alcun ulteriore raccoglimento.

I primi tre termini formano però il quadrato di un binomio e tra gli
altri due possiamo raccogliere~$-2$, quindi~$(\mmevid{ev_rosso}{x+y})^{2}-2\cdot(\mmevid{ev_rosso}{x+y})$,
raccogliendo~$(x + y)$ tra i due termini si ottiene

\begin{equation*}
x^{2}+y^{2}+2xy-2x-2y=\left(x+y\right)\cdot \left(x+y-2\right).
\end{equation*}
 \end{esempio}

 \begin{esempio}
 $8a+10b+\left(1-4a-5b\right)^{2}-2$.

Tra i monomi sparsi possiamo raccogliere~2 a fattore comune
\[2\cdot \left(4a+5b-1\right)+\left(1-4a-5b\right)^{2}.\]

Osserviamo che la base del quadrato è l'opposto del polinomio contenuto
nel primo termine. Ma poiché numeri opposti hanno
lo stesso quadrato, possiamo cambiare il segno alla base del quadrato riscrivendo:
\[2\cdot\left(4a+5b-1\right)+\left(-1+4a+5b\right)^{2}.\]
Quindi si può mettere a fattore comune il termine $(4a+5b-1)$ ottenendo
\begin{align*}
8a+10b+(1-4a-5b)^{2}-2&=\left(4a+5b-1\right)\cdot\left(2-1+4a+5b\right)\\
&=\left(4a+5b-1\right)\cdot\left(1+4a+5b\right).
\end{align*}
 \end{esempio}

 \begin{esempio}
 $t^{{3}}-z^{{3}}+t^{2}-z^{2}$.

Il polinomio ha~4 termini, è di terzo grado in due variabili.
Poiché due monomi sono nella variabile $t$ e gli altri due nella
variabile $z$ potremmo subito effettuare un raccoglimento
parziale:~$t^{{3}}-z^{{3}}+t^{2}-z^{2}=t^{2}\cdot\left(t+1\right)-z^{2}\cdot \left(z+1\right)$,
che non permette un ulteriore passo. Occorre quindi un'altra
idea.

Notiamo che i primi due termini costituiscono una differenza di cubi e
gli altri due una differenza di quadrati; applichiamo le regole:

\begin{equation*}
t^{{3}}-z^{{3}}+t^{2}-z^{2}=\left(t-z\right)\cdot
\left(t^{2}+tz+z^{2}\right)+\left(t-z\right)\cdot
\left(t+z\right).
\end{equation*}
Ora effettuiamo il raccoglimento totale del fattore comune~$(t-z)$

\begin{equation*}
t^{3}-z^{3}+t^{2}-z^{2} = \left(t-z\right)\cdot
\left(t^{2}+tz+z^{2}+t+z\right).
\end{equation*}
 \end{esempio}

 \begin{esempio}
 $P(x)=x^{{3}}-7x-6$.
 \end{esempio}

Il polinomio ha~3 termini, è di~3{\textdegree} grado in una variabile.
Non possiamo utilizzare la regola del trinomio particolare poiché il
grado è~3. Procediamo con la regola di Ruffini: cerchiamo il numero che annulla 
il polinomio nell'insieme dei divisori del termine
noto~$D=\{\pm~1\text{,~}\pm~2\text{,~}\pm~3\text{,~}\pm~6\}$.

Per~$x=+1$ si ha~$P(+1)=(+1)^{3}-7\cdot (+1)-6=1-7-6\neq~0$.

Per~$x=-1$ si ha~$P(-1)=(-1)^{3}-7\cdot (-1)-6=-1+7-6=0$.

Quindi~$P(x)=\left(x+1\right)\cdot Q(x)$ con~$Q(x)$ polinomio di
secondo grado che determiniamo con la regola di Ruffini:

\begin{wrapfloat}{figure}{r}{0pt}
 % (c) 2012 Dimitrios Vrettos - d.vrettos@gmail.com

\begin{tikzpicture}[x=5mm,y=5mm]
\matrix (a)[matrix of nodes, nodes in empty cells,nodes={ text width=8mm, text depth=1mm, text centered}]{
&1&0&$-7$&$-6$\\
$-1$&&$-1$&1&6\\
&1&$-1$&$-6$&0\\
};  
\begin{scope}[blue]
\draw(a-1-2.north west)--(a-3-1.south east);
\draw(a-2-1.south west)--(a-2-5.south east);
\draw(a-1-4.north east)--(a-3-4.south east);
      \end{scope}
\end{tikzpicture}

\end{wrapfloat}
Pertanto:~$P(x)=x^{3}-7x-6=\left(x+1\right)\cdot \left(x^{2}-x-6\right)$.

Il polinomio quoziente è un trinomio di secondo grado; proviamo a
scomporlo come trinomio notevole.
Cerchiamo due numeri~$a$ e~$b$ tali che~$a+b=-1$ e~$a\cdot b=-6$.
I due numeri vanno cercati tra le coppie che hanno~$-6$ come prodotto,
precisamente~$(-6\text{,~}+1)$, $(-3\text{,~}+2)$, $(+6\text{,~}-1)$, $(+3\text{,~}-2)$. La coppia che fa al
caso nostro è~$(-3\text{,~}+2)$ quindi si
scompone~$Q(x)=x^{2}-x-6=\left(x-3\right)\cdot \left(x+2\right)$.

In definitiva~$x^{{3}}-7x-6=\left(x+1\right)\cdot (x-3)\cdot (x+2)$.

 \begin{esempio}
 $\left(m^{2}-4\right)^{2}-m^{2}-4m-4$.

Il polinomio ha~4 termini di cui il primo è un quadrato di un binomio;
negli altri tre possiamo raccogliere~$-1$;

\begin{equation*}
\left(m^{2}-4\right)^{2}-m^{2}-4m-4=\left(m^{2}-4\right)^{2}-\left(m^{2}+4m+4\right)
\end{equation*}

Notiamo che anche il secondo termine è un quadrato di un binomio, quindi:
\begin{equation}\label{eq:es_diff_quad}
\left(m^{2}-4\right)^{2}-\left(m+2\right)^{2}
\end{equation}
che si presenta come differenza di quadrati, allora diviene:
\[\left[\left(m^{2}-4\right)+\left(m+2\right)\right]\cdot
\left[\left(m^{2}-4\right)-\left(m+2\right)\right].\]

Eliminando le parentesi tonde~$\left(m^{2}+m-2\right)\cdot
\left(m^{2}-m-6\right)$.

I due fattori ottenuti si scompongono con la regola del trinomio. In
definitiva si ottiene:

\begin{align*}
(m^{2}+m-2)\cdot (m^{2}-m-6)&=\left(m+2\right)\cdot\left(m-1\right)\cdot \left(m-3\right)\cdot
\left(m+2\right)\\
&=\left(m+2\right)^{2}\cdot \left(m-1\right)\cdot
\left(m-3\right).
\end{align*}

Allo stesso risultato si poteva arrivare anche considerando che la \ref{eq:es_diff_quad} è sì una differenza di quadrati, ma a sua volta il termine $m^2-4$ è anch'esso una differenza di quadrati. Quindi si ha
\[\left(m^{2}-4\right)^{2}-\left(m+2\right)^{2} = (m+2)^2\cdot (m-2)^2-(m+2)^{2}\]
e mettendo in evidenza il fattore $\left(m+2\right)^{2}$ si può scrivere

\[\left(m+2\right)^{2}\cdot \left[ \left( m-2 \right)^2 -1 \right]. \]

Svolgendo le operazioni all'interno delle parentesi quadre si ottiene

\[\left(m+2\right)^{2}\cdot \left[ m^2-4m+4 -1 \right] = \left(m+2\right)^{2}\cdot \left[ m^2-4m+3 \right]. \]

A questo punto, per scomporre il fattore $m^2-4m+3$ si deve cercare una coppia di numeri interi, tali che la loro somma sia $-4$ ed il loro prodotto sia 3. La coppia di valori è $(-3\text{,~}-1)$ e quindi di può scrivere

\begin{align*}
\left(m^{2}-4\right)^2-m^2-4m-4 &=\left(m+2\right)^2\cdot \left( m^2-4m+3 \right)\\
&=\left(m+2\right)^{2}\cdot \left(m-1\right)\cdot \left(m-3\right).
\end{align*}


 \end{esempio}

 \begin{esempio}
 $\left(a-3\right)^{2}+\left(3a-9\right)\cdot\left(a+1\right)-\left(a^{2}-9\right)$.


\begin{equation*}
\left(a-3\right)^{2}+\left(3a-9\right)\cdot\left(a+1\right)-\left(a^{2}-9\right)%
=\left(a-3\right)^{2}+3\cdot \left(a-3\right)\cdot%
\left(a+1\right)-\left(a-3\right)\cdot \left(a+3\right).
\end{equation*}
Mettiamo a fattore comune~$(a-3)$:
\begin{equation*}
(a-3)\cdot \left[\left(a-3\right)+3\cdot
\left(a+1\right)-\left(a+3\right)\right].
\end{equation*}
Svolgiamo i calcoli nel secondo fattore e otteniamo:

\begin{equation*}
(a-3)(a-3+3a+3-a-3)=(a-3)(3a-3)=3(a-3)(a-1).
\end{equation*}
 \end{esempio}

 \begin{esempio}
 $a^4+a^{2}b^{2}+b^4$.

Osserva che per avere il quadrato del binomio occorre il doppio
prodotto, aggiungendo e togliendo~$a^{2}b^{2}$ otteniamo il doppio
prodotto cercato e al passaggio seguente ci troviamo con la differenza
di quadrati:

\[a^4+2a^2b^2+b^4 -a^2b^2=\left(a^2+b^2\right)^2-\left(ab\right)^2%
 =\left(a^2+b^2+ab\right)\left(a^2+b^2-ab\right).\]

 \end{esempio}

 \begin{esempio}
 $a^{5}+2a^{4}b+a^{3}b^{2}+a^{2}b^{3}+2ab^{4}+b^{5}$.

 \begin{align*}
  a^{5}+2a^{4}b+a^{3}b^{2}+a^{2}b^{3}+2ab^{4}+b^{5}&=a^{3}\left(a^{2}+2ab+b^{2}\right)+b^{3}\left(a^{2}+2ab+b^{2}\right)\\
  &=\left(a^{3}+b^{3}\right)\left(a^{2}+2ab+b^{2}\right)\\
  &=\left(a+b\right)\left(a^{2}-ab+b^{2}\right)\left(a+b\right)^{2}\\
  &=\left(a+b\right)^{3}\left(a^{2}-ab+b^{2}\right).
 \end{align*}
 \end{esempio}

 \begin{esempio}
 $a^{2}x^{2}+2ax^{2}-3x^{2}-4a^{2}-8a+12$.

 \begin{align*}
  a^{2}x^{2}+2ax^{2}-3x^{2}-4a^{2}-8a+12&=x^{2}\left(a^{2}+2a-3\right)-4\left(a^{2}+2a-3\right)\\
  &=\left(x^{2}-4\right)\left(a^{2}+2a-3\right)\\
  &=(x+2)(x-2)(a-1)(a+3).\\
 \end{align*}
 \end{esempio}

\end{exrig}
\newpage
% (c) 2012-2013 Claudio Carboncini - claudio.carboncini@gmail.com
% (c) 2012-2014 Dimitrios Vrettos - d.vrettos@gmail.com

\section{Esercizi}

%\subsection{Esercizi dei singoli paragrafi}

%\subsubsection*{17.1 - Equazioni di grado superiore al primo riducibili al primo grado}

\begin{esercizio}[\Ast]
\label{ese:17.1}
Risolvere le seguenti equazioni riconducendole a equazioni di primo grado.
\begin{multicols}{2}
\begin{enumeratea}
 \item $x^{2}+2x=0$;
 \item $x^{2}+2x-9x-18=0$;
 \item $2x^{2}-2x-4=0$;
 \item $4x^{2}+16x+16=0$.
\end{enumeratea}
\end{multicols}
\end{esercizio}

\begin{esercizio}[\Ast]
\label{ese:17.2}
Risolvere le seguenti equazioni riconducendole a equazioni di primo grado.
\begin{multicols}{2}
\begin{enumeratea}
 \item $x^{2}-3x-10=0$;
 \item $x^{2}+4x-12=0$;
 \item $3x^{2}-6x-9=0$;
 \item $x^{2}+5x-14=0$.
\end{enumeratea}
\end{multicols}
\end{esercizio}

\begin{esercizio}[\Ast]
\label{ese:17.3}
Risolvere le seguenti equazioni riconducendole a equazioni di primo grado.
\begin{multicols}{2}
\begin{enumeratea}
 \item $-3x^{2}-9x+30=0$;
 \item $-{\dfrac{3}{2}}x^{2}+\dfrac{3}{2}x+63=0$;
 \item $7x^{2}+14x-168=0$;
 \item $\dfrac{7}{2}x^{2}+7x-168=0$.
\end{enumeratea}
\end{multicols}
\end{esercizio}

\begin{esercizio}[\Ast]
\label{ese:17.4}
Risolvere le seguenti equazioni riconducendole a equazioni di primo grado.
\begin{multicols}{2}
\begin{enumeratea}
 \item $x^{4}-16x^{2}=0$;
 \item $2x^{3}+2x^{2}-20x+16=0$;
 \item $-2x^{3}+6x+4=0$;
 \item $-x^{6}+7x^{5}-10x^{4}=0$.
\end{enumeratea}
\end{multicols}
\end{esercizio}

\begin{esercizio}[\Ast]
\label{ese:17.5}
Risolvere le seguenti equazioni riconducendole a equazioni di primo grado.
\begin{multicols}{2}
\begin{enumeratea}
 \item $x^{3}-3x^{2}-13x+15=0$;
 \item $x^{2}+10x-24=0$;
 \item $2x^{3}-2x^{2}-24x=0$;
 \item $x^{4}-5x^{2}+4=0$.
\end{enumeratea}
\end{multicols}
\end{esercizio}

\begin{esercizio}[\Ast]
\label{ese:17.6}
Risolvere le seguenti equazioni riconducendole a equazioni di primo grado.
\begin{multicols}{2}
\begin{enumeratea}
 \item $-x^{3}-5x^{2}-x-5=0$;
 \item $\dfrac{3}{4}x^{3}-\dfrac{3}{4}x=0$;
 \item $-4x^{4}-28x^{3}+32x^{2}=0$;
 \item $-{\dfrac{6}{5}}x^{3}-\dfrac{6}{5}x^{2}+\dfrac{54}{5}x+\dfrac{54}{5}=0$.
\end{enumeratea}
\end{multicols}
\end{esercizio}

\begin{esercizio}[\Ast]
\label{ese:17.7}
Risolvere le seguenti equazioni riconducendole a equazioni di primo grado.
\begin{multicols}{2}
\begin{enumeratea}
 \item $-4x^{3}+20x^{2}+164x-180=0$;
 \item $5x^{3}+5x^{2}-80x-80=0$;
 \item $-3x^{3}+18x^{2}+3x-18=0$;
 \item $4x^{3}+8x^{2}-16x-32=0$.
\end{enumeratea}
\end{multicols}
\end{esercizio}

\begin{esercizio}[\Ast]
\label{ese:17.8}
Risolvere le seguenti equazioni riconducendole a equazioni di primo grado.
\begin{multicols}{2}
\begin{enumeratea}
 \item $x^{3}+11x^{2}+26x+16=0$;
 \item $2x^{3}+6x^{2}-32x-96=0$;
 \item $2x^{3}+16x^{2}-2x-16=0$;
 \item $-2x^{3}+14x^{2}-8x+56=0$.
\end{enumeratea}
\end{multicols}
\end{esercizio}
%\newpage
\begin{esercizio}[\Ast]
\label{ese:17.9}
Risolvere le seguenti equazioni riconducendole a equazioni di primo grado.
\begin{multicols}{2}
\begin{enumeratea}
 \item $2x^{3}+12x^{2}+18x+108=0$;
 \item $x^{4}-10x^{3}+35x^{2}-50x+24=0$;
 \item $-2x^{3}-12x^{2}+18x+28=0$;
 \item $-5x^{4}+125x^{2}+10x^{3}-10x-120=0$.
\end{enumeratea}
\end{multicols}
\end{esercizio}

\begin{esercizio}[\Ast]
\label{ese:17.10}
Risolvere le seguenti equazioni riconducendole a equazioni di primo grado.
\begin{multicols}{2}
\begin{enumeratea}
 \item $\dfrac{7}{6}x^{4}-\dfrac{161}{6}x^{2}-21x+\dfrac{140}{3}=0$;
 \item $(x^{2}-6x+8)(x^{5}-3x^{4}+2x^{3})=0$;
 \item $\left(25-4x^{2}\right)^{4}\left(3x-2\right)^{2}=0$;
 \item $(x-4)^{3}\left(2x^{3}-4x^{2}-8x+16\right)^{9}=0$.
\end{enumeratea}
\end{multicols}
\end{esercizio}

\begin{esercizio}[\Ast]
\label{ese:17.11}
Risolvere le seguenti equazioni riconducendole a equazioni di primo grado.
\begin{multicols}{2}
\begin{enumeratea}
 \item $(x^{3}-x)(x^{5}-9x^{3})(x^{2}+25)=0$;
 \item $x^{5}+3x^{4}-11x^{3}-27x^{2}+10x+24=0$;
 \item $2x^{2}-x-1=0$;
 \item $3x^{2}+5x-2=0$.
\end{enumeratea}
\end{multicols}
\end{esercizio}

\begin{esercizio}[\Ast]
\label{ese:17.12}
Risolvere le seguenti equazioni riconducendole a equazioni di primo grado.
\begin{multicols}{2}
\begin{enumeratea}
 \item $6x^{2}+x-2=0$;
 \item $2x^{3}-x^{2}-2x+1=0$;
 \item $3x^{3}-x^{2}-8x-4=0$;
 \item $8x^{3}+6x^{2}-5x-3=0$.
\end{enumeratea}
\end{multicols}
\end{esercizio}

\begin{esercizio}[\Ast]
\label{ese:17.13}
Risolvere le seguenti equazioni riconducendole a equazioni di primo grado.
\begin{multicols}{2}
\begin{enumeratea}
 \item $6x^{3}+x^{2}-10x+3=0$;
 \item $4x^{4}-8x^{3}-13x^{2}+2x+3=0$;
 \item $8x^{4}-10x^{3}-29x^{2}+40x-12=0$;
 \item $-12x^{3}+68x^{2}-41x+5=0$.
\end{enumeratea}
\end{multicols}
\end{esercizio}

\begin{esercizio}[\Ast]
\label{ese:17.14}
Risolvere la seguente equazione riconducendola a una equazione di primo grado.

$(x^{4}+3x^{3}-3x^{2}-11x-6)(4x^{6}-216x^{3}+2916)=0$;
\end{esercizio}

\subsection{Risposte}
%\begin{multicols}{2}
 \paragraph{17.1.}
a)~$\{0\text{,~}-2\}$;\quad b)~$\{-2\text{,~}+9\}$;\quad c)~$\{2\text{,~}-1\}$;\quad d)~$\{-2\}$.
\paragraph{17.2.}
a)~$\{5\text{,~}-2\}$;\quad b)~$\{2\text{,~}-6\}$;\quad c)~$\{3\text{,~}-1\}$;\quad d)~$\{2\text{,~}-7\}$.
\paragraph{17.3.}
a)~$\{2\text{,~}-5\}$;\quad b)~$\{7\text{,~}-6\}$;\quad c)~$\{4\text{,~}-6\}$;\quad d)~$\{6\text{,~}-8\}$.
\paragraph{17.4.}
a)~$\{0\text{,~}+4\text{,~}-4\}$;\quad b)~$\{1\text{,~}+2\text{,~}-4\}$;\quad c)~$\{2\text{,~}-1\}$;\quad d)~$\{0\text{,~}+2\text{,~}+5\}$.
\paragraph{17.5.}
a)~$\{1\text{,~}+5\text{,~}-3\}$;\quad b)~$\{2\text{,~}-12\}$;\quad c)~$\{0\text{,~}-3\text{,~}+4\}$;\quad d)~$\{1\text{,~}-1\text{,~}+2\text{,~}-2\}$.
\paragraph{17.6.}
a)~$\{-5\}$;\quad b)~$\{0\text{,~}+1\text{,~}-1\}$;\quad c)~$\{0\text{,~}+1\text{,~}-8\}$;\quad d)~$\{-1\text{,~}+3\text{,~}-3\}$.
\paragraph{17.7.}
a)~$\{1\text{,~}+9\text{,~}-5\}$;\quad b)~$\{-1\text{,~}+4\text{,~}-4\}$;\quad c)~$\{1\text{,~}-1\text{,~}+6\}$;\quad d)~$\{2\text{,~}-2\}$.
\paragraph{17.8.}
a)~$\{-1\text{,~}-2\text{,~}-8\}$;\quad b)~$\{4\text{,~}-4\text{,~}-3\}$,\quad c)~$\{1\text{,~}-1\text{,~}-8\}$;\quad d)~$\{7\}$.
\paragraph{17.9.}
a)~$\{-6\}$;\quad b)~$\{1\text{,~}2\text{,~}3\text{,~}4\}$;\quad c)~$\{-1\text{,~}2\text{,~}-7\}$;\quad d)~$\{1\text{,~}-1\text{,~}-4\text{,~}+6\}$.

\paragraph{17.10.}
a)~$\{1\text{,~}-2\text{,~}+5\text{,~}-4\}$;\quad b)~$\{0\text{,~}1\text{,~}2\text{,~}4\}$;\quad c)~$\left\{\frac{5}{2}\text{,~}-\frac{5}{2}\text{,~}\frac{2}{3}\right\}$;\quad d)~$\{4\text{,~}+2\text{,~}-2\}$.

\paragraph{17.11.}
a)~$\{0\text{,~}1\text{,~}-1\text{,~}3\text{,~}-3\}$;\quad
b)~$\{1\text{,~}{-1}\text{,~}{-2}\text{,~}3\text{,~}{-4}\}$;\quad c)~$\left\{1\text{,~}-\frac{1}{2}\right\}$;\quad d)~$\left\{-2\text{,~}\frac{1}{3}\right\}$.

\paragraph{17.12.}
a)~$\left\{\frac{1}{2}\text{,~}-\frac{2}{3}\right\}$;\quad b)~$\left\{1\text{,~}-1\text{,~}\frac{1}{2}\right\}$;\quad c)~$\left\{-1\text{,~}2\text{,~}-\frac{2}{3}\right\}$;\quad d)~$\left\{-1\text{,~}-\frac{1}{2}\text{,~}\frac{3}{4}\right\}$.

\paragraph{17.13.}
a)~$\left\{1\text{,~}\frac{1}{3}\text{,~}-\frac{3}{2}\right\}$;\quad b)~$\left\{3\text{,~}-1\text{,~}\frac{1}{2}\text{,~}-\frac{1}{2}\right\}$;\quad c)~$\left\{2\text{,~}-2\text{,~}\frac{3}{4}\text{,~}\frac{1}{2}\right\}$;\quad d)~$\left\{5\text{,~}\frac{1}{2}\text{,~}\frac{1}{6}\right\}$.

\paragraph{17.14.}
$\{-1\text{,~}+2\text{,~}+3\text{,~}-3\}$.
%\end{multicols}
\cleardoublepage
