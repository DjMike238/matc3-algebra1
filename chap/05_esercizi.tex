% (c) 2012 Dimitrios Vrettos - d.vrettos@gmail.com
% (c) 2012 Claudio Carboncini - claudio.carboncini@gmail.com
\section{Esercizi}
\subsection{Esercizi dei singoli paragrafi}
\subsubsection*{\thechapter.1 - Insiemi ed elementi}

\begin{esercizio}
 \label{ese:5.1}
 Barra con una crocetta i raggruppamenti che ritieni siano degli insiemi.
 \begin{multicols}{2}
 \begin{enumeratea}
\item I fiumi più lunghi d'Italia;
\item le persone con più di~30 anni;
\item i numeri~1, 20, 39, 43, 52;
\item i libri più pesanti nella tua cartella;
\item i punti di una retta;
\item gli animali con~2 zampe;
\item le vocali dell'alfabeto italiano;
\item i professori bravi;
\item i gatti con due code;
\item i calciatori che hanno fatto pochi gol.
\end{enumeratea}
\end{multicols}
\end{esercizio}

%%%%%%%%%%%%%%%%%%%%%%%%%%%%%%%%%%%%%%%%%%%%%%%%%%%%%%%%%%%%%%%%%%%%%
\begin{esercizio}
 \label{ese:5.2}
Considerando l'insieme $A$ delle lettere dell'alfabeto italiano, per ciascuno dei seguenti casi inserisci il simbolo adatto fra ``$\in$'' e ``$\notin$''.

b \ldots $A$, i \ldots $A$, j \ldots $A$, e \ldots $A$, w \ldots $A$, z \ldots $A$.
\end{esercizio}

\begin{esercizio}
\label{ese:5.3}
Le vocali delle parole che seguono formano insiemi uguali, tranne in un caso. Quale?
\begin{center}
 \boxA\quad sito\quad\boxB\quad micio\quad\boxC\quad zitto\quad\boxD\quad fiocco\quad\boxE\quad lecito\quad\boxF\quad dito.
\end{center}
\end{esercizio}

\begin{esercizio}
\label{ese:5.4}
Individua tra i seguenti insiemi quelli che sono uguali:
\begin{multicols}{2}
\begin{enumeratea}
 \item vocali della parola ``SASSO'';
 \item consonanti della parola ``SASSO'';
 \item vocali della parola ``PIETRA'';
 \item vocali della parola ``PASSO''.
\end{enumeratea}
\end{multicols}
\end{esercizio}

\begin{esercizio}
\label{ese:5.5}
Quali delle seguenti frasi rappresentano criteri oggettivi per individuare un insieme? Spiega perché.
\TabPositions{8.5cm}
\begin{enumeratea}
\item Le città che distano meno di~100 km da Lecce; \tab\boxV\quad\boxF
\item i laghi d'Italia;  \tab\boxV\quad\boxF
\item le città vicine a Roma; \tab\boxV\quad\boxF
\item i calciatori della Juventus;  \tab\boxV\quad\boxF
\item i libri di Mauro;  \tab\boxV\quad\boxF
\item i professori bassi della tua scuola;  \tab\boxV\quad\boxF
\item i tuoi compagni di scuola il cui nome inizia per M; \tab\boxV\quad\boxF
\item i tuoi compagni di classe che sono gentili; \tab\boxV\quad\boxF
\item gli zaini neri della tua classe.  \tab\boxV\quad\boxF
\end{enumeratea}
\end{esercizio}

\begin{esercizio}
\label{ese:5.6}
Scrivi al posto dei puntini il simbolo mancante tra ``$\in$'' e ``$\notin$''.

\begin{enumeratea}
\item La Polo \ldots\ldots all'insieme delle automobili Fiat;
\item il cane \ldots\ldots all'insieme degli animali domestici;
\item la Puglia \ldots\ldots all'insieme delle regioni italiane;
\item Firenze \ldots\ldots all'insieme delle città francesi;
\item il numero~10 \ldots\ldots all'insieme dei numeri naturali;
\item il numero~3 \ldots\ldots all'insieme dei numeri pari.
\end{enumeratea}
\end{esercizio}
\pagebreak
\begin{esercizio}
\label{ese:5.7}
Quali delle seguenti proprietà sono caratteristiche per un insieme?
\TabPositions{8.5cm}
\begin{enumeratea}
\item Essere una città italiana il cui nome inizia per W; \tab\boxV\quad\boxF
\item essere un bravo cantante; \tab\boxV\quad\boxF
\item essere un monte delle Alpi;  \tab\boxV\quad\boxF
\item essere un ragazzo felice; \tab\boxV\quad\boxF
\item essere un numero naturale grande;\tab\boxV\quad\boxF
\item essere un ragazzo nato nel~1985; \tab\boxV\quad\boxF
\item essere un alunno della classe~1\textsuperscript{a}C; \tab\boxV\quad\boxF
\item essere una lettera dell'alfabeto inglese; \tab\boxV\quad\boxF
\item essere una retta del piano; \tab\boxV\quad\boxF
\item essere un libro interessante della biblioteca; \tab\boxV\quad\boxF
\item essere un italiano vivente nato nel~1850; \tab\boxV\quad\boxF
\item essere un italiano colto. \tab\boxV\quad\boxF
\end{enumeratea}
\end{esercizio}
%eliminato da Antonio
%\begin{esercizio}
%\label{ese:5.8}
%Scrivi al posto dei puntini il simbolo mancante tra ``$=$'' e ``${\neq}$''.
%\begin{enumeratea}
%\item L'insieme delle lettere della parola ``CANE'' e della parola ``PANE'' sono \ldots\ldots;
%\item l'insieme delle vocali della parola ``INSIEME'' e della parola ``MIELE'' sono \ldots\ldots;
%\item l'insieme delle consonanti della parola ``LETTO'' e della parola ``TETTO'' sono \ldots\ldots;
%\item l'insieme delle lettere della parola ``CONTRO'' e della parola ``TRONCO'' sono \ldots\ldots;
%\item l'insieme delle vocali della parola ``LIBRO'' e della parola ``MINISTRO'' sono \ldots\ldots;
%\item l'insieme delle vocali della parola ``DIARIO'' e della parola ``RAMO'' sono \ldots\ldots;
%\item l'insieme delle lettere della parola ``MOUSE'' e della parola ``MUSEO'' sono \ldots\ldots;
%\item l'insieme delle consonanti della parola ``SEDIA'' e della parola ``ADESSO'' sono \ldots\ldots;
%\item l'insieme dei numeri pari minori di~5 e l'insieme vuoto sono \ldots\ldots;
%\item l'insieme dei numeri pari e l'insieme dei multipli di~2 sono \ldots\ldots
%\end{enumeratea}
%\end{esercizio}

\begin{esercizio}
\label{ese:5.8}
Le stelle dell'universo formano un insieme. Le stelle visibili a occhio nudo formano un insieme? Spiega il tuo punto di vista.
\end{esercizio}

\subsubsection*{\thechapter.2 - Insieme vuoto, insieme universo, cardinalità}
\begin{esercizio}
\label{ese:5.9}
Indica se gli insiemi~$G =\{\text{gatti con~6 zampe}\}$ e~$P = \{\text{polli con~2 zampe}\}$ sono o non sono vuoti.
\end{esercizio}

\begin{esercizio}
\label{ese:5.10}
Barra con una croce gli insiemi vuoti.
\begin{enumeratea}
 \item L'insieme dei numeri positivi minori di~0;
 \item l'insieme dei numeri negativi minori di~100;
 \item l'insieme dei numeri pari minori di~100;
 \item l'insieme delle capitali europee della regione Lombardia;
 \item l'insieme dei triangoli con quattro angoli;
 \item l'insieme delle capitali italiane del Lazio;
 \item l'insieme dei punti di intersezione di due rette parallele.
 \end{enumeratea}
\end{esercizio}

\begin{esercizio}
\label{ese:5.11}
Quali delle seguenti scritture sono corrette per indicare
l'insieme vuoto?
\begin{center}
 \boxA\quad~$\emptyset $ \quad\boxB\quad~0 \quad\boxC\quad~$\{\emptyset \}$ \quad\boxD\quad~$\{0\}$ \quad\boxE\quad \{ \}.
\end{center}
\end{esercizio}
\pagebreak
\begin{esercizio}
\label{ese:5.12}
Quali dei seguenti insiemi sono vuoti? Per gli insiemi non vuoti indica la cardinalità, ossia il numero di elementi che contiene.
\begin{enumeratea}
\item L'insieme degli uccelli con~6 ali;
\item l'insieme delle lettere della parola ``VOLPE'';
\item l'insieme dei cani con~5 zampe;
\item l'insieme delle vocali della parola ``COCCODRILLO'';
\item l'insieme delle vocali dell'alfabeto italiano;
\item l'insieme degli abitanti della luna;
\item l'insieme dei numeri sulla tastiera del telefonino.
\end{enumeratea}
\end{esercizio}

\begin{esercizio}
\label{ese:5.13}
Scrivi per ciascun insieme un possibile insieme universo.
\begin{enumeratea}
\item l'insieme dei rettangoli;
\item l'insieme dei multipli di~3;
\item l'insieme delle lettere della parola ``MATEMATICA'';
\item l'insieme dei libri di matematica;
\item l'insieme dei ragazzi che hanno avuto un'insufficienza in matematica.
\end{enumeratea}
\end{esercizio}

\begin{esercizio}
\label{ese:5.14}
Dato l'insieme~$A = \{\text{0, 2, 5}\}$ determina se le seguenti affermazioni sono vere o false.
\TabPositions{2.5cm}
\begin{enumeratea}
\item $0\in A$. \tab\boxV\quad\boxF
\item $5\in A$. \tab\boxV\quad\boxF
\item $\emptyset \in A$. \tab\boxV\quad\boxF
\item $A\in A$. \tab\boxV\quad\boxF
\item $\np{3,5}\in A$. \tab\boxV\quad\boxF
\end{enumeratea}
\end{esercizio}
